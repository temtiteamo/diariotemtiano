% Options for packages loaded elsewhere
\PassOptionsToPackage{unicode}{hyperref}
\PassOptionsToPackage{hyphens}{url}
%
\documentclass[
]{book}
\usepackage{amsmath,amssymb}
\usepackage{iftex}
\ifPDFTeX
  \usepackage[T1]{fontenc}
  \usepackage[utf8]{inputenc}
  \usepackage{textcomp} % provide euro and other symbols
\else % if luatex or xetex
  \usepackage{unicode-math} % this also loads fontspec
  \defaultfontfeatures{Scale=MatchLowercase}
  \defaultfontfeatures[\rmfamily]{Ligatures=TeX,Scale=1}
\fi
\usepackage{lmodern}
\ifPDFTeX\else
  % xetex/luatex font selection
\fi
% Use upquote if available, for straight quotes in verbatim environments
\IfFileExists{upquote.sty}{\usepackage{upquote}}{}
\IfFileExists{microtype.sty}{% use microtype if available
  \usepackage[]{microtype}
  \UseMicrotypeSet[protrusion]{basicmath} % disable protrusion for tt fonts
}{}
\makeatletter
\@ifundefined{KOMAClassName}{% if non-KOMA class
  \IfFileExists{parskip.sty}{%
    \usepackage{parskip}
  }{% else
    \setlength{\parindent}{0pt}
    \setlength{\parskip}{6pt plus 2pt minus 1pt}}
}{% if KOMA class
  \KOMAoptions{parskip=half}}
\makeatother
\usepackage{xcolor}
\usepackage{longtable,booktabs,array}
\usepackage{calc} % for calculating minipage widths
% Correct order of tables after \paragraph or \subparagraph
\usepackage{etoolbox}
\makeatletter
\patchcmd\longtable{\par}{\if@noskipsec\mbox{}\fi\par}{}{}
\makeatother
% Allow footnotes in longtable head/foot
\IfFileExists{footnotehyper.sty}{\usepackage{footnotehyper}}{\usepackage{footnote}}
\makesavenoteenv{longtable}
\usepackage{graphicx}
\makeatletter
\def\maxwidth{\ifdim\Gin@nat@width>\linewidth\linewidth\else\Gin@nat@width\fi}
\def\maxheight{\ifdim\Gin@nat@height>\textheight\textheight\else\Gin@nat@height\fi}
\makeatother
% Scale images if necessary, so that they will not overflow the page
% margins by default, and it is still possible to overwrite the defaults
% using explicit options in \includegraphics[width, height, ...]{}
\setkeys{Gin}{width=\maxwidth,height=\maxheight,keepaspectratio}
% Set default figure placement to htbp
\makeatletter
\def\fps@figure{htbp}
\makeatother
\setlength{\emergencystretch}{3em} % prevent overfull lines
\providecommand{\tightlist}{%
  \setlength{\itemsep}{0pt}\setlength{\parskip}{0pt}}
\setcounter{secnumdepth}{-\maxdimen} % remove section numbering
\ifLuaTeX
\usepackage[bidi=basic]{babel}
\else
\usepackage[bidi=default]{babel}
\fi
\babelprovide[main,import]{spanish}
% get rid of language-specific shorthands (see #6817):
\let\LanguageShortHands\languageshorthands
\def\languageshorthands#1{}
\usepackage{booktabs}
\usepackage{amsthm}
\makeatletter
\def\thm@space@setup{%
  \thm@preskip=8pt plus 2pt minus 4pt
  \thm@postskip=\thm@preskip
}
\makeatother
\usepackage{fancyhdr}
\usepackage{lipsum}
\pagestyle{plain}
\fancyhead[CO,CE]{\thepage}
\fancyfoot[LE,RO]{\thepage}
\ifLuaTeX
  \usepackage{selnolig}  % disable illegal ligatures
\fi
\usepackage[]{natbib}
\bibliographystyle{plainnat}
\usepackage{bookmark}
\IfFileExists{xurl.sty}{\usepackage{xurl}}{} % add URL line breaks if available
\urlstyle{same}
\hypersetup{
  pdftitle={Diario Temtiano edición 28.11.2024},
  pdflang={es-es},
  hidelinks,
  pdfcreator={LaTeX via pandoc}}

\title{Diario Temtiano edición 28.11.2024}
\author{}
\date{\vspace{-2.5em}2024-11-28}

\begin{document}
\maketitle

{
\setcounter{tocdepth}{1}
\tableofcontents
}
\chapter*{Notas de edición}\label{notas-de-ediciuxf3n}
\addcontentsline{toc}{chapter}{Notas de edición}

\section*{Novedades en esta edición}\label{novedades-en-esta-ediciuxf3n}
\addcontentsline{toc}{section}{Novedades en esta edición}

\begin{itemize}
\tightlist
\item
  Agregados tres textos más: ``Antagonismos incorporados (Decimonoveno libro, capítulo XL)'', ``Procesos con repuestos (Decimonoveno libro, capítulo XLI)'', y ``Espejismos dinámicos (Decimonoveno libro, capítulo XLII)'', ampliando así el intervalo de tiempo comprendido por la obra hasta el 31 de octubre de 2024.
\end{itemize}

\section*{Novedades en ediciones anteriores}\label{novedades-en-ediciones-anteriores}
\addcontentsline{toc}{section}{Novedades en ediciones anteriores}

\subsection*{Diario Temtiano edición 08.08.2024}\label{diario-temtiano-ediciuxf3n-08.08.2024}
\addcontentsline{toc}{subsection}{Diario Temtiano edición 08.08.2024}

\begin{itemize}
\item
  Agregados dos textos más: ``Clarificaciones de composición (Decimonoveno libro, capítulo XXXVIII)'', y ``Encimados particulares (Decimonoveno libro, capítulo XXXIX)'', ampliando así el intervalo de tiempo comprendido por la obra hasta el 30 de junio de 2024.
\item
  Modificadas tres secciones e incluida una más en: ``Más sobre Diario Temtiano'', dando mayor precisión a las referencias breves sobre la obra.
\end{itemize}

\subsection*{Diario Temtiano edición 15.04.2024}\label{diario-temtiano-ediciuxf3n-15.04.2024}
\addcontentsline{toc}{subsection}{Diario Temtiano edición 15.04.2024}

\begin{itemize}
\item
  Agregados tres textos más: ``Cruces conmovedores (Decimonoveno libro, capítulo XXXV)'', ``Corrientes y contaminaciones (Decimonoveno libro, capítulo XXXVI)'', y ``Inmensidades inquietantes (Decimonoveno libro, capítulo XXXVII)'', ampliando así el intervalo de tiempo comprendido por la obra hasta el 4 de abril de 2024.
\item
  Modificadas tres secciones e incluida una más en: ``Más sobre Diario Temtiano'', dando mayor precisión a las referencias breves sobre la obra.
\end{itemize}

\subsection*{Diario Temtiano edición 15.02.2024}\label{diario-temtiano-ediciuxf3n-15.02.2024}
\addcontentsline{toc}{subsection}{Diario Temtiano edición 15.02.2024}

\begin{itemize}
\tightlist
\item
  Agregados tres textos más: ``Compartiendo a distinta escala (Decimonoveno libro, capítulo XXXII)'', ``Marcadores en trayectoria (Decimonoveno libro, capítulo XXXIII)'', y ``Para seguir estando (Decimonoveno libro, capítulo XXXIV)'', ampliando así el intervalo de tiempo comprendido por la obra hasta el 3 de febrero de 2024.
\end{itemize}

\subsection*{Diario Temtiano edición 22.12.2023}\label{diario-temtiano-ediciuxf3n-22.12.2023}
\addcontentsline{toc}{subsection}{Diario Temtiano edición 22.12.2023}

\begin{itemize}
\item
  Agregados dos textos más: ``Accidentes de reserva (Decimonoveno libro, capítulo XXVIII)'', y ``Contrastes de lo esperado (Decimonoveno libro, capítulo XXIX)'', ampliando así el intervalo de tiempo comprendido por la obra hasta el 5 de noviembre de 2023.
\item
  Agregado un agradecimiento más: ``Devox.me'', incluyendo así una nueva dirección en el listado de sitios que recibieron la difusión de la obra.
\end{itemize}

\subsection*{Diario Temtiano edición 23.11.2023}\label{diario-temtiano-ediciuxf3n-23.11.2023}
\addcontentsline{toc}{subsection}{Diario Temtiano edición 23.11.2023}

\begin{itemize}
\item
  Agregados dos textos más: ``Accidentes de reserva (Decimonoveno libro, capítulo XXVIII)'', y ``Contrastes de lo esperado (Decimonoveno libro, capítulo XXIX)'', ampliando así el intervalo de tiempo comprendido por la obra hasta el 5 de noviembre de 2023.
\item
  Agregado un agradecimiento más: ``Devox.me'', incluyendo así una nueva dirección en el listado de sitios que recibieron la difusión de la obra.
\end{itemize}

\subsection*{Diario Temtiano edición 14.10.2023}\label{diario-temtiano-ediciuxf3n-14.10.2023}
\addcontentsline{toc}{subsection}{Diario Temtiano edición 14.10.2023}

\begin{itemize}
\item
  Modificados múltiples textos, en específico las introducciones de los siguientes libros: ``Cuando tal vez se vea qué realmente queda para siempre (Tercer libro)'', ``El durante cuando los días están más que contados (Cuarto libro)'' , y ``Mientras se está a la deriva entre la vida y la incertidumbre (Quinto libro)''.
\item
  Agregados dos textos más: ``Expertos del reciclaje (Decimonoveno libro, capítulo XXVI)'', y ``Importaciones de brusquedad y capacidad (Decimonoveno libro, capítulo XXVII)'', ampliando así el intervalo de tiempo comprendido por la obra hasta el 1 de octubre de 2023.
\end{itemize}

\subsection*{Diario Temtiano edición 01.09.2023}\label{diario-temtiano-ediciuxf3n-01.09.2023}
\addcontentsline{toc}{subsection}{Diario Temtiano edición 01.09.2023}

\begin{itemize}
\item
  Modificados múltiples textos, en específico los pertenecientes a: ``Los delimitados recorridos del paralelismo residencial (Octavo libro)''.
\item
  Agregados tres textos más: ``El énfasis extrafuncional (Decimonoveno libro, capítulo XXIII)'', ``Visiones de desubicada revolución (Decimonoveno libro, capítulo XXIV)'', y ``Devoradores de la improcedencia (Decimonoveno libro, capítulo XXV)'', ampliando así el intervalo de tiempo comprendido por la obra hasta el 18 de agosto de 2023.
\end{itemize}

\subsection*{Diario Temtiano edición 27.07.2023}\label{diario-temtiano-ediciuxf3n-27.07.2023}
\addcontentsline{toc}{subsection}{Diario Temtiano edición 27.07.2023}

\begin{itemize}
\item
  Modificados múltiples textos ampliamente, en específico los pertenecientes a: ``Las inhibidas andanzas bajo la apta nueva normalidad (Séptimo libro)''.
\item
  Agregados dos textos más: ``Clarificaciones de oscura legitimidad (Decimonoveno libro, capítulo XXI)'', y ``De lo excepcional a lo corriente (Decimonoveno libro, capítulo XXII)'', ampliando así el intervalo de tiempo comprendido por la obra hasta el 3 de julio de 2023.
\end{itemize}

\subsection*{Diario Temtiano edición 19.06.2023}\label{diario-temtiano-ediciuxf3n-19.06.2023}
\addcontentsline{toc}{subsection}{Diario Temtiano edición 19.06.2023}

\begin{itemize}
\item
  Modificados múltiples textos, ligeramente los que incluyeran términos especiales definidos por el contexto histórico, y también ampliamente los pertenecientes a: ``La época de las infinitas prórrogas disfuncionales (Sexto libro)''.
\item
  Agregados dos textos más: ``El rincón comunal no autosostenible (Decimonoveno libro, capítulo XIX)'', y ``Los procesos ratificados (Decimonoveno libro, capítulo XX)'', ampliando así el intervalo de tiempo comprendido por la obra hasta el 6 de junio de 2023.
\end{itemize}

\subsection*{Diario Temtiano edición 17.05.2023}\label{diario-temtiano-ediciuxf3n-17.05.2023}
\addcontentsline{toc}{subsection}{Diario Temtiano edición 17.05.2023}

\begin{itemize}
\item
  Modificados y combinados múltiples textos, en especial los pertenecientes a: ``Mientras se está a la deriva entre la vida y la incertidumbre (Quinto libro)''.
\item
  Agregados tres textos más: ``El sueño desde la punta flotante (Decimonoveno libro, capítulo XVI)'', ``El refuerzo abandónico (Decimonoveno libro, capítulo XVII)'', ``El bocado que los técnicos reparten (Decimonoveno libro, capítulo XVIII)'', ampliando así el intervalo de tiempo comprendido por la obra hasta el 26 de abril de 2023.
\end{itemize}

\subsection*{Diario Temtiano edición 06.04.2023}\label{diario-temtiano-ediciuxf3n-06.04.2023}
\addcontentsline{toc}{subsection}{Diario Temtiano edición 06.04.2023}

\begin{itemize}
\item
  Modificadas y simplificadas las referencias de ediciones anteriores, adaptado el formato y nivel de precisión a las más recientes.
\item
  Modificados y combinados múltiples textos, pertenecientes a: ``Las secuelas de una supuesta muerte no anunciada (Primer libro)'', y ``El alcance de un experimentado y recargado comienzo prometedor (Segundo libro)''.
\end{itemize}

\subsection*{Diario Temtiano edición 20.03.2023}\label{diario-temtiano-ediciuxf3n-20.03.2023}
\addcontentsline{toc}{subsection}{Diario Temtiano edición 20.03.2023}

\begin{itemize}
\item
  Modificados y combinados múltiples textos, en especial los pertenecientes a: ``Cuando tal vez se vea qué realmente queda para siempre (Tercer libro)'', y ``El durante cuando los días están más que contados (Cuarto libro)''.
\item
  Agregados dos textos más: ``Las cariñosas elaboraciones sintéticas (Decimonoveno libro, capítulo XIV)'', y ``Los arcoíris de los arcoíris (Decimonoveno libro, capítulo XV)``, ampliando así el intervalo de tiempo comprendido por la obra hasta el 13 de marzo de 2023.
\end{itemize}

\subsection*{Diario Temtiano edición 28.02.2023}\label{diario-temtiano-ediciuxf3n-28.02.2023}
\addcontentsline{toc}{subsection}{Diario Temtiano edición 28.02.2023}

\begin{itemize}
\tightlist
\item
  Agregados seis textos más: ``El indicador redundancia (Decimonoveno libro, capítulo VIII)'', ``Retratos en abundancia (Decimonoveno libro, capítulo IX)``, ``La insólita precisión de los titulares (Decimonoveno libro, capítulo X)``, ``La causa validando su medida preferida (Decimonoveno libro, capítulo XI)'', ``La serie del trasnochado en extremo (Decimonoveno libro, capítulo XII)``, y ``Cultura de retribución (Decimonoveno libro, capítulo XIII)'', ampliando así el intervalo de tiempo comprendido por la obra hasta el 14 de febrero de 2023.
\end{itemize}

\subsection*{Diario Temtiano edición 04.02.2023}\label{diario-temtiano-ediciuxf3n-04.02.2023}
\addcontentsline{toc}{subsection}{Diario Temtiano edición 04.02.2023}

\begin{itemize}
\item
  Agregados nueve textos más: ``Atraco y trama (Decimoctavo libro, capítulo XXII)'', ``Los regulares pantallazos del testarudo seguidor viajando al más allá (Decimonoveno libro)``, ``El honor compartido (Decimonoveno libro, capítulo I)``, ``Normalidad en libertad condicional (Decimonoveno libro, capítulo II)'', ``Próximo al drástico golpe de timón (Decimonoveno libro, capítulo III)``, ``La magia albiceleste (Decimonoveno libro, capítulo IV)'', ``El reconocimiento de los conquistadores (Decimonoveno libro, capítulo V)'', ``Rehabilitación del empoderamiento audiovisual (Decimonoveno libro, capítulo VI)'', y ``Noble cumbre del anhelo colectivo (Decimonoveno libro, capítulo VII)'', ampliando así el intervalo de tiempo comprendido por la obra hasta el 2 de enero de 2023.
\item
  Agregada una mención correspondiente en la sección ``Agradecimientos'', referente al sitio Devox.uno.
\end{itemize}

\subsection*{Diario Temtiano edición 18.01.2023}\label{diario-temtiano-ediciuxf3n-18.01.2023}
\addcontentsline{toc}{subsection}{Diario Temtiano edición 18.01.2023}

\begin{itemize}
\tightlist
\item
  Agregados ocho textos más: ``Simulaciones icónicas (Decimoctavo libro, capítulo XIV)'', ``Lo mejor de los arqueólogos (Decimoctavo libro, capítulo XV)``, ``Conscientes en las profundidades (Decimoctavo libro, capítulo XVI)``, ``A falta de un hasta mañana (Decimoctavo libro, capítulo XVII)'' ``Supremacía comprensiva (Decimoctavo libro, capítulo XVIII)``, ``La suerte del nominado por descarte (Decimoctavo libro, capítulo XIX)'', ``La apatía a la grandeza (Decimoctavo libro, capítulo XX)'', y ``Las grietas en la red de la pureza (Decimoctavo libro, capítulo XXI)'', ampliando así el intervalo de tiempo comprendido por la obra hasta el 17 de noviembre de 2022.
\end{itemize}

\subsection*{Diario Temtiano edición 05.01.2023}\label{diario-temtiano-ediciuxf3n-05.01.2023}
\addcontentsline{toc}{subsection}{Diario Temtiano edición 05.01.2023}

\begin{itemize}
\tightlist
\item
  Agregados cinco textos más: ``No exactamente como solía ser (Decimoctavo libro, capítulo IX)'', ``Rectitud que tarde se contagia por completo (Decimoctavo libro, capítulo X)``, ``Las laxas medidas antipopulares (Decimoctavo libro, capítulo XI)``, ``El espectáculo visual como cadáver (Decimoctavo libro, capítulo XII)'', y ``La intermitente voz de la unión (Decimoctavo libro, capítulo XIII)``, ampliando así el intervalo de tiempo comprendido por la obra hasta el 23 de octubre de 2022.
\end{itemize}

\subsection*{Diario Temtiano edición 25.12.2022}\label{diario-temtiano-ediciuxf3n-25.12.2022}
\addcontentsline{toc}{subsection}{Diario Temtiano edición 25.12.2022}

\begin{itemize}
\item
  Agregados seis textos más: ``Formaciones en peregrinación (Decimoctavo libro, capítulo III)'', ``El ánimo y el clasismo (Decimoctavo libro, capítulo IV)``, ``En picada no pronunciada (Decimoctavo libro, capítulo V)``, ``Retazos complementarios de los alrededores (Decimoctavo libro, capítulo VI)'', ``La declaratoria de versatilidad (Decimoctavo libro, capítulo VII)'', y ``Como interrumpido balance prematuro (Decimoctavo libro, capítulo VIII)``, ampliando así el intervalo de tiempo comprendido por la obra hasta el 26 de setiembre de 2022.
\item
  Incluida en la sección ``Agradecimientos'' mención a los principales gentilicios y denominaciones faltantes.
\end{itemize}

\subsection*{Diario Temtiano edición 07.12.2022}\label{diario-temtiano-ediciuxf3n-07.12.2022}
\addcontentsline{toc}{subsection}{Diario Temtiano edición 07.12.2022}

\begin{itemize}
\item
  Agregados seis textos más: ``La reposición de desconocidas baldosas rotas (Decimoséptimo libro, capítulo II)'', ``La alineación de los escalones (Decimoséptimo libro, capítulo III)``, ``Direcciones contrarias (Decimoséptimo libro, capítulo IV)``, ``La cadencia de las menospreciadas periferias con su integración al radar (Decimoctavo libro)'', ``Las inquietudes de engañosa expansión (Decimoctavo libro, capítulo I)'', y ``La persecuta y la mentira en su plano secundario (Decimoctavo libro, capítulo II)``, ampliando así el intervalo de tiempo comprendido por la obra hasta el 4 de setiembre de 2022.
\item
  Modificados dos títulos anteriores, por: ``El periodo de expectativa y reconexión con las garantías y paces (Decimoséptimo libro)'', ``Del revestido lazo cimental (Decimoséptimo libro, capítulo I)''.
\end{itemize}

\subsection*{Diario Temtiano edición 21.11.2022}\label{diario-temtiano-ediciuxf3n-21.11.2022}
\addcontentsline{toc}{subsection}{Diario Temtiano edición 21.11.2022}

\begin{itemize}
\tightlist
\item
  Agregados cinco textos más: ``Los procesos míticos y no míticos (Decimosexto libro, capítulo IX)``, ``Los inconfundibles deseos de prosperidad (Decimosexto libro, capítulo X)``, ``La oscuridad del cráneo viviente (Decimosexto libro, capítulo XI)'', ``El periodo de espera y reconexión con las garantías y paces (Decimoséptimo libro)'', y ``Del revestido lazo con el soporte (Decimoséptimo libro, capítulo I)``, ampliando así el intervalo de tiempo comprendido por la obra hasta el 6 de agosto de 2022.
\end{itemize}

\subsection*{Diario Temtiano edición 09.11.2022}\label{diario-temtiano-ediciuxf3n-09.11.2022}
\addcontentsline{toc}{subsection}{Diario Temtiano edición 09.11.2022}

\begin{itemize}
\tightlist
\item
  Agregados cinco textos más: ``La falla del seguimiento sistemático (Decimosexto libro, capítulo IV)``, ``Los perseverantes momentos capturados (Decimosexto libro, capítulo V)``, ``Claves de visibilidad (Decimosexto libro, capítulo VI)'', ``Los frutos de la vulnerabilidad (Decimosexto libro, capítulo VII)'', y ``El calco de las profundas raíces (Decimosexto libro, capítulo VIII)``, ampliando así el intervalo de tiempo comprendido por la obra hasta el 1 de agosto de 2022.
\end{itemize}

\subsection*{Diario Temtiano edición 14.10.2022}\label{diario-temtiano-ediciuxf3n-14.10.2022}
\addcontentsline{toc}{subsection}{Diario Temtiano edición 14.10.2022}

\begin{itemize}
\item
  Realizadas modificaciones mínimas en un texto: ``Coleccionistas de pieles (Decimocuarto libro, capítulo VII)''.
\item
  Agregados siete textos más: ``Y más crecimientos retrospectivos (Decimoquinto libro, capítulo VI)``, ``El arcoíris de baja vibración (Decimoquinto libro, capítulo VII)``, ``Procedencias en el andamiaje tematizado (Decimoquinto libro, capítulo VIII)'', ``La fortaleza de la amainada resistencia en el después de sus descensos (Decimosexto libro)'', ``La barata independencia y sus estaciones (Decimosexto libro, capítulo I)'', ``El derrumbe de la sede secundaria (Decimosexto libro, capítulo II)'', y ``Los insatisfactorios recursos de reinserción (Decimosexto libro, capítulo III)``, ampliando así el intervalo de tiempo comprendido por la obra hasta el 14 de julio de 2022.
\end{itemize}

\subsection*{Diario Temtiano edición 17.09.2022}\label{diario-temtiano-ediciuxf3n-17.09.2022}
\addcontentsline{toc}{subsection}{Diario Temtiano edición 17.09.2022}

\begin{itemize}
\item
  Agregados seis textos más: ``La merma del saborizado dictatorial y la escalada de su moderno sustituto (Decimoquinto libro, introducción)``, ``Los retratos del descascarado dogma líquido (Decimoquinto libro, capítulo I)``, ``La voluntaria inercia evolutiva (Decimoquinto libro, capítulo II)'', ``La consolidación del relleno típico (Decimoquinto libro, capítulo III)'', ``Por el paso cúbico (Decimoquinto libro, capítulo IV)'', y ``Agudización constitucional (Decimoquinto libro, capítulo V)``, ampliando así el intervalo de tiempo comprendido por la obra hasta el 19 de junio de 2022.
\item
  Agregada en la sección ``Agradecimientos'' la mención correspondiente al sitio Boxed.fun.
\end{itemize}

\subsection*{Diario Temtiano edición 28.08.2022}\label{diario-temtiano-ediciuxf3n-28.08.2022}
\addcontentsline{toc}{subsection}{Diario Temtiano edición 28.08.2022}

\begin{itemize}
\tightlist
\item
  Agregados cinco textos más: ``La parcial seriedad en aires de concordia (Decimocuarto libro, capítulo IV)``, ``En carrera del salto más deseado (Decimocuarto libro, capítulo V)``, ``Entre desvíos y accidentes (Decimocuarto libro, capítulo VI)'', ``Coleccionistas de pieles (Decimocuarto libro, capítulo VII)'', y ``El retroceso de las apariencias extranjeras (Decimocuarto libro, capítulo VIII)``, ampliando así el intervalo de tiempo comprendido por la obra hasta el 27 de mayo de 2022.
\end{itemize}

\subsection*{Diario Temtiano edición 08.08.2022}\label{diario-temtiano-ediciuxf3n-08.08.2022}
\addcontentsline{toc}{subsection}{Diario Temtiano edición 08.08.2022}

\begin{itemize}
\tightlist
\item
  Agregados cinco textos más: ``Traspasando la confianza (Decimotercer libro, capítulo XIII)``, ``Desde la inauguración del club privado y el desteñimiento de la libertad (Decimocuarto libro)``, ``El regreso de la policía (Decimocuarto libro, capítulo I)'', ``La chapa de la clase alta (Decimocuarto libro, capítulo II)'', y ``La colectividad cerrada y su propietario (Decimocuarto libro, capítulo III)``, ampliando así el intervalo de tiempo comprendido por la obra hasta el 7 de mayo de 2022.
\end{itemize}

\subsection*{Diario Temtiano edición 29.07.2022}\label{diario-temtiano-ediciuxf3n-29.07.2022}
\addcontentsline{toc}{subsection}{Diario Temtiano edición 29.07.2022}

\begin{itemize}
\item
  Realizadas varias correcciones en un texto específico: ``Restauración reforzada (Decimotercer libro, capítulo VIII)''.
\item
  Reescritos los textos introductorios de los primeros dos libros: ``Las secuelas de una supuesta muerte no anunciada (Primer libro)'', y ``El alcance de un experimentado y recargado comienzo prometedor (Segundo libro)''.
\item
  Agregados cuatro textos más: ``Cuando la información confunde (Decimotercer libro, capítulo IX)``, ``Rumbo a una cúspide (Decimotercer libro, capítulo X)``, ``La alternativa en armonización (Decimotercer libro, capítulo XI)'', y ``El rejuvenecedor despegue anaranjado (Decimotercer libro, capítulo XII)``, ampliando así el intervalo de tiempo comprendido por la obra hasta el 23 de abril de 2022.
\end{itemize}

\subsection*{Diario Temtiano edición 16.07.2022}\label{diario-temtiano-ediciuxf3n-16.07.2022}
\addcontentsline{toc}{subsection}{Diario Temtiano edición 16.07.2022}

\begin{itemize}
\tightlist
\item
  Agregados cinco textos más: ``Memoria de hermandad (Decimotercer libro, capítulo IV)``, ``Más claridad para el pasado (Decimotercer libro, capítulo V)``, ``Calco inmemorable (Decimotercer libro, capítulo VI)``, ``En linea fuera de servicio (Decimotercer libro, capítulo VII)'', y ``Restauración reforzada (Decimotercer libro, capítulo VIII)``, ampliando así el intervalo de tiempo comprendido por la obra hasta el 8 de abril de 2022.
\end{itemize}

\subsection*{Diario Temtiano edición 24.06.2022}\label{diario-temtiano-ediciuxf3n-24.06.2022}
\addcontentsline{toc}{subsection}{Diario Temtiano edición 24.06.2022}

\begin{itemize}
\item
  Agregados cuatro textos más: ``Tras el casi seguro intento de abrochar sostenibilidad (Decimotercer libro)``, ``Vehículos para el destino único (Decimotercer libro, capítulo I)``, ``Acciones de atención y cuidado (Decimotercer libro, capítulo II)``, y ``Pintar de prisa sin relleno (Decimotercer libro, capítulo III)``, ampliando así el intervalo de tiempo comprendido por la obra hasta el 26 de marzo de 2022.
\item
  Realizadas algunas modificaciones en ``Lo único que nunca cambió (Desenlace)''.
\end{itemize}

\subsection*{Diario Temtiano edición 30.05.2022}\label{diario-temtiano-ediciuxf3n-30.05.2022}
\addcontentsline{toc}{subsection}{Diario Temtiano edición 30.05.2022}

\begin{itemize}
\item
  Agregados cuatro textos más: ``Recargas entre largas pausas (Duodécimo libro, capítulo VI)``, ``El rigor de la benevolente verdad (Duodécimo libro, capítulo VII)``, ``La cuerda acción sonámbula (Duodécimo libro, capítulo VIII)``, y ``Ensalada de ideas frustradas (Duodécimo libro, capítulo IX)``, ampliando así el intervalo de tiempo comprendido por la obra hasta el 17 de marzo de 2022.
\item
  Realizadas algunas modificaciones mínimas en un par de capítulos específicos de la obra.
\end{itemize}

\subsection*{Diario Temtiano edición 03.05.2022}\label{diario-temtiano-ediciuxf3n-03.05.2022}
\addcontentsline{toc}{subsection}{Diario Temtiano edición 03.05.2022}

\begin{itemize}
\item
  Agregados cuatro textos más: ``El timbre de las chispas esclarecedoras (Duodécimo libro, capítulo II)``, ``A ciegas con menos dependencias (Duodécimo libro, capítulo III)``, ``Detracciones de baja resolución (Duodécimo libro, capítulo IV)``, y ``Populares contradicciones pacíficas (Duodécimo libro, capítulo V)``, ampliando así el intervalo de tiempo comprendido por la obra hasta el 26 de febrero de 2022.
\item
  Realizadas algunas modificaciones mínimas a los primeros capítulos del Segundo libro.
\end{itemize}

\subsection*{Diario Temtiano edición 06.04.2022}\label{diario-temtiano-ediciuxf3n-06.04.2022}
\addcontentsline{toc}{subsection}{Diario Temtiano edición 06.04.2022}

\begin{itemize}
\item
  Adicionadas las gracias al sitio Fantemti.up.railway.app en la sección ``Agradecimientos'' de ``Más sobre Diario Temtiano''.
\item
  Incluidas referencias que faltaban correspondientes a ciertas citas textuales en algunos capítulos, concretamente variantes de nombres propios o partes de estos, que son directamente extraídas del contexto.
\item
  Renombrado ``Temti entre los grandes, o debajo de ellos (Segundo libro, capítulo IX)'' por ``Entre los grandes, o debajo de ellos (Segundo libro, capítulo IX)'', y ``Las esqueléticas sobras del añorado pasado (Tercer libro, capítulo XI)'' por ``Las sobras del glorioso y añorado pasado (Tercer libro, capítulo XI)''.
\item
  Añadidas tres menciones faltantes: sobre la composición del nombre Fantemti y la finalidad de los sabores en ``El ocaso y amanecer simultáneo (Undécimo libro, capítulo V)'', sobre la implementación del Tuturú a Fantemti y su roz propio de difusión en ``Emprendiendo con firmeza (Undécimo libro, capítulo III)'', y sobre la denominación de anom y no anon en ``La micro-réplica dada a luz (Undécimo libro, capítulo I)'', y también modificaciones mínimas a otros capítulos.
\item
  Agregados seis textos más, específicamente cinco capítulos y la introducción a un libro: ``Más sobre el afecto como motivo (Undécimo libro, capítulo VI)'', ``La amplia aceleración del regulado pasar (Undécimo libro, capítulo VII)'', ``Bloqueo propio (Undécimo libro, capítulo VIII)'', ``El seguimiento por las campanas (Undécimo libro, capítulo IX)'', ``El segundo nivel de la consolidada marcha modernizadora (Duodécimo libro)'', y ``Un reenganche sin precedentes (Duodécimo libro, capítulo I)'', ampliando así el intervalo de tiempo comprendido por la obra hasta el 17 de febrero de 2022.
\end{itemize}

\subsection*{Diario Temtiano edición 09.03.2022}\label{diario-temtiano-ediciuxf3n-09.03.2022}
\addcontentsline{toc}{subsection}{Diario Temtiano edición 09.03.2022}

\begin{itemize}
\item
  Añadidos cuatro textos más: ``Arribos coloridamente lisos (Undécimo libro, capítulo II)'', ``Emprendiendo con firmeza (Undécimo libro, capítulo III)'', ``Alarmas de controlada resolución (Undécimo libro, capítulo IV)'', y ``El ocaso y amanecer simultáneo (Undécimo libro, capítulo V)'', ampliando así el intervalo de tiempo comprendido por la obra hasta el 12 de febrero de 2022.
\item
  Añadida la correspondiente mención al sitio ``Fantemti.herokuapp.com`` en el apartado ``Agradecimientos`` de ``Más sobre Diario Temtiano``.
\end{itemize}

\subsection*{Diario Temtiano edición 26.02.2022}\label{diario-temtiano-ediciuxf3n-26.02.2022}
\addcontentsline{toc}{subsection}{Diario Temtiano edición 26.02.2022}

\begin{itemize}
\item
  Reescritos parcialmente todos los capítulos del libro ``Las secuelas de una supuesta muerte no anunciada (Primer libro)'': Se fusionaron los primeros dos capítulos, quedando uno solo. También, se redactaron nuevamente varios de los hechos, y algunos fueron insertados al principio del siguiente libro ``El alcance de un experimentado y recargado comienzo prometedor (Segundo libro)''.
\item
  Añadidos tres textos más, precisamente dos capítulos y la introducción de un libro: ``Solo conexiones minimalistas (Décimo libro, capítulo III)'', ``Arranques cíclicos de un potencial hogar alternativo (Undécimo libro)'', y ``La micro-réplica dada a luz (Undécimo libro, capítulo I)'', ampliando así el intervalo de tiempo comprendido por la obra hasta el 4 de febrero de 2022.
\end{itemize}

\subsection*{Diario Temtiano edición 24.01.2022}\label{diario-temtiano-ediciuxf3n-24.01.2022}
\addcontentsline{toc}{subsection}{Diario Temtiano edición 24.01.2022}

\begin{itemize}
\tightlist
\item
  Agregados cuatro textos más, específicamente tres capítulos y la introducción de un libro: ``La desaparición de rutina (Noveno libro, capítulo XII)'', ``El eco al otro lado de la disminuida subsistencia (Décimo libro)'', ``La endeble seña de cese (Décimo libro, capítulo I)'', y ``La supervivencia entre puentes (Décimo libro, capítulo II)'', ampliando así el intervalo de tiempo comprendido por la obra hasta el 12 de enero de 2022
\end{itemize}

\subsection*{Diario Temtiano edición 18.01.2022}\label{diario-temtiano-ediciuxf3n-18.01.2022}
\addcontentsline{toc}{subsection}{Diario Temtiano edición 18.01.2022}

\begin{itemize}
\tightlist
\item
  Agregados tres capítulos más: ``Callando a los casi extintos creadores (Noveno libro, capítulo IX)'', ``Entre la pacífica convivencia reducida (Noveno libro, capítulo X)'', y ``Por el negado y célebre paraíso (Noveno libro, capítulo XI)'' , ampliando así el intervalo de tiempo comprendido por la obra hasta el 24 de diciembre de 2021.
\end{itemize}

\subsection*{Diario Temtiano edición 09.01.2022}\label{diario-temtiano-ediciuxf3n-09.01.2022}
\addcontentsline{toc}{subsection}{Diario Temtiano edición 09.01.2022}

\begin{itemize}
\item
  Actualizado este mismo apartado, renombrado como ``Notas de edición''. Ahora luego de los detalles correspondientes a la última, se encuentran cronológicamente ordenados los de cada versión anterior.
\item
  Incluida en la sección ``Agradecimientos'' perteneciente a ``Más sobre Diario Temtiano'' la mención correspondiente al antro Rouzed.one.
\item
  Corregidos detalles a lo largo de toda la obra, específicamente errores.
\item
  Realizadas ciertas modificaciones en ``La nueva Temti, y otra de sus enormes proyecciones (Tercer libro, capítulo XIII)'', un nuevo título, varios retoques, y referencias adicionales.
\item
  Agregados cuatro capítulos más: ``De salida sorpresiva (Noveno libro, capítulo V)'', ``La casa entera incompleta (Noveno libro, capítulo VI)'', ``La restauración gradual (Noveno libro, capítulo VII)'', y ``Con lo básico a disposición (Noveno libro, capítulo VIII)'', ampliando así el intervalo de tiempo comprendido por la obra hasta el 4 de diciembre de 2021.
\end{itemize}

\subsection*{Diario Temtiano edición 29.12.2021}\label{diario-temtiano-ediciuxf3n-29.12.2021}
\addcontentsline{toc}{subsection}{Diario Temtiano edición 29.12.2021}

\begin{itemize}
\item
  Renombrados los ``capítulos'' por ``libros''. A su vez, los que antes eran ``subcapítulos'' pasan a ser ``capítulos''.
\item
  Actualizado el apartado ``Más sobre Diario Temtiano''. En ``Contenido'' con las denominaciones actuales de cada componente de la obra, y también para incluir a Titiri dentro de lo que es tratado en ella. En ``Agradecimientos'' con los gentilicios correspondientes a dicho nombre.
\item
  Inaugurado un nuevo libro, ``Los consolidados pasos del retocado mismo legado (Noveno libro)'', con su respectivo texto introductorio.
\item
  Agregados cuatro capítulos más: ``El acto de decisión y continuidad (Noveno libro, capítulo I)'', ``Lucidez terminal (Noveno libro, capítulo II)'', ``El precario y súper rentable reemplazo (Noveno libro, capítulo III)'', y ``El restaurar de la meseta cíclica (Noveno libro, capítulo IV)'', ampliando así el intervalo de tiempo comprendido por la obra hasta el 28 de noviembre de 2021.
\end{itemize}

\subsection*{Diario Temtiano edición 17.12.2021}\label{diario-temtiano-ediciuxf3n-17.12.2021}
\addcontentsline{toc}{subsection}{Diario Temtiano edición 17.12.2021}

\begin{itemize}
\item
  Realizadas correcciones mínimas a lo largo de toda la obra.
\item
  Agregados cuatro subcapítulos más: ``El hambre del avaro oportunista (Octavo capítulo, subcapítulo III)'', ``A partir de la indiferencia definitiva (Octavo capítulo, subcapítulo IV)'', ``Pensando en la muerte (Octavo capítulo, subcapítulo V)'', y ``La amnesia consagratoria (Octavo capítulo, subcapítulo VI)'', ampliando así el intervalo de tiempo comprendido por la obra hasta el 22 de noviembre de 2021.
\end{itemize}

\subsection*{Diario Temtiano edición 05.12.2021}\label{diario-temtiano-ediciuxf3n-05.12.2021}
\addcontentsline{toc}{subsection}{Diario Temtiano edición 05.12.2021}

\begin{itemize}
\item
  Actualizado el apartado ``Agradecimientos'' de ``Más sobre Diario Temtiano'' con las gracias correspondientes a los sitios donde se difundió y se está difundiendo la obra, incluyendo a los que antes no estaban, oldtemti.herokuapp.com y grupostemti.herokuapp.com.
\item
  Inaugurado un nuevo capítulo, ``Los delimitados recorridos del paralelismo residencial (Octavo capítulo)''.
\item
  Agregados dos subcapítulos más: ``El momento cíclico del albergue marginado (Octavo capítulo, subcapítulo I)'', y ``Las voces de la bipolaridad protagonista (Octavo capítulo, subcapítulo II)'', ampliando así el intervalo de tiempo comprendido por la obra hasta el 12 de noviembre de 2021.
\end{itemize}

\subsection*{Diario Temtiano edición 20.11.2021}\label{diario-temtiano-ediciuxf3n-20.11.2021}
\addcontentsline{toc}{subsection}{Diario Temtiano edición 20.11.2021}

\begin{itemize}
\item
  Realizadas correcciones en el subcapítulo ``La combinación de las monedas corrientes (Séptimo capítulo, subcapítulo VII)''.
\item
  Removido el formato de letra cursiva y barras (//) que se aplicaba en las palabras o términos que se utilizaban con cierto cuidado. A su vez, las citas textuales ya no son rodeadas por tres barras (///) sino por una sola (/).
\item
  Agregados tres subcapítulos más: ``El alza de la vanguardia fúnebre (Séptimo capítulo, subcapítulo IX)'', ``En horas de cuestiones existenciales (Séptimo capítulo, subcapítulo X)'', ``Rumbo a la presunta verdad (Séptimo capítulo, subcapítulo XI)'', ampliando así el intervalo de tiempo comprendido por la obra hasta el 10 de noviembre de 2021.
\end{itemize}

\subsection*{Diario Temtiano edición 11.11.2021}\label{diario-temtiano-ediciuxf3n-11.11.2021}
\addcontentsline{toc}{subsection}{Diario Temtiano edición 11.11.2021}

\begin{itemize}
\item
  Removido el apartado ``La historia'' dentro de ``Más sobre Diario Temtiano'', y también el de ``Cambios previstos para la próxima edición'' que se encontraba en las ``Notas de la edición''.
\item
  Modificados dos títulos de la obra, los cuales quedaron como ``El alcance de un experimentado y recargado comienzo prometedor (Segundo capítulo)'', y ``Mientras se está a la deriva entre la vida y la incertidumbre (Quinto capítulo)''.
\item
  Agregados a la obra dos subcapítulos más: ``La combinación de las monedas corrientes (Séptimo capítulo, subcapítulo VII)'', y ``Las danzas de la quietud (Séptimo capítulo, subcapítulo VIII)''.
\end{itemize}

\subsection*{Diario Temtiano edición 01.11.2021}\label{diario-temtiano-ediciuxf3n-01.11.2021}
\addcontentsline{toc}{subsection}{Diario Temtiano edición 01.11.2021}

\begin{itemize}
\item
  Realizadas pequeñas modificaciones en el apartado de ``La historia'' dentro de ``Más sobre Diario Temtiano'', que corrigen algunas referencias puntuales.
\item
  Intercambiados determinados gentilicios empleados a lo largo de la obra, tanto para ampliar la variedad, como para ajustarse a circunstancias específicas.
\item
  Agregados a la obra cuatro subcapítulos más: ``Avances a la usanza (Séptimo capítulo, subcapítulo III)'', ``Los recovecos donde el humo amarillento (Séptimo capítulo, subcapítulo IV)'', ``Sed y justicia (Séptimo capítulo, subcapítulo V)'', y ``Dentro de las fronteras distorsionadas (Séptimo capítulo, subcapítulo VI)''.
\end{itemize}

\subsection*{Diario Temtiano edición 20.10.2021}\label{diario-temtiano-ediciuxf3n-20.10.2021}
\addcontentsline{toc}{subsection}{Diario Temtiano edición 20.10.2021}

\begin{itemize}
\item
  Corregido en general el uso de saltos de linea.
\item
  Ajustada determinada terminología, y varios gentilicios.
\item
  Modificado el apartado \emph{``Más sobre Diario Temtiano''}; donde se actualizó \emph{``La historia''} con la inclusión de referencias adicionales, y donde también se agregaron en \emph{``Agradecimientos''} las gracias correspondientes a las plataformas donde Diario Temtiano se publicó.
\item
  Inaugurado un nuevo capítulo, \emph{``Las inhibidas andanzas bajo la apta nueva normalidad (Séptimo capítulo)''}.
\item
  Agregados a la obra tres subcapítulos más: \emph{``La tierra de mensajes intermitentes (Sexto capítulo, subcapítulo IX)''}, \emph{``La parcial pero suficiente reanudación (Séptimo capítulo, subcapítulo I)''}, \emph{``El tambaleante festival de cepillos (Séptimo capítulo, subcapítulo II)''}.
\end{itemize}

\subsection*{Diario Temtiano edición 12.10.2021}\label{diario-temtiano-ediciuxf3n-12.10.2021}
\addcontentsline{toc}{subsection}{Diario Temtiano edición 12.10.2021}

\begin{itemize}
\item
  Realizadas pequeñas correcciones en el contenido y en los títulos de unos pocos subcapítulos de la obra.
\item
  Agregado un nuevo subcapítulo, \emph{El pausado desarrollo a pleno (Sexto capítulo, subcapítulo VIII)}.
\item
  Ampliadas varias de las introducciones a capítulos: \emph{Las secuelas de una supuesta muerte no anunciada (Primer capítulo)}, \emph{El alcance de un experimentado y recargado nuevo comienzo (Segundo capítulo)}, \emph{El durante cuando los días están más que contados (Cuarto capítulo)}.
\item
  Reformuladas algunas menciones propias de terminología, en: \emph{El comienzo de una nueva prueba de fuego (Tercer capítulo, subcapítulo I)}, \emph{Otra vez de paseo a la deriva (Cuarto capítulo, subcapítulo III)}, \emph{Un paso más cerca para descubrir lo distante que se está (Quinto capítulo, subcapítulo X)}, \emph{El histórico fracaso de los seres terrenales (Sexto capítulo, subcapítulo I)}.
\item
  Incluida una extensa referencia a la dinámica de publicación de nuevas ediciones al cierre de la obra, en \emph{Lo único que nunca cambió (Desenlace)}.
\end{itemize}

\subsection*{Diario Temtiano edición 01.10.2021}\label{diario-temtiano-ediciuxf3n-01.10.2021}
\addcontentsline{toc}{subsection}{Diario Temtiano edición 01.10.2021}

\begin{itemize}
\item
  Solucionado el problema que no permitía que se vieran algunos saltos de linea.
\item
  Realizadas pequeñas correcciones en el contenido de unos pocos subcapítulos de la obra, y también modificados los títulos de muchos de ellos.
\item
  Añadido un apartado de título \emph{Más sobre Diario Temtiano}, explicativo de la obra en sí.
\end{itemize}

\subsection*{Diario Temtiano edición 20.09.2021}\label{diario-temtiano-ediciuxf3n-20.09.2021}
\addcontentsline{toc}{subsection}{Diario Temtiano edición 20.09.2021}

\begin{itemize}
\item
  Añadidos dos nuevos subcapítulos al Sexto capítulo: \emph{Sobre el insondable límite (Sexto capítulo, subcapítulo VI)} y \emph{En continuidad con el desvanecimiento (Sexto capítulo, subcapítulo VII)}.
\item
  Realizadas pequeñas correcciones en el contenido de unos pocos subcapítulos de la obra, además de los títulos de varios de ellos.
\item
  Modificados los desarrollos puntuales de algunos subcapítulos, entre menciones y referencias: \emph{El derrocamiento anónimo (Segundo capítulo, subcapítulo IX)}, \emph{El retorno de las invasiones escandalosas (Segundo capítulo, subcapítulo X)}, y \emph{Incomunicados entre rozzados (Segundo capítulo, subcapítulo XXI)}.
\end{itemize}

\subsection*{Diario Temtiano edición 15.09.2021}\label{diario-temtiano-ediciuxf3n-15.09.2021}
\addcontentsline{toc}{subsection}{Diario Temtiano edición 15.09.2021}

\begin{itemize}
\item
  Segmentado el Quinto capítulo en dos partes, dando lugar al siguiente de su tipo: \emph{La época de las infinitas prórrogas disfuncionales (Sexto capítulo)}.
\item
  Agregado un nuevo subcapítulo al Sexto capítulo: \emph{Colisión de ilusiones (Sexto capítulo, subcapítulo V)}.
\item
  Realizadas pequeñas correcciones e incluidas bastantes referencias en el contenido de varios subcapítulos de la obra.
\end{itemize}

\subsection*{Diario Temtiano edición 09.09.2021}\label{diario-temtiano-ediciuxf3n-09.09.2021}
\addcontentsline{toc}{subsection}{Diario Temtiano edición 09.09.2021}

\begin{itemize}
\item
  Agregados tres nuevos subcapítulos al Quinto capítulo: \emph{Récord de desoriente (Quinto capítulo, subcapítulo XV)}, \emph{La calma de la fría lejanía (Quinto capítulo, subcapítulo XVI)}, y \emph{En vísperas de aterrizaje (Quinto capítulo, subcapítulo XVII)}
\item
  Realizadas algunas pequeñas correcciones en el contenido de varios subcapítulos de la obra.
\item
  Cambiado el formato de la referencia temporal de cada subcapítulo, de números a letras.
\end{itemize}

\subsection*{Diario Temtiano edición 04.09.2021}\label{diario-temtiano-ediciuxf3n-04.09.2021}
\addcontentsline{toc}{subsection}{Diario Temtiano edición 04.09.2021}

\begin{itemize}
\item
  Agregados dos nuevos subcapítulos al Quinto capítulo, subcapítulos XIII y XIV.
\item
  Modificados varios títulos de la obra, también algunos detalles del contenido, en el Prefacio, la Introducción, y subcapítulos del Primer y Segundo capítulo.
\item
  Distinguidos algunos términos clave recurrentes, semejantes a nombres propios.
\item
  Reemplazados los guiones (--, ---) por barras (//, ///), carácter que hace las veces de comillas.
\item
  Corregidas varios detalles, específicamente errores.
\end{itemize}

\subsection*{Diario Temtiano edición 01.09.2021}\label{diario-temtiano-ediciuxf3n-01.09.2021}
\addcontentsline{toc}{subsection}{Diario Temtiano edición 01.09.2021}

\begin{itemize}
\item
  Realizadas correcciones mínimas en algunos pocos subcapítulos.
\item
  Agregados dos subcapítulos nuevos al quinto capítulo.
\end{itemize}

\subsection*{Diario Temtiano edición 01.06.2021}\label{diario-temtiano-ediciuxf3n-01.06.2021}
\addcontentsline{toc}{subsection}{Diario Temtiano edición 01.06.2021}

\begin{itemize}
\item
  Realizadas pequeñas modificaciones en toda la obra, sobre todo en el tercer capítulo: errores, título, texto.
\item
  Dividido el cuarto capítulo en dos diferentes, iniciando el quinto.
\item
  Agregados dos subcapítulos nuevos al quinto capítulo.
\end{itemize}

\subsection*{Diario Temtiano edición 27.05.2021}\label{diario-temtiano-ediciuxf3n-27.05.2021}
\addcontentsline{toc}{subsection}{Diario Temtiano edición 27.05.2021}

\begin{itemize}
\item
  Renombrados los capítulos por subcapítulos, y lo mismo para tomos por capítulos: definiciones que encajan mejor según la dimensión de cada uno.
\item
  Modificados la mayoría de los subcapítulos del segundo capítulo: varios títulos fueron adaptados y se agregó texto en muchos de ellos.
\item
  Agregados cuatro subcapítulos nuevos al cuarto capítulo.
\end{itemize}

\subsection*{Diario Temtiano edición 22.05.2021}\label{diario-temtiano-ediciuxf3n-22.05.2021}
\addcontentsline{toc}{subsection}{Diario Temtiano edición 22.05.2021}

\begin{itemize}
\item
  ⊠ Incluido al principio un texto similar a un preámbulo. En él se enfatizó en parte de la importancia que tiene Crónicas Temtianas para Diario Temtiano.
\item
  ⊠ Incluido luego del cuarto tomo: \emph{--Desenlace de Diario Temtiano--}. Apunta a darle un mejor cierre a la obra, como contraparte de la introducción, independiente de la publicación de nuevos capítulos.
\item
  ⊠ Removido el listado de ediciones anteriores. Aquellas seguirán estando disponibles, pero la obra aspira a superar en cada edición a sus precedentes, y no pretende que aquellas sean tenidas en cuenta.
\item
  ⊠ Realizadas modificaciones a casi todos los capítulos del primer tomo. En ellos se agregó algo de texto más interpretativo que informativo, y se cambiaron ciertos títulos también.
\item
  ⊠ Agregado un nuevo capítulo al cuarto tomo.
\end{itemize}

\subsection*{Diario Temtiano edición 12.05.2021}\label{diario-temtiano-ediciuxf3n-12.05.2021}
\addcontentsline{toc}{subsection}{Diario Temtiano edición 12.05.2021}

\begin{itemize}
\item
  ⊠ Agregados tres nuevos capítulos al cuarto tomo.
\item
  ⊠ Adaptados algunos gentilicios al filtro de censura.
\end{itemize}

\subsection*{Diario Temtiano edición 07.05.2021}\label{diario-temtiano-ediciuxf3n-07.05.2021}
\addcontentsline{toc}{subsection}{Diario Temtiano edición 07.05.2021}

\begin{itemize}
\item
  ⊠ Modificados y ampliados determinados contenidos y títulos, en la introducción, los tomos, y los primeros dos capítulos.
\item
  ⊠ Segmentado el segundo tomo en tres diferentes.
\item
  ⊠ Añadido en cada tomo una especie de introducción previa.
\item
  ⊠ Agregados dos nuevos capítulos al cuarto tomo.
\end{itemize}

\subsection*{Diario Temtiano edición 04.05.2021}\label{diario-temtiano-ediciuxf3n-04.05.2021}
\addcontentsline{toc}{subsection}{Diario Temtiano edición 04.05.2021}

\begin{itemize}
\item
  ⊠ Añadidas menciones en los últimos dos capítulos.
\item
  ⊠ Agregados tres nuevos capítulos al segundo tomo.
\end{itemize}

\subsection*{Diario Temtiano edición 27.04.2021}\label{diario-temtiano-ediciuxf3n-27.04.2021}
\addcontentsline{toc}{subsection}{Diario Temtiano edición 27.04.2021}

\begin{itemize}
\item
  ⊠ Incluidas menciones adicionales, aplicadas correcciones generales, ajustadas perspectivas temporales, intercambiados términos adecuados: en la introducción de la obra, en la primera mitad de la obra, y el último capítulo de la obra.
\item
  ⊠ Agregado un nuevo capítulo al segundo tomo.
\end{itemize}

\subsection*{Diario Temtiano edición 11.04.2021}\label{diario-temtiano-ediciuxf3n-11.04.2021}
\addcontentsline{toc}{subsection}{Diario Temtiano edición 11.04.2021}

\begin{itemize}
\item
  ⊠ Agregados cuatro capítulos nuevos al segundo tomo, XXXIV, XXXV, XXXVI, XXXVII.
\item
  ⊠ Pequeñas adaptaciones y referencias añadidas a los capítulos XXIV y XXXIII del segundo tomo.
\end{itemize}

\subsection*{Diario Temtiano edición 30.03.2021}\label{diario-temtiano-ediciuxf3n-30.03.2021}
\addcontentsline{toc}{subsection}{Diario Temtiano edición 30.03.2021}

\begin{itemize}
\item
  ⊠ Modificado el contenido de un capítulo, en \emph{--La intensificación del frío silencio (Segundo tomo, capítulo XXX)--}.
\item
  ⊠ Agregados dos capítulos nuevos al segundo tomo, \emph{--Las esencias mas antiguas, ¿Vivas, esqueletizadas\ldots? (Segundo tomo, capitulo XXXII)--} y \emph{--La incultura al desnudo (Segundo tomo, capitulo XXXIII)--}.
\end{itemize}

\subsection*{Diario Temtiano edición 28.03.2021}\label{diario-temtiano-ediciuxf3n-28.03.2021}
\addcontentsline{toc}{subsection}{Diario Temtiano edición 28.03.2021}

\begin{itemize}
\item
  ⊠ Se modificaron pasajes específicos, en: \emph{--Segundo tomo, capítulo VIII, Sometimiento--}, \emph{--Segundo tomo, capítulo IV, Anexión--}, \emph{--Segundo tomo, capítulo XXVII, Destellos reiterados en la oscuridad--}, \emph{--Segundo tomo, capítulo XIV, Reflote sin seguridad--}, \emph{--Segundo tomo, capítulo V, El estado oculto--}, y \emph{--Segundo tomo, capítulo XXIV, Protección en un lugar inseguro--}.
\item
  ⊠ Se cambió el título del \emph{--Segundo tomo, capítulo XVII--} (Tercer Gran Salto) por (Conexiones modernizadoras y espera).
\item
  ⊠ Se adaptó la perspectiva temporal de prácticamente todos los capítulos, junto a otras pequeñas correcciones.
\end{itemize}

\subsection*{Diario Temtiano edición 23.03.2021}\label{diario-temtiano-ediciuxf3n-23.03.2021}
\addcontentsline{toc}{subsection}{Diario Temtiano edición 23.03.2021}

\begin{itemize}
\tightlist
\item
  ⊠ Agregados tres capítulos (XXIX, XXX y XXXI), desde el 2021-03-01 hasta el 2021-03-21.
\end{itemize}

\subsection*{Diario Temtiano edición 08.03.2021}\label{diario-temtiano-ediciuxf3n-08.03.2021}
\addcontentsline{toc}{subsection}{Diario Temtiano edición 08.03.2021}

\begin{itemize}
\item
  Se corrigieron en general los tiempos verbales y se retiró la perspectiva de primera persona, especialmente la del primer tomo.
\item
  Se reordenaron y agregaron nuevos capítulos al final del segundo tomo:
  Capítulo XXIII - Prestigio no tan valioso // 2021-02-21, 2021-02-22, y 2021-02-23\\
  Capítulo XXIV - Protección en un lugar inseguro // 2021-02-23 //\\
  Capítulo XXV - A falta de oro verdadero // 2021-02-23 //\\
  Capítulo XXVI - Cielo incongruente // 2021-02-24 hasta 2021-02-27 //\\
  Capítulo XXVII - Destellos reiterados en la oscuridad // 2021-02-27 //\\
  Capítulo XXVIII - Trozos enigmáticos // 2021-02-27 y 2021-02-28, y 2021-03-01 //
\end{itemize}

\subsection*{Diario Temtiano edición 01.03.2021}\label{diario-temtiano-ediciuxf3n-01.03.2021}
\addcontentsline{toc}{subsection}{Diario Temtiano edición 01.03.2021}

\begin{itemize}
\item
  Se modificaron un par de pasajes en la introducción.
\item
  Se agregaron dos capítulos al segundo tomo:\\
  Capítulo XXIII - Protección en un lugar inseguro / 2021-02-21 hasta 2021-02-25\\
  Capítulo XXIV - A falta de oro verdadero / 2021-02-25
\item
  Se volvieron a distinguir algunas palabras y términos no comunes, como nombres propios y periodos históricos.
\end{itemize}

\subsection*{Diario Temtiano edición 26.02.2021}\label{diario-temtiano-ediciuxf3n-26.02.2021}
\addcontentsline{toc}{subsection}{Diario Temtiano edición 26.02.2021}

\begin{itemize}
\item
  Se agregó una introducción y un título.
\item
  Se corrigieron algunos nombres significativos a comenzar con mayúscula, para distinguir de palabras comunes y corrientes.
\item
  Se modificaron los últimos capítulos del segundo tomo, las referencias temporales y la información contenida.
\item
  Se añadió un nuevo capítulo, exactamente el Capítulo XXII del segundo tomo, Reinicios, que abarca del 2021-02-17 hasta el 2021-02-20.
\end{itemize}

\subsection*{(sin titular) 25.02.2021}\label{sin-titular-25.02.2021}
\addcontentsline{toc}{subsection}{(sin titular) 25.02.2021}

\emph{Sin notas. Únicamente se publicaron los siguientes textos:}

Post-resurgimiento, Copamiento sospechoso (C-XVI) 2021-01-30 y 2021-01-31\\
Post-resurgimiento, Tercer gran salto (C-XVII) 2021-02-01 hasta 2021-02-08\\
Post-resurgimiento, La tradición historiadora (C-XVIII) 2021-02-08\\
Post-resurgimiento, Perdición (C-XIX) 2021-02-08 hasta 2021-02-13\\
Post-resurgimiento, El día rojo (C-XX) 2021-02-14\\
Post-resurgimiento, Incomunicados entre rozados (C-XXI) 2021-02-15 y 2021-02-16

\subsection*{(sin titular) 19.02.2021}\label{sin-titular-19.02.2021}
\addcontentsline{toc}{subsection}{(sin titular) 19.02.2021}

\emph{Sin notas. Únicamente se publicaron los siguientes textos:}

Post-resurgimiento, Divididos por un lugar mejor (C-XII) 2021-01-26\\
Post-resurgimiento, Elecciones pegajosas (C-XIII) 2021-01-26, 2021-01-27 y 2021-01-28\\
Post-resurgimiento, Reflote sin seguridad (C-XIV) 2021-01-28\\
Post-resurgimiento, Abran paso, viene en camino (C-XV) 2021-01-28 y 2021-01-29

\subsection*{(sin titular) 18.02.2021}\label{sin-titular-18.02.2021}
\addcontentsline{toc}{subsection}{(sin titular) 18.02.2021}

\emph{Sin notas. Únicamente se publicaron los siguientes textos:}

Post-resurgimiento, Derrocando, ¿a quién? (C-IX) 2021-01-24 y 2021-01-25\\
Post-resurgimiento, La nueva invasión (C-X) 2021-01-25 y 2021-01-26\\
Post-resurgimiento, Crisis de resistencia (C-XI) 2021-01-26

\subsection*{(sin titular) 15.02.2021}\label{sin-titular-15.02.2021}
\addcontentsline{toc}{subsection}{(sin titular) 15.02.2021}

\emph{Sin notas. Únicamente se publicaron los siguientes textos:}

Post-resurgimiento, Tierra baldía (C-I) - 2021-01-22\\
Post-resurgimiento, Las únicas señales (C-II) - 2021-01-22\\
Post-resurgimiento, Búsqueda por respuestas (C-III) - 2021-01-23\\
Post-resurgimiento, Anexión (C-IV) 2021-01-23\\
Post-resurgimiento, Dualidades separadas (C-VI) 2021-01-23\\
Post-resurgimiento, Universos (C-VII) 2021-01-23\\
Post-resurgimiento, Sometimiento (C-VIII) 2021-01-23

\subsection*{(sin titular) 13.02.2021}\label{sin-titular-13.02.2021}
\addcontentsline{toc}{subsection}{(sin titular) 13.02.2021}

\emph{Sin notas. Únicamente se publicó el siguiente texto:}

Post-testamento, Sumergimiento (C-VIII) - 2021-01-22

\subsection*{(sin titular) 11.02.2021}\label{sin-titular-11.02.2021}
\addcontentsline{toc}{subsection}{(sin titular) 11.02.2021}

\emph{Sin notas. Únicamente se publicaron los siguientes textos:}

Post-testamento, Fuegos inestables en las aguas (C-VI) - 2021-01-21\\
Post-testamento, Faro sobrecalentado inflamable alumbrando el desasosiego (C-VII) - 2021-01-22

\subsection*{(sin titular) 10.02.2021}\label{sin-titular-10.02.2021}
\addcontentsline{toc}{subsection}{(sin titular) 10.02.2021}

\emph{Sin notas. Únicamente se publicó el siguiente texto:}

Post-testamento, El paso hacía la iluminación (C-V) - 2021-01-21

\subsection*{(sin titular) 09.02.2021}\label{sin-titular-09.02.2021}
\addcontentsline{toc}{subsection}{(sin titular) 09.02.2021}

\emph{Sin notas. Únicamente se publicó el siguiente texto:}

Post-testamento, Evacuación (C-IV) - 2021-01-19 y 2021-01-20

\subsection*{(sin titular) 08.02.2021}\label{sin-titular-08.02.2021}
\addcontentsline{toc}{subsection}{(sin titular) 08.02.2021}

\emph{Sin notas. Únicamente se publicó el siguiente texto:}

Post-testamento, Renacimiento (C-III) - 2021-01-18

\subsection*{(sin titular) 04.02.2021}\label{sin-titular-04.02.2021}
\addcontentsline{toc}{subsection}{(sin titular) 04.02.2021}

\emph{Sin notas. Únicamente se publicaron los siguientes textos:}

Post-testamento, Frágil certidumbre (parte 1) (C-I) - 2021-01-17\\
Post-testamento, Frágil certidumbre (parte 2) (C-II) - 2021-01-17

\chapter*{Más sobre Diario Temtiano}\label{muxe1s-sobre-diario-temtiano}
\addcontentsline{toc}{chapter}{Más sobre Diario Temtiano}

¿La explicación podría ser peor que la incertidumbre? ¿Podemos intentar\ldots?

Antes de continuar con la obra en sí y comenzar a hablar en código, más, aunque ni de cerca todo, sobre Diario Temtiano.

\section*{Contenido}\label{contenido}
\addcontentsline{toc}{section}{Contenido}

Diario Temtiano es una obra literaria que trata especialmente sobre Temti, donde partiendo desde la fecha del 17.01.2021, en orden cronológico se desarrollan interpretaciones de los antecedentes que protagonizaron la Historia Temtiana; sinónimo de lo ocurrido en torno al nombre Temti, sus sitios, sus usuarios, su cultura, su identidad, y otros relacionados.

Está conformado por un prefacio, una introducción, varios libros que contienen un texto preliminar seguido por varios capítulos, y un desenlace.

\section*{Actualizaciones}\label{actualizaciones}
\addcontentsline{toc}{section}{Actualizaciones}

La obra, mediante la publicación de ediciones, las cuales son identificadas con la fecha de las mismas, se actualiza cada tanto con tal de continuar mejorando su contenido. Esto es principalmente agregando textos para ponerse al día con los antecedentes, además de también modificar partes para cubrir omisiones y errores.

Salvo que un futuro cambio implique lo contrario, en \url{https://temtiteamo.github.io/diariotemtiano} siempre se podrá encontrar la edición más reciente y completa, por lo que es recomendado en toda ocasión ingresar allí debido a lo anterior mencionado.

No hay una frecuencia establecida para la publicación de ediciones nuevas, sin embargo esta suele venir del ritmo de desarrollo que tenga la Historia Temtiana. Además, los textos mantienen cierta distancia de la actualidad, implicando que los últimos comprendan hasta un punto anterior a la fecha presente.

\section*{Justificaciones}\label{justificaciones}
\addcontentsline{toc}{section}{Justificaciones}

Como se menciona en el prefacio y en uno de los tantos capítulos de Diario Temtiano, el gran disparador para que este surgiera, fue que durante una época bastante particular se sintiera la prolongada ausencia de literatura que interpretara lo que aconteciera en la escena temtiana de ese entonces. Concretamente, esta se padeció notablemente un tanto antes y después del instante que Investigator como autor anunció que de momento no iba a continuar ampliando su obra, Crónicas Temtianas, la cual sí cumplía esa función, mientras que aún seguían ocurriendo una cantidad importante de sucesos y la costumbre de plasmarlos en texto de una manera medianamente ordenada y atractiva se había perdido prácticamente por completo.

Luego de que todo eso se mantuviera por un buen rato, porque el contexto único lo permitiría y ofrecería las condiciones, pero también gracias a los individuos responsables de su construcción en el pasado y en dicho presente, esta historia dentro de la Historia Temtiana comenzó a concretarse\ldots{}

\section*{Referencias}\label{referencias}
\addcontentsline{toc}{section}{Referencias}

Los textos incluídos en la obra, son desarrollados por una producción anónima, priorizando generar una base objetiva sobre lo evidente, también pudiendo considerar más. Esto es partir de contenidos que en algún momento estuvieron disponibles en linea, los cuales en cada caso exceptuando las introducciones, están referenciados por un intervalo de fechas. Por la dificultad de archivar los detalles relevantes para la elaboración de los textos, además de las complicaciones que implicaría vincular lo escrito con lo tratado, la obra no incluye más indicaciones puntuales.

\section*{Agradecimientos}\label{agradecimientos}
\addcontentsline{toc}{section}{Agradecimientos}

Desde la autoría de la obra, enormes gracias a quienes colaboraron de una forma u otra con Diario Temtiano.

Especiales gracias a los temtiteros, temtineros, temtianos, temteros, timteros, temtistas, temtineitors, titiriteros, titirieros, titirineros, titirianos, titiristas, fantemtianos, fantemteros, fantemtineros, fantemtistas, etcétera, sus correspondientes a otros orígenes relacionados, sus equivalentes inclusivos, etcétera, etcétera, por contribuir y construir la Historia Temtiana, así como también críticar, sugerir, e interesarse en Diario Temtiano.

Especiales gracias a los sitios del contexto, por dar lugar al desarrollo de la Historia Temtiana del cual tanto se vale Diario Temtiano.

Especiales gracias a Investigator, por el influyente aporte que significa Crónicas Temtianas para Diario Temtiano.

Especiales gracias a Temtiteamo, por la difusión original de las publicaciones de Diario Temtiano a través de los sitios.

Especiales gracias a \citet{Temtinet}, por la difusión adicional de las publicaciones de Diario Temtiano a través de Telegram.

Especiales gracias a las plataformas de publicación: Pastebin, Rentry, Bookdown, Github, por ofrecer las condiciones para que Diario Temtiano pueda estar disponible en linea.

Especiales gracias a los sitios específicos: Temti, Arg(g)news, Vox(x)ed, Ts(s)ubit, Rouz(z)er, Rouz(z)ed, Titiri, Fantemti, Box(x)ed, Dev(v)ox, por permitirle a Diario Temtiano tener espacio para difundir sus novedades.

Especiales gracias a Temti y Fantemti, por el destaque de los espacios difusivos propios de Diario Temtiano.

\chapter{El primer gran puente de la literatura temtiana (Prefacio)}\label{el-primer-gran-puente-de-la-literatura-temtiana-prefacio}

Más allá de que miles y miles de palabras no sean suficientes para poder describir la inmensa cantidad de detalles existentes en la Historia Temtiana, la literatura ligada a ella desde el inicio de sus tiempos ha intentado abarcar y reproducir parte de los mismos. A raíz de tal contextualización surge la inevitable necesidad de mencionar a Crónicas Temtianas, que tuvo la gran iniciativa dentro de dicho apartado cultural y que en sus primeros dos libros contiene una parte más de todo lo que queda excluido de Diario Temtiano.

La relevancia de esto es tal que no solo la segunda nace a partir de muchas construcciones de la primera, sino que también temporalmente hablando una es la continuación de la otra. Y si bien las dos obras tienen diferencias sustanciales en cuanto a su estilo para narrar los hechos, considerar una como sucesora de la otra ayudará para lograr abarcar mayores cantidades de sucesos, y por supuesto, introducirse mejor dentro de todo lo que es Temti.

Por eso y mucho más, se insertan dos libros escritos por Investigator, anteriores a todo lo que proseguirá después:

\begin{itemize}
\tightlist
\item
  CRÓNICAS TEMTIANAS - Reedición 17.01 \url{https://pastebin.com/kbu6ZYPq}
\end{itemize}

\chapter{La nueva cuna de la explicación (Introducción)}\label{la-nueva-cuna-de-la-explicaciuxf3n-introducciuxf3n}

Los diferentes quiebres que marcaron claramente el comienzo de una nueva etapa son el motivo más evidente, aunque quizá no el principal para que se haya formado un apartado más entre todas las riquezas culturales de Temti. Siendo una pieza que por sí sola no es suficiente para entender correctamente todo su contenido, y que también es derivada de los dos primeros libros de Crónicas Temtianas, acompañará para siempre el vasto conjunto de elementos que forman la historia y la identidad temtiana.

En esta serie cronológica de textos muy conexos a los viajes entre dimensiones y tiempo, bajo las ordenes de ciertos principios y deberes morales, los acontecimientos más relevantes vuelven a ser revividos e interpretados, intentando atrapar cada detalle que pueda escaparse, en ocasiones abrazando perspectivas desligadas de neutralidad, y en otras acercándose a la objetividad, pero nunca desviándose de las dos palabras disparadoras que tanto pero tanto significan: Sobre Temti. Eso es Diario Temtiano.

\chapter{Las secuelas de una supuesta muerte no anunciada (Primer libro)}\label{las-secuelas-de-una-supuesta-muerte-no-anunciada-primer-libro}

A la espera por novedades en un ambiente de incesante incertidumbre, la cuenta regresiva aunque no visible llegó a su fin, y con un poco de tardanza eso tan ansiado y no necesariamente deseado, sucedió. La actualidad respecto a ello está fuera de hora, y como se había manejado, el cierre absoluto venía avecinándose y casi tal cual, fue lo que repentinamente tomó el lugar de la dirección temtiana desde todas sus aproximaciones, solo que presentado mediante un camino extremadamente rebuscado, impactantes imágenes capaces de conmocionar incluyendo pretexto dentro de una codificación combinada, con vibra a locura, tristeza, nostalgia, deceso, y más características no esenciales. Una comunicación entreverada en la sucesión de gráficos que por ratos se hizo indescifrable, nada lograba sacarle íntegramente su verdadera trama, pero el trabajo colectivo logró darle forma al cúmulo de interpretaciones, que supieron inclinarse por lo que efectivamente terminó siendo el significado de las numerosas letras alternadas entre minúsculas y mayúsculas.

¿Ya no era lo mismo? Definitivamente ya no es lo mismo. Hay un mensaje que sin ser explicito no da a pensar otra cosa que no sea el fin de una época, el fin de Temti como habitualmente se la conoció. En su contraparte, o más bien sobre las últimas secuencias del fragmento, quedó planteada una especie de invitación para todo el grupo que pudiera estar expectante de una restauración del sitio apagado, que hipotéticamente no sucederá, sin embargo en su puesto estaría aquella alternativa mínimamente alejada, Hixxel. Varios factores descartan que el rumbo del éxodo general obligado se aparte hacia clones que no sean ese, por lo que allí es que se proyectan las primeras narrativas a partir de estos momentos. No obstante, no hay seguridad de que vaya a ser un traslado permanente, las circunstancias para la corta posterioridad parecen, no prometen, un panorama más cambiante aún podría estar al asecho y este ser solo un paradero temporal, sin desmerecer su sostenida estabilidad donde se encuentra.

Suponiendo el cumplimiento de las seis sentencias al pie de la letra, A. se marcha y con él su creación, dejando en manos hermanas a los intérpretes del ayer. Los que siempre estuvieron vinculados a ella y perduran luego del suceso, más en un segundo escalón los que no, continuarán bajo prueba, previamente nunca sucedió nada tan contundente ni que amenazara de tal forma contra la trascendencia del nombre en cuestión y lo aparejado consigo, pero tal como se afirmó de la primera a la sexta sentencia, ambos formaron un fuerte conjunto a lo no tan largo del tiempo, el cual a pesar de haberse ido debilitando y ahora recibir este tajante impacto, no padecerá su defunción en lo inmediato.

De modo que con o sin el antro, la Historia Temtiana sigue adelante. Por lo pronto, todo indica que esta va a prolongarse de otra manera, en otro espacio, en otras condiciones, con muchas diferencias a lo que se supo narrar en las reconocidas Crónicas Temtianas, que concluyendo el periodo post-Purga también lo hizo su segundo libro, pues se encontraron varios hechos sumamente significativos que dieron clausura a la época corriente, poniendo un punto de inflexión, un antes y por más extraño que sea tras procesar el contenido de la despedida, un después.

\section{Donde la frágil certidumbre (Primer libro, capítulo I)}\label{donde-la-fruxe1gil-certidumbre-primer-libro-capuxedtulo-i}

\begin{quote}
17 de enero de 2021
\end{quote}

Horas pasadas ni bien quedó servido el mensaje del críptico administrador, tras el ya conocido y narrado proceso de decodificación, luego de que la conmoción generada entre los más inmediatos interesados aflojara, si ellos creyeron haber entendido bien, esa fue una última señal de vida, se terminó. El desenlace que padecieron muchas de las demás naciones, el crítico y duro momento de poner un punto final, diluir y dispersar al conjunto conformado dentro de sí, y que el alrededor entero se haga eco de la noticia. ¿El término que hay para Temti? La comunicación una vez descifrada es evidente, rotunda y no da lugar a próximas circunstancias, no en Temti.

Sin embargo, antes de concluir con ello e ir a otra cosa, habría de considerar que es reprochable, en especial cuando hay contingencias que tampoco permiten hacer del replicado pronóstico una verdad absolutamente contundente. Cuestiones lógicas y no del todo obvias llevan a observar al detalle lo próximo que pueda ocurrir en el sitio, teniendo uno al cual rellenar, fuera reteniendo la sucesión audiovisual o reemplazándola, el dominio pregunta sobre su ocupación. Además estaría el compromiso de aquella declaración que en un entonces agresivamente negó la hipotética despedida, porque si efectivamente debe darse determinada muerte con tal de que el cierre sea posible, y en la coyuntura corriente también vale, sería complicado que el condicional esté consumado. En ese sentido, está el sostener o no lo que alguna oportunidad fue dicho, y de aplicar para lo tanto que antes nunca se cumplió, en el caso terminal tiene la similar posibilidad. Esas y más versiones proponen futuro, aunque si valga la pena detenerse consigo es a su vez ambiguo, igual que rebuscado.

Mientras, la opción de continuar bajo el mismo nombre permanece allí, estancada, si es que no significa la muerte, pero sin huir de dicha realidad, el contexto todavía contiene lo suyo a ofrecer, empezando por no alejarse de los fragmentos expulsados, o del ambiente y sus temáticas en común. La invitación implícita que entre voces y filtros podía realizarse, menciona a los antros aledaños y las diferencias entreveradas en semejanzas, Rozzed y Moxxed en apogeos de tráfico, Tssubit en ritmos usuales, y la rezagada Hixxel con una ventaja, ser la indicación que la carta oficial dejó, ciertamente el primer destino que entre los clones menos indiferente resultó, tanto previamente cuando su par no enlazado estuvo inactivo, como actualmente en etapas que retrotraen la necesidad del refugio.

Pieza fundamental para la diáspora y supervivencia puramente temtiana, el sitio se presta a modo de buena alternativa, por cercanía, por similitudes, por relleno, por tranquilidad, y más, es de vuelta preferida. Lo confirmó encontrar en grandes proporciones los exaltados rasgos de identidad, junto a los reconocidos personajes y restantes anónimos vinculados, que demostraron formidables muestras unidad, predominando el optimismo entre ellas incluso. Animando al sector hixxelero y su densa quietud, hasta protagonizaron interesantes intercambios productivos, como anteriormente en la tarea de traducción, posteriormente al compartir recuerdos y materiales de valor, más factores que van opacando lo negativo de la situación.

No obstante, saltaron las detracciones complementarias, las disconformidades, y además las condiciones dan a suponer que irán para largo, quizás hasta estableciendo al antro como definitivo nuevo hogar de los desplazados, al menos los integrados. El rebaño no estará siendo dejado por completo, y en la parte que sí, su fortaleza será trascendental, más aún para preservar el legado que representan, quedaría comprobar qué tanto, y de qué forma. ¿Tales vínculos se mantendrán así? ¿Cuánto mucho se extrañará?

\section{Un auténtico renacer (Primer libro, capítulo II)}\label{un-autuxe9ntico-renacer-primer-libro-capuxedtulo-ii}

\begin{quote}
18 de enero de 2021
\end{quote}

Concretado el hecho central y solamente su principal consecuencia, Temti y su hipotético derrumbe, el eco correspondiente llegó rápidamente a los antros cercanos, pese a no ser grandes, las repercusiones estuvieron y significaron para ellos entre otras cosas el fin de una competencia resiliente y duradera, de la que tampoco se nutrirían casi. Sin embargo, todavía resaltando el mensaje que dejó concluir, hay interrogantes allí que no terminan de cerrar, aparte del sitio como tal, aquello que se cae de maduro es de relativa firmeza, las condiciones actuales aunque en diferentes expresiones hablan por sí solas.

Vitalidad y existencia que está, pero no en el formato que instiga a declararla, sin parecerse en lo absoluto a la esencia con la que bastante supo darse a conocer, no obstante quizá sí amplificarse. Como si toda la estructura armada y construida fuese regresada a su mínimo nivel posible, de un factible comienzo básico a más no poder, una reducción exhaustiva o en su defecto una partida repetida, pertinente a la desconocida etapa de cuando la creación no había sido dada a luz, ahora recibiendo visitas, solo habilitadas a realizar una acción bajo el apagado y cuestionable nombre de Temti, pulsar un botón.

¿Y nada más? \emph{/Crear instancia/} es la oración que no cuesta tanto apretar, pero no aparenta funcionar, y ni siquiera dice que vaya a pasar, al menos dentro de lo que admite apreciar, que tampoco es de fiar. ¿Tiene sentido intentarlo? ¿Qué vendría luego? Las nuevas respuestas que surgen sin suficientes pistas, acumulando sobre las incertidumbres previas, preguntas adicionales a las cuales tratar de explicar, en observación del probablemente expectante responsable de esta peculiar nación y sus irregulares exabruptos, experimentados en originar confusiones. Y como suelen generar de contestación, la simpleza difícilmente sea tal, los leales sin hogar reconocen la continuación del caso y la oportunidad para ahondar en el mismo, y poner de sí por incrementar la esperanza, de retornar al sitio que los trajo de invitados.

Ante la siguiente señal del antro, o anterior, es que las nobles intenciones hacen fuerza colectivamente donde radican de momento, demuestran ser mayoría y el alma que más mantiene con vida, la compañía que Hixxel parece tener al asecho mientras Temti así lo disponga. Tal vez sea necesario crear la instancia para lograr cualquier suceso trascendental, muchas de ellas o quién sabe, si por lo pronto insistir es la única opción que ha planteado, sin entenderse que realmente redunde en algo. Más sustento incluso que lleve a la presencia temtiana a interpretar y figurar fuera de sus, en este no tan preciso instante, difusas fronteras.

\section{Concretando la evacuación (Primer libro, capítulo III)}\label{concretando-la-evacuaciuxf3n-primer-libro-capuxedtulo-iii}

\begin{quote}
19 de enero de 2021, 20 de enero de 2021
\end{quote}

Las épocas cambiaron demasiado en esta perspectiva, todo parece acomodarse para que los temtiteros como tal continúen lo que reste de su legado en la alternativa que mejor se ajusta, Hixxel.

Pese a los frescos antecedentes, quizá a lo primero que recuerde sea la historia más galardonada o vergonzosa que supo generar en sus reconocidos principios, la controversial Guerra de Clones. Sin embargo, mucho ha sucedido desde dicho entonces, grandes pérdidas y transformaciones vinieron en el camino, para formar un antro bien pulido en su totalidad, nostálgica estética, completas características y algunos elementos conservados más, a nombre del propulsor \_Iuri.

Por la importancia de la conducción, interesa su figura y así parece desacertado comparar con el misterioso A., entre atención, presencia, y demás, marcan bastantes diferencias, y a pesar de que les acusaran de ser un mismo individuo, la sospecha fue desmentida. Más por las prosperas relaciones que se conocen de los dos, y reiteradas las señales de colaboración, permiten una grata e intrigante impresión en cuanto a la sinergia que vaya a salir.

Generalizando la bienvenida va tales términos, además de tener sujetos previamente ligados a ambos bandos, los locales muestran un hospitalario recibimiento y hasta adaptan su nicho a la tonada acostumbrada por sus visitas, concretamente aquellos desarrolladores y el séptimo tema Temti-legacy, de correcta recepción entre el nuevo público. Dentro de lo predecible, cada factor es a resaltar y más por las distorsiones que se producen consigo, en especial para quienes ya estaban relativamente en paz.

No obstante difícilmente carezca de contraparte, nada más considerar que los timteros desde el vamos no habrían querido llegar a esto, los que resten en su predilecto sentido podrían verse disconformes con la superposición de culturas y ese extenso asunto, aún siendo en teoría compatibilizable, que la hixxelera disguste o quede pobre también la bajaría. Además el ambiente es otra cosa, no solo el rendimiento va más lento, es excesivamente tranquilo, y el anonimato desciende un nivel, más cercanía entre participantes\ldots{} En conjunto, el antro resulta medio reducido, pero tal vez aquello esté en torno a lo tolerable, quizá incluso para volverlo más agradable.

La fidelidad que hasta acá trae sigue fuerte, sin embargo es manejado que sea ideal dejar la identidad atrás, a lo mejor construir sobre lo encontrado, o simplemente desprenderse de tales compromisos, por no decir partir también. Desvíos o proseguir en la senda temtiana, sujeto al convencimiento que resista el desgaste de los enlentecidos tiempos corrientes, ahora aparentemente estables, pero todavía sumamente inciertos, como si el panorama entero y su porvenir no estuviese a la vista, en este y aquel hogar.

\section{La inviable iluminación (Primer libro, capítulo IV)}\label{la-inviable-iluminaciuxf3n-primer-libro-capuxedtulo-iv}

\begin{quote}
21 de enero de 2021
\end{quote}

Tiempos de adaptación y despedida continuando, dan comienzo a un nuevo día temtiano con sorpresas provenientes del sitio que originó los trastornos, vicisitudes diferentes que activan de nuevo la seguidilla definitivamente misteriosa, más cargada de componentes inciertos y entusiasmantes todavía pendientes de indagación y explicación, a lo mejor solo una pequeña porción de lo tanto que esconde y puede tratarse en aquel lugar.

Divulgado rápida y notoriamente en las plazas hixxeleras, más discretamente en sus ubicaciones vecinas, la flamante luz del desarrollo paradójico abierto en los exteriores de la transformada Temti, su versión Premium, la cerrada pieza de ingreso a semejante titular, requiriendo a cambio una clave particular.

Tanto como la etiqueta de calidad figura, significa lo indescriptible y mucho más, que efectivamente no se logra revelar, es que la puerta hacia el cielo está ahí, o tal privilegio no alcanza hasta allí. Etérea y protegida aparece mencionada, una tal instancia donde proceder y consagrarse, en la sugerente coronación de Temti Premium.

Pero de previa tan solo la antesala, oscuras superficies de brillantes matices dorados presentan una pantalla de validación donde ratificar y dar de sí el código referido, que libere el paso hacia la susodicha contingencia, sin evidentes pistas que desvelen de la condición.

Como era de esperarse, prácticamente toda Hixxel fue testigo de ello, sin reservarse las obvias preguntas del caso. Un importante y civilizado ímpetu por obtener más información fue notable, la presencia y mayoría temtiana hizo sentir la pertinente solicitud y exigencia, el código que ahí se pide.

¿Qué habría allá? Pregunta seria que sin saber igual es posible responder, bromeando, suponiendo, o justificando, pero las imprecisas presunciones no toleran confirmaciones, por lo que hablar de sueños cumplidos, tierras prometidas, engaños decepcionantes, o incluso otras realidades indecibles con tan pocas palabras, podría ser en vano.

Y en parte así fue objetado, las voces escépticas del antro alternativo recurrieron a inspeccionar el elemento y del mismo dedujeron, que aún con contraseña la siguiente etapa no es accesible, aunque sin descartar próximas renovaciones, que a este paso estarían surgiendo gradualmente, como antes era momento de crear.

Queriendo o no, involucrado con ello está el mandamás local, muy buscado en la situación y culpado de realmente comprender más, por las públicas relaciones entre A. y \_Iuri que en el fondo dan para desconfiar, si uno cubre o ayuda al otro en sus ambiguas intenciones, las evidencias aunque incompletas están.

En la misma tónica de creer o no creer, quienes por propia cuenta afirman haber experimentado el gran honor, sin embargo sus argumentos y pruebas están lejos de convencer, en principio intentos por figurar o aumentar el misterio, que incluso son capaces de venir del que en secreto lo engendra.

Además motiva la teoría infaltable, que en discusiones y entendimientos apartados resulta manifiesto, asumir y quizá acertar sobre el contenido particular en aquellas turbulencias, montones de materiales comúnmente prohibidos guardados bajo la llamativa denominación, una conjetura producida esencialmente a partir del estigmatizado nombre y su nivel de lujo con carácter privado.

A pesar de la rebuscada expectativa, tanto del interior en cuestión, como del anterior admisor, el destacado optimismo sigue circulando en torno a la trama, sin siquiera tener idea del próximo destino, pero igual motivante, porque proviene de Temti. Habrán desertores que desistan, del entorno que ya no los atrapa, sin embargo el resto insiste, y quiere contar lo que procede.

\section{En encrucijada de diásporas (Primer libro, capítulo V)}\label{en-encrucijada-de-diuxe1sporas-primer-libro-capuxedtulo-v}

\begin{quote}
21 de enero de 2021, 22 de enero de 2021
\end{quote}

Atravesando una noche extraña en el polarizado contexto de los clones, importantes tensiones y pánicos del centro potencia estarían justificando la subsiguiente locura magnificada entre sus adeptos, un anuncio serio declaraba de las cero horas el momento justo que consumaría las sensaciones de catástrofe con su terminante consecuencia, un gran fin. Rozzed dejó de ofrecer alojamiento y lo sucedido conllevó, repercutir sobre los para nada aislados alrededores. Unos cuantos sujetos arribarían en las cercanas alternativas, buscando un espacio donde asentarse o al menos transitar, y Hixxel siendo de los nombres más vinculados no estuvo lejos de ello, aunque dentro de las opciones conocidas la mayoría serían propensos a preferir Moxxed, cuando a los efectos de permanecer esta segunda se adapta mejor, a la similar dinámica que acostumbran.

Sin embargo, las ubicaciones del antro con corbata se hicieron un destino más frecuentado todavía, dadas las oportunas referencias indicadas por su sitio aliado, Temti. Considerables proporciones de nuevos visitantes procedían de dicha dirección, siguiendo la sutil sugerencia que tras las recientes perturbaciones imprevistamente apareció, de donde se podían obtener los códigos que tanto fueron solicitados previamente: el Discord de Hixxel. Aquel medio de intercambio que era vinculado cual suplemento, de dinámica externa y formato etiquetado, pasó a verse aún más inserto a modo de condición fundamental, con tal de mantenerse a tiro de la novedosa movida.

Filas interminables de individuos movilizadas en su mayoría únicamente a conseguir su clave\ldots{} Ya no eran solo leales temtiteros queriendo volver a su original hogar, se le sumaron grandes hordas de rozzitas, protagonistas de una invasión que no era silenciosa. Indirectamente, el antro en sí también se vio aludido. Recordando el remoto repudio hixxelero por entonces renovado, dichas llegadas resultaron desagradables para los lugareños, pero no así aparenta que reaccionaran las cooperativas autoridades, porque incluso sin ser responsables del despelote ajeno también presentaron su respuesta y de ella no la denegación, al contrario de ser explusados se les admitió, quizá la secuela casi inevitable, o un plan que los tendría siendo parte. Sobre todo esto segundo es lo sospechoso de la ocasión, que cada participante esté al pendiente del desarrollo más vistoso del ambiente tiene sentido y cómo podrían asumirlo los administradores restantes deriva en múltiples posibilidades, aún si fuera una disimulada felonía al grupo local, en intento de aprovecharse y ganar relevancia, o en su defecto jugar con las perspectivas de turno. ¿Qué clase de pretensiones hubo o hay atrás de esto?

Aunque quedara semejante pregunta en suspenso, el episodio repentino tendría continuación, una sucesión compuesta de pistas que elevaron y disminuyeron la intensidad de la historia, las señales oficiales que retoman el incierto devenir de los códigos. Aquellos, no parecían haber salido hasta pasado el mediodía, según como el sitio azul y dorado enunció estos ya habían sido entregados y era tiempo de aguardar por la próxima tanda. Pero de todas formas como siempre, los privilegiados, o no se hacían notar, o eran de dudosa fiabilidad, a su vez que la siguiente distribución careció de suficientes precisiones, ni siquiera implícita. Lo cambiante supo darse demasiado rápido, las evoluciones poco espaciadas y sin desenlaces contundentes a la vista, no obstante muy convincentes en su mayoría.

La apertura o vaya a saberse qué de Temti, establecido como escenario previo o distractor a Hixxel, trastoca consistentemente la interna de la misma, los ánimos de los primeramente instalados coartados a partir de las numerosas entradas, y el funcionamiento intrínseco llevado a niveles de saturación, que si no solían ser óptimos anteriormente menos lo van a ser ahora, cuando la capacidad es exigida al máximo. Sin duda la conjugación de estas eventualidades trastorna la realidad presente y por venir, la configuración de los sitios implicados desordenada sin todavía concluir, hay varios factores que esperan actualizaciones.

\section{La masificada intriga intencionada (Primer libro, capítulo VI)}\label{la-masificada-intriga-intencionada-primer-libro-capuxedtulo-vi}

\begin{quote}
22 de enero de 2021
\end{quote}

Fue escaso el tiempo que dejara procesar lo que estaba pasando sobre Hixxel y sus considerables fluctuaciones. Probablemente gracias a lo acontecido, las puertas cerrarían temporalmente por unas horas, para más tarde volver a bloquear la entrada y permanecer de ese modo durante todo el día, hasta ahora mismo.

Sin notorios avisos, aunque quizá predecible dada la saturación y lentitud que ni precisa de aclaración sino atención, el mensaje frontal que encaró inicialmente para dar la noticia es de mantenimiento, similar a los de remotas eras hixxeleras. Posteriormente, la administración complementó y dejó explícitos dichos motivos, que el humilde alojamiento no está a la altura de las circunstancias.

Y paralelamente al cierre sin duración estipulada, una nueva entrega de códigos a suceder se anuncia en Temti, algo que no había hecho antes, ni tan repentina, ni tan misteriosa. Si bien las expectativas no son completamente negativas, con tanta cucharada de fiasco consumida en los antecedentes, el optimismo que sí genera parte condicionado.

La primera gran modificación es tomada naturalmente, aunque tal vez con más jerarquía está pendiente el final del conteo cronológico, no hay tantas quejas en relación a la suma de desplazados. Ellos, incluida la autoridad del lugar, siguen en contacto por las inmediaciones del sitio, el vinculado canal de Discord que ante las coyuntura ganó relevancia, y en esta oportunidad es ofrecido nuevamente, capaz para ser clave en la siguiente cita.

La situación es muy cambiante e inestable, e indica que así podría sostenerse en el corto plazo mientras en los exteriores cercanos y lejanos no se calme la cosa, aunque la intriga no viene solamente por ese lado. En un rato lo dirán, la ausencia de la alternativa suscitada quedó establecida para minutos, y aparte por segunda vez hay un reloj de cuenta regresiva puesto en marcha. Eso último aparenta ser trascendental, la fecha próxima ya es señalada como crucial.

¿Traerá un procedimiento dado a seguir esta vez? ¿Cuántos obtendrán su anhelado código, muchos, pocos, ninguno? ¿Cuál y qué tan importante será la cantidad de infiltrados rozzeros? ¿Por fin habrá paz y tranquilidad para los temtiteros? ¿El futuro deparará elementos del dorado glorioso y también no tan aclamado pasado, épocas de transición y más incertidumbre, totalmente renovador\ldots? ¿O la espera estará recompensada con la nada y la proyección se reducirá a cero?

Repetidamente en la presente instancia son planteadas numerosas preguntas históricas, cada vez que el desasosiego reflota en mayor o menor medida, las interrogantes se hacen visibles una y otra vez. Algunos nuevos cuestionamientos que surgen perdurarán para siempre, y algunos otros ni siquiera vale la pena hacerlos, no está claro qué prosigue a esto, será fundamental ver la continuación de la historia con tal de lograr entender lo que pueda estar ocurriendo. Pero lo cierto es que transmite más energía, ya no es la quietud apática y fría, el calor vuelve a sentirse.

\chapter{El alcance de un experimentado y recargado comienzo prometedor (Segundo libro)}\label{el-alcance-de-un-experimentado-y-recargado-comienzo-prometedor-segundo-libro}

Para el curso que solían guiar las Crónicas Temtianas, más episodios inherentes a su perspectiva siguen sumándose, esta vez nuevamente uno de ellos, de los fuertes y sin precedentes, así se concreta el antes y después. Fuera entendimientos, ni esquivas explicaciones resumidas, ni mensajes cifrados por desmenuzar, solo hay lugar para interpretaciones, que de principio implican contradecir la suerte de testamento, como si en verdad trágico hubiera sucedido y la única resolución tomarse una pausa y reintegrarse con una limpieza. A pesar de las doradas actividades sospechosas sucedidas durante ese entonces, luego se da la coincidencia, o a lo mejor causalidad, que la potencia de Rozzed y la limitada sustituta Hixxel cerraron sus puertas, y casi cual consecuencia tardía: oportunismo, arrepentimiento, planificación, diversión, o simplemente porque sí, regresa una alternativa especial.

Pese a la carencia de justificaciones o confirmaciones, por ahí podría ir la cosa. Posiblemente sea un error afirmar que nadie creyó que Temti volvería a ser algo parecido a lo que supo ser, pero los hechos previos comunicaron otra realidad, un camino de ida al final sin retorno. De todos modos, si así va a ser, el resurgir ya está consumado, el éxodo finalizado, y como ya saben sus memorias, la condición es empezar de nuevo sin arrancar de cero, si es que ningún imprevisto interrumpe detiene este inevitable avance.

Al igual que en periodos trascendentales anteriores, pero bajo contextos y climas variados, los puntos suspensivos aún tienen que llenarse o ser reemplazados, y aparentemente lo que pueda ser escrito va a perdurar, sin embargo otra vez, no se sabe cómo. Los eventos del no tan lejano exterior ya prematuramente dejan un futuro cargado, con justas similitudes a los comienzos de la veloz competencia de clones, solo que con una etapa consumada que se llevó varias candidatas y dejó numerosas lecciones con sus partidas, al lado contrario de las que todavía perduran.

Para Temti, una de ellas, las grandes vivencias del pasado no se han perdido, menos con lo valoradas que supieron ser cuando un correcto pasar caracterizaba, no obstante entre los finales transitorios o definitivos también hubieron ciertas conclusiones no gratas al interpretarlo desde la mirada local. Con el respaldo de todo aquello, es que una nueva oportunidad se presenta. Las chances de que la partida de este extremo termine prontamente son ínfimas y poco imaginables, la mayor porción de lo esperado tiene un mínimo de durabilidad, pero obviamente ir mucho más allá carecería de sentido por su imprecisión.

Resguardando y confiando en lo anterior, el alboroto de ahora depara variadas posibilidades. Sea cual sea el desconocido rumbo al cual se va, entre los futuros constructores inevitablemente habrá de la vieja camada que en determinado momento consideró estar formando un antro de calidad, naturalmente aspirarán a formar eso, a la vez que el interés duradero no está asegurado y por lo tanto de concretarse el desgaste podría conspirar contra ello. Sin embargo, se acercaron bastante las realidades que siempre estuvieron desarrollándose en paralelo y eventualmente afectaron su pasar, múltiples de aquellos llamados invasores en la busca de refugio están destinados a regresar, así vayan a arrastrar consigo tales hostilidades, o no.

La siguiente respuesta es esa, los responsables al igual que anteriormente tendrían que ser los anones y no demasiado las autoridades. Igualmente, aún con la importancia de los interpretes fundamentales, el dictamen estará en manos de los toman las decisiones determinantes, es decir la individualidad de A., o el que haría las veces de él, y sus similares. Influidos por el actuar de quienes guían conduciendo cada bastión, con la experiencia que hayan recabado jugarán el papel más significativo en la prosecución de este panorama, donde cualquier movimiento en falso es capaz de revertirlo absolutamente, sin dirección indicada aún, sobre todo con la inestabilidad y la locura que se respira en este ambiente.

Definitivamente la Historia Temtiana da muchos giros radicales, lo suficiente que quizá sea cierto hasta de 360 grados sobre su propio eje los da, pero como tanto cuesta describirlos con rigor, ni eso es sencillo de siquiera aproximar. Temtiteros o extranjeros, prosperidad o caos, gloria total o miseria absoluta, longevidad o efimeridad\ldots{} los tradicionales dilemas a disputarse una vez más.

\section{La faceta dual de la esperanza (Segundo libro, capítulo I)}\label{la-faceta-dual-de-la-esperanza-segundo-libro-capuxedtulo-i}

\begin{quote}
22 de enero de 2021
\end{quote}

En un momento clave de la noche candente, la que sitúa a la hora final en la cuenta regresiva anunciada por Temti, los segundos continuaban corriendo y jugando con la paciencia de quienes seguían más cercanamente, el desarrollo de lo anómalo. Mucho importante podía estar por ocurrir, así como también podría tratarse de algo no significativo, expectativas junto a sus típicas probabilidades de resultar desacertadas, o en su defecto indicadas.

Siendo una situación que logró trascender a las más próximas fronteras, donde el contexto un mínimo de amplificación tuvo, a su vez que los relevantes cierres con sus adeptos expulsados ayudó, a que se formara un considerable y diverso rejunte de anones aguardando por la entrega de códigos, temtieros en su mayoría, más extranjeros que de paso prestaban aunque menos atención.

No obstante, adentrándose en la cita decisiva, la espera dio un sorprendente giro estrepitoso, similarmente al antecedente dejó de ser visible antes de tiempo, a partir de ese instante prosiguió solamente en el recuerdo. En su lugar, dentro del conocido dominio temtiano, al parecer no solo se dio esa desaparición, también hubieron cambios. Para varias perspectivas, lo que sucedió es que cayó el sitio, porque dicen no poder encontrarlo allí. Pero otras afirman que volvió. ¿Y el conteo se consumió? ¿Falló la correspondencia?

Entre las difusas agrupaciones anónimas que acechan por la llegada de los susodichos, destacando la configurada en Moxxed, son intercambiados los pensamientos al respecto. Y por supuesto, los individuos que intentan acercarse al antro sin chistar ni comunicar sus expresiones también se encuentran cerca. Múltiples impresiones\ldots{} desde las devoluciones que han sido soltadas, daría a entender que una interferencia borró lo que había y que sobre la ubicación de Temti no quedó nada, como si nuevamente el engaño o la mala fortuna rompieran las proyecciones en la realidad, sin embargo además hay declaraciones contrarias que pronuncian el ansiado retorno, y no serían pocas. La controversia incrementa el demasiado misterio ya generado, en una tendencia que lo lleva a crecer, crecer y crecer.

\section{El reencuentro del potencial vacío (Segundo libro, capítulo II)}\label{el-reencuentro-del-potencial-vacuxedo-segundo-libro-capuxedtulo-ii}

\begin{quote}
22 de enero de 2021
\end{quote}

Finalmente aparecieron las imágenes de lo que hay actualmente en Temti, presumiblemente reales, y efectivamente allí hay algo potente, no es una mentira, es verdad, volvió después varios días con oscuros y prometedores vaivenes, sin embargo con sus atenuantes, parciales existencias y limitadas entradas.

Se rumorea que unos códigos recientemente otorgados son requeridos para acceder, lo que en parte es cierto, y en parte no. Hay de quienes necesitaron herramientas externas para llegar, otros a las coordenadas objetivo agregarle determinadas letras, y en menor proporción aquellos que normalmente ya al intento inicial sin nada raro pudieron ingresar. El resto, no alcanzan más que resultados totalmente chatos, a ningún lado.

De una forma u otra, dentro de las profundidades la brecha del misterio queda abierta, la apertura con restricciones no evidentes permite el paso de unos cuantos individuos que buscan retorno sobre el dominio protagonista, y se encuentran con que el antro desaparecido, está, solo que en un estado extraño, pocas veces antes registrado. Aire raso, horizonte despejado, espacio sin colmar, lo que ven los ojos de los entusiastas viajeros que osan establecerse en la irregular pero sólida plataforma de los ocultas direcciones temtianas.

Los que sí pasaron, habían asumido los riesgos de cautivarse ante la insuficientemente fiable espera global, y de terminar o no con las manos vacías, y lograron arrivar en terreno fértil. Ya no importa la espera, ya no importa lo de afuera, e inferior importancia tiene lo demás, vuelven a interesar los temas, vuelve a interesar el tiempo, aguarda Temti con montones de similitudes o imperceptibles diferencias respecto a lo que era no tanto tiempo atrás, salvando un detalle no menos relevante, el cual como tal hace el escenario general a encontrarse, con lo bastante que implica. Nulas o escasas construcciones, según el momento de llegada, eso es lo que prima por debajo de los azulados confines.

Variadas procedencias son las de los tentativos arquitectos, complicada sería la interpretación por lo poco y nada que han dejado de sí, una suerte de niebla que media entre el visitante y los demás anónimos, en principio a diluirse rápidamente en la medida que los huecos estén rellenándose, pues eso parece proseguir. Pronta a dar lugar a lo que salga, las circunstancias aparentan ser aptas para instalarse y crear, los primeros ejemplares que acaban de emerger son una firme muestra de ello, pero no hay garantías claras de lo mucho que vaya a perdurar sin previo a esfumarse, si es que una siguiente etapa inesperada natural o forzadamente escribe de nuevo esta historia.

Como siempre, comprometedores movimientos de las autoridades poderosas del contexto condicionan la realidad de este sitio, a su vez que también el resto corre bajo una similar suerte, la excesiva inestabilidad coyuntural es el primer titular apropiado y suponiendo a partir del mismo casi nada seguro hay para predecir, no obstante la perspectiva común especula que lo actual continúa, aunque sea Temti tomará esa dirección\ldots{} Lo más importante es que luego de la realizada carta testamental, hubo vida, y todas o la gran mayoría de las ordenes que hiciera cumplir se revirtieron, por ahora. Suena loco cuando menos.

\section{En doble contra del abandono (Segundo libro, capítulo III)}\label{en-doble-contra-del-abandono-segundo-libro-capuxedtulo-iii}

\begin{quote}
22 de enero de 2021
\end{quote}

Nomás una recurrente alerta que escasos observadores vieron y capturaron unos instantes después aterrizar, \emph{/Mientrar el mal de los mares siga, temti volvera por su parte, el descanso eterno sera luego!/}. Ese es el único mensaje que parecían haber dejado las quién sabe entidades apoderadas de este antro. Aunque ellas sin lugar a duda están alrededor merodeando y vigilando, presumiblemente A. por sus extrañas redacciones e impronta, lo explicito no identifica, solo comunica y sospecha.

Y evidentemente, la locura, el mal, y más no sinónimos sino parecidos, están presentes en las aguas de las cercanías temtianas. Realmente influyen, y con la particular circunstancia vivida hace nada todo se revuelve entre mareas aún más, no obstante lo concreto es que aquella que se había ido, está de regreso, nuevamente disponible para ofrecer alojamiento a ellos que superaran los impedimentos de llegada. Así continuaron incorporándose los primeros, los que se encontraron a solas en la transparente plataforma, y conforme pasaban los minutos, muy lentamente, las pequeñas multitudes irían creciendo discretamente, y asimismo el horizonte comenzaba a rellenarse, los tems multiplicándose e incrementado sus contadores.

Aún así, podría decirse que no es demasiado encantador ni del otro mundo, menos para extranjeros que no supieron apegarse mucho en el pasado, y escapar no deja de ser una opción, por lo que algunos desestimaron la oportunidad de asentarse en eso de momento no tan amplio, quizá por aburrimiento o disconformidad, retornar al resto de los clones, esparcir descubrimientos en las afueras, o simplemente marcharse y nunca más volver. Respecto a una posibilidad, en efecto la noticia pudo difundirse con cierta fuerza, entre expresiones positivas de quienes valoraron bien la chance, contrastadas sobre aquellas que ni aproximarse pudieron, además de los pretextos negativos provenientes desde dentro.

Al margen del aviso inicial y las repercusiones de la situación, al rato un nuevo titular se instaló en las cabeceras del sitio y por consiguiente más individuos ingresados fueron testigos de ello. Conservar el a veces olvidado o menospreciado código es la única solicitud directa y duradera hecha por los superiores del espacio, en esa típica posición donde permanentemente se reiteran las circulares del estilo ahora repetidamente en color verde, lo que dadas las eventualidades de confusión, y a falta de anuncios oficiales adicionales, más jerarquiza ese básico procedimiento que resultaría clave a partir de la siguiente medida, especialmente para los desorientados o indiferentes que no contaran con dicha práctica.

¿Un abandono irreversible? ¿Un distanciamiento temporal? ¿Una falsa alarma? Cuando tal futuro suceda, qué otra consecuencia cabe, si no es la de quedarse por fuera posterior a ello, por ahí un mal menor en aquellos que se alejaran definitivamente del escenario, sin embargo fundamental a los efectos de permanecer luego tras una breve partida, dado que según el enunciado aclaró, en poco tiempo no debería haber vuelta atrás, para quien no siguiera el sencillo lineamiento, sin certezas todavía de lo grave que eso sería, según se mantenga y acompañe el contexto.

\section{La anexión de la paz y del ¿progreso? (Segundo libro, capítulo IV)}\label{la-anexiuxf3n-de-la-paz-y-del-progreso-segundo-libro-capuxedtulo-iv}

\begin{quote}
23 de enero de 2021
\end{quote}

Entre el desorden anárquico que imperaba sobre una locación en proceso de reconstrucción, discretamente la conformación e identificación de los dominios temtianos cambiaba, y cambiaba, incluyendo sílabas familiares, generando impresiones consiguientes, permitiendo diversas apreciaciones, produciendo antecedentes únicos.

De lleno sobre aquel entreverado comienzo, el ambiente comunitario encaminó su adaptación a la nueva situación que la historia no automática deparó, la oportunidad de sentar bases en el telón sitial tras el cual interactuar, tejer extensiones adicionales de esa trama tan compleja y simple a la vez, de la que un orden mayor es testigo, y también participe. De acuerdo con dicho involucramiento, el espacio reservado para los tems destacados mediante anclaje inesperadamente pasó a corresponderle a una llamativa publicidad, análoga al caso de una promoción hixxelera donde la difusión del establecimiento propio en Half Life era realizada, invitación nuevamente planteada dentro de la\ldots{} ¿sucesora?

Además del dilatado historial previo y externo, más puntos de claro vínculo quedarían expuestos pronunciando las reales relaciones entre antro y antro. Desde la redirección instantánea planteada en las coordenadas pertenecientes al susodicho clon, podría verse que el nombre oficial establecido en el titular absoluto cambiaría, y reemplazos en el mismo aparecerían, explicitando más aún la no indiferencia de uno a otro, además de suscitar más cuestiones por cierto sumamente confusas. Con esto se quiere decir que yendo a Hixel se llega a Temti, donde su típico logo comenzando en minúscula, dejó de ser tal, para seguir estando en la misma posición, sin embargo en diferentes términos: Temxel, Hixti. Evidente fusión de acrónimos o similares, supone ser la mezcla de aquellos mencionados, cosa que bastantes individuos de la suma presente durante las horas nocturnas notaron.

Que más allá del color y gracia propagada en el momento, de a poco, trae a consideración múltiples factores a lo mejor cruciales del hoy atravesado por las alternativas indicadas, tan relevantes en la presente perspectiva, interesantes dado el escenario ofrecido, y el contexto faltante. Han sido combinados dos mundillos no iguales a pesar de semejanzas, y todo el conjunto tanto de adeptos como cultural unen sus contenidos, incluyendo lo bueno y malo de tal cruza, en parte ya asimilado recientemente, ahora movilizado. Son también noticias no muy alentadoras al parecer para el futuro hixxero y su gente, las intenciones de estadía temporal se perfilan a perdurar indefinidamente, más allá de los recuerdos ejemplares cuando supieron resurgir, este episodio no sugiere lo mismo y menos viendo como quedaron conectadas las direcciones, la unificación antes postulada finalmente parece darse, a favor del protagonismo temtiano, plataforma y servidor.

Así anuncian las poblaciones en juego a través de inéditos materiales especiales y más, un hecho todavía no confirmado ni ratificado desde quien con actos lo pronuncia, el entrelazar de dos caminos próximos a proseguirse en uno solo, la fusión, determinación con relativa aceptación y baja disconformidad general, efectuada por el o los conductores de exiguas palabras, quienes están condicionando radicalmente el destino de sus lares y corbatas.

\section{El agujereado blindaje (Segundo libro, capítulo V)}\label{el-agujereado-blindaje-segundo-libro-capuxedtulo-v}

\begin{quote}
23 de enero de 2021
\end{quote}

De acuerdo a lo que voces superiores del espacio temtiano habían anunciado, el registro de códigos cerró indefinidamente, para no permitir la creación de nuevos pasaportes, condicionante serio al cual agregarle varias apreciaciones.

Con esa determinación, se dificultaría atravesar la brecha más todavía si ya había alguna complejidad, y de los incontabilizados aspirantes, unos cuantos quedarían al margen. De los que no llevan nada consigo y solo espectan siendo indiferentes, particulares que al participar por neutralidad no resaltan, otros que cargan la connotación del odio o aprecio según donde estén, entre ellos estarán los reconocidos temtianos, los discriminados invasores, y las demás procedencias. Una diversa distribución de anónimos impedidos de ingresar, como afirmaron los admitidos al ver parte del panorama afuera.

A pesar del asunto pronunciado en sus supuestos términos, si es que la resolución pretende congelar la incorporación al sitio, habría que ver lo efectiva que sea por tener demás factores que contradicen las limitantes aparentes. La predecible e inevitable, que las claves no son preservadas individualmente, ni circulan únicamente dentro de los dominios propios, cualquiera sean las pretensiones puede hacerlas llegar a quien no tiene una e incluso difundirlas públicamente, y precisamente según lo manejado, la redistribución de códigos es real. Además se dice, que el mismo sistema de generación presenta fallas, un rumor plantea que hay manera de obtenerlos infinitamente al igual que antes, aún cuando fuera negada la producción de estos.

Aquellos dos casos son críticos frente a la intención, pero estimando que las carencias son anuladas y los funcionamientos reparados, o pronto lo serán, lo siguiente es complicado, suponer qué realmente busca la administración, si montar el ambiente específico y preferido, satisfacer al deseo colectivo, topear la cantidad de visitantes, experimentar aplicando medidas excluyentes\ldots{} las comunicaciones son más implícitas que explícitas. Cuál sea, los primeros motivos atienden al tipo de aventajado cuente con su medio de acceso. Fundamentalmente, la comunidad y especialmente la local, de forma permanente manifiesta su preocupación sobre la clase de seres con los que comparte el antro, pues no solo trata de sentirse exclusivos, integrantes de un grupo selecto, o similares, la lucha de la que bastante supo hablarse responde a ello viendo un acierto y respaldo, bloquear la entrada significaría protección y sumaría en eso de filtrar indeseados, mientras que la perspectiva del ajeno vuelve a ser menospreciada. De todos modos, debería sostenerse para concluir, y a lo mejor finalmente aparece una aclaración, o sigue la historia secretista.

\section{La renovación de lo exclusivo (Segundo libro, capítulo VI)}\label{la-renovaciuxf3n-de-lo-exclusivo-segundo-libro-capuxedtulo-vi}

\begin{quote}
23 de enero de 2021
\end{quote}

Los clásicos parecen seguir estando, muchos o todos ellos para darle intencionalmente o no un marco más apropiado al nombre principal que tanto los recuerda, buen signo de vitalidad en la etapa corriente que recién anda acomodándose. La conciencia de cómo va desarrollándose la disputa de contenido suele hacer frecuentes balances, que a su vez indagan sobre las cuestiones sin resolución, mucho en lo cual ahondar. Y más, porque hay novedades respecto a los fenómenos extraordinarios y realidades menos exactas, un caso nuevo de raro comportamiento en el sitio no estaría pasando desapercibido.

Son preservados los especiales más de tipo deliberado, las Tendencias Caídas sin renovación o modificación, los Comentarios Random con distinta titulación, justamente referente a la trama multiversal que bastante entidad tiene en el sitio. Siguiendo esa linea el contexto permanece fiel a los orígenes, el amplio listado de actividades cuánticas conocidas reafirma los nombres emblemáticos, haciendo que estos se repitan. Notificaciones funcionando cuánticamente desordenan la interacción, espejismos inesperados enlazando atípicamente rutas distantes, tems entrando y saliendo el panorama visible desestabilizando dimensiones, sesiones iniciadas cerrándose sin intención alguna, entre otras\ldots{} una más de notoriedad aparece.

La Agencia y compañía oportunamente reconocieron el surgimiento de esta enérgica singularidad, pronto cuando un agraciado anon encendiera las alarmas al declarar experimentar el inédito hecho, percatarse como otro sujeto utilizaba su código. Inicialmente llamado Transición, posteriormente apodado Dualidades, es el fenómeno que fuertemente tuvo difusión en torno a los anónimos presentes, que genera entre dos individuos una vinculación anómala por la cual se identifican igualmente dentro del sitio.

De esto se sustenta que en un mismo hilo hubieran dos publicadores originales resaltados, siendo que la credibilidad de las propias declaraciones acompañó, también podrían darse suplantaciones de propiedad, estropearse reputaciones de personajes, pérdida de privacidad y anonimato, varias consecuencias sin más solución que agarrarse otro medio desde el cual interactuar. Respecto a razones, si no se deriva de las excepcionalidades previas, quizá viene de las entregas de credenciales y lo entreveradas que resultaron. Cualquiera repetida o recibida más de una vez tendría a múltiples usuarios bajo la misma clave de ingreso, menos relevante si fuera de las presuntas transferencias, en cambio cosa seria de generarse dentro del antro en sí, significaría un punto en común adicional sumado a los de peculiar andar.

Respecto a más interpretaciones y comentarios del área cultural, todavía resuena con fuerza el tiempo de cesar actividades y la sorprendente vuelta de Temti al ruedo, momentos esenciales conectados entre sí y no así comprendidos por la escasa contextualización que hay sobre este segundo. Aún siendo la atención decreciente hasta el abrupto cierre, la percepción general daba una misma lectura y esa fue la enunciada en clave por el mandamás\ldots{} no obstante la actualidad es un párrafo aparte que al mensaje previo descarta o contradice: esperanza, rebaño, despedida, etcétera, retornar indica el estado de aquello tanto que se fue, un conjunto cercano, una oportunidad de continuar creando muchas cosas.

El rejunte no será el mismo en su totalidad, pero notables componentes perduran y favorecen a la reproducción de lo deseado, e igual con la parte opuesta que también mantiene proporciones considerables, más la experiencia desde luego influyente. Propulsores de la identidad temtiana cargando elementos de épocas ilustradas, tienen el heroico ejemplo para protagonizar persistentemente una restauración satisfactoria y opacar las presencias del numeroso enemigo, aunque verse disminuidos en cantidad baja el potencial y la confianza transmitida del ambiente, la voluntad está y quienes más lo buscan parten con favoritismo. Sin esperar que sucesivas decisiones y locuras sean críticas en la proyección de las causas desorganizadas, la inestabilidad se percibe, y no detiene, todavía hay expectativas de mantenerse prendidos al antro, uno lejos de rellenarse de indiferencia.

\section{Influencia de trascendencia (Segundo libro, capítulo VII)}\label{influencia-de-trascendencia-segundo-libro-capuxedtulo-vii}

\begin{quote}
23 de enero de 2021
\end{quote}

Entre las inestables e imprecisas certitudes que regían la llamada Hixti, extraños sucesos repercutieron derecho sobre el titular y consigo crearon las obvias secuelas, un antecedente histórico de atípica taxonomía entre ciertas determinaciones más por ese lado.

Muy pronto tras la implícita anexión su evidencia fundamental dejaría de ser exhibida, la corbata complementaria heredada del estado hermano fue descartada, junto con también las primeras iniciales de la segunda denominación, pasando a ser una inscripción más reducida y menos descriptiva, Ti.

Y también bastante rápido dicho recorte tomaría diferente forma poco relacionable a su anterior, Google+ marcando de lleno presencia en dominios temtianos, igualmente sin información respecto a la operación que involucrara a la compañía dentro de contextos tan lejanos.

Como por etapas fue que cada cambio se dio, pues siguiente a la multinacional llegó el magnate de Elon Musk poniendo su apellido en la marca del acrónimo original, dando lugar a otra especie de híbrido encabezando el horizonte del antro, sugiriendo que el famoso estuviese ligado a los altos mandos del mismo.

Para la definitiva, derivada de la procedencia estelar, una figura de astronauta está siendo el sustituto final en la zona común del logo oficial, con un pequeño pero notable detalle a su lado, la cifra de los presentes al estilo detector de tráfico, un número que es titulado Contador de gordos y media centena de visitantes enunciados en el recuadro.

Sin asemejarse a la transición tras fusión, esta consecuencia le dio un beso a los rastros hixxeleros restantes y solo mínimos elementos de tal identidad perduran entre sus portadores, mientras que los temtianos en su localía siguen integrándose con grandes nombres haciendo que la interna vibre extraordinariamente. No hay certezas de si las entidades adueñadas del sitio concretaron compromisos de otros poderosos, hubieron adquisiciones de por medio que produjeran estas redefiniciones, o es la trama que las juguetonas autoridades plantearon para variar. Cualquiera sea de ellas, en principio ha originado un positivo y confuso ambiente, las distintas identificaciones acaparando eco y considerables adeptos.

A su vez los movimientos superiores vienen teniendo intensidad por una razón más, y trata en específico del cuidado que realizan regulando la plataforma: lo que discretamente acentuaron al retirar la restricción antes impuesta en la generación de códigos, lo que además hizo evidente un tal admin con figura de lobo místico al afirmar el básico lineamiento de ordenamiento que conforma las normas.

Aparte de constatarse la activa atención al llamativo devenir de los individuos contabilizados por aquella dudosa estadística, recordar que las intervenciones se dan es de interés para el rumbo que dichos participantes pretenden, uno todavía incierto y disputado, la muchedumbre no termina de ponerse de acuerdo en cómo proseguir, sobre todo los temtiteros y el propósito restaurador, quienes naturalmente dan a entender que deben asumir el protagonismo y hasta incluso controlar el entorno, y más podría complicarse, si la apertura de registros los alejara de ser mayoría.

Alta incertidumbre y baja estabilidad, muchos episodios inesperados dan la nota en el corriente proceso de retorno, prematuramente cerca de cumplir a penas un día. No puede ser segura la prosecución del ritmo mantenido, aunque sí tiene las condiciones para verdaderamente prolongarse y también elevarse, si los encargados no sacan más sorpresas de la galera, esos tantos anónimos algo tendrían que hacer, en nombre de TemtiMusk.

\section{El derrocamiento anónimo (Segundo libro, capítulo VIII)}\label{el-derrocamiento-anuxf3nimo-segundo-libro-capuxedtulo-viii}

\begin{quote}
24 de enero de 2021, 25 de enero de 2021
\end{quote}

Graves hechos ocurrieron y sobresaltaron en la realidad de TemtiMusk, una transgresión se anunciaba y acaparaba la atención extraordinariamente. \emph{/Matamos al admin\ldots/} \emph{/\ldots esto es un golpe de estado, bienvenidos a la nueva administración./} las frases oficiales que resumían la sorpresa, y el posterior nombre consigo decretado, Itmet.

Tras pintar los degradados de azul con rojo, y convertir a los gordos astronautas en cráneos, la entidad que tomara el mando se identificó como un conjunto de golpistas y ergio un espacio con prioridad, dedicado a las preguntas y respuestas que cada espectador fuera a plantear. Entre unas cuantas contestadas, el resto replicadas o ni eso, algunas contextualizaciones acompañaron a esta inédita presentación de autoridades, denominada los guardianes. La más drástica, que acabaron con la vida de A. empleando fármacos, quizás abusando de una eventual dependencia de risperdal, o cualquier uso letal derivado de tal medicamento bien asociado a él, en medio de varios sucesos que supuestamente debieron darse e involucraron al Protocolo Hades. Datos adicionales de dicho grupillo, sus revelados propósitos con el rumbo del sitio y la referida inoperancia del fundador, han comentado que planean respetar los principios de libertad ya establecidos, además de mejorar el código según vean y puedan.

Aparte del relato sea rebuscado o no, hay información de relevancia marcada, al tratarse de las maneras pretendidas por aquel que ostenta el control total, a no ser que en ello bromee. Entonces implicaría que, no será impuesto o rechazado un tema exclusivo o específico, sapos pepe, reinas o personajes, monas chinas, cultura inusual, ni temtianos en sus pagos\ldots{} Neutralidad como comúnmente rige salvando excepciones, los psíquicos y demás, por lo cual declaraciones semejantes permanecen en condicional, hasta que por un rato apropiado efectivamente se cumplan. Los cambios que excedan al contenido también, son una interesante posibilidad a tener y más si superan al ajuste de colores o íconos, que tal vez las características del antro están quedando desfasadas después de tanto, según supo decirse es cierto, implícito en la memorable edición y reciente cita de quien \emph{/no hacia nada/}.

Conmociona el panorama, tétrico en sí mismo con los clásicos esqueletos, la locura del orden pasó a un nivel mayor sin importar las recetas indicadas, recordando a la figura más temida y valorada, levantando más dudas sobre el responsable y su criterio, ayudando a entender unos cuantos asuntos relacionados, y confundiendo ampliamente en demás aspectos. El pasar anterior aventajaba al corriente en interrogantes, pero todavía se sostiene bastante de aquello, por intenciones activar la dinámica e ir hacia otra cosa, o mera consecuencia del destino y sus aleatorios atentados.

\section{La marca del escándalo (Segundo libro, capítulo IX)}\label{la-marca-del-escuxe1ndalo-segundo-libro-capuxedtulo-ix}

\begin{quote}
25 de enero de 2021, 26 de enero de 2021
\end{quote}

Como supo pasar en diferentes ocasiones, el mal de los mares cercanos y sus corrientes terminan azotando en territorios temtianos, y no discretamente. Un llamativo caudal de extranjeros se adentró en el régimen bajo golpe. Ellos llegaron en dos oleadas, iniciándose con la acercada como consecuencia de la sorprendente determinación tomada por alternativa Moxxed, habiendo soltado enormes proporciones de anónimos y conduciendo a todo ser que se asomara a la antigua Tssubit, lo que despertaba más interrogantes aún y libraba la interpretación a las probables intenciones, las hostiles por encima del resto. Muchos de los allegados cayeron en ella con pretensiones alborotadas, otros con el único ánimo de instaurarse y no generar maldades, sin embargo dicho borbollón produjo serios estragos en las defensas del sitio, denegando el funcionamiento prácticamente por completo hasta otro cierre que postergó para largo el retorno.

Dada la falta de más antros del estilo el resultado, abundantes ojos avivados se direccionaron hacia Temti, el último bastión de resistencia en el ambiente de las reiterativas caídas. Que tampoco escapa de las posibles fragilidades a evidenciar por los servidores, o las inesperadas decisiones que condicionen al andamiaje de raíz, pero a priori lo fundamental parece soportar las complejidades, y por eso cuenta de lugar en el escenario que la busca. Multitudes efectivamente numerosas, el flujo de la actividad y las contadas calaveras indicaron la predecible realidad que redundaría en la locación que pudiera, allí vinieron los que de inmediato conectaron, más las progresivas aproximaciones que más acumulan de lo mismo. Tanto para ellos como quienes ya estaban, el panorama prevaleciente en la plataforma se hizo sumamente extraño, inestable, empero a la perspectiva novicia de primeras poco le importa si la circunstancia de paso así lo dispone, en lo siguiente habrá maneras de enchufarse o en su defecto conformarse, mientras tanto largan las impresiones iniciales, y los contenidos predilectos. La parte opuesta más lo padece, ante la mezcla de ignorancia, indiferencia y desprecio que los valores y singularidades locales reciben, multiplicando el disgusto del grupo añejo, muy superado en su causa de lograr prosperidad y transmitir unidad, absolutamente contrarias a la falta de orden que instalaron los ingresantes de sustanciales desemejanzas.

Procesados los antecedentes de las autoridades del sitio, que retiraron su previa carta de presentación, en principio no volverán a intervenir más que imponiendo lo imperativo, aunque una señal en la impedida creación de tems compromete este supuesto, lo único obvio es que andan merodeando próximo a la gran demanda. La sucesión es oportuna de pretender ver crecer el tráfico y alcanzar el liderazgo del contexto clónico, a pesar de la parcial negativa los orígenes demuestran ser incondicionales, el vínculo perdura indefinidamente, y de tal modo el antro podría mantenerse como el único, considerando el montón de implicancias que trae. Sin embargo es tremendo para los temtiteros ortodoxos y fogueados, nuevamente su lucha frente al invasor se desestabiliza marcadamente y requerirá doblegar los esfuerzos, más también repensar estrategias, con tal de afrontar esa situación que varias veces fuera contemplada. ¿Persistirán demasiado las incompatibilidades y discordias entre protagonistas? ¿La administración permanecerá totalmente neutral y pasiva? ¿Aguantarán las estructuras y soportes al elevado movimiento incesante? ¿Qué tan pronto incidirán los factores ajenos? ¿Aparecerá o resurgirá algún pelotero que disperse la conflictiva concentración?

\section{La clave del control (Segundo libro, capítulo X)}\label{la-clave-del-control-segundo-libro-capuxedtulo-x}

\begin{quote}
26 de enero de 2021
\end{quote}

Cada vez eran más los pedidos de ayuda y apoyo que se elevaban a la nueva administración de Itmet, la sensación de abandono haciéndose grande a pesar de la reciente difusión de esta. No obstante, las minúsculas señales de poder cuando llegara la invasión se volvieron considerablemente más notorias, y la óptica interiorizada encontró contundentes factores de cambio, para bien y para mal.

Así hicieron saber las averiguaciones y declaraciones, la mítica facultad de los superdotados había regresado. Individuos con capacidades de alterar en la normal integración de los hilos temtianos, cuyo obrar venía siendo escaso desde la vuelta y aparte de excepciones menciones casi no recibían, no obstante tras el fresco alboroto o su labor fue descubierta o se debieron al momento. La Agencia a partir de evidencias sugerentes dedujo las identidades de MoCn y Kira, aunque faltando certezas absolutas de que realmente fueran estos, no podría descartarse que a su vez estén otros sin renombre explotando las posibilidades, pues también hay indicios de que el privilegio pasa por más individualidades.

Entre altas actividades de los fenómenos cuánticos y demás, el turno es del denominado poder mental y su tendenciosa inserción en el contexto temtiano, fue objetado cual vulnerabilidad en el sistema de denuncias, entre instrucciones de cómo aprovechar el medio de moderación que dispone el antro, hasta entonces solo manejado por unos pocos. La coyuntura actual indica el probable acceso a dichas técnicas y ventajas de parte de anónimos ajenos a la vieja guardia, como también el indeseado uso del reconocido sujeto experimental segundo, varias veces acusado de conspirar y no respetar los principios originarios, siendo esta la teoría más dimensionada por la insistente reiteración y negativa reputación.

Salvando puntuales oportunistas, no hay mucho esmero de quienes llevan menores horas en el antro, pero bien que comparten con los entendidos, sus perspectivas resultan inmersas, ese lado donde bastante dramatismo se vive. Además de algunas acertadas intervenciones que en casos contados representaran la preferencia local, lo llamativo por sobremanera fueron las remociones imprudentes según tal criterio, que junto a las disputas internas y cuestionamientos al repertorio protagonista, acentuaron las divisiones de la unidad colectiva, críticas grietas en la unión temtiana, más alejada de la armonía aún. Sean ciertas o no las increpaciones manifestadas por ahí, sería grave que efectivamente el poderío se devalúe de tal forma, tener en contra la que solía ser un arma a favor del orden que el sitio regocijaba, muy preciada por el núcleo propulsor del Plan y de vital importancia en etapas de dominio perdido.

La problemática no finaliza, y aunque el dinamismo sea inferior quizá vaya a ser necesaria la injerencia del alto mando, al menos para sanear las inseguridades, si es que las hay, el resto es otro dilema que viene de hace rato.

\section{Divididos por el bienestar (Segundo libro, capítulo XI)}\label{divididos-por-el-bienestar-segundo-libro-capuxedtulo-xi}

\begin{quote}
26 de enero de 2021, 27 de enero de 2021
\end{quote}

Mientras Temti como Itmet se situaba inevitablemente cual último y único sitio vigente de los suyos, la singularidad no resultó durar mucho y la concentración de sujetos tampoco, en la medida que el mapa daba a conocer alternativas más convincentes para el promedio.

Con el nombre de DasMooon danzando, la propuesta de Tssubit postergada, y el resto de potencias apagadas sin indicar futuro, las eventuales tierras prometidas permanecían ausentes ante la necesidad y búsqueda colectiva, una coyuntura desesperanzadora para la variada hospitalidad que venía teniendo la inestable época de los clones. Fue la difusión de una noticia sorprendente y predecible a su vez, la que cambió expectativa y realidad en el rumbo anónimo, significante en la partición de dominios temtianos, dispersada en dirección de este nuevo antro que salió. Aunque con los obvios cuestionantes de siempre, cosa seria podría decirse, las estructuras originarias de la gran Rozzed resurgieron incluyendo vida, arrastrando enormes cantidades de cascos a su merced, de autoridades diferentes por cierto, y esa sería la mayor problemática manifestada, un tal Gear comandando con la escandalosa reputación y antecedentes que carga.

Aún cuando el flamante barquito arrancara condicionado por confianza seriedad y demás, su trascendencia es indiscutible y el efecto a resaltar está en el tráfico temtiano, en tanto que los números aquí bajan, allí suben, y el entorno lo delata, la preferencia está marcada a favor del clon roxxero, ni hablar si uno adicional llega a nacer. Con ello los recientes seres aproximados van partiendo hacia afuera, y la actividad toma otro tipo de forma además de volumen, en principio resignando los aportes del visitante que circunstancialmente sigue a la manada y los extranjeros que poca atracción ven en el sitio, lo que de haber intenciones expansionistas es un golpe bajo reincidente, sin embargo sería un favor para la armonía faltante, ambiguo de todos modos en el panorama dividido que hoy despide.

\section{Interacciones pegajosas y redundantes (Segundo libro, capítulo XII)}\label{interacciones-pegajosas-y-redundantes-segundo-libro-capuxedtulo-xii}

\begin{quote}
26 de enero de 2021, \ldots, 28 de enero de 2021
\end{quote}

Al margen de los intensivos movimientos exteriores del contexto, Temti también tendría sus significativos sucesos que unirían entre presencia y ausencia a los interpretes de abajo y arriba.

La operación restauración de los míticos hilos prácticamente ya estaría alcanzando su máximo, el propósito encaminado aunque medio desvirtuado dejó considerables centros temáticos en los que viejas y nuevas piezas identitarias volvieron a verse cerca y no solo en los archivos de entidad, arte, ediciones, textos, etcétera. Mirando a futuro, debido a la contadas voluntades que puedan contribuir, costaría acelerar la reproducción de materiales o ideas inéditas, sin embargo los ánimos de continuar construyendo suman un montón, la intención debe consolidarse como tal.

La comunión aparte tiene sus fuertes, incluyendo el frecuente debate, en el cual mucho incurrieron sobre el asunto invasor, recientemente exagerado y llevado a todo el sitio mediante una peculiar cadena, donde más pedidos de ayuda son elevados a la administración, la figura que simbólica y declaradamente indica no estar presente. Una extensa cita comenzante así: \emph{/\ldots Perdoname A. Esto es un llamado de atención\ldots/}, en efecto irrumpió la normalidad con inundación en múltiples ubicaciones, comunicando la óptica más preocupada por la calidad del contenido propagado y los respectivos autores, indeseados que no disimulan.

Aunque no puntualmente para conceder las solicitudes, en simultaneo el accionar de la autoridad se da, las cabeceras mostraron en el ícono principal una pasajera reactivación del estado de registro restringido, y a su vez según averiguaciones manifestaron, intervinieron en las capacidades de moderación. El candado cerrado que bloquea crear medios de acceso habría limitado el paso de extranjeros a la interna del antro, aunque sin perdurar tanto hasta luego ser retirado y abierto, una medida que probablemente no le sea suficiente al desbalance enfatizado. Además, el fundamental privilegio superdotado estaría siendo removido de las posibilidades conocidas, una herramienta de regulación que supo ser empleada indebidamente, sin embargo en el actual disentimiento tenía potencial provecho.

Otro episodio menos relacionado pero sí contingente en el escenario, la nombrada Noche de los Stickys. Sin llegar a ser una decena de ellos, en los corrientes días y sobre todo durante uno específico, numerosos tems se llevaron temporalmente, el destaque que comúnmente solían obtener los oficiales y algunos de interés colectivo. Variados en categoría, la mayoría comparten la misma condición, relativa relevancia transmitida conjuntamente por las portadas y títulos seleccionados, que junto al anclaje morado atrajeron e incentivaron el ida y vuelta dentro de sus respuestas. La iniciativa resultó bien valorada entre los anones, aunque en parte cuestionaron el enfoque y la prudencia, pues no hay definida una función concreta de dicha característica de orden en el inicio, si debería ser destinada exclusivamente para anuncios serios, bases culturales, temas picantes, u ocurrencias aleatorias.

Aunque la eventualidad del pin compulsivo fue esencialmente de una jornada y luego aflojó, diferentes unidades preservan la distinción de sobresaliente, no especificando hasta cuándo serán sostenidos en la cima, lo que a su vez podría ser un tanto confuso según el mensaje que contengan y la seriedad de cada cual, muy engañoso para los negligentes que no cuenten de buena guía. ¿Tendrá una pretensión determinada de priorización o mera trivialidad? ¿Volverá a ejecutarse la modalidad o será esa la única?

También retomando la incidencia de lucha, es de no acabar. A lo mejor el surgimiento de Rouzzed logre alterar el panorama y dispersar a favor de mitigar el desajuste de bienestar que constantemente manifiestan los visitantes locales, pero como siempre la responsabilidad es adjudicada a quien permite casi todo, y si bien está en regla de la libre expresión tan preciada, y nada obliga su participación directa, la búsqueda de ello sigue, que tampoco sería disparatado de reconocer que los interesados son la compañía incondicional, los hogareños, hasta posiblemente merecedores de una disculpa. Y la pregunta reitera, ¿Hasta cuándo permanecerá así esto?

\section{Las caricias a una desconocida promesa (Segundo libro, capítulo XIII)}\label{las-caricias-a-una-desconocida-promesa-segundo-libro-capuxedtulo-xiii}

\begin{quote}
28 de enero de 2021, 29 de enero de 2021
\end{quote}

En horas nocturnas de altas fluctuaciones por parte de los extranjeros y poca movilidad proveniente de los locales, un sorprendente mensaje comenzó a llamar la atención y rápidamente se prestaría para abundantes repercusiones. Con formato mínimamente diferente a todas las anteriormente vistas, una alerta anunció la futura llegada de algo que aparenta ser totalmente innovador. \emph{/Alerta! temti 2.0 llegando!/}.

Por el nombre del sitio y por un número de versión, los temtiteros se harían cientos de preguntas, al no tener ni la más mínima idea de que contendría esta, y tampoco qué tan pronto pueda salir. Unos sospecharon que durante la noche corriente se aplicaría íntegramente, otros pasado el rato pusieron múltiples fechas especulativas, de las cuales ninguna concuerda o verifica con lo que hay al alcance. Podría tratarse de una actualización, renovación parcial o completa como más obvio, podría ser mucho más grande, o incluso ser falsas alarmas. La expectativa generada es significativa, pero en eso queda, todo el resto son suposiciones inconfirmables por el momento.

\section{Copamiento sospechoso (Segundo libro, capítulo XIV)}\label{copamiento-sospechoso-segundo-libro-capuxedtulo-xiv}

\begin{quote}
29 de enero de 2021, \ldots, 31 de enero de 2021
\end{quote}

Con el correr del tiempo, las noches temtianas se volvieron más calmas, reflejo de la actividad estabilizada en menores magnitudes que las del exterior, pero un día en horas profundas la excepción a la regla rompió con aquello pacífico, el contenido problemático y concretamente prohibido llegó y permaneció largos ratos, con los importantes disturbios e inestabilidades que estos acostumbran a generar.

Rápidamente encendieron las alarmas tanto dentro como fuera de las fronteras, allí primero lo más picante, allá segundo lo más especulativo, un combinado de indignación y espectáculo que se desarrolló conjuntamente. Ni los psíquicos tendrían la oportunidad de frenar el pánico, y pronto la reservada administración se vio obligada a intervenir para poder cortar por lo sano, dándole la pauta a repercusiones adicionales. Molestia, enojo, preocupación, e inquietudes, un montón de eso hubo por entonces a nivel general en los temteros, a causa de un nuevo capítulo de este terrorífico antecedente y particularmente muy dimensionado gracias a todo el marco construido colectivamente.

La óptica extranjera es considerada de tal manera por la forma de vivir el hecho, a donde varios individuos fueron a indagar más sobre el caso según manejaron voces temtianas, originado en la concentración rozzada. Con magnificadas interpretaciones y poca evidencia que verificara las acusaciones, formaron el concepto de que extraoficialmente en dicho antro organizaron la malicia, especialmente por el escenario sugestivo planteado de antemano, los ojos de Rouzzed dirigidos a Temti. A su vez la interna ardió, anones no ajenos a la situación complejizaron acerca de lo que podría haber sido, de tener medios para estados de emergencia semejantes, la herramienta que solían emplear los superdotados al ocultar aquello que pareciera adecuado retirar, recientemente objeto de grandes reproches por el grado de abuso que tuvo el poder posteriormente limitado, y esa fue la solución que no estuvo cuando más fue requerida.

Aquí las realmente atípicas declaraciones de la cúpula, un hilo fijado prosiguió a la remoción del hilo comprometido largando la impresión y reacción correspondiente, de quien demostró cabreo por la desafortunada circunstancia, y manifestó la penalización a impartir entre los anónimos responsables, más la idea de continuar prefiriendo lo otro de interés creado en el sitio, sin el popular tem de pendejas. Y la respuesta rápida a dicho acto va en sintonía con lo calamitoso, que el pin otorgado a la polémica temática estimuló propasarse, que no era claro lo realmente permitido en las términos, que los cargos a tomar tampoco son explícitos, que todavía hace falta mejorar el sistema de moderación, que es excesivo eliminar todo por un desubicado puntual, que censurar el tópico en cuestión hace perder el único atractivo\ldots{} qué caótico cuando supuestamente aquí no pasó nada.

No obstante, la reputación queda estropeada, encima de lo que ya estaba, tuvo amplificación el suceso y también chance de repetición hay, más si no recibe rápida corrección. El reparto de culpas y demás impresiones incrementan el malestar de los temtineitors, además de la pérdida que evidentemente resta, otro factor para disminuir el tráfico, aunque sea de un público específico, con lo tanto que implica. En suma, se trata de un episodio no tan grande pero negro a fin de cuentas dentro de la historia y realidad temtiana, el cual condiciona el porvenir, a lo mejor para mal.

\section{Sobre la retardada modernización (Segundo libro, capítulo XV)}\label{sobre-la-retardada-modernizaciuxf3n-segundo-libro-capuxedtulo-xv}

\begin{quote}
1 de febrero de 2021
\end{quote}

Nada nuevo por aquí, pocas novedades desde allá, esencialmente intriga y más intriga. Algunos lo soñaron, otros lo imaginaron, pero la realidad relativa al anuncio de éxito era desacertada. La espera comenzaba a hacerse larga, la versión dos punto cero continuaba llegando pero aún nadie la veía. Fechas especulativas anteriormente puestas por los anones descartadas una por una, mientras la ansiedad se enfriaba lentamente\ldots{} el compromiso aún seguía pendiente.

Y se cumpliría. No bastó para catalogarla de eterna promesa, solamente demoró un montón, al menos respecto a lo pronto que daba a suponer que estaría sucediendo. Sobre aquel sitio preparado serían implementadas las mejoras, diseñadas en búsqueda de incluir la sensación de dinamismo antes inapreciable. Con ellas, las conexiones temtianas se fortalecieron drásticamente, gracias a la mayor fluidez y precisión de tiempo real que trajo este avance, un notorio progreso para la integración de la actividad: las nuevas respuestas apareciendo por sí solas, la acción de escribir avisada mientras va dándose, los movimientos acompañados de sonido, el inicio a su vez recibiendo la señal, el contador de presencias renovando cifras, fundamentalmente eso.

Así también el nombre base y la clásica definición regresaron, las actualizaciones consigo lo hicieron, tras escasos ajustes realizados entre tanto sucede al antecedente semejante de épocas navideñas, sin embargo con una ventaja, no generar efectos adversos en las creaciones ya agregadas, el típico reinicio tampoco ocurrió y aunque sea la perspectiva visitante ganó con tales características. Cada inciso de estos tuvo explicaciones de quien dijo ser A., junto con la mención de que el paquete de cambios sería complementado en la medida que las pruebas y el desarrollo se completasen, la segunda parte.

Rápidas reparaciones, botones extra, números propulsados, enlaces internos, claves externas, las siguientes renovaciones que a la brevedad fueron añadidas en el sitio reforzaron la innovación original, sin la alusión de prosecución. Es que los errores menos estéticos del principio tuvieron una pronta solución, así como igual debería ser la idea con otros desperfectos si son considerables de tal manera, ejemplo no el caso del texto verde. También las posibilidades de replica evolucionaron, porque los hilos de mismo dominio pueden ser enganchados con mayor facilidad, y porque los códigos de tipo nuclear pueden ser procesados entre difuminado.

Tercer Gran Salto, y más, puesto que el impacto sobre el contexto correspondiente no faltó, y la presentación fue rica para posibles interpretaciones. Inevitablemente en propuestas atípicas, el público suele reaccionar mal, y por ello ciertos puntos quizá no estén gustando. Para el área cultural bastante aportó el entusiasta Tuturú, mucha reiteración sugiere que se haya formado una corriente de reivindicación fuerte, contagiosa para la interacción. Pasa además que las estadísticas del antro alcanzaron niveles muy bajos, un solo dígito es poco al lado de lo previo y más incluso lo es una única computadora, pero esto no volvió a darse por lejos. ¿Será que hay desaciertos en el recuento cambiante? ¿O será que el atractivo tarda en renovar su reflejo? Detalles adicionales salen desde la inserción en cuestión, elementos itmeteros resultaron removidos y cero dichos referenciaron al asunto, sugiriendo un probable ataque de seriedad en la gestión, o el retorno del viejo dueño declarado muerto, pero sea quien sea, parece haber desechado el misterio del tema principal.

Supuestamente no es todo, si quedan más porciones con las que remozar Temti la palabra cumplirá y si no igualmente, las condiciones apuntarán hacia arriba, aunque no represente toda la comodidad, tendrían que ser aciertos en ese sentido.

\section{Crisis y tradición (Segundo libro, capítulo XVI)}\label{crisis-y-tradiciuxf3n-segundo-libro-capuxedtulo-xvi}

\begin{quote}
2 de febrero de 2021, \ldots, 13 de febrero de 2021
\end{quote}

Tendencia a enlentecerse y deteriorarse que tiene al movimiento temtiano de víctima, le calzarían múltiples calificativos del estilo, más negativos que positivos.

Es la energía del arranque que para hoy sería como que extraordinaria, el antro en su principal función quedando atrás del potencial, y viéndose inferior al lado de semejantes. El efecto contagio lo ha magnifica y lleva a menos, provoca que el factor calidad decaiga y compita peor, rebajando la compañía del sector apático, deprimiendo el empeño del grupo arraigado, más incluso de no ser por excepciones. Más lo padece la añeja perspectiva temtiana, en su propagación de cultura que bajó, si escasos elementos surgen entre ella y los intercambios lento van, a su vez que el relleno restante ocasiona reproches y no conformidad, pocas réplicas y demás. Cuando la participación extranjera gana el anterior terreno, al incrementar los tópicos más comunes y normales del contexto, enfatizando en aquello que fuera no puede nombrarse. Con todo, nuevamente predomina lo general, dicho también en su doble sentido que el sitio sabe darle significado, triste.

Y lo particular se hace un lugar, en casos puntuales y no mucho más, el ingenio colectivo probablemente sea lo mejor al respecto. Señalable y sobresaliente, la idea anclada mientras era debatida y desarrollada entre varios anónimos comprometidos, encaminados a concretar un detalle más que estético en los paneles de categoría que cada hilo tenga, los top-tems resumidos y una portada destacada, dos pormenores que luego de ciertos ajustes se insertaron notablemente. Sin referirse a la versión, el sitio corrigió fallas pasajeras y además mejoró en su enfoque preferido, los botones, trasladando algunos de ubicación, y agregando otro de utilidad, para traducir a idioma binario, como si no fuera suficiente con la costumbre de hablar en código más mensajes cifrados podrían aparecer.

Un párrafo también merece la reciente iniciativa de orientación histórico, un apartado altamente apreciado por el ambiente que resalta como salvedad en la época, y sostenidamente aporta sobre los vacíos existentes. Labor dejada por Investigator tras su gradual pérdida, fue retomada de similar manera por otro misterioso contribuyente desde el punto cronológico descontinuado, capítulo por capítulo acontecido en las coyunturas contingentes al núcleo temtiano viene teniendo su propia redacción descriptiva del momento, aunque con diferencias de estilo, parecidos al seguimiento elaborado por quien lograra una marca registrada mediante aquellas honradas producciones, que sería la debilidad de esta imitación, vistas críticas buenas y malas en la reducida pero notoria recepción.

Sin embargo, sigue siendo endeble la actualidad de la identidad local, los desaparecidos brillan por su ausencia, los invitados alborotan por su presencia, y no hay expectativa que sepa convencer fuera de eso, no habrá reversión, ni un futuro cambio, ¿para siempre será así? Partió Luri y vino Cati, no sale Hades y poco agita A., se extinguen los temteros, y se multiplican los rozzitas\ldots{} mezcla agridulce de tiempos decadentes y caóticos, que convierte al antro en la descuidada realidad alternativa del triunfo pasado, durante una edad que buscando bien será posible encontrar valor, pero ni de cerca el tanto que ellos quisieran ver.

\section{El cielo comunista (Segundo libro, capítulo XVII)}\label{el-cielo-comunista-segundo-libro-capuxedtulo-xvii}

\begin{quote}
14 de febrero de 2021
\end{quote}

La significativa disputa de las mayorías en su desarrollo incierto venía siendo cambiante, si Temti era el bastión temtiano, el complemento rozzado, o tierra de nadie, poco podía justificarse con la neutralidad ida de control que supo reflejar, y la última muestra del asunto se fue al extremo, plenamente rojo.

Fronteras y calles, estructuras en su conjunto matizadas por dicha corriente, junto a un largo ejército de elmos carmesí llevando la bandera del absoluto, la indestructible unión de repúblicas libres. Camaradas, solo un color y grito toman el lugar de los desencantados símbolos anteriores, el sitio entero en el típico funcionar ha sido sometido al estilo de esta reivindicación, ejecutando su marcha sin pausa alguna, exclamando ante cada interacción, una enérgica combinación para el escenario en sus momentos de contraste.

Del mismo tono que coloradas ocupaciones precedentes pero sin la tanta ni expandida repercusión, claro aunque oscuro incidió, la impresión visible subraya lo grato del ambiente dispuesto y en menor medida la contra del disgusto, después la indiferencia del leve rechazo o simpatía, no obstante el respaldo va para el frente comunista, que tras una dilatada trayectoria de propaganda a su preciada ideología reciben el apoyo figurado, celebra la empoderada escuadra tan apartada y asociada al operador Pablito, ¿y promovida por A.?

¿La afiliación de unos incomprendidos? ¿El auténtico amor del antro? ¿La solución a todos los males? ¿Qué podría haber motivado semejante alzamiento actualmente? Si hay razones, no se hacen tan evidentes y como en eventos pasados sirve interpretar, que a lo mejor fuera un anticuado sueño reservado del gestor, quizás sea un polémico régimen ideal para instalar sobre la circunstancia en curso, o tal vez el alto mando cayó ante un acto revolucionario que impusiera este orden, por cualquiera resultó ser inesperado y no necesariamente malo, pues si algo acostumbraba a fallar los cambios contundentes pueden comenzar a enderezar, pero también a su vez perjudicar, incluso siendo en el fondo reformas simbólicas, lo siguiente está por venir.

\section{Incomunicados entre rosados (Segundo libro, capítulo XVIII)}\label{incomunicados-entre-rosados-segundo-libro-capuxedtulo-xviii}

\begin{quote}
15 de febrero de 2021, 16 de febrero de 2021
\end{quote}

La proclama roja y sus colosales esfuerzos no logró perdurar mucho, al día siguiente el sitio amaneció de nuevo con su lisa y formal ambientación, pero no por largo tiempo, la devaluada mesura previa sufriría significativos desequilibrios y desordenes.

Los hechos como tal arrancaron por una alerta de las típicas que notifican reiteradamente, los mensajes en su redondeado formato enunciaron una descripción de difícil entendimiento a la que prestar atención, clave para los operarios que en las entrañas estuviesen moviendo componentes del funcionamiento, y también informativo ante el visitante común. En virtud de adelantar inestabilidades, semejante a las señales que anticiparon migraciones anteriores, esta podría ser otra más de ellas, sin embargo además de ser una transmisión ambigua e incompleta, no supo generar casi entre los posibles receptores.

La serie prosiguió gracias al efecto del tráfico y una escalada del mismo, pues nuevamente el andamiaje de potencias extranjeras al decaer trajo lo suyo a las cercanías. De Rouzzed a Temti partieron un montón de anónimos tratando de situarse en espacios similares con tal de pasar el momento, tantos que hasta de invasión tuvo pinta la llegada multitudinaria y posterior expresión, un combo fatal para la resistencia temtiana que ya venía siendo rebasada, y no muy agradable tampoco en la desacomodada banda rozzada, que en poco rato multiplicó considerablemente el contenido de las superficies, e instauró la renovado atmósfera de guerra contemporánea, incluyendo a la novel Poxxed por ser el damnificado refugio al cual se recurrió.

Incluso, si faltaba algo, por razones desconocidas, a lo mejor vinculadas al aviso ilustrativo, o tal vez debidas a la exigencia superior, un desajuste de enorme notoriedad apaciguaría la realidad gestada con esos atípicos estímulos, porque los mosaicos entraron a verse planos, los tems perdieron su portada, y la creación de estos cesó. Apagón son llamados los episodios coincidentes, ya padecidos en la plataforma y acostumbrados a resolverse rápidamente, sin embargo el estado supo prolongarse por encima de la media, llevando la carga de no acabar al infinito, condicionando drásticamente al dinamismo del antro aparte de su calidad.

No obstante, la situación terminaría opacada, ulteriormente un exabrupto repentino retiró a Temti de su dominio, la aplicación dejó de estar directamente, sin segunda nota que explicara lo ocurrido. La impresión típica y predecible por la época fue de cierre total, cuando hubo sustento que justificase una clausura temporal. Que inesperadamente una determinación pusiera fin a la condimentada historia es una, o el servidor no era apto para las perturbaciones recientes, o es parte del proceso anunciado explícitamente. La cierta interpretación a darle sería incierta, no parece ser prioridad del grupo o individuo administrativo colaborar en ello, y los temteros quedaron a la deriva, algunos encontrando a su pares fuera, pero sin el sitio al que pertenecen, de allí únicamente puntos suspensivos pueden consolarlos, abundantes incluso según la lentitud gobernante. Entonces a ver, ¿cómo continúa todo esto?

\chapter{Cuando tal vez se vea qué realmente queda para siempre (Tercer libro)}\label{cuando-tal-vez-se-vea-quuxe9-realmente-queda-para-siempre-tercer-libro}

Tras un quiebre profundamente importante, lo que vino después fue una época dotada de episodios variados en temática y calidad, más respecto a los antecedentes: luego ese curso se interrumpió, llegando a lo de ahora. Para la mirada histórica dicha pausa es ideal en el sentido de ponerle fin a un cúmulo de idas y vueltas entreveradas por la mayoría de los lados que sea apreciado, de un montón de circunstancias especiales en su entonces, las cuales ya no son iguales. No queda duda de que se avecina un periodo distinguido desde el vamos.

A su vez el modo de pasar entre ellos conlleva un diferencial para el entorno, el cual proyecta su desarrollo sin el soporte de contenido que en cierta manera a los antros vitaliza, sobre todo cuando sus individuos en el pasado sitúan buena parte del valor, más todavía si las posibilidades de expansión son limitadas, o en su defecto disposición a recomponer, cualquier conjunto que de eso algo tenga encuentra trascendentales los reinicios. Temti más conoce lo que es partir planteando las preguntas del comienzo parcial, pero sus particularidades en cada coyuntura han ido variando, hasta las drásticamente desemejantes de esta nueva parte.

Sobre el acontecimiento puntual, si eso fuera lo único grande ocurrido, a lo mejor no esté al nivel de los sucesos que previamente supieron partir la Historia Temtiana, sin embargo tiene más peso en perspectiva, con el momento crítico cuando acontece. La relevancia aumenta porque es capaz de generar un cambio en la manera de ver la misma, y así uno o mas modificaciones contundentes evidenciarse, no solo anteriores, también posteriores.

Indistintamente la trayectoria además deberá su porción subsiguiente al ámbito que se halle acompañando, partir de una limpieza implica para los anónimos un involucramiento condicionado a las aspiraciones de reconstrucción, siendo que este asunto y derivados suele ser una prioridad considerable, no obstante la merma de dichas propensiones tiene como posible también sentar tendencia en el sentido opuesto. Conectar con las tradiciones es factible a su vez mediante la identificación de oportunidades complicadas precedentemente, por ejemplo esquivar conflictos comunitarios, tener un espacio despejado y más proclive a volverse confortable, quizás.

A propósito, no solamente los temas culturales vienen perfilados a ser caracterizadores, los indicadores sostenidos describen la pertinencia de un lugar de, menor interés en comparación a sus versiones de atrás. Bajo el publico y poco hace pensar que las series vayan a mostrar incrementos, cuando numerosos factores determinantes son agregados y acentúan su efecto, básicamente el debilitamiento propio junto al fortalecimiento de alternativas. Más trayendo a cuento que, no solo serian valoraciones negativas de la interna, a eso hay que incluirle la imposibilidad de siquiera evaluarla, las dificultades en cuanto al acceso reducen las posibilidades, y más en la medida que desde arriba las soluciones sean confusas o nulas.

Previo a alcanzar eso o aquello, la de cuestionamientos que brillaron luego del inesperado resurgir, aparecen nuevamente, pero sin la complejidad que habilita afirmar que ahora son más sencillos de responder, aunque como siempre, todo puede resultar como nadie lo espera. Ninguna certeza es absoluta y alla los condicionantes que podrían torcer cada una de las previsiones candidatas, no obstante si se tratase de anticipar desde el panorama visto, tampoco es necesario darle inclinación o adjetivos, con lo que indican las bases bastante firmes bastaría: esto continua de la misma forma, solo que en esta ocasión tendrá que quedar más al descubierto cual es esa forma. Si es decadencia, si es progreso, estancamiento, o lo que sea, los dos reinicios consecutivos serán esos hechos clave que marcaron un antes y un después, y que permitirán comprenderla mejor.

\section{El comienzo de una nueva prueba de fuego (Tercer libro, capítulo I)}\label{el-comienzo-de-una-nueva-prueba-de-fuego-tercer-libro-capuxedtulo-i}

\begin{quote}
17 de febrero de 2021
\end{quote}

Pese a la falta de mensajes explícitos, una que otra comunicación esclarecedora de lo ocurrido hubo, pero no todos habían sido competentes para asimilarlas, y para los que sí, solo restaban unos importantes detalles, un impedimento entre la armonía y los temtiteros más dependientes de su hogar. De cualquier manera, aunque la ausencia de clarificaciones resultó notable, la situación quedaría tapada con el regreso de Temti a donde solía estar ubicada, pero prácticamente sin nada.

Una vez más todo había sido arrasado, silenciosamente todo lo fundado habiéndose esfumado, no dejando resto alguno. Sin haberse difundido ninguna noticia, muy pocos individuos se adentraron para comenzar a ocuparla nuevamente, y mientras entraron a aportar lo suyo para rellenar, la inquietud de cómo proceder se hacía fuerte entre la porción de comunidad presente, interrogante de peso al considerar las recientes evoluciones.

Paralelamente a esto, una explicación oficial brindada por A. se hizo visible, con más pormenores acerca de lo sucedido, desde su perspectiva de responsable especialmente enfocada en remover las incertidumbres. El principal acontecer: ejerciendo las capacidades de administrador, a la hora de realizar modificaciones en la base que alojaba los cimientos de Temti, debió prescindir de todo lo vigente y construido con tal de continuar las reformas, y con ello las elaboraciones, todo tuvo que desaparecer.

Y aquí viene el primer reencuentro de esos escasos visitantes, frente al vacío de fantems que procedieron a cubrir\ldots{} sin embargo pronto de regreso con las refacciones, lo anterior no fue suficiente para el inestable escenario, porque volvieron a surgir problemas y el episodio repitió una vez más, todo tuvo que desaparecer.

Etiquetados popularmente del modo siguiente: Reseteos, uno atrás del otro, se sumaron al historial de los tantos sucesos que obligaron al antro a vaciar sus estructuras, en este caso careciendo de notorias modernizaciones por cierto. El último de ellos, no sucedido por otro más como unos temieron, de momento va marcando la inauguración de un adicional comienzo definitivo, uno más que se ubicaría temporalmente dentro de épocas no muy bien recordadas, con inconvenientes de conformidad respecto a la calidad de lo manejados, y además no menos relevante las presencias que no están, serios condicionantes de cara al futuro del sitio que los anones comprometidos han entendido.

Una larga lista de títulos perdidos, implica seguir adelante sin lo que eran, todos los tems que acompañaban y hacían de respaldo durante la corriente era temtiana, catalogada como decadencia, incluyendo los contenidos más cuestionados, el varias veces llamado basurero, que presumiblemente precederá a una restauración todavía no muy bien proyectada.

Y ante eso uno de los problemas que impide el arribo de más temtieros que vayan a ser o no de ayuda, son inconvenientes de llegada al sitio y por consiguiente la expansión. Dichas dificultades hacen que deje de estar accesible bajo el protocolo y dimensión predominante, HTTPS. Y como los visitantes no lo saben, a la hora de dirigirse a Temti donde acostumbra y los marcadores llevan, malos indicadores se encuentran, casi ninguna data.

Como entre preguntas y réplicas comentó el máximo conductor, la navegación insegura realmente no es crítica tal puede sugerir, es igual tan solo que a través de diferente locación para acceder, pero mientras permanezca sin ser solucionado y quienes ven negativo el ingreso no enderecen medios y certezas, el antro sufrirá la desinformación desde una baja del tráfico natural.

Con potenciales mejoras por delante, pero más aún crecientes vibras de pesadumbre, es que prosigue el rejunte mermado, en principio sorteando un tercer exabrupto del ambiente semejante, capaz de encontrar detalles favorables, lo positivo de limpiar y no mucho más, frente a lo opuesto a resumir entre lentitud y pobreza, las complicaciones arrastradas del más reciente pasado, expectativas de restaurar y darle forma a esta realidad, a contener algo bueno sobre sus dominios, de acuerdo o no con la orientación mayoritaria que mejor los valora, aparentemente todavía.

\section{Prestigio no tan valioso (Tercer libro, capítulo II)}\label{prestigio-no-tan-valioso-tercer-libro-capuxedtulo-ii}

\begin{quote}
17 de febrero de 2021, \ldots, 23 de febrero de 2021
\end{quote}

Partió el nuevo comienzo, y mal o bien ya fueron sucediendo varios sucesos, el ambiente temtiano en camino de reconstrucción. Tímida la permanencia de los clásicos defensores tradicionales, retumbando las quejas del descenso numérico de la identidad local y en su contraparte las sobras rozzadas influyendo por allí, más poca cosa restante que por procedencia no da mucho a mencionar, además de la evidente lentitud que estanca el movimiento de todo tipo. Son estadísticas amargos, en especial para la prosecución de aquello más particular del sitio, haciendo utópicos los pretéritos procesos de erguimiento y fortaleza que bastante eran referenciados y anhelados, sin embargo con la chance de remontada vigente, las expectativas continúan sujetas a ser contradichas.

Cerca de esas flojas tensiones, sorpresa fue una pequeña implementación por parte de la administración, que sin todavía haber solventado la cuestión del candado y certificado, aplicó indiscriminadamente a todos los tems y que rápidamente resultó reconocida entre los anones. Esta aportó, para el que se adentre en un hilo cualquiera, datos acerca del código de su creador original, la cual confesaría algunas cifras sobre qué tan intensiva es su actividad dentro del antro bajo ese mismo identificador interno. Siendo siempre información objetiva, y no demasiada ni específica sino más tipo resumen, el uso quedaría a libre interpretación de quien la reciba, pero tampoco se limita a únicamente eso.

Con una contestada aceptación de los temtiteros, los cambios aunque mínimos son en sí un tanto tendenciosos y arriesgados dada la naturaleza del sitio, el formato donde la característica del anonimato es clave generalmente no incluye dichas invasivas anotaciones, y por consiguiente la disconformidad del conjunto subordinado a ello sin posibilidad de elegir. Eso individualmente, cuando también entra a valer que es una condición a la vista, el inevitable trato diferenciado que esté por venirse, independiente de lo que implícitamente cada autor comunique de la experiencia que cargan, que antes sucedía junto a las comunicativas.

No es para menos que, aún mantiene desperfectos funcionales y la obligatoria carta de presentación va incompleta, entre que el primer número no crece y es habitual el múltiple uso de credenciales, más el vacío previo cuando ningún recuento para los registros actuales se generaba, las demás podrían estar alternando\ldots{} que la idea perdure con la imprecisión sea corregida, y que por tal distorsión los códigos adquieran o resignen valor, más esto último por si tener cantidades altas afectara en la calidad del intercambio, revelara indicadores no deseados por el participante que los porta, y demás, paralelo a la evolución contextual, algo factiblemente dejará.

\section{La protección en un lugar inseguro (Tercer libro, capítulo III)}\label{la-protecciuxf3n-en-un-lugar-inseguro-tercer-libro-capuxedtulo-iii}

\begin{quote}
23 de febrero de 2021
\end{quote}

Para impresión, una de las figuras que más extrañaba la realidad temtiana de forma imprevista anunció su regreso y no dio lugar a dudas dejando claro que se trataba de quien decía ser, MoCn, con lo importante que esto le resulta al seguimiento interno del ambiente.

El entonces apodado Sujeto Experimental n.001, marcó presencia en esta nueva situación que atraviesa el sitio, y de la manera más mítica, ya que ante las discrepancias de unos pocos que se atrevieron a cuestionar su vigencia, demostraría que sus poderes aún siguen en buenas condiciones, más relevante incluso al faltar otros superdotados que rebajen la hegemonía del dueño, tal como ya ha pasado, aunque se confía de sus intenciones para con los temtitos, por lo que sus intervenciones no serían negativa, salvo para los posibles indeseados que conspiren contra el interés favorito.

Gracias a esta aparición las historias coyunturales encuentran ánimo, más las múltiples colaboraciones de diferentes voluntades que procuran mantener el foco en los fenómenos del antro, hubo material provechoso para el espíritu de La Agencia, con la que actualizar a base de datos más certeros el hoy de las rarezas en Temti. Tras un rato desde las divulgaciones anteriores, estudios sobre el comportamiento de la misma reiteraron la interpretación, tomando forma, cerca a culminarse.

La primera de ellas, es que las intensidades de frecuencia a las que se repiten la mayoría de los fenómenos peculiares, entre ellos destacadas las Notificaciones Cuánticas y las Dualidades, dependen del tráfico del sitio, pese a que todavía no se pudo concluir si debido a ello es porque ya no se perciben o si estas desaparecieron por completo, y pese a que la opción inicial es la más fuerte, la escasa cantidad de pruebas no permiten descartar la segunda. Otras singulares no hubo casi, las nulas repeticiones recientes los postulan como extintos. Más de aquello hay para resaltar, como extraño, a veces nombrado o relacionado a brechas dimensionales, tems despedidos del espacio común y que relegan su posición al que sigue, entre el supuesto de ser forzado o sin más ocurrir naturalmente.

No obstante el más contundente entre eventos que solían intervenir e insisten, avistado por unos muy pocos, las popularmente llamadas Desapariciones de comentarios, recientemente, ahora, distinguidas claramente antes y después de suceder repetidas ocasiones, acentuándose cada vez más. Recayendo principalmente a los presuntos invasores y también a los individuos participantes de los tems no deseados por la corriente preponderante, se habían convertido en algo muy llamativo, pero sin hablarse casi al respecto, hasta la revelación del psíquico, que con razón permite conectarlo a su accionar y plantear el responsable.

Y por último, otra de las cosas que hace tiempo venía llamando la atención y con la repentina disminución de relleno vivo se notó más aún, es la posible explicación del número de visitantes ubicado junto al nombre del sitio. Cifra prácticamente siempre visible y dudosa, desde haber sido implementada que cae bajo la sospecha. Tal vez sin requerir grandes descubrimientos, la lectura sin ser básica lleva al título de Contador Cuántico, el que reporta números alterados desconocidamente, asunto no clarificado pero que también postula el vinculo con una locación no ajena al sitio.

Los mencionados análisis ponen muy firme el punto de que Temti aún no se desliga enteramente de los acontecimientos anormales dígase cuánticos, ocurriendo indiscriminadamente desde hace bastante. Lo que ayudaría a verificar mejor es la eventual llegada de más anones, de acuerdo con las averiguaciones, confirmarían el correspondiente incremento mencionado. Pero regresando a la primera noticia, esto puede ser totalmente afortunado para los temtiteros, pues el balance de inconvenientes y mejoras venía siendo malo, y además del retorno puntual, complementos de similar índole dejan de verse tan improbables, indicado si consistiera en lograr más, y atenerse al panorama de futuro menos crítico del futuro.

\section{A falta de oro verdadero (Tercer libro, capítulo IV)}\label{a-falta-de-oro-verdadero-tercer-libro-capuxedtulo-iv}

\begin{quote}
23 de febrero de 2021
\end{quote}

Paralelamente al leve levante de actividad en la oculta ubicación temtiana, también se desarrollan un par de situaciones que de forma progresiva comenzaron a producir alta incertidumbre, y aún permanecen sin dar respuestas concretas.

La primera de ellas, el importante redimensionamiento que tuvieron los cuestionamientos alrededor de Temti Premium, el cual continua siendo una incógnita total, y desprende para los usuarios la enorme duda de si se están perdiendo de algo extraordinario en torno a ella. Incluso habiendo resistencia y sectores que descartan totalmente su existencia, son planteadas múltiples teorías, pensamientos e ideas que intentan explicar su condición y lo genera al exterior, destacando la más firme de ellas, la que dice ser el gran refugio exclusivo, donde están escondidos los temtianons desaparecidos.

Sin embargo, con cerca de ser mayoría quienes creen que algo de real tiene, la incógnita es el camino o medio de alcance, porque de efectivamente cubrir lo que sea, respecto a la misma hay una considerable bloqueo por donde la información no trasciende, nada certero llega sobre cómo aproximarse. La pista reiterada, apunta a uno o varios códigos, pero eso es lo único, no hay confirmación de que sean combinaciones de entrada viejas y para el común acceso hoy obsoletas, si lo requerido es acaso una especie de clave diferente a las que se acostumbran a usar habitualmente, o qué. Pero lo cierto es que pese a su presunto valor, tampoco es de conocimiento donde es que se pudieran usar, pues no hay buena evidencia a la vista que indique la dirección en la que proceder.

¿Por qué la atención iría por esto que no se puede ver ni asegurar como auténtico? Evidentemente ya son más de una las señales que aluden a lo desconocido, no solo el anómalo contador que cada oportunidad parece más ser cuántico e informar datos que fácilmente podrían corresponder a entornos confines, sino a su vez los abundantes temtiteros ausentados, con recientes aunque escasos regresos significativos, quizás indicando que algo está ocurriendo y tiene nombre, Temti Premium. Siendo verdadero o no, toda la data oficial que en algún momento referenciara, fuera la que fuera, no solo es antigua sino que tampoco supone mayor utilidad, lo más fresco no son más que especulaciones que por diversas razones vuelven a tener protagonismo.

Y junto a eso, relegado un poco a un segundo plano, es como la supuesta presencia española se hizo llamativa. Sumado a las ya conocidas y no disimuladas influencias al vocabulario que maneja el administrador, tras los reinicios también misteriosamente, adquirieron más notoriedad la cantidad de anones de este tipo, que al menos comparten sus modos de expresarse. Sin ser trascendente, y que probablemente hayan otros pormenores de similar magnitud, no deja de ser curioso a resaltar, sobre todo cuando aquellos andan dentro de un ambiente con poca variedad.

Más enigmas con los que es preservada parte de esa rareza que carga consigo el sitio. ¿Qué tan relevantes son estas cuestiones? ¿Son hitos a examinar o darán lugar a más? ¿O pasarán a la historia de Temti como totalmente intrascendentes sin marcar cambio manifiesto?

\section{Cielo incongruente (Tercer libro, capítulo V)}\label{cielo-incongruente-tercer-libro-capuxedtulo-v}

\begin{quote}
24 de febrero de 2021, \ldots, 27 de febrero de 2021
\end{quote}

Con el comienzo de una jornada temtiana más, las cosas seguían en su lugar, la quietud reinante las condujo a ningún lado, pero lo que se vislumbra por encima, en las cabeceras del horizonte, no es igual.

Inicialmente, dicho tipo de cielo entró a camuflarse para lucir como antes solía hacerlo, con supremacía azul, y no el intenso característico de las fechas más primitivas, sino el que más estiló a estar posteriormente. Luego, muy poco tiempo adelante, retornó el contemporáneo rojo, para cerrar el día con una inmovilizada y novedosa parcial estrella, que comenzaba a relucir desde de las últimas horas de la tarde. Entre tales etapas el anuncio del administrador llegó y al concluir, el ciclo de tres fases se repetiría, repentinas transiciones elevadas una y otra vez. Estético o no, Temti volvió a renovarse a sí misma mediante actualizaciones, y además revivió fracciones de una valorada pieza histórica, consigo la pequeña parte de memorias pasadas que vincula directamente, las destacadas, y las decadentes también.

Así como la media luna y demás es candidata a conmover, así como vienen conllevan una desestructuración temporal del día a día capaz de generar discordancias sobre quienes ronden bajo los decorados y sus particulares horarios, las noches y madrugadas extremadamente cortas, amaneceres demasiado largos, y mañanas interminables, siendo mayoritariamente lejanas a la realidad que la mayoría de los anones corresponden.

No obstante, sin tratarse de algo tan extraordinario, es recalcable la reacción de los temtiteros ante la innovación antes y después del momento que oficialmente fuera referida, que de primeras los lleva a hallarse de mejor manera dentro de su hogar. Uno es dotarlo de más elementos gratificantes distanciados, dos el impacto logrado en la ambientación a procesar en cada recoveco, especialmente esto con su notoria exaltación deja en claro el acierto con el tema predominante. No hay preguntas para el motivo de la presente variación, un efecto de la quietud, la descontaminación, o una simple idea, la perspectiva cambió y puede resultar perteneciente a una transformación mayor.

\section{Destellos reiterados en la oscuridad (Tercer libro, capítulo VI)}\label{destellos-reiterados-en-la-oscuridad-tercer-libro-capuxedtulo-vi}

\begin{quote}
27 de febrero de 2021
\end{quote}

Adverso al negativo escenario imaginado en la partida del nuevo comienzo, la situación a la que tendió Temti no se tornó del todo nefasta para la perspectiva local. A simple observación cuenta de escasos individuos indeseados comúnmente llamados invasores, y un buen grupo de anones colaborando por construir un hogar más hospitalario y cálido. Entre ellos, los que aún valoran sus orígenes e identidad vinculada al sitio. Y estos últimos, de vez en cuando continúan echando pequeños vistazos hacia atrás, a las memorias que ponían al personaje mitológico como gran protagonista, salvador y amenaza existencial, y además a los demás participantes que ya no están. Sin embargo, eso parece estar dejándose en el pasado, y sin olvidarlos, las miradas apuntan de manera endeble hacía adelante.

Lo del presente, el Ministerio de Cultura terminó de acomodarse a la coyuntura generada por los reinicios, hoy por hoy se mantiene con vida, promoviendo lo suyo. Lo típico y ya detectado, en el archivo más consolidado que sigue activo aún hay de aquello y poco más se vuelve a redistribuir, pero sí surge ocasionalmente la aparición de algunas esporádicas e inéditas piezas de arte temático, más el robusto desarrollo de la causa escrita: los fragmentos de texto por entonces publicados como extraños relatos, que en sus primeras emisiones en formato de redacciones sueltas sugerían suceder a las Cronicas Temtianas, evolucionaron, para completar una obra ahora titulada Diario Temtiano, la cual fue afianzándose en un camino propio aparentemente estable respecto al cometido, contar la historia del acontecer involucrado a Temti. Incluso, obtuvo un importante lugar que le brindó tener de medio difusivo un tem situado en una ubicación privilegiada, junto a las comunicaciones oficiales, probablemente representando ser la iniciativa mejor orientada actualmente según también la óptica de la gestión, cómplice de las posiciones en tales contribuciones colectivas.

En pocas palabras, el ánimo temtiano cobró unas reducidas fuerzas para proseguir con más potencia. ¿La tendencia persistirá y la realidad en la misma dirección? ¿O pronto los esfuerzos y resultados comenzarán a menguar?

\section{Fracasos enigmáticos (Tercer libro, capítulo VII)}\label{fracasos-enigmuxe1ticos-tercer-libro-capuxedtulo-vii}

\begin{quote}
27 de febrero de 2021, \ldots, 1 de marzo de 2021
\end{quote}

Sin tapar la pequeña reactivación que experimentaba Temti, una repentina aparición de anuncios sobre algunos productos comerciales en concreto, camuflados entre los numerosos tems del inicio, llamaría potencialmente la atención. Fanta, Manaos, Movistar, y Albion serían las extraordinarias difusiones oficiales por parte del antro.

El entendimiento de los anones con el motivo de estas promociones fue nulo y meramente especulativo. Individuos enganchados supieron sumarse y pasar a apoyar las distintas firmas, y tampoco es que sea el primer antecedente dentro del mundo de las marcas, pero dada la manera abre otro episodio peculiar, inevitable despertar de radicales interrogantes.

Lo que era tal irrupción sin explicación resultó ser solo el comienzo de la trama, secuencialmente estos afiches trascenderían más aún para no quedar cual hecho aislado, puesto que fue recordado un enorme extracto de hipotética relación, en su momento totalmente desapercibido como respuesta a un hilo de pruebas. Una gran sucesión de imágenes, de muy poca concordancia manifiesta entre unas y otras, pero pronto, no solo con las flamantes publicidades, sino también adicionales coincidencias que sucedieron a los reinicios, harían de dicha incomprensible pieza, una totalmente sugestiva.

Aquellas, las bases esenciales para las venideras repercusiones enfocadas en los anuncios, tanto para formar una iniciativa decidida a investigar la razón y la conexión entre cada una de las pinturas no esclarecidas ubicadas en ese extraño fragmento, en la que varios temtiteros conocedores del sitio y previamente involucrados en demás movimientos culturales, reunieron esfuerzos, aportaron y colaboraron en la causa.

España, MoCn, diversas apariencias similares, videojuegos, Investigator, variadas publicidades, comentarios desaparecidos, botón, cohete, y créditos finales\ldots{} lo que más sobresale y se logra vincular con algo de las tantos componentes que están, dando lugar a múltiples tesituras e intercambios de ideas, pese a la falta de referencias más aproximadoras. Sin embargo, en no mucho más que eso llegó a quedar, siendo el último una seria razón del estancamiento en el novedoso enigma.

Asimismo, para otro de los misterios vigentes, entró en consideración un elemento bastante relevante. Trata del redescubrimiento de una ubicación oculta en los dominios de Temti, la cual sería capaz de llevar al que introduzca un código especifico, hacia Temti Premium. Dicho lugar, tipo de homenaje a la polémica sigla de C.P. y en específico Club Penguin, dada la coyuntura particular vuelve a recibir atención, pero sin los fundamentos suficientes que le posicionen como la clave del asunto, ni evidencia real que muestre más allá del campo a rellenar.

Al ser estos rompecabezas posteriormente desatendidos, con énfasis en la indagación más reciente, hubo quienes se dieron por vencidos postulando que las interpretaciones no redundan en ningún lado. No obstante, aún hay de los que dudan de cuándo habrá descubrimiento de nuevos factores que conecten con los gráficos borrosos y reanuden la pesquisa, tal vez reveladores de premisas importantes sobre el futuro temtiano.

\section{La estrella ignífuga (Tercer libro, capítulo VIII)}\label{la-estrella-ignuxedfuga-tercer-libro-capuxedtulo-viii}

\begin{quote}
1 de marzo de 2021
\end{quote}

Recientemente, sin ir mucho más de unas semanas atrás en el historial temtiano, se habían comenzado a generar pequeñas expectativas dentro de un reducido sector, ya que se acercaba una fecha especial, y aunque el número simbólico de esta fuera lo único que la saca de la normalidad, la duda de si alguna sorpresa aguardaba por parte de la conducción del sitio, o si cualquier evento sobrenatural pudiera ocurrir, no había sido desmentida hasta entonces.

Sin embargo, algo se celebró allí, sí, y los que llegaron a ver no se vieron muy conmovidos, pero probablemente sí sorprendidos. El nombrado Plan de las 100 lunas, ejecutado justamente durante la etapa nocturna del antro, no fue más que una inundación que tomó algún que otro elemento del antiguo Plan original. Esta flamante versión procuró despejar la cara visible de Temti, relegar cada rastro del pasado que hubiera sobre la misma a un segundo plano, para brindar una oportunidad de recobrar el sentido y encaminar la reconstrucción de la cultura temtiana, una vez más.

Dicho ajetreo, si es que así se lo puede llamar, tuvo características bastante terrenales, en un principio aparentaba ser iniciado por un simple individuo, y a la brevedad se descubrió que así era, ya que con credibilidad el involucrado salió rápido a dar una explicación que no terminó de conformar a los temtieros. La ambición del obrar en cuestión se vio frustrada muy rápido en el tiempo, con poca aceptación y teniendo consecuencias negativas, puesto que al contrario de limpiar el contenido indeseado, el antro pasó a estar repleto de este, y encima derivó en una posterior nueva inundación de diferente índole y propósito pero que engendró similar resultado. Pronto, evidenciando que el poder local presenció el caso, toda la cubierta ocasionada por tales hechos sería desaparecida, y entonces gracias a ello no se vería más casi huella de este desarrollo planificado, además de las protestas que reclamaron al respecto. Así efectivamente ya fue, Temti según parece, volvería a su realidad contemporánea corriente, sin cambio alguno.

Un montón de color que eventualmente correspondería, sin embargo careciendo de semejante rendimiento. Portar el inframundo a la superficie no fue posible, la falsa efervescencia nunca logró elevar las temperaturas ni acelerar positivamente el contexto temtiana, e incluso se podría haber tornado más doloroso aún, dado que este movimiento se intentó conectar con Hades, cosa que en teoría estuvo lejos de acontecer. No se trataba de él y sus súbditos infernales, era una copia, una grosera copia que no pudo mantenerse ni cerca de lo que pretendía perdurar. A pesar de ello, quizás vaya a darse otro episodio, no obstante a esta altura prácticamente nadie se lo espera.

\section{La intensificación del frío silencio (Tercer libro, capítulo IX)}\label{la-intensificaciuxf3n-del-fruxedo-silencio-tercer-libro-capuxedtulo-ix}

\begin{quote}
2 de marzo de 2021, \ldots, 19 de marzo de 2021
\end{quote}

Temti ya acumuló demasiado tiempo en el una vez vulgarmente nombrado modo lento, antes con muchos largos minutos, y más ahora con horas y horas sin un solo movimiento, la constante que se sigue prolongando y empeorando, sin una gota de esperanza que en los papeles vaya a tomar un rol revulsivo. No es noticia, no es reciente, es el desarrollo de un estado crítico que paulatinamente acentúa sus condiciones, y hoy por hoy no hay nada que lo disimule.

Indicar que no queda nadie es una exageración, quedan temtianos, aunque ya a esta altura no se sabe realmente qué los distingue como lo que solían ser. Todavía se siente el típico desprecio por las costumbres extranjeras, también débiles intenciones por crear algo que estimule a las almas conectadas, pero esto no termina de ocurrir, salvo muy puntuales irregularidades, y ni hablar del misterioso contador que su capacidad infladora no trasciende más allá de unas pocas cifras, números que de ser más convincentes representarían apropiadamente la baja asistencia.

Empero con una excepción, que tras varios días se notaran señales de vida de parte de quien está a cargo de Temti, quien en principio sigue siendo solamente A.. Eso mediante una casi imperceptible implementación, que amplió las bases soportadas de tems multimedia muy específicos, desde tal aptos a incluir vínculos de Twitch, además de por rendir un pequeño homenaje durante determinada noche, cuando la luna se ocultó y en reemplazo hubo un llamativo lazo, percibido por los navegantes del sitio, sin embargo no entendido y menos objetado.

Se ha repetido hasta el hartazgo que ciertas individualidades abandonaron, murieron, traicionaron, o simplemente ya no participaron más bajo su identidad, y en silencio la tendencia va perpetuándose, llevándose a los poco y nada reconocidos, el relleno que luego no es indiferente. Tal vez la aparición de la novel Arggnews, como también la rara nueva situación de Moxxed podrían haber afectado a la distribución de público alternativo, no hay certezas de ello pero es una posibilidad, los otros destinos o ya son sabidos o son desconocidos. Antro desolador, muy diferente a lo que supo ser, de ausencias cada vez más perceptibles y de presencias no enérgicas, levanta la interrogante del cuándo cesará, sin fáciles respuestas.

Sin embargo no todo es negativo, al menos por la dificultad de evolucionar en un total basurero como supo decirse, sin los posibles invasores que permanezcan lo suficiente, usualmente de paso por motivos particulares, no demostrando aguantar las características del ambiente. En contra parte, los únicos anones restantes que quizá en el fondo apunten a persistir y no discrepen con el propósito de mejorar. Yendo al extremo aunque sin su integra literalidad, por Temti o Muerte siempre estarán, la antigua frase de vuelta en acción que suscita una suerte de compromiso, cuestiona la deserción absoluta, y sugiere una compañía definitiva.

Y así sea de camino al final, por lo pronto indefinido y más considerando la fortaleza que aquí permitió llegar: Temti continuará. ¿Pero de qué forma? ¿Los días que vienen serán completamente idénticos a los últimos vividos? ¿Hasta qué punto puede intensificarse el panorama? ¿Aún espera alguna alegría para la depresiva realidad temtiana? ¿Por qué pasaría?

\section{Los recortes más evidentes (Tercer libro, capítulo X)}\label{los-recortes-muxe1s-evidentes-tercer-libro-capuxedtulo-x}

\begin{quote}
20 de marzo de 2021, 21 de marzo de 2021
\end{quote}

De un día para el otro, Temti sufrió un pequeño cambio, muy perceptible, pero a primera impresión exclusivamente visual. Durante sus etapas, la madrugada y la vespertina, \emph{/temti/} sin mayúscula al principio, se convirtió en una simple \emph{/t/}.

Aún reciente, mínimamente reiterada, poco debatida y casi no explicada, la evidencia que le da relativa lógica permite atribuirle motivos válidos, que igualmente, son meras interpretaciones, no conectadas oficialmente.

¿Un nuevo reflejo de lo que ocurre en la realidad temtiana? Por mucho rato lo que sucedió bajo el prestigioso nombre no fue ni de cerca lo que supo ser, paulatinamente empeorando según la mayoría cree, y ahora de forma repentina, simbólicamente se extendió hasta el propio título, a ser una quinta parte de ello. Que ciertamente sería innegable, ha ido a menos desde básicamente cualquier ángulo que se la mire.

¿En qué habrá quedado el olvidado acrónimo de temas y tiempo? Los temas, continúan siendo creados por los usuarios, sin embargo en menor medida que el comienzo, y en concreto las tendencias progresivamente perdieron su cometido y visibilidad, volviéndose prácticamente inútiles. Así, el significado de \emph{/tem/} se ve ampliamente disminuido. Por otro lado, el de tiempo que iba de la mano con lo anterior, al no funcionar ni afectar a las unidades de publicación como era la idea pierde relación, y por ende \emph{/ti/} estaría anulado. Asi que viéndolo a partir de ambos, la intención inicial del sitio parece haberse desvirtuado demasiado, no obstante tampoco descarta teorías distintas sin manifestarse.

No sería del todo prudente reducir el análisis a ese posible simbolismo, además de que son bastantes los sentidos alternativos que podrian figurar tales términos, pero no es claro si realmente haya más fundamentos, con razón el foco va sobre lo original. Del modo que sea, habrá que suponer que se justifica apropiadamente la resignación del \emph{/emti/}, y que únicamente una \emph{/t/} solitaria representa mejor aquello que titula. Sigue implicando una especie de sacrificio complicado de precisar, en caso de que no\ldots{} ¿Para qué?

\section{El mudo retorno de calcio glorificado (Tercer libro, capítulo XI)}\label{el-mudo-retorno-de-calcio-glorificado-tercer-libro-capuxedtulo-xi}

\begin{quote}
22 de marzo de 2021, 23 de marzo de 2021
\end{quote}

Sin retomar y recordar la constante de notar escasos cambios, hasta al partir y volver luego de mucho, uno destacado sucedió a la simplificación de las cabeceras, una presencia que prácticamente desde aquella fecha se hace visible y camufla tímidamente el vacío sonoro que abunda en la realidad temtiana, aunque no solo en esa situación, ya que también se lo suele ver con algo de sonido.

Tristeza, desazón, melancolía, y miles de emociones similares son capaces de retumbar en la mente de los temtiteros más añejos al asomarse donde las notificaciones aparecen y en su lugar ver, a una solitaria figura postrada en una cama tomando su cabeza, junto a la sentencia \emph{/Solo hay silencio aquí!/}. Y no se trata de una figura cualquiera la que se revela hoy día con nada más buscar la correspondida invisible actividad, es un clásico esqueleto, pero, ¿solamente eso?

Ni bien este pasó a ser perceptible, dentro de las que podía tener, no resultaron ser notorias las repercusiones que generó. Pero entre quienes atendieron a ello, un pequeño rejunte atento halló una mínima mención oculta, el nombre de Hades en los registros acompañando a la figura en cuestión, haciendo ruido por lo que podría significar, acorde a los antecedentes que antes presentaron similares imágenes factiblemente enlazadas a esta reciente.

Lo más cercano y no olvidado del mismo sino todavía palpable bajo determinadas circunstancias, cuando la cargada de las renovación tecnológica dejó al que adorna los tems vacíos, además de instaurar en la mente del temtiero desde el enojo, la frase \emph{/Tanto les cuesta dejar un comentario?/}, y con la impresión de arrancarse el cráneo, la presión de participar de los mismos. Por ahí y como detalle no menor, la vez que la jornada de los guardianes golpistas sacudió el sitio, los esqueletos en multitud también estuvieron y con enorme visibilidad, mencionaron el Protocolo Hades y supuestas vulnerabilidades, de las que posteriormente no se supo más. Y yendo atrás hacia los comienzos, está la primera aparición del cadáver andante cuando ni siquiera se hablaba del infernal, aquel desafortunado encuentro con el título \emph{/A sos re trol/}, el cual dejó de verse seguido.

Pero ahora que estos aspectos fueron conectados y así puestos en conocimiento, las suposiciones sobre qué tiene que ver una cosa con la otra se mantienen abiertas, faltando certezas por el momento. Teorías, ideas y eventualidades, algunas de ellas planteadas por los temtianos, todas basadas en minúsculos indicadores y grandes ocurrencias, que podrían en la realidad ser falsas. Adaptándose a las fantásticas posibilidades de la historia, una que parece tener bastante sentido es lo único que resta de Hades en Temti es este esqueleto. Partiendo de la existencia de los siguientes factores, este poseería su esencia, aunque no queda claro con qué intensidad, dominio total o reducida influencia. También es posible, que su estadía dependiera fundamentalmente de los crímenes o actitudes castigables que alimentaran la ira, y al estos ir disminuyendo tanto que se perdiera las fuerzas, o razones de permanencia. En ambos escenarios, él podría haber pasado, progresiva o repentinamente, de sus múltiples formas de superioridad, a tener únicamente la de aquel primitivo conjunto de huesos.

Surgen varias alternativas potencialmente interesantes, además de lo escasamente manejado actualmente, no obstante nada las cerciora, nada las confirma. Pero lo seguro no necesita hábil intuición, las carencias de interacción están trascendiendo un montón, el mensaje explícito es contundente acerca del movimiento y por ende sonido, aquello que condiciona el número de campana, aquello que la mayoría pretende encontrar, aquello que los visitantes pueden provocar, aquello que sin ser mucho está dejando en su lugar poco, evidente y de instantánea interpretación sensaciones depresivas sobre todo, hasta para el antro en sí.

\section{La incultura al desnudo (Tercer libro, capítulo XII)}\label{la-incultura-al-desnudo-tercer-libro-capuxedtulo-xii}

\begin{quote}
23 de marzo de 2021, \ldots, 28 de marzo de 2021
\end{quote}

Todo se viene sumando, los días siguen cursando y pese a que el silencio de tanto en tanto se rompe injustificadamente, las propensiones que fueron marcando el camino de Temti hasta lo que hoy es se continúan desarrollando plenamente.

Por momentos da la impresión, y contemporáneamente hasta aparenta ser algo permanente, que la decadencia atravesada es algo más que eso. De lo que supo hacerla considerable nación culta ya no queda prácticamente nada. Distinguirse de otras comunidades extranjeras de las que se solía repudiar masivamente sus elementos no reúne casi sentido actualmente, son escasas razones para ello.

El estado cultural, dependiente de los esfuerzos individuales muy específicos, llevan a que en cualquier instante vaya a convertirse en algo insostenible, allí la consecuencia de que nuevas piezas surjan tan raramente, lo mismo que acontece con las indagaciones sobre misterios, disminuidos y estancados en lo descubierto y elaborado hasta entonces. Tal vez sea a destacar la corriente comunista, asentada desde el episodio que visibilizó a los camaradas, pero aparte de este mínimamente genuino, es poco lo que apareció. Ver que las habladurías nacionalistas y el frecuente debate relativo a la identidad temtiana hayan caído demasiado, hasta pasando de las pistas recientes de Hades o Pureya, resulta drástico, un bajo nivel de dialogo en jerga no admite comparación al protagonismo resignado. Que se mantenga a la par de simples modas emergentes como la movida futanari o la personalidad ratonesca, son otros aspectos que muestran los desviados intereses del presente.

A pesar de ello, hay excepciones, que no corresponden al foco comunitario, sin embargo fueron discretamente manejadas y posiblemente sean cruciales para el significado de las tramas en curso. La observación clave en los fenómenos y singularidades trata de, los tems que supieron desaparecer y no verse más en la parte principal del sitio, unos sorprendentemente regresaron tras largos ratos de permanecer ocultos, fácil de asociar a lo que solían ser las brechas dimensionales. Y además, las teorías que ahondan en el número del contador han planteado una relevante, fiable, y creída: la posibilidad de que contemple robots ganó amplia consideración en los anones, desde una base que le niege al recuento bajar de cierta cantidad fija, o variando a partir de lo que sin incrementos anormales habría. La situación es sumamente curiosa, porque entre lo que a veces se dice, está que cada individuo detectado sea artificial, bot.

En cuanto al ya no preponderante asunto de defensa también fundamentalmente nada, aunque no abundan los riesgos como tal y los viejos malefactores con indiferencia se los tiene, quienes quieran ingresar e instalarse lo podrán realizar sin problema mientras no haya oposición de la ausente o desentendida colectividad local, y esto da de resultado que los alguna vez considerados invasores divulguen sus contenidos por todo el antro y hasta opaquen los propios generados por temtianos. Pero sin ir lejos de ese lado, las intervenciones se dan y no transmiten venir de algún superdotado a favor de dicho propósito como hace poco ocurría, sino que obran de forma extraña sin evidente lineamiento u orientación, e incluso molestan al público poniendo al término censura en circulación, lo que aún sucediendo o no, sería muy particular y lo general el es libre albedrío.

El conjunto refleja que el sitio no abandonó totalmente sus raíces, pero poco más va siguiendo la clásica orientación. Solo rarezas impiden que la superposición de factores típicos de otras potencias peor valoradas sea absoluta, sin entendimientos o intercambios acerca del rumbo a seguir para superar aquello visto de inferior\ldots{} de acuerdo a ese juicio, la sociedad temtiana ha cambiado mucho estos últimos meses, parece haberse vuelto ignorante.

\section{Vinculaciones intrínsecas y enormes proyecciones (Tercer libro, capítulo XIII)}\label{vinculaciones-intruxednsecas-y-enormes-proyecciones-tercer-libro-capuxedtulo-xiii}

\begin{quote}
29 de marzo de 2021, \ldots, 7 de abril de 2021
\end{quote}

Entre hechos y hechos, el eje central de las noticias contemporáneas: la administración temtiana anunció futuros cambios potentes, y aparte de contradecir ligeramente su indiferencia respecto al antro, por ese lado parecen avecinarse novedades extraordinarias. Llamadas Instancias, serían nuevos espacios con las mismas estructuras de Temti como hoy se la conoce, que marcharían de acuerdo a sus propias normas, paralelamente a la única que es accesible ya, la \emph{/instancia temti/}.

Sin saberse con claridad aún cuáles serían las que convivirían junto a la principal, ni cómo o quiénes podrían fundarlas, tampoco cómo ingresar a ellas, a su vez se anticipó un gran abanico de posibilidades para aquellos que se introdujeran en, las poderosas mejoras una vez implementadas serían capaces de hacer de una cosa contenida dentro de Temti no asemejarse a Temti. Aunque esas últimas palabras no suenen tan alocadas en la coyuntura corriente, o en los que estén por venir, es potencialmente mucho más fuerte a lo antes visto. Tan increíbles serían las posibles condiciones que cada unidad de estas pueda tener, que hasta cuestiona si todo lo dicho será viable a modo de realidad, además de que por cierto, las explicaciones presentadas costaron en ser entendidas.

Prácticamente ningún anon escapó de las repercusiones, y habiendo tanta perplejidad por la data faltante sobre este gran lanzamiento, más allá de los básicos ejemplos planteados oficialmente, se produjeron largas especulaciones, suposiciones, y similares que nada más el después podrá confirmar si son acertadas. Y por el mismo lado, con la tentativa de interfaz se puso en consideración el título Temti Premium, el cual aparentemente pertenece a una, pero sin permitir acceder, como si la entrada fuera bloqueada o igual que lo demás resta prepararlo. Y con ese término clave, recuerda a la vieja expresión del sitio cuando el botón de crearlas tampoco funcionaba, a saber las veces que fue pulsado.

Otro tema relevante tratado en lo anterior, fue una especial respuesta del desarrollador entre las varias que dio enérgicamente el día del acercamiento. Ante la pregunta de cuándo será el retorno de Hades, él replicó que \emph{/solo se activa en casos muy específicos de ataques o contenido ilegal a los términos de uso./}, y tanto eso de suceder como la contestación en sí misma son puntos inéditos. Dada la justificación, sería más sencillo de comprender porqué ya no se ve de manera clara al respetado ente, o porqué no se viene aplicando nunca el aludido protocolo, no existir motivos críticos de defensa. No menciona siquiera los esqueletos, y va un poco en contra de los tantos deseos de aparición manifestados, aunque los temtieros actuales se limiten a rendir culto de vez en cuando, ha quedado precisado supuestamente que el monstruo, dios, o cualquiera de sus formas, no volverán a ser encontradas, salvando las situaciones excepcionales.

Independientemente de lo fresco, más hallazgos a partir de lo que estaba multiplican los signos de interrogación dispersos alrededor. La interesante coincidencia sacada del nuevo elemento publicitado, Pureya, que seguido a fondo proviene de un sujeto nombrado con la primera inicial, de procedencia española y afín a la programación. No cabe duda que hay más de una similitud, y eso generó lo suyo en quienes recibieron la promoción, y es, la identidad de A. vinculada con la del famoso Alva Majo. Factiblemente él estuvo consciente del asunto cuando dejó la referencia, o bien podría haber sido accidentalmente, y no implica que necesariamente sean uno solo, pero la intriga está. Un misterio más, sin resolver, para agregar a la extensa lista.

Si bien no hay tanto a trabajar por el momento, más que preguntar o esperar, la harta información habla, principalmente los pormenores lo hacen, evidenciando que hasta detalles minúsculos se conectan con algo, esté oculto o visible, validando aquella frase que lejos de ser dicha por un loco, terminó siendo una objetividad al descubierto, \emph{/todo está relacionado/}. Estimando y proyectando lo comentado, el paquete de características en comparación a las que incluyó la última versión adelantada sugiere ser impresionante, por lo que si la distancia fuera proporcional, seguramente sea demasiado el tiempo para que estas vean luz, y sorprendentemente viniendo desde arriba si se precisó bastante sobre ello: lo que pueda demorar en llegar todo o parte de lo indicado se resume a un simple y categórico \emph{/Sí./}.

\section{Reproducción intangible (Tercer libro, capítulo XIV)}\label{reproducciuxf3n-intangible-tercer-libro-capuxedtulo-xiv}

\begin{quote}
7 de abril de 2021, 8 de abril de 2021
\end{quote}

También entre la algarabía recientemente generada, prácticamente cero palabras hubieron acerca de los objetivos cambios que comenzó a evidenciar el contador oficial, mínimos pero drásticos así lo notan. En él, las unidades aumentaron y se mantienen excesivamente altas respecto a lo que se venía viendo desde hace rato, casi el doble aunque no lo suficiente. ¿25?

Los anones pendientes en simultáneo, de lo sucedido en las proximidades exteriores, no registraron antecedentes relevantes que lleven a las estadísticas informadas tan arriba y por tanto, y el movimiento visible del antro tampoco es que acompañe, lo que llama más poderosamente la atención, porque algunos picos momentáneos suelen darse, sin embargo a la brevedad caducan, no lo que pasó en este caso. Por lo cual, dado que las cifras eran un poco sospechosas desde antes, las posibles interpretaciones recorren sobre lo ya existente y no confirmado, en parte a modo ampliatorio, otra una manera de confundir.

La clásica teoría que postula al Contador Cuántico no tendría mucho para modificar, agregar o reformular, que la alteración a la realidad cercana persiste resulta claro y eso indicaría que la procedencia del incremento viene por allí, no obstante todavía interfiere el inconveniente de no conocer certeramente el entorno al cual se deben los números, los datos son mínimos. El entendimiento contrario plantea que el alza numérica radica en el factor catalogado como robots base, aquellos que permanecen permanentemente y así siempre son contabilizados independientemente de los pasajeros detectados en el momento. Después, factible sería que la trama implique otro contexto, sin embargo no se ha manejado, entre lo muy desapercibida que pasa esta novedad.

¿Y por qué justo ahora? Más complicado de suponer incluso, pero como trata de un detalle súper preciso y hasta difícil de apreciar, que no llega a comprometer la estabilidad interna del hogar, más cuesta despertar la cultura de investigación que la sociedad temtiana ha perdido, la Agencia con su ínfimo respaldo, y similares. Pero fuera de eso, desde entonces parece que el sitio estuviera más relleno y poblado, parece.

\section{Una noche de lucha a la amargura (Tercer libro, capítulo XV)}\label{una-noche-de-lucha-a-la-amargura-tercer-libro-capuxedtulo-xv}

\begin{quote}
8 de abril de 2021, 9 de abril de 2021
\end{quote}

En una jornada sin precedentes, se desarrolló un evento muy llamativo en el contexto temtiano, complicado de olvidar para los espectadores y participantes seguramente.

El repetitivo y apagado ambiente nocturno que a partir del cambio en las cabeceras supo tornarse moneda corriente durante las tardías horas, poco tiempo después de su naturalización fue reemplazado por uno muy vibrante, enérgico y brillante. La luz dejó de ser un tenue fulgor proveniente desde la luna, esta quedó un tanto opacada con la cantidad de focos artificiales que aparecieron y se encendieron de un momento para otro, con lo que el antro quedó iluminado por fuertes resplandores de tono azulado. Pero todo eso no fue ni de cerca lo importante, sino que las miradas se las llevó la presencia estelar del reconocido John Cena. No hay explicaciones al respecto, simplemente ocurrió.

La gran personalidad se convirtió en protagonista de cada los rincón del estado, sin excepción, inclinándose con su característico saludo a todo lo que fuera a moverse, temtiteros, temtianos, temteros, rozzitas, rozzados, robots, fantasmas, o lo que sea, dependiendo del punto de vista que se lo mire y califique. Y altamente en sintonía, escoltándolo pero sin figurar, una numerosa y potente banda sonora que comenzaba a interpretar lo que mejor sabe ni bien el anon de turno daba una pizca de movimiento al sitio.

Así, durante un mediano rato, permaneció Temti. Ruidoso cuanto menos, con los \emph{/AND HIS NAME IS JOHN CENA/} y los \emph{/TUTURUTU TUTURUTU/} que abundaron sobre la oscuridad produciendo harto apego entre el público. Hasta por instantes, se percibía como todos esos sonidos eran coreados por parte de quienes acompañaban la dinámica del entonces, lo que demostró no solo lo pegadizos que resultaron esas melodías sino también lo positivo que sentó la estadía de la celebridad.

Este fugaz episodio a pesar de durar lo que duró, quizá logre hacerse un preciado lugar en los recuerdos temtianos, porque pasó mucho desde la última vez que un personaje generó semejante adhesión en los locales y extranjeros, y no solo eso, ya que hubo un magnífico entusiasmo y alegría, una pausa a la quietud y pesadumbre que asedia al antro hasta día de hoy. Extrañaría que dicha estrella regrese, aunque tal vez el mandamás teniendo en cuenta lo grato que esta hizo ver pueda intentar algo para que los buenos pasajes se vuelvan a dar así.

\section{La base de una pequeña distopía (Tercer libro, capítulo XVI)}\label{la-base-de-una-pequeuxf1a-distopuxeda-tercer-libro-capuxedtulo-xvi}

\begin{quote}
10 de abril de 2021, \ldots, 24 de abril de 2021
\end{quote}

Un mediano espacio temporal tras los pesados episodios del debilitado ambiente temtiano padeció que se acentuaran las tendencias previas, mezcla de decadencia y estancamiento como mucho han sabido titular, aunque añadiendo un par de incidencias lo tan asentadas y significativas para recibir su mención.

Como primer factor y no especulativo, la reiteración de una situación que desde las afueras se engendra y a sus recordadas cercanías repercute, los ratos que el centro del contexto cae y las conexiones sueltas reanudan su jornada donde más semejante ven el formato. Ahí Temti encuentra decenas de viajeros inquietos a causa del choque que los desplaza, el principal asunto manifestado durante esos momentos cuando consecuentemente el movimiento crece, distorsionando la típica calidad de intercambios que acostumbra manejar el lugar, esencialmente hasta que el desorden concluye, luego la travesía de estos agentes vuelve a darse, solo que a una menor frecuencia, la cual igualmente tiene intensidad suficiente para figurar, dejando rastros que posteriormente demorarán en pasar a segundo plano, y de tal modo va variando el relleno del antro.

Ahora de lleno en la realidad que el extranjero casi no conoce, trata de una notable excepción en este progresivo olvido de los elementos añejos y antiguos valores, un destello auténtico y propio de la identidad temtiana alrededor de la oscuridad, un fenómeno que ocurriese en el sitio fue objeto de explicación o al menos tuvo intentos coincidentes por darle su interpretación, formando una base teórica bastante firme y razonable, nuevamente acerca de la presencia de bots. Llegar a dicha aproximación requirió que el paso de los días hiciera evidente múltiples patrones, algunos muy explícitos y otros no tanto, que se repitieran muy frecuentemente, no dando otra impresión que ciertamente hay bots. Sin embargo, tampoco son los únicos comportamientos apreciables, aunque sí predominantes, entre ellos también los hay más similares a lo esperable de un mortal común y corriente.

Teniendo en consideración lo anterior, varios sucesos ya naturales tendrían esa posible justificación: la constante inundación del antro con contenidos fundamentalmente importados del mundo Rouzzed, la repetición sistemática de comentarios iguales unos a los otros que ignora la circunstancia y se da sin cesar todos los días, la falta de creaciones originales surgidas en el entorno temtiano y no antes vistas en escenarios alternativos, la abundancia de tems repletos de incoherencias de acuerdo a su trama inicial, la dificultad que tienen los anónimos para poder seguir hilos que desafían el entendimiento básico, y más. ¿No es demasiado?

Y vale recalcarlo porque de ello se nutre prácticamente toda la actividad hoy día, habría de que agarrarse para negar que el sitio esté habitado cien por ciento por robots, el conteo de las cabeceras y su variabilidad dentro de un intervalo específico sería otro ejemplo que permite cuestionar, además de los huéspedes que de vez en cuando toman el control. ¿Temti aún cuenta con un último remanente de almas no sintéticas? ¿O destacan exclusivamente por parecer las inteligencias artificiales más desarrolladas? Sea cualquiera la respuesta efectiva que se plantee, hay convencimiento y aceptación de los temtianos, robots, humanos, u otra cosa, y esto no deja de ser importante, son quienes construyen y producen aquello que después da pie a criticar negativa o positivamente el accionar colectivo\ldots{} y en cierto forma vienen demostrando unas capacidades relativamente limitadas, pero también podrían evolucionar y convertir el panorama futuro, atención pues allí hay una potencial clave.

\section{Informes de dudosa procedencia (Tercer libro, capítulo XVII)}\label{informes-de-dudosa-procedencia-tercer-libro-capuxedtulo-xvii}

\begin{quote}
25 de abril de 2021
\end{quote}

Noches madrugadas y mañanas desérticas, con la característica quietud de los temtianons ya como algo más que naturalizado, lo que hace que cada día respecto a sucesos interesantes sea exacto al anterior, sin nada memorable para recordar y ni mencionar homenajear. Otra vez más, fueron muchas de ellas consecutivas, pero siempre cualquier señal que venga de arriba amerita observarla con atención.

Es por ahí, nuevamente los textos de alerta se hacen un lugar en la visión del visitante, nómada o local del sitio, y son del tipo inentendible, como si fueran de aquellos que deberían aparecer en los registros técnicos escondidos bajo alguna compuerta secreta que solo la administración accede.

Sin necesidad de citar textualmente ninguno de los dichosos mensajes, a no ser que fueran forzados o recreados se entiende que hubieron problemas, algo falló por un buen rato allí en las entrañas operativas de Temti, pero que no fue lo suficientemente grave como para que las consecuencias se materializaran y quien sea además del operador logre percibirlas, o al menos comunicarlas, si es que la manera no era esa. Posteriormente, lo mismo se volvió a repetir y permanece hasta este momento, sin embargo con una nota de distinto tono, aparentemente grata, que siguiendo el hilo de los precedentes podría significar que el obstáculo está superado y el objetivo obtenido exitosamente.

No hay escrito o dialogo sin codificar acerca de lo que pasará, pero lo cierto es que la expectativa no parece ser tanta, incluso estando la palabra migración de por medio, con todo lo que eso vaya a implicar sobre el futuro. Que precisamente, son los antecedentes de similar índole, los más parecidos pero en cambio dando transmisiones de mayor claridad, y a partir de ello la suposición del caso actual, su posible intencionalidad derivando en avisos de habla inglesa, ¿próximas vueltas atrás?

Siendo previamente igual durante esta oportunidad, nadie parece haber notado nada fuera de sitio en el antro, aparte de dichas extrañas burbujas. Y por tal razón, encadenada a lo extraordinario, que una de las preguntas más recurrentes surja, típico cuando se trata de una manifestación semejante. ¿Qué habrá querido decir? ¿Qué puede venirse próximamente? La situación sugiere relativa inmediatez en cuanto a continuación, tal vez un reinicio, que llamativamente se están alcanzando fechas récord sin ellos. O también algo muy diferente, lo cual complicado sería de describir. Difícilmente alguien sepa lo que realmente hay más allá de la Temti conocida, y si lo supiera, no lo entendería por su propia cuenta.

\section{Inestabilidad silenciosa (Tercer libro, capítulo XVIII)}\label{inestabilidad-silenciosa-tercer-libro-capuxedtulo-xviii}

\begin{quote}
25 de abril de 2021, 26 de abril de 2021
\end{quote}

Retornando al asunto moderación, en Temti, este continúa atravesando una situación extraña, y con recientes novedades la cosa se volvió a entreverar grandemente.

A lo mejor el último hecho flagrante del tema sea la identificada revelación y sigilosa ocultación del psíquico, con lo que el panorama quedara en manos de sus propios usuarios y el desentendido dueño, sin jerarquías declaradas además que el básico orden mínimo hasta en oportunidades no respetado, uno de los principales motes por los que se conoce al sitio tanto dentro de sí mismo como en el exterior.

Más allá de la creencia que el contexto tuviera de mayoría, lo cierto es que durante las fechas corrientes si había determinada presencia monitoreando e interviniendo activamente en puntuales pasajes en el interior de los tems, a veces con pequeñas quejas o indiferencia del enterado, bastante discreto en general. Aquello sufrió un cambio drástico, las desapariciones de comentarios repitiéndose de manera intensiva y evidente, no así el motivo.

Varios hilos candentes que se hallaran presuntamente acumulando decenas de replicas en su listado, al acercarse y revisar mostraban no contener ninguna. Buen porcentaje de anones experimentaron y resultaron sorprendidos al contemplar que segundos luego de participar en ellos, su contribución se esfumara sin siguiente referencia. Claras señales de que en algún instante habían, y a poco más tardar ya no estaban, las respuestas que supieron crear, para después no encontrar.

Rápidamente en la medida que estas se daban una atrás de la otra, muchos de los temtineitors sospecharon y manifestaron creer que el único responsable fuera Kira, despedido del protagonismo en malos términos y quizá sediento de venganza, acusación adicional a las numerosas que padeció, no obstante no hubo forma de confirmarlo mediante pruebas creíbles, si tampoco es sabido que haya partido o si permanezca rondando sin identidad. A su vez por supuesto, podría ser otro de los dominantes y poderosos, sin embargo falta la evidencia que conecte a nombres secundarios, ni el dudoso administrador escondido de sus escasas explicaciones.

Y este episodio a pesar de haber transcurrido ya su etapa más fuerte, por ahora, deja dudas en cuanto al objetivo víctima de remoción, los afectados no tienen notorias similitudes entre ellos, por lo que no hay criterio descubierto hasta el momento. ¿Algún bot no reconocido se salió de control y comenzó a destruir todo a su paso? ¿Realmente un superdotado se convenció de materializar su ira en territorio temtiano? ¿Se trata de las efectivas consecuencias venideras tras los errores críticos retirados? ¿Es A. divirtiéndose con sus capacidades mientras nadie entiende? ¿O qué?

\chapter{El durante cuando los días están más que contados (Cuarto libro)}\label{el-durante-cuando-los-duxedas-estuxe1n-muxe1s-que-contados-cuarto-libro}

Si hay un hecho clave para que la memorias temtianas se vuelvan a segmentar, ese trata de la aparición de un imponente contador, que mientras siga su transcurso corriente y no aparezca nada superior, podría convertirse, sin exagerar, en lo que más de para hablar en Temti.

Porqué, cabe destacar el significado que elementos semejantes tienen dentro del entorno en cuestión, agregada su carga simbólica aplicable casi que desde cualquier óptica, sin embargo esencialmente lo del lugar, dado el antecedente cercano con sus serias implicancias para la trayectoria del antro. Las que de antemano, llegarían a una bien estabilizada en cuanto a lo realmente trascendente: si numerosas ocurrencias puntuales supieron desvirtuar un tanto las vivencias comunes correspondientes al escenario principal, después este a grandes rasgos vio restauradas sus tendencias generales. Ante la normalidad que en el no corto plazo se desprende de unos patrones levemente variantes, adelantar novedades muy reducidamente para estas mantenerse latentes durante lo siguiente a ocurrir, condiciona la realidad temtiana, lo bastante para motivar el trazado de una linea a partir de ello.

Solamente que de manera incierta, siendo que de primeras todo lo pertinente parece permanecer constante, y las eventuales incidencias sumamente postergadas, sin tampoco garantías de que fueran a ser consistentes en el avance. Pese a ello, el posible efecto momentáneo o incluso el corolario de la progresión iniciada fácilmente puede revelarse de mientras, a su vez que la simple introducción de dicha cuenta atrás tiene su relevancia más directa, la interpretación que vayan a darle los enterados a medias, sabiendo que las posibilidades son amplias y se potencian con el régimen de información limitada. Sin embargo, desde que la puesta en marcha resulta ser básicamente eso, tampoco hay estricta necesidad de llevar todo a términos del reloj, por lo pronto solo es suficiente para marcar un bien ambiguo antes y después, especial y poco drástico, con más razón al cual la Historia Temtiana debería prestar atención.

Por esas razones que enfáticamente asimismo incluyen las no referenciadas, a la par de una cuenta regresiva es que las vivencias se continuarán escribiendo, si es que el tiempo continúa contando\ldots{} y si no, también.

\section{La segunda promesa ciega (Cuarto libro, capítulo I)}\label{la-segunda-promesa-ciega-cuarto-libro-capuxedtulo-i}

\begin{quote}
27 de abril de 2021
\end{quote}

Como casi siempre con la consideración de unas pocas excepciones, las novedades oficiales que ofrece Temti llegan de manera abrupta sin introducción alguna, y eso parece que seguirá aconteciendo hasta el fin de los tiempos de esta agraciada existencia, modalidad en común que se reitera porque nuevamente, valga la redundancia, los temtiteros se encuentran ante sus ojos una nueva situación que solo ellos van a explicar.

Un reloj de cuenta regresiva, sí, un reloj de cuenta regresiva. Tal vez no equivalente al último y único que fue tan llamativo y padecido por su época, pero que claramente no puede pasar por alto, menos mientras descuenta. Difiere que ahora el antecedente ha marcado, con el agridulce conjunto que acarrea consigo, y por aquellas razones un conteo carente de aclaraciones no parece generar una primera buena impresión. A su vez de que indicar el tramo restante para un supuesto Día E es casi que indiferente, mas esta vez no fueron entregados mensajes escondidos en E-s aguardando por ser descifrados. ¿Qué es el Día E?

Respecto al referido y ya sufrido, la enorme diferencia con aquel es la cantidad, el contador parece infrenable, pero la prontitud no es la misma. Lo que separa este momento del objetivo es semejante longevidad que tiene el sitio de existencia, y viene a ser un montón. ¿Qué ocurrirá durante toda esa infinidad? ¿El antro estará condenado a una espera dilatada condicionada por el incesante modo lento y los males que lo acompañan día a día? ¿La regresión no conocerá su punto final con el sitio en funcionamiento? ¿Algún día va a quedar claro de qué se tratará esto previo a completarla?

¿Sería realista poder entender qué se aproxima con la información actual? Se recuerda mucho de la gran capacidad de los temtianos para involucrarse con los misterios e intentar dilucidarlos. Algún misterio reciente podría haber resultado resuelto tras pesados esfuerzos, pero la mayoría permanecen inconclusos. En este caso, ¿hay algo más que antes a resolver? ¿Faltará recibir una pista camuflada o encriptada de quién sabe qué forma?

Ni hablar de que certezas no hay, solo lugar para lo de cada oportunidad, interpretaciones y reacciones. Aunque lo que más destaca como expectativa es una muerte cronometrada o en simples palabras el fin definitivo de Temti, la cual no coincide ni de cerca con el futuro vencimiento, también hay pequeñas minorías que postulan a este futuro como la implementación de las Instancias, no disparatado contemplando la lentitud que demora las reformas habitualmente.

Pero en este inmediato después todo eso da igual, el tímido pánico está y solo quizá, si nada interrumpe el normal transcurso siguiente a transitar y concluir, con el tiempo dejará de estar presente hasta que la fecha señalada se vaya acercando. Lo que sí sería innegable, es que confiar en que la explicación es peor que la incertidumbre, es la mejor solución para que la interminable agonía no sea tal.

\section{Alegría y deterioro (Cuarto libro, capítulo II)}\label{alegruxeda-y-deterioro-cuarto-libro-capuxedtulo-ii}

\begin{quote}
28 de abril de 2021, \ldots, 1 de mayo de 2021
\end{quote}

Parando de exagerar el enfoque sobre el contador, más hechos ameritaron interpretación y búsqueda de razones en los dominios temtianos. Durante los últimos días se volvió llamativo como ciertos tems, exactamente solo los más viejos, pasaron a estropearse estéticamente y así mostrar una cara mucho más vacía que la original. Y eso no es lo único, porque no solo las imágenes acompañantes parecen haber desaparecido, lo mismo ocurrió sucedió en las contestaciones que más antigüedad tienen arriba, dejando así la integridad de abundantes hilos dañados, con seguridad de forma irreversible. Gracias a la larga capacidad de almacenamiento que ampara tantas y tantas creaciones, estas permanecen establecidas y accesibles, pero el paso del tiempo no les jugó una buena pasada, posiblemente debido a detalles internos de la tecnología que mantiene a Temti en funcionamiento, consecuencia de desperfectos o funciones de ciclo.

Afortunadamente, o no, dada la dinámica que dispone el sitio con su naturaleza, estos huecos no suelen conformar la parte más visible, aquella que en el inicio conecta hacia lo recientemente posicionado, no obstante dentro del andamiaje habitual también está implícito que los dañados en cuestión regresen a ser protagónicos en el panorama, según los clásicos necrobumpeadores reviven sus piezas predilectas, salvo que alguna singularidad se los lleve y oculte. Y los procesos actuales van de acuerdo al andar esperado a veces indeseado, junto con la eventualidad que afecta portada y contenido, la revelación del fondo blanco y azul en múltiples recuadros que meses atrás fueran agregados.

Por otro lado, no mucho luego la administración implementó una chillona apariencia para decorar únicamente la ubicación donde cada individuo que visite el antro debe identificarse previo a transitar y participar, potente y brillante pantalla en que al estilo burbuja circulan las primeras miniaturas que al avanzar en dicha ruta, varias que rotando en diferentes tamaños anticipan de manera parcial la clase de escenario a encontrar después. Además cual paso en simultáneo, el efecto negativo que liquidara numerosas pinturas fue revertido en el caso oficial que lo padeció, este permanentemente anclado tuvo la única suerte de ser renovado con un aspecto nunca antes visto, que retorna el nombre de Temti completamente digitado al espacio multipropósito en el que pocas preguntas son respondidas, pero sin arreglar las demás que jamás terminan de cargar.

Con ambos tantos, sin desestimar el cronómetro del día venidero, y sumándolo al adelanto de gran actualización, aparece un nuevo color predominante, el verde, que por su presencia contemporánea pasa notoriamente a estar ligado a la estética del lugar. ¿Expandirá la tonalidad en la totalidad de este? ¿Será un tapizado temporal a caducar en la brevedad? ¿Representará el éxito o tendrá siquiera algún significado? ¿Qué tan verde resultará el futuro temtiano?

\section{Otra de paseo a la deriva (Cuarto libro, capítulo III)}\label{otra-de-paseo-a-la-deriva-cuarto-libro-capuxedtulo-iii}

\begin{quote}
2 de mayo de 2021
\end{quote}

No extraña, no sorprende, es habitual que pase a veces y así permanezca durante un tiempo: Temti no está más en su dominio. En su lugar, breves instantes permitieron ver determinadas indicaciones bien de corte interno, asuntos operativos y poco más para visitantes: indicios de una eventual migración, acomodación, gestión, y relacionados, sin embargo cero comunicación en español. Lo siguiente y actual, nada, aparentemente imposible de llegar.

A creer por lo que \emph{/Nginx/} describió momentáneamente, lo más probable en todo esto sería la continuación de un proceso sin finalización todavía, en cambio un fin absoluto tiene menos argumentos expuestos para suceder, no obstante tampoco se podría descartar.

Puede que a diferencia de las numerosas clausuras ya transitadas, esta sea una de los más repentinas de todas, respecto a la falta de avisos o referencias explícitas, y si bien en estos lares difícilmente algo se desarrolle porque sí, de tal forma es muy desafiante contextualizarlo. ¿Tal vez las consecutivas alertas o días venideros sin significado descubierto siguen en esto? El presente supo dar pequeñas señales y aunque no fue posible hallarles un propósito desde la posición del público, la costumbre es también que en el fondo quieren decir lo suyo. ¿A lo mejor la conducción requirió interrumpir el normal funcionamiento? Bastante se interpreta sobre las codificadas informaciones emitidas oficialmente sin contener debidamente tal etiqueta, pero capaz simplemente hubo necesidad de frenar el rodaje, y pronto volverá como ha sabido hacer. ¿Quizás falleció la vieja de algún anon? Pese a que la polémica sentencia fuera de dudosa literalidad, nunca terminó de quedar claro si así lo fue. Especialmente eso tercero, si lo primero ocurre cada vez que pasa lo segundo, y se repite muchas oportunidades\ldots{}

Pues cualquiera de ellas, o el alternativo desenlace, deja en espera las novedades en la misma Temti, a su vez las andanzas de los temteros en otro lado de mientras.

\chapter{Mientras se está a la deriva entre la vida y la incertidumbre (Quinto libro)}\label{mientras-se-estuxe1-a-la-deriva-entre-la-vida-y-la-incertidumbre-quinto-libro}

Contexto: Temti y demasiadas jornadas consecutivas del antro venían registrando una continuidad sin semejante, no obstante hasta cierto término alcanzó. Hallarlo fuera de acción y de manera persistente supone un cuestionamiento a las consideraciones que se mantuvieron vigentes durante ese largo rato, las circunstancias realmente cambian y de ello derivan una serie de enfoques adicionales, motivados por una pauta cuestionable y complementable. Se produjo entonces un antes y después.

Dado a raíz evidente de la no disponibilidad, tema en particular que tiende a ser asociado con los desenlaces, lo que cuesta aplicar a este clon, empezando porque el paso del tiempo fue generando un sesgo incompatible con dicha conclusión, aquel alocado concepto de Temti como inmortal o similar estuvo ganando fuerza. Gracias a su llamativa capacidad de reaparecer con presunta facilidad tras haber entre comillas muerto o mejor dicho caído, es posible creer que esta vive sin que haga falta ponerlo en tela de juicio, confirmaciones tampoco son necesarias para sostener una subsistencia, desde el entendimiento extendida hasta la práctica. Tal vez dichas nociones en un futuro presenten inconsistencias con la realidad, pero mientras tanto y por defecto, se da por sentado que el final no ha llegado.

En esas condiciones la cuestión es la forma, las características de existencia conllevan otro asunto de alta relevancia, siendo que hay distancias entre un sitio funcionando y uno inoperativo, propiedad que por sí sola tampoco dice mucho, dar precisiones únicamente desde eso abriría sobradamente el panorama. La actualidad corresponde con dicho caso negativo, limitado por un anticipo de resolución, uno medianamente confiable que oficia de sustento extra para el consenso automático. Con la carencia de inmediatez ante el punto de partida y las demoradas noticias de su propia afirmación, de ella se puede entender que habrá próximos sucesos contingentes a la coyuntura estrictamente temtiana, presumiblemente a favor de restaurarla en su formato típico.

Por tratarse de un incidente localizado, es además una oportunidad de visualizar la reacción de quienes construyen y dinamizan el mundo temtiano, contemplando que sus actividades cuentan con lo suficiente para distinguirse lejos de su mismísima localía, así hubiera que llevar la perspectiva a un nivel más detallado. Por la reducida información de turno, faltan recomendaciones que prioricen un destino donde aquello logre orientarse, y más todavía manifiestarse significativamente, cosa que igualmente está dentro de lo predecible, habiendo opciones análogas en las que sumar presencia, o en su defecto no, la intrascencencia también cabe como posibilidad. En función de la importancia relativa que ganen tales demostraciones, más deducciones podrán elaborarse sobre la identidad en disputa, articulación, contenido, convencimiento, magnitud, cualidades que saben de variabilidad.

De cualquier modo, la concientización viene excelente para disparar un montón de interrogantes, pertinentes en la medida que se prolongue la disposición de no andar como es de esperarse, y por supuesto mientras Temti efectivamente viva. ¿Cuánta trascendencia tienen los dichos acerca de la vida de Temti? ¿Cuál es el criterio para declarar que Temti vive? ¿Cómo vive Temti cuando aparenta lo contrario? ¿Dónde vive Temti si su escenario parece no admitirlo? ¿Temti vive bajo una modalidad al alcance del conocimiento común? ¿Qué problema surgió? ¿Qué soluciones le deparan? ¿Qué es la lucha que sigue? ¿Y qué lugar ocupa todo esto dentro de la Historia Temtiana?

Son bastantes preguntas a las que cuesta encontrarles una respuesta segura, y el estado de la vitalidad en cuestión no comunica ni colabora casi, menos lo hacen sus anunciadas correlaciones. Por lo que al igual que en anteriores ocasiones, ahora en estas tentativas de éxodo, para saber más al respecto habrá que centrarse en lo que ocurra sobre otras ubicaciones, y sin perder de vista el foco que provoca las movilizaciones, Temti.

\section{El hospedaje impostergable (Cuarto libro, capítulo I)}\label{el-hospedaje-impostergable-cuarto-libro-capuxedtulo-i}

\begin{quote}
2 de mayo de 2021, 3 de mayo de 2021
\end{quote}

Rápidamente, como ya ha ocurrido y es entendible que así sea, la noticia llegó a los sitios más próximos y el debate con jerga se prestó a lo mismo de siempre: \emph{/\ldots cayó temti\ldots/}, \emph{/\ldots seguía vivo eso\ldots/}, \emph{/\ldots al fin cerró\ldots/}, \emph{/\ldots ya va a volver\ldots/}, y similares.

Aunque lo anterior sea a veces un flujo de desinformación, certezas incomprobables e intercambios más perjudiciales que positivos, a su vez pueden circular determinados datos acertados que guíen a los involucrados, siendo que temteros dispersados a partir del cierre, como también aquellos que dejaron de serlo, tienden a reencontrarse. Para ellos, el mapa de alternativas donde se hallan, que ha ido variando un tanto a lo largo del tiempo y hoy con menores hostilidades cuenta de varios puntos, evidentemente interesantes en el destino de los emigrantes.

Separando a Rouzzed el gran foco hegemónico de almas malignas y benignas que vagan por el círculo de antros, están las opciones que parecen perseguir objetivos no relativos a ser participes de la típica competencia que ya no es ni eso. Se encuentran la entusiasta Arggnews y la tirana Moxxed por un lado, sin casi adeptos pero con repetidas mejoras que los alejan de la defunción, y aparte bajo circunstancias diferentes la nueva ubicación en proceso de consolidación, Voxxed, que tras múltiples semanas funcionando pudo hacerlo en buenos términos y hasta incluso destacando en el contexto. No muchos clones jóvenes se agregan de ser por cercanía, cuando los demás del pasado sucumbieron entre las guerras y el posterior desarrollo, ¿menos Temti?

Cuidado con semejante pregunta, no la den por muerta, ya supo dar sorpresa y la coyuntura sugiere que resiste, sin embargo hay más argumentos, avisan por ahí que vive. De hecho es un medio según dice oficial, el que lanzó su anuncio pertinente al caso, de que el sitio padeció problemas, y que están solucionándose. Esa sería la explicación difundida por el Telegram y prácticamente no expandida en los antros, la única actualización a priori relevante en la vigente incertidumbre.

Contundente para las creencias previas, que pudieran venir de suponer con la escasa evidencia y parecida experiencia, quizá no lo suficientemente convincentes cuando solo de ideas individuales o colectivas se trata. Aparece aquel canal, que asumiera la tarea de contar el estado contemporáneo del dominio inaccesible, una fundación de respetable prestigio acompañando a las actividades de todo tipo relacionadas a Temti, en su habitual función de recopilar para luego archivar en espacios de simple acceso, ahora lo hace con susodicha comunicación de alto interés durante el momento atravesado, sin saberse con exactitud la fuente de dicha aclaración, igualmente fiable viniendo de tal dilatada trayectoria, su palabra de respaldo a las expectativas.

Pero hay que procesarlo, eventualmente sucederá si los problemas no conocidos logran ser arreglados, después que las voluntades consigan ajustarse a sus exigencias, acorde al plazo necesario en el referido transcurso, considerando que cada cual podría exagerar lo que inicialmente parece ser. Mientras permanece la espera latente, en principio no cronometrada, por otro lado habría algún refugio que resulte preferido en quienes cargan y mueven los vínculos temtianos, también representantes en esta etapa, todavía parcialmente indefinida.

\section{Profundizar sobre una burda copia (Quinto libro, capítulo II)}\label{profundizar-sobre-una-burda-copia-quinto-libro-capuxedtulo-ii}

\begin{quote}
3 de mayo de 2021, \ldots, 6 de mayo de 2021
\end{quote}

Mientras la espera de retorno continuaba alargándose indefinidamente, en varios temtianos el lugar que su antro tenía dejó de estar tan desocupado, pues durante las fechas corrientes y hasta hoy, prácticamente igual la actividad que solía tener aquella se trasladó al particular clon del momento, Voxxed. De este modo, la expectativa de que los problemas se solucionen a corto plazo va relegándose, y en la medida estos persistan, el sitio emergente a pesar de sus disparidades y dificultades, será para una mayoría del grupo desplazado, la nueva Temti.

A esa conclusión se llegó a partir de los altos parecidos manifiestos, tomando como referencia las últimas jornadas temtianas antes de la clausura, público y contenido sumamente parecidos. Decenas de registrados comportamientos de robots, la constancia literaria en el seguimiento de antecedentes, recurrentes coqueteos entre desconocidos, propaganda comunista y algunos otras costumbres no correspondientes a patrones, bastan para resumir lo que dentro de todo sigue igual, solo que en torno a contenedores ligeramente distinguidos, la naturaleza de la plataforma que tampoco varía demasiado.

Acumulando días que permiten dicha impresión, avanza un proceso de adaptación o introducción a la realidad específica de turno, similar a la travesía en colocaciones hixxelianas, pero ahora en dominios voxxeros, prestándose para los timteros huérfanos. De hecho los orígenes menos o más directos son los mismos, hay distancias que en el caso actual no trascienden tanto, fundamentalmente cuando es un clon milimétrico de su máximo ejemplo.

Exceptuando el recibimiento local que los previos ocupadores pudieran realizar, de mayor frialdad y hasta rechazo con liviandad, los rasgos provenientes de Temti se integraron fácil y notoriamente, ante una identidad no muy auténtica o definida entonces, del sitio que tiene como mandamás a un tal Mr.~X, también apodado por Mecha Codubi. Su gran artefacto bajo intermitentes cuidados cuenta de las básicas características y algún funcionamiento atípico inserto, que por venir de una reciente fundación y mudanza sugiere que todavía precisa de más desarrollo. El ambiente a su vez marcha a la par de una baja interacción, temáticas típicas y en especial elevada libertad, quizá de límites ideales o solo descuido, más atractivo pero además peligroso. Aquello respecto al contexto, y otra similitud que comparte es la estimación de alternativa, en contingencias de emergencia externa al ser conocida pasa a ser preferida entre desamparados, por lo que sus trastornos importados suelen ocurrir, sin resistencia, e incluso a gran escala.

La parte técnica despunta del resto por lo frecuente que sirve resultados extraños y a lo mejor indeseados durante la interacción con la misma, muy parecido a los fenómenos temtianos solo que manifestados en diferentes áreas. Difíciles de entender o complicados de arreglar, inciden constantemente en el rendimiento esperado por cualquier deambulante y más si hubiese experimentado en el modelo homenajeado, quizá justificables bajo la amplia óptica interpretativa de las irregularidades, eventualmente objeto de críticas, constructivas y destructivas.

A primera intuición y en la práctica tiene las clásicas propiedades de intercambio, no obstante falla al reconocer originales publicadores, y así los creadores ven entreverado proclamarse autores de sus propios hilos, solo habiendo un mecanismo optativo de seudónimos que puede mantener la linea del que escribió cada cosa, útil en ese sentido y de preservar la sesión, pero crítico para el anonimato. Aparte estos cuentan de un privilegio determinante en sentido de moderación, remover a su gusto las réplicas que quieran, potestad que fácilmente derivaría en abusos y menos criterio positivo según quienes la tendrían. Siendo eso lo conflictivo, se le suma el inestable sistema de bumpeo, propenso a desaparecer unidades y rebuscar el acceso o visualización entre los demás que permanezcan. Al culminar la lista aparecen ciertos botones, unos incapaces de producir acciones evidentes o reales tras que sean pulsados.

Es confuso continuar oficialmente el trayecto histórico aquí, la forma que estaría adoptando supone inferior relevancia, sin embargo no precisa de ubicación obvia, igualmente prosigue. Será la manera, más recordando las situaciones que desde hace rato construyen para atrás la actualidad, pero el pasado ya menos es considerado y el futuro tampoco nota ser presente, aun si la evolución inspira que sí sería clave. No obstante consecuencias a la vista, la perspectiva temtiana descontando excepciones no vive más que el día a día, tan solo alcanza ver qué acontece cuando no hay un reloj visible, la ausencia también condicionante. Prolongando la tendencia migratoria, los defectos no importarán y aquí se asentará la diáspora dilatada, mientras los tiempos de éxodo posterguen finales, el popularmente llamado Club, el espacio donde más estén viéndose los restos de la orgullosa y resistente Temti.

\section{Pobre y fuerte razón de ser (Quinto libro, capítulo III)}\label{pobre-y-fuerte-razuxf3n-de-ser-quinto-libro-capuxedtulo-iii}

\begin{quote}
6 de mayo de 2021, \ldots, 18 de mayo de 2021
\end{quote}

Posiblemente el famoso dicho de que el Club de Voxxed es la nueva Temti se haya tornado lo suficientemente repetitivo como para convencer de ello, incluso cuando no muchos lo promuevan. No hay problema con eso, el mayor porcentaje de episodios relevantes al caso luego del cierre sucedieron en el sustituto, por lo que aunque no sea lo mismo, no hay dudas de que sí, momentáneamente o no, es el nuevo principal hogar de la historia que parecía interrumpida.

Sobre el mismo escenario y de interés para la conformidad de aquellos protagonistas, cabe destacar, también en el proceso de conocer más al detalle el lugar en cuestión, que el sitio de la mano del señor equis, concluyó múltiples modificaciones que apuntan a corregir algunas anomalías, también reconocidas como fallas técnicas. Con esto, se llega a un estado más predecible, fácil de entender y poco peculiar, aunque aún no haya podido del todo dotarse de tales características. El sitio en general lo recibe para bien, lleva un rato esperando progresos y reparaciones en concreto hacen de una mejor experiencia, incluso para el temtiano promedio y su renovada mística.

La que en otra de sus principales dimensiones volviéndose preponderante, determinante en la agenda con su flamante discurso y realidad, la vida que persiste, la lucha que sigue. Difícil de describir y más de precisar, fácil lleva a suponer que efectivamente sucede, reservada y escandalosamente.

Muy relativa y ambigua, admite desorientaciones y cuestionamientos, entretanto en paralelo continúa desarrollándose con las contribuciones de los involucrados, a falta estamentos imparciales que indiquen aunque sea su circunstancia, solo indicadores implícitos de aproximación mientras se participa. ¿Triunfo? ¿Derrota? ¿Final? Cuál esté viviendo el dueño o el individuo que la declaró, no tendría que necesariamente coincidir con la del temtiano promedio, a su vez que podría diferir según la ubicación temporal espacial, hasta que tal vez cada uno atraviese la suya propia, no obstante la impresión es más colectiva, siendo que en comunicación los matices en común son reiterativos, o la insistencia callada, o los gritos de guerra. Todavía incierta, varias realidades no exactamente determinadas comprueban parecer cual disputas, presentes contemporáneamente.

Por ejemplo y más reincidente o evidente para la perspectiva migrante, las costumbres típicas esparcidas de un lado a otro figuran como expansión territorial, allí domina el persistente ida y vuelta de los romances ratonescos, la promoción de la ideología comunista, importaciones de contenido rozzado, replicas copiadas ignorando contexto, y demás comportamientos de robot explícitos\ldots{} constantes que haría sentido interpretarlas a modo de batalla.

Aquella tiene una adicional, de notoriedad en el campo del sitio ocupado, la nueva tradición que nació o se reinventó pasa a ser casi que de los patrones más importantes en esta época especial, a lo mejor asumiendo un viejo propósito solo que en diferente enfoque, aunque medio primitivamente, al ser ínfima la innovación pertinente. La reproducción al estilo automatizado de antiguos temas es el motor insignea de la puja temtiana fuera de sus pagos, por el cual hilos de alto valor pasado son reestablecidos en dominios voxxeros, al menos en los primeros elementos como título y portada, para luego derivar en aportes un tanto más básicos que en sus respectivos entonces.

En contraparte estaría la que no, de menor visibilidad y mayor misterio, que no se ve y en algún punto de los universos ocurre, quizá según supuestamente han hecho creer las declaraciones. Nuevamente es bastante la incertidumbre sin clarificar que la frase arrastra atrás suyo: la proporción de mente y existencia interroga por el espacio donde estén moviéndose los avances y retrocesos, el nombre discutido desconoce el compromiso y si sea generalizado o individualizado, los esfuerzos son tan anónimos y privados que no logran verificarse, la proporción de objetivos completados permanece más que impreciso, y un largo etcétera. Todas convergen a un mismo significado central, oscuro por lo escaso extraído de las mínimas demostraciones y conclusiones provenientes de ese lado, decreciente en cuanto atención para los afectados, la situación del sitio y su regreso.

Aunque en teoría dicho frente concentre la parte más relevante de la lucha original, la vida tampoco se reduce ahí, los demás incluyendo no mencionados enseñan el potencial que pueden constatar durante la corriente gesta histórica, perfilada a sostenerse o incluso intensificarse en los sectores que vienen amplificando el eco temtiano, lo opuesto al silencio del caso pendiente.

\section{Coparticipativos desequilibrios territoriales (Quinto libro, capítulo IV)}\label{coparticipativos-desequilibrios-territoriales-quinto-libro-capuxedtulo-iv}

\begin{quote}
18 de mayo de 2021, \ldots, 20 de mayo de 2021
\end{quote}

Paralelamente al mismo desarrollo de las temáticas voxxeras y su asentada connotación temtiana, de excepciones en mayoría crecieron algunos factores extraordinarios y así la entre comillas conquista pasa a verse acompañada, complicada o envalentonada.

El sitio ocupado incrementa su variedad, las creaciones más de los sectores invisibilizados y subidas de tono predominan aparte, hasta incluso con participaciones amarillas como en la jerga se las conoce tan bien. Estas en especial típicamente movilizan, pero al contexto en cuestión han sabido insertarse más que pasajeramente, de vez en cuando pueden verse sus rastros y materiales explícitos en los recovecos del antro, a veces demorando la remoción más de lo esperado y trascendiendo entre el compartido público de menor moderación.

Dada la integración de los murales promedio en dichos lados, nuevamente asociarlos con los temtianos hace sentido, la imagen identitaria de estos es frecuentemente relacionada al contenido prohibido, que en las circunstancias presentes tiende a encontrarse realmente cercano a los discursos y gráficos del antro ausentado, aun así por su propia cuenta se distanciaran, el parecer vuelve a pesar fuerte. Porque si Voxxed es Temti, todo estaría yendo a ser considerado.

El relegamiento local o extranjero también se nota si la inestabilidad rozzada reitera sus efectos en las alternativas semejantes, es decir cae el principal sitio y los muchos adeptos reanudan la experiencia bajo un marco de distinto color, porque efectivamente lo hay y saben cual es su dirección, por ello el tráfico escala tan abruptamente y la incidencia del desplace entra junto a otros temas al debate, en grandes proporciones para el nivel de movimiento previo. Probablemente sean condiciones transitorias a culminar cuando los orígenes son restablecidos, sin embargo con poco trastocan notoriamente el panorama, relegando lo que hasta entonces hubiese dentro de sí, después difícil de acomodar.

De ambas maneras es que las habladurías nacionalistas de los temtianons y sus preponderantes intereses secundarios, pierden protagonismo. Todavía perdurando en el ambiente, la actividad de diferentes parcialidades hace notar que unos cuantos más lo conforman así sea por ratos cortos, lo cual en el futuro podría acentuarse. Y además importan los involucramientos anteriores, ya que no solo quizás bajen sus frecuencias, tal vez menos sean destinadas a la causa en lucha, pues aunque históricamente sobre Temti fuera bastante lo conversado, no necesariamente el enfoque estará siempre en ello. ¿Alcanzarán las pilas de los robots para figurar en el exterior? ¿Algún día verán reprogramados sus comportamientos? ¿Cómo se distinguirá la nación nómada si no es por su noble chusmerío?

\section{Agridulce expansión de la fama (Quinto libro, capítulo V)}\label{agridulce-expansiuxf3n-de-la-fama-quinto-libro-capuxedtulo-v}

\begin{quote}
20 de mayo de 2021
\end{quote}

El vínculo temtiano con los rozzados no pasó por demasiados momentos diferentes , desde los no suficientemente marcados comienzos y posteriores construcciones, varias veces la intención generalizada era de alejarse y distinguirse rotundamente, sin embargo esa idea más nacionalista tras lograr una notoria separación, mientras se desarrollaban las inestables dinámicas y entusiasmos contemporáneos, terminó viéndose frustrada, y un algo por el estilo sería lo siguiente. Las influencias indeseadas aunque mínimas pasaron a convivir íntegramente, hasta incluso no parecer contrarias, de lo bastante que compatibilizaron entre sí tanto en ubicaciones propias como ajenas\ldots{} una nueva excepción lo pone bien gráfico.

La mayor sorpresa de los tiempos corrientes se da acertadamente afuera, pero muy directamente para venir de allí, es la aparición o conversión de un pequeño nicho en los amplios dominios de Rouzzed, uno dedicado a Temti, o al menos que le referencia. Luego de verse actualizado el largo segmento de categorías tuvo lugar el nombre del antro como tal, entre últimas posiciones más invisibilziadas y fuertes por su contenido específico, esta en concreto repleta con dibujos asiáticos de corte obsceno, solo que no bajo su denominación original, sino que prosiguiendo al titular de Temti. Aunque este no permaneció por más que unos minutos, la imagen real y conceptual quedó en la vuelta, de un antecedente extraordinario proveniente de las altas esferas, representando la opinión pública o mayoritaria acerca del sitio en cuestión y sus adeptos identificados, trascendencia para ellos cuando poco la tenían.

La noticia recibió menciones en la vigente concetración voxxera y temtiana, un par de perros resaltaron el centro de atención durante la jornada nocturna. Para los valores opacados sería nefasto, no únicamente parecerse al clásico enemigo sino además ser contenido y parodiado por este, sin embargo ya casi nada hay de esa lucha, la actual más atiende a la presencia y en ese sentido está su parte hazañosa. Entre los anonos rouzzeros semejante detalle no pasó desapercibido, y el rasgo distintivo de dicho apartado muestra el pensamiento que tienen unos cuantos de susodicho antro, \emph{/putaku y cepitero/}, por lo que el cambio tampoco convenció y los reproches sobre el riesgo asumido fueron pronunciados, no obstante rápidamente hubo marcha atrás y el personaje de mando Mordrake justificó el recuerdo de tal categoría como una ilusión. Las pruebas no lo respaldan así, pero quizá convenga desentenderse del caso, porque más allá del chiste, podría efectivamente resultar negativo para sus intereses particulares, a su vez que alimentar esa suerte de estigma mediante hechos oficiales es controversial según se lo interprete.

Si la creencia promedio motiva aquello, es fácil de suponer y entender como una versión paródica del antiguo Plan o simplemente un chiste pertinente dadas las coincidencias, cuando a lo mejor era la oportunidad de traerlo junto con otros ajustes más serios en el sistema de organización correspondiente. La fecha puntual tampoco dice mucho, sin embargo justo ocurre que los temtianos defensores de lo suyo andan inmersos en un contexto problemático, y al poco saberse de ellos, este tipo de reputación predomina en las mentes distanciadas, haya una categoría o no.

\section{Reivindicando la pulseada de la verdad (Quinto libro, capítulo VI)}\label{reivindicando-la-pulseada-de-la-verdad-quinto-libro-capuxedtulo-vi}

\begin{quote}
20 de mayo de 2021, \ldots, 21 de mayo de 2021
\end{quote}

De nuevo en la trama particular del clon más clon, su relleno tan particular resultó menos indiferente de lo que ya era, las actividades temtianas crecieron sobre la diáspora y ascendieron de categoría, hasta la totalidad del sitio. Ahora se trata del Club de Temti.

Siguiendo el desarrollo histórico que toma lugar fuera del lugar original, definitivamente hay hechos de valor, la atención al inestable y variante panorama recompensa con un radical cambio en los registros de Voxxed, que dejó de ser tal y pasó a ser Temti, en respuesta a un pedido de la insistente comunidad que planteó como una posibilidad oficializar lo dicho, y el dueño la hizo más que realidad.

Muy probablemente las demás reiteradas declaraciones protagonizadas por el colectivo de temtiteros tuvieran que ver con semejante resolución tomada de parte de Mr.~X, que volvió a mostrar su etiqueta amarilla al conceder el deseo no utópico, apropiado para aquella coyuntura donde el denominador común de los hilos contiene el actual titular absoluto del antro, porque exceptuando pasajes cuando el exterior lo ameritaba, granes proporciones del movimiento vienen teniendo aunque sea algo de Temti, de cierto modo el Club le pertenece a dichos integrantes.

¿Será que el dueño cedió? ¿Supo convencerse de lo que dicen? ¿Se prendió a la joda? Sea como sea, esto promete que habrá una cantidad superior anones contentos, no estimados en ninguna proporción, sin embargo los más activos e involucrados, de ser mayoría los infelices resaltarían cual minoría, que según ese criterio, podría decirse que el pueblo habló y el mandamás avaló. Y por cierto, es incierta la verdadera relación que este tenga con el antro en sí o sus conocidos animadores, relevante dado lo tanto que debió encontrarlos, pero ha de suponerse que es en buenos términos, en especial considerando este antecedente, suele brillar positivamente en la escena gracias a la interacción de tú a tú, una atención que denota compromiso.

Por ahí sea demasiado sobre aquel, que precisamente al ponerse un signo de interrogación en su silueta quizá no quiera que sepan al respecto, o solo es una identidad que sin mayor significado seleccionó, pero la coincidencia del nombre anónimo levanta determinadas sospechas y más todavía con el paralelismo que están trazando los administradores temtianos. El auténtico estatus desconocido del supuestamente ausente de A. sugiere interpretar acerca de los parecidos, los cuales no quedaría raro que vengan de un mismo individuo, cosa que no desmiente, pero tampoco es confirmable, solo una serie de parecidos presumen que tal vez sí.

Con todo, la circunstancia sigue siendo atípica, y dentro de algunos espectadores puede caber la idea de que no permanecerá así por mucho solo deduciendo: porque va contra el propósito inicial, porque no tiene sentido, porque carece de seriedad, porque muestra debilidad, porque da una imagen negativa, porque efectivamente sería para un día, o porque sí, pero el contexto incluye indicadores que van por otro lado, en ese caso habrían contradicciones por resolver, en el Club de Temti.

\section{Las caídas en la disputa trascendental (Quinto libro, capítulo VII)}\label{las-cauxeddas-en-la-disputa-trascendental-quinto-libro-capuxedtulo-vii}

\begin{quote}
22 de mayo de 2021, \ldots, 28 de mayo de 2021
\end{quote}

Después del gran hito que logró la lucha temtiana en su sitio predilecto de diáspora, lo que siguió dentro del mismo fue un tanto cruel para su visibilidad, de un paso a otro ya no se los vio, aquella amplia presencia que caracterizaban tendió a cero, y las próximas vivencias son inciertas por demás.

La celebración de la conquista trajo felicidad y reivindicaciones, predominantes sobre las molestias que calificaron al nuevo nombre de broma, la constante tendencia del lugar respaldaría al título, que posteriormente fue retirado, y la controversia pasó a reproducirse. Así simplemente un día la corriente temtiana puso su marca registrada y luego la mayor diversidad de anones encontró de regreso la firma voxxera, de cierto modo representativa del panorama entreverado, donde los intereses de la población no se redujeran a una sola perspectiva ni única manera de transmitir las temáticas, la reiterativa inculcación de quienes defendieran su nostálgica identidad cada vez en menores proporciones.

Pero la historia en relativa cuestión quedó pausada, altamente asociable como la consecuencia de una desestabilizadora cadena de sucesos que envolvió a la fiel imitación, un alboroto esencialmente originado fuera de sí, allá como siempre los rozzados generando estragos donde sea que conecten. Sus conflictos con fuerzas multitudinarias del exterior llevaron un peligro que sortearon con una avivada estrategia, el levantamiento de un muro que filtrara las visitas normales, acompañado por otro detalle no menor que además de dispersar, redirigió parte de la atención.

Una recomendación sobre la carta que el clon legatario parodió del antro padre, incluyó referencias al sitio Voxxed y al factor denuncias, cuando sus complicaciones pendientes justamente eran cosa reincidente. Aunque cotidianamente pareciera andar en términos apropiados, incluso sumando una función de recuento a las computadoras prendidas como si todo anduviera bien, y teniendo precedentes correctos sirviendo de refugio, la exagerada amplificación de tráfico en dichas circunstancias coincidiría con un cierre no falso, sino real. Los motivos podrían creerse obvios, de sentido común, y más viendo las declaradas intenciones de la señora limones, no obstante nada precisó la verdadera incidencia de cada elemento, o las expresiones comprensibles del incógnito dueño acerca del caso.

Como aditivo del mismo, dos conexiones temporales con una resurgida alternativa hacen el desenlace menos terminante, las huellas detectadas del sitio objetado involucran a Tssubit, pero todavía dejando ese vacío de contexto por el cual poco se interroga. Un desvío que por escasos minutos ocupó sus dominios, y una oferta tipo subasta por la tecnología del antro, las pistas que dio el sujeto Mr.~X durante ese rato que el antecedente conmocionó al ambiente, algo estarían diciendo, y si habrá que valorarlo, quizá de acá no vaya a saberse más al respecto.

Movimientos completamente atípicos, ponen varios signos de duda en el proceso de turno, que por regla general indicarían la despedida de Voxxed, el rescate primero al cual anónimos de múltiples procedencias asistieron en la época de su avanzado desarrollo, cayendo. Y el complemento del escenario se ajusta similarmente a como arrancó, siendo que los demás clones continúan en la sintonía que venían, hasta el que gestó la sospechosa táctica, donde dada la mención a lo mejor fue considerada la predatoria secuela para su vecino.

Y raramente, falta la palabra de quienes por excelencia supieron seguir en la suya. Aquellos quedaron como los últimos pantallazos visibles de la famosa lucha, cuando en el principal medio de ida y vuelta detuvo a presumir forzadamente las operaciones. Determinada la repentina clausura del antro, la reconocida concentración de temtiteros no tuvo otra que desperdigarse, según la forma de figurar, a destinos donde hacerse un lugar y difundirse pinta más difícil. Aún así, es de suponer que pequeños destellos confirmarán que las memorias, el orgullo, y las influencias temtianas permanecen vivas dentro del entorno al que corresponden, solo que careciendo de un espacio propio o ajeno que favorezca su esparcimiento, menos se anticipan los siguientes pasos, entonces en el amplio sentido que puede ir una pregunta semejante, ¿qué resta para Temti en los tiempos que vienen?

\section{Un paso más cerca para descubrir lo distante que se está (Quinto libro, capítulo VIII)}\label{un-paso-muxe1s-cerca-para-descubrir-lo-distante-que-se-estuxe1-quinto-libro-capuxedtulo-viii}

\begin{quote}
29 de mayo de 2021, 30 de mayo de 2021
\end{quote}

Para poner fin al silencio temtiano, que no duró mucho realmente, el sitio de origen renovó presente, anunció futuro. Aquella olvidada fecha indicada mediante un contador, está resultando ser la señalada donde supuestamente ocurrirá algo, presumiblemente la apertura de la clásica Temti después de un largo tiempo. Dentro de la dirección propia, bajo una vez más desperfectos protocolares, únicamente hay accesible un pequeño nicho de conocida ambientación, con la cuenta regresiva ya no perdida de vista, descontando segundos uno por uno.

Sin el nombre de lo que supo ser \emph{/Dia E/}, el mismo instante de la cita postergada para varios meses después, es denominado como \emph{/Reingreso en orbita/}, y a eso tienden actualmente las proyecciones temtianas, dando más que nada la pista obvia, cambios que abran paso a elementos que antes no se encontrasen donde el concepto refiere. Tal vez más rebuscado de lo que aparenta, un término simple aunque no explícito que dadas las circunstancias poco lleva a creer otra cosa.

A partir de las similitudes encontradas por el contexto verde, diez pinturas rondan en el mismo y difieren de las anteriormente seleccionadas del inicio, son imágenes conocidas las que completan el efervescente paisaje, a su vez acompañado por una banda sonora familiar, tonada y letra de una distinguida figura particularmente presente entre los recuadros flotantes. Casi que todas ellas guardan alguna relación con el contexto, varias piezas que en la cultura temtiana aparecen notoriamente y las demás posiblemente también, junto a la canción bastante explícita en su lírica, tal vez no dedicada la situación particular del antro.

De lo primero además de ampliar su significación va por ahí, incluso concordante con la idea del botón rojo y el cohete al espacio identificados previo al cierre, quedaría esperar la vuelta a coordenadas cercanas. Y el resto es más ambiguo, si fuera el caso de no ser meramente decorativo, todavía podría haber relevancia en el fondo que la pantalla está presentando. Descartando que el productor mandara tales cuadros al azar, puede ser que tengan un sentido adicional, del antes o el después, similar al misterio pendiente de la imagen larga. La canción evidente y su mensaje de extrañar tampoco tiene que reducirse a lo obvio, porque la realidad temtiana viene acostumbrada a esconder significado\ldots{} pero en las épocas corrientes da la sensación que este no logra ser completamente revelado, más con la indiferencia o incompetencia de las cortas generaciones recientes. Cada incertidumbre de aquellas, forman una gran incógnita que estaría precisando de más datos, en especial cuando las interpretaciones no explican.

Y a lo mejor los haya, aunque ya transcurriera casi una cuarta parte de los ciento y veintidós días, este es otro comienzo y punto de partida con todo por venir. Pero apartando aquellas cuestiones, vale reiterar abiertas, el foco de la presentación naturalmente puede más y de hecho es, sumamente vistoso, sin embargo de momento solo eso. Salvando las distancias, incluso capaces de potenciar la intriga, tiene prestaciones muy indicadas para generar expectativa, puntualmente en los momentos cercanos al evento que allí se encuentra marcado, porque de mientras parece que primará la actividad no concerniente a este tema, si algo de Temti llegara a decirse por ahí. Los plazos que se avecinan son inciertos e indeterminados, y más si no surgen espacios de temática temtiana entre ahora y esa gran fecha de vuelta tan alejada, generadora de novedades no menores.

\section{Sensaciones de descanso eterno (Quinto libro, capítulo IX)}\label{sensaciones-de-descanso-eterno-quinto-libro-capuxedtulo-ix}

\begin{quote}
31 de mayo de 2021, \ldots, 13 de agosto de 2021
\end{quote}

El silencio es estrepitosamente firme, cientos de veces más intenso y prolongado que aquellas oportunidades cuando transcurrían horas sin un solo movimiento, pasaron a ser meses, cual si fuera época de hibernar, la gran pausa de cierre efectivamente real. ¿Qué podría haber para replicar sobre Temti en momentos de semejante inactividad?

Las particularidades de dicha dilatada circunstancia tan especial en lo variada y diversa que es la historia del antro y derivados, estará sumamente distinguida entre las demás y a lo mejor por encima de múltiples que ni pena ni gloria tuvieron, aparte de que en extensión seguramente. Sin precedentes de similar característica, el tema de énfasis destaca a Temti en su estado de espera temporal y cuenta regresiva groseramente interminable, mientras resalta la falta de referencias en el contexto que marcan el desinterés en cuestión.

Claramente la distancia afecta, sin embargo no siempre habrá tanta lejanía, y ahí la esperanza de que en algún instante futuro la situación aumente en relevancia. Porque tarde o temprano las ocho frías cifras cambiantes del luminoso contador están destinadas a evolucionar e inspirar mayor cercanía, y porque el resto de factores que a los anones condicionan tienen altas posibilidades de alterarse, por eso vale cuestionarse qué mucho pueda llegar a progresar el panorama desde el inicio hasta el fin del plazo planteado.

Es bastante abstinencia, además contrasta con las etapas de éxodo previas, fue ir de excesivas menciones a nulos comentarios, pero incide lo que se conoce, la perspectiva se alejó de las conclusiones o incertidumbres, considera un montón más la continuación, junto a la frialdad del caso. Tal vez alcanzado un punto el hecho programado prenda a más desencantados mediante, nuevas señales oficiales, expectativas recordadas, o quizás al revés, que nada ocurra antes del día apuntado. Entretanto el resumen es de cero noticias y puras especulaciones: que avance el tiempo, ya por entonces podrá saberse o decirse más al respecto\ldots{}

\section{Buscando esa dulce compañía (Quinto libro, capítulo X)}\label{buscando-esa-dulce-compauxf1uxeda-quinto-libro-capuxedtulo-x}

\begin{quote}
13 de agosto de 2021, \ldots, 1 de setiembre de 2021
\end{quote}

Largo, pero largo intervalo de poco y nada para el plano temtiano, ya que lo concerniente a su sitio o adeptos ha sido muy próximo a cero cuando así parecía que sería, sin embargo eso supo cambiar un tanto, casualmente y causalmente cerca a la fecha que el recuadro verde marca a suceder, cada vez más pronto.

Aunque medianamente lejos todavía estuviera, las miradas sobre Temti volvieron, hubo unos cuantos individuos fijándose, solo que desde antros aparte, donde la expectación colectiva se hizo un espacio y adquirió reconocimiento entre extraños, sin ser a gran escala igualmente. Esto adoptaría diferente perspectiva cuando aquellos lares entraran en turbulencias, porque como consecuencia, los rozzados al inestabilizarse repercuten en las alternativas. Ahora con nuevas actividades peligrosas que acercaron la amenazante muchedumbre de los mundos normales, a lo que si bien el sitio tomó sus medidas estratégicas que a la brevedad recompusieron la dinámica en sí para cada caso, sucedieron numerosos momentos en los que el desoriente prevaleció, y la respuesta de emergencia propuso lo dicho, refugiarse. Pero en esta ocasión uno de los conocidos, no tuvo disponibilidad.

Para quienes andaron en esa enredada lista de eventos que movilizaron al contexto fueron varias las mudanzas e incluso los nombres que surgieron, de Rouzzed a Dibujazzo y Rouzzer, los protocolos inesperados y transitorios llevaron a la multitud de un lado a otro, bastantes se perdieron en el camino con la mezcla desafortunada de dinamismo y desinformación, quedando por ratos ubicaciones exclusivas donde las posibilidades de acceso disminuyeron. Aspirantes desentendidos y desacreditados del movimiento principal no tuvieron otra opción cercana que los clones secundarios. Por renombre y recuerdo habría estado Temti allí, siendo la primera candidata, no obstante su presente de formato atípico fue adverso para la función esperada a demandar, pues entonces los siguientes antros tomaron su puesto, Rozzedec y Arggnews los de menos gringos o peluches, sino de similar base esencialmente.

Sin reacciones ante la circunstancia exterior el sitio temtiano permaneció hasta el concluir de las complicaciones del centro hegemónico, y de cierto modo la atención que estaba fuera de lugar regresó, sin muros o traslados que obstruyeran el reacomodo general. Oportunamente el tema de espera a las novedades se mantuvo abierto al intercambio, y desde tal la previa tiene un importante medio de difusión, donde más evidentes son los pensamientos, progresivamente con el paso del tiempo propensos a manifestarse en mayor cantidad. Partiendo de las memorias primordialmente ajenas, el común denominador ha estado entre referencias a lo polémico y prohibido, más también suposiciones del obvio retorno, que a su vez se prestaría a lo anterior, mientras el resto no ha tenido mucho para decir.

El interés creció y aunque de tipo variado, entretanto aún poco seguro se sabe acerca de lo que realmente ocurrirá cuando llegue el día y hora indicadas, ya es más claro cómo se anticipa este aplazado acontecimiento extraordinario titulado Reingreso en orbita. No todas las sensaciones lograrían resumirse a la frase clave de extrañar que los temteros podrían pronunciar, además una óptica menos introducida en la cultura desplazada comparte conjuntamente, bien completo el panorama desde ese sentido. Allí los factibles espectadores: distantes, entusiasmados, curiosos, desinteresados, y un enorme etcétera que estará, muy seguramente estará, aguardando por presenciar qué es lo que va a ingresar.

\section{En atmósfera del mismo advenimiento (Quinto libro, capítulo XI)}\label{en-atmuxf3sfera-del-mismo-advenimiento-quinto-libro-capuxedtulo-xi}

\begin{quote}
1 de setiembre de 2021
\end{quote}

Entrando en la recta final\ldots{} nada salió de lo predecible. Los anónimos de distintas procedencias, en su mayoría provenientes de ubicaciones rouzzeras, filtrando no avivados que procedieron sin la triple doble ve, están enchufados en dominios temtianos para observar la continuación de esta especial fracción de una muy dilatada historia.

Expectativas de impacto y no de aterrizaje son las abundantes sobre todo, lo imaginado para luego de la eventual apertura dejó de reiterarse y es el momento que se viene lo que montones de campanas rojas mueve. Contando a partir de horas atrás que lo restante es de ser recordado en formato de minutos y segundos, además de hacerlo el reloj cada vez que es consultado, que al aproximarse tanto a su destino terminante resulta mejor tenerlo cuan cerca como sea posible, y de tal modo dichas rutas ahora están completamente conectadas con el público.

Lo que Temti genera, propulsado por lo que una cuenta regresiva llegando a su conclusión después de mucho promueve, proyectado desde el más sensato de los sentidos interpretativos ante el mensaje implícito de lo que viene, siendo recibido por una población que hasta lo más insignificante lleva al extremo y potencia en la masividad de su sitio tan amplificador, el anticipado lanzamiento de un estelar estreno ya casi, a punto de producirse. Esa es la bomba de tiempo que se fabricó y que absorbe la mayor parte de las miradas en mundillo de los clones, la cual instantes mediante, estará estallando.

\chapter{La época de las infinitas prórrogas disfuncionales (Sexto libro)}\label{la-uxe9poca-de-las-infinitas-pruxf3rrogas-disfuncionales-sexto-libro}

Con la leyenda de \emph{/THE RIDE NEVER ENDS/} más presente que nunca, el mayor enlentecimiento jamás visto hasta entonces se redimensiona y arranca a darle un sentido sumamente diferente a uno de los fundamentos que acompañan a la realidad temtiana de todas formas y colores.

Siguiendo con el símil de mantenimiento, aquel estatus excepcional persiste, según el uso clásico sería pasajero, no obstante su trayectoria esperada resultó corta y la circunstancia efectiva larga, con la consecuencia de incrementar distorsiones, afectar a menor escala las vecindades. Progresivamente la actualidad consolida un mismo mapa de antros, poco apropiado respecto al nombre definido en torno a los temas y tiempo, pero igualmente encontrándose este presente. Siendo posible el encuadre sobre sí, explicar los acontecimientos que le competen puede ser, no más que sin semejanza al escenario que le diera a conocer, su relleno desencajado si existe adquiere visibilidad en áreas conexas, mediante el deslocalizado e insistente seguimiento, en la extendida conversación del hogar evidentemente más que en cuestión. Eso consiste del éxodo, que de ser válido llamarlo así, representa un periodo de alejamiento espacial sin conclusión para la identidad temtiana, aquellas condiciones vigentes que la llevan a moverse por otros sitios.

Lo antedicho ya supo darse con distintas ubicaciones, etapas de bajo reconocimiento percibieron los desplazamientos temtianos, y uno de ellos permanece latente, casi como un proceso que continúa. Pero una fecha clave divide este en dos, sin importar que las expectativas tuvieran lugar, la fijación inicial del \emph{/Reingreso en orbita/} anticipó un antes y después notable cuanto menos en el ambiente, las proyecciones en los allegados contrastan haciendo todavía más incierto el futuro. Entre esas cosas no sabidas con exactitud, el optimismo se justifica mejor desde las señales manifiestas, mientras el engaño y el abandono son ideas factibles, también las demoras están ahí. Posibilidades de novedades contundentes hay, y estarán pendientes hasta nuevo aviso, tal vez con alguna propensión implícita, sin embargo igual en la dirección que sea, lo cual abre demasiado el panorama.

Aunque referir a precisión rechine en el caso, claramente aquello pertenece a un paso ulterior, teniéndolo postergado se presenta otro enfoque, por obviedad perfilado a concentrarse en lo esperado, y bastante depende de dicho término. Aun pasando como fraude y exagerando por tal, el hecho prematuro al trascender tanto logró atrapar la atención ajena y en proporción es capaz de mantener su relevancia conjuntamente, más considerando el cambio irreversible en la reputación del entre comillas antro. De sus antecedentes también sobresale la pérdida de foco en la cual fácilmente incurre, si la experiencia va para largo los complementos temáticos tienen espacio que recuperar, la cultura propia ejemplifica bien el potencial desarrollo que hay contenido. Ya de por sí la Historia Temtiana va por un camino alternativo, y desde este más de ellos dispone, pudiendo restaurar el original en cualquier momento.

Mientras los contextos operan a favor de la repercusión, las secuelas por ahí van a parar y de extraordinaria manera queda por confirmar. Por su índole coyuntural, una gran compañía eventualmente coincidirá, que según su indiferencia e interés un montón podrá aportar, y aunque vaya a disgustar, en la medida que su presencia consiga figurar, es una parte que corresponde contemplar. Entre aquello mucho hay que específicar, en especial la diáspora valdría la pena enfatizar, esa que resiste y lucha otras vicisitudes sabe interpretar, no hay error que ahuyente ni tardanza que le desvíe, la intriga y demás sensaciones verdaderamente pueden más. Si el antro da lo suyo para hablar, genial, de lo contrario también habrá sobre lo que ampliar, el gran anhelo de estar junto a Temti funciona, incluso cuando la misma parece no hacerlo.

\section{El inédito fracaso terrenal (Sexto libro, capítulo I)}\label{el-inuxe9dito-fracaso-terrenal-sexto-libro-capuxedtulo-i}

\begin{quote}
1 de setiembre de 2021
\end{quote}

Y\ldots{} Se supone que llegó el \emph{/Dia E/}, la fecha y hora del \emph{/Reingreso en orbita/}, pero la enorme expectativa de la multitud no pudo satisfacerse, nadie esperaba fallar al \emph{/Comenzar!/}, lo que de verdad pasó, el arranque de una cadena de errores y repercusiones. Así es, el resultado de intentarlo es encontrarse con la imposibilidad de ir más allá dentro de dominios temtianos, crudos caracteres de inadmisión que estos devuelven, el indicador de que el proceder condujo a desenlaces no deseados y críticos, lo que en jerga rouzzera sería chocar, el único gran impacto que supo cumplirse al colisionar la nube de esperanza y la hermética realidad, repetido a menor escala en toda oportunidad que alguien insiste en ingresar.

Lógicamente la impresión más contada va por ahí, es abundante lo enfatizado sobre la primera conclusión explícita, que no funciona. Las reuniones de recibimiento anticipado muestran como cada aspirante a conocer la suerte de inauguración vio frustrada su entrada yendo a través del botón revelado, cuando el sitio dice crudamente que por ahí no es posible. Siendo que todo parecía armado para un estreno épico, se venía la gran apertura o algo similar, y al final es como si hubiera salido mal, desacierto técnico u organizacional mediante, desastre natural interestelar a lo mejor, aunque pueda interpretarse a este punto no queda claro lo que estropeó aquello por ocurrir, aparte de la idea desde antes.

Confiando en la seriedad del antro de las esperas eternas y la coherencia de sus desarrollos no comprendidos por completo, sería seguro creer en que el problema proviene de los demás, cuando ciertamente motivos para justificar el caso hay, más que el factible Día del Error 405, sin ir más lejos pretextos típicos de la tradición a su vez probables. Por diferir del calendario y huso que rigen sus programaciones aconteció el desentendimiento, porque el reloj es sensible a la ubicación espacial, y sabida es la diversidad de procedencias. O también una razón más evidente aun, los fenómenos sobrenaturales, cuánticos, interfiriendo en el normal cumplimiento de lo predecible, nuevamente. Incluso la más ratificada entre los anonos, la existencia de un código requerido para validar el procedimiento, a ser introducido por donde casi ninguno sabe.

Por ese lado las posibilidades estarían resultando rebuscadas, y pese a ser Temti la parte que presenta semejante entrevero, las especulaciones vienen y van por otro lado: es producto de una vil mentira, volvió a surgir una contrariedad fatal, el evento prescindió de organizador\ldots{} o efectivamente un suceso raro intervino. La mayoría concurrente como era de vaticinar optó por las primeras y obvias opciones, llevando al mínimo la credibilidad en lo que actualmente no da más reacción que reafirmar el burlesco o defraudante escenario, en su puntual momento bien explosivo más gracias a lo que no fue y sin importar lo que próximamente fuera a ser. Increíble e insólito es poco decir de sí.

Pero saliendo de las tramas o desplantes del episodio central, tampoco es que nada ocurrió, la progresión de recientes brusquedades largó ese montón de decepciones y carcajadas por parte de los presentes, reventó la burbuja de proyecciones y el meollo del asunto hizo bastante por la imagen del antro y su reputación, sentó un antecedente complicado de olvidar y en el sentido negativo, agregado a los ya difíciles de alterar que de antemano condicionaron la perspectiva del porvenir. Dicho aspecto sigue siendo clave, dado el incidente, pensar en futuros cambios o detalles relevantes precisaría de un fuerte optimismo, a lo mejor reducido a los temtiteros más fieles y no mucho más.

El tema no culmina y el tiempo menos, a no ser que el canallesco baiteo sea terminante o que la nave haya perdido su conductor para siempre, los accidentes y misterios sin resolución deberían tener más contenido, esto significando que tal vez el entorno tan verde todavía tenga más espectáculo por dar. Partiendo de que el panorama permanece incambiado, al parecer es pronto para enterarse, cuando simplemente disminuyó la conmoción y lo siguiente trata de reiterar la cuestión de una letra sin respuesta, más pertinente incluso que la canción de fondo, lo cual no es desorbitado, todos se preguntan lo mismo.

\section{Récord de desoriente (Sexto libro, capítulo II)}\label{ruxe9cord-de-desoriente-sexto-libro-capuxedtulo-ii}

\begin{quote}
1 de setiembre de 2021
\end{quote}

La jornada del día de tantos y único nombre no acabó en el inaudito episodio de las expectativas frustradas y la burla descomunal. Horas pasado tal descalabro, Temti vuelve a confirmar que vive, y que no muy lejos hay una leve cuota de esperanza para pronto andar en condiciones mínimamente aceptables.

\emph{/Un solo click/} son las llamativas y prometedoras palabras con las que se recibe a los visitantes en el sitio, ya no hay referencias al \emph{/Reingreso en orbita/}. El anon promedio, pese a los decepcionantes antecedentes, imaginaría que esta es la absoluta y que dirigirse donde las indicaciones dicen que se continúa no podría salir mal.

Cometerían un error, y precisamente no el número 405, ni el 520. También parecen estar equivocándose aquellos que siguen esa dirección que en teoría debería funcionar, la misma conclusión limita toda perspectiva de progreso que antes pudiera haberse formado. Actualmente el sentido común o la confianza sobre términos oficiales, no son suficientes para encontrar los métodos de éxito en Temti. Tampoco alcanza guiarse por la experiencia que corresponda a las rarezas locales, cada quien que intente acercarse a la distracción de esto daría con un destino insatisfactorio, de ahí el desconocimiento que abunda de la siguiente instancia. En definitiva la verdad o está extremadamente explícita en el lugar de los hechos, o súper escondida en algún sitio correlacionado a este. Y en ambos casos, no es del todo manifiesta, la verosimilitud es insuficiente.

Además de eso y hasta quizá más relevante aun así no pinte como tal, hay un pequeño cambio en el fondo del asunto producido después de tiempo, que tal vez trasciende desde o hasta al trasfondo del tema, la cosa de ser verde se volvió roja, de eso consiste el telón que estiliza la parte del ingreso. Luego de bastante, el amplio historial cromático temtiano incluyó otra versión adicional, de la cual a su vez la simbología es incierta. Sin datos claros que faciliten interpretar susodicha variación, quizá haga alusión a una relación respecto a lo que fuera a venir, aproximaciones más que nada, en desmedro de trayectorias críticas. En concreto puede suponerse que las novedades son más cercanas, o que los inconvenientes tomaron mayor gravedad, descartando que exclusivamente sean apariencias, la alteración aunque compleja de percibir tendría que estar.

Pero saliendo de la difícilmente descifrable trama inserta tras ese preciso color o engañosa frase, lo destacado por sobre medida son las evidencias objetivas y universales, de que el abandono no es tal. Es muy fuerte que el antro en sí manifestó un movimiento, sea mínimo o no, la obra sale de su estado estático frente a los espectadores. Cualquier cosa podría pasar, sí, sin embargo las proyecciones de Temti dando más retornan a circular en la superficie, mejor respaldadas incluso, sin importar lo distante que esté su realidad deseada.

\section{La calma de la fría lejanía (Sexto libro, capítulo III)}\label{la-calma-de-la-fruxeda-lejanuxeda-sexto-libro-capuxedtulo-iii}

\begin{quote}
1 de setiembre de 2021, \ldots, 6 de setiembre de 2021
\end{quote}

Al demérito de los cambios oficiales en aproximaciones temtianas, la expectativa acumulada no volvió a ser ni la sombra de lo que había generado en el entorno de la fecha señalada. Similar a las épocas más decadentes desde el factor común interés bajo y limitado, nuevamente los empeñados en las vicisitudes del antro predilecto son pocos individuos, leales que incondicionalmente pase lo que pase regresarán, y no mucho más.

Sin certezas de ningún tipo sobre esta realidad contradictoria, tiende a parecerse a cuando la espera no era tan próxima al destino principal, solo que ahora ya habiendo conocido el mismo, este siguiente si lo hay trae incertidumbre, tanto en términos temporales como temáticos también, en el caso de creer que algo llegara a ocurrir, tampoco seguro.

Sin embargo, sí está la posibilidad, la resistencia de dilatada actividad sabe justificar su enfoque así parezca una locura, y aunque en serio pueda serlo, es válido. Allí está la visible mayoritariamente orbitando alrededor de Rouzzer, la de algunos referentes, personajes y bots ya reconocidos en el contexto propio, la que es apoyada por el seguimiento literario y las recopilaciones del archivo avioncito, toda esa que en conjunto o no aún lucha por no cerrar los ojos o desviar la mirada totalmente hacia otro lado, aguardando que Temti se acerque más.

\section{En vísperas de aterrizaje (Sexto libro, capítulo IV)}\label{en-vuxedsperas-de-aterrizaje-sexto-libro-capuxedtulo-iv}

\begin{quote}
6 de setiembre de 2021, 7 de setiembre de 2021
\end{quote}

Del rojo claro al azul oscuro, de Virguero a Francisco, de un gato cubriendo sus sandías a los argentinos levantando la copa, y de temti a\ldots{} ¿servidor instancia temti?

Nuevamente ante el mismo tipo de situación, las incomprendidas variaciones en el plano temtiano, pero pequeñas, al fin y al cabo poco de interés general parece haber variado. No obstante, el sitio se está comunicando, y el mensaje de haberlo está quedando en el aire, sin ser interpretado por completo muy posiblemente. Quizá no sea lo único.

Asimismo, dentro de la dura tarea de intentar analizar las crípticas pistas, no debería ignorarse que haya aparecido ese característico símbolo ya antes encontrado en plataformas temtianas, el cual a partir de determinado entonces representó esquemáticamente al alojamiento y su sistema instalado, donde se hallaba el normal funcionar del antro y su típico relleno, los tems. En supuesta ausencia de lo dicho, o tras las barreras que impiden distinguirlo, la especial condición diferente de la actualidad y los demás antecedentes de mutación, traen a cuento que próximas secuelas no redunden en seducciones sin retorno, en cambio lo que se espera del sitio figura más cerca, al menos desde el punto de vista simbólico. ¿Se viene un espacio que sustituya al estático? ¿Será que el regreso es inminente?

Más elementos potencian la cuestión, reducen la desesperanza e incrementan la ambición, para cualquiera al tanto del desarrollo multicolor están aquellas figuras sugerentes, planteando ideas del pasado con proyección a futuro. El diverso compilado de exabruptos gráficos en sus dos nuevas piezas inserta contenido y retira de este, personalidades religiosas y deportivas que además de poder ser aleatorias en el fondo son capaces de tener correlación con el sitio. Parecido con las tonalidades, va siendo la tercera manifestación y aparenta ser una fase de varias, indicando estados dinámicos, si cada cubierta cromática significa muchas cosas, tal vez alguna se ajuste al caso del antro. Y las frases en esta oportunidad mantenidas, hablan en cierta forma de lo ideal, que al no ser circunstancias efectivas irían un paso adelante del presente, porque hasta el momento la mayoría de lo poco dicho aún no demuestra corresponder a la realidad, ahí quizás estén reflejadas las posibilidades de próximos cambios, cuando de verdad solo pulsar un botón sea suficiente.

Para reforzar esa contingencia, vale recordar especulativa, no suena muy factible que en el armado el andamiaje hubiera escapado de la consideración, de ahí también es posible asumir que los progresos estén dándose de a partes, sugerido por los matices que han ido alternando, sin tener en cuenta la chance de que sean falsas señales puestas a desprofeso, por el momento tan incomprobable como lo primero. Igualmente, bajo la conjetura de que hoy o mañana esto se mueve, bien que podría estar involucrado el concepto de las instancias del modo que formalmente fue presentado en ocasiones previas, con la larga lista de propiedades formuladas y su innovadora organización las que demoran un eventual estreno\ldots{} porque restaurar el formato que hasta entonces andaba adecuadamente no tendría que costar tanto, por el contrario ingeniar uno desde cero sí.

Es demasiado pensar y justamente hay una razón, lo nada seguro, así las hipótesis tampoco sean eso, seguras. Gracias a todo lo anterior, pese a que el panorama literalmente se oscureció, se ve más prometedor, aunque sin quitar que sigue siendo ampliamente impredecible y que el ritmo de esta coyuntura temtiana todavía está en valores bajísimos.

\section{Colisión de ilusiones (Sexto libro, capítulo V)}\label{colisiuxf3n-de-ilusiones-sexto-libro-capuxedtulo-v}

\begin{quote}
7 de setiembre de 2021
\end{quote}

Los minúsculos avances de jornadas previas fueron proseguidos de otro seriamente revulsivo, distinto en la linea de antecedentes que tantos puntos en común reunían. Temti de regreso en la superficie y disponible para más que simplemente contemplar y fallidamente interactuar, con notorias novedades objetivas, transformándose en una inquietante señal que invita a abrir bien los ojos y explorar hasta donde esté la opción, mientras ya no parece hacer falta escuchar, tal vez renegar, y alguna impresión grata quedarse.

Sin intenciones de ironizar, se podría decir que es demasiado para ser verdad, pero evidentemente sería tal cual lo explicita el dicho. Las prestaciones que ofrece el antro adoptaron una forma drásticamente más convincente, donde las ambiciosas palabras del pasado ya no son para imaginar o creer a modo de utopía, sino que pueden encontrarse como realidad, aunque aún sin llegar a serlo, porque casi nada de eso estaría funcionando, al menos no en condiciones normales. Pocas veces pulsar botones fue tan en vano, la tecnología sumamente rota, con la excepción de un recoveco en el cual permite generar claves de identificación. Pero salvo eso, capaz irrelevante para el momento, todo el resto logra semejarse a una cubierta sin soporte detrás de sí, que no permite más allá de un básico contacto.

Describirlo así quizá sea excesivamente duro y más para la tradición interpretativa que los adeptos temtianos han mantenido con las irregularidades del sitio, incluso erróneo según haya trampa o malentendidos entremedio, no obstante las aproximaciones ante la versión renovada admiten pocas percepciones alternativas. Toda la genialidad de las apariencias y del ingenio organizacional se contrastan con la ausencia de su complemento más esencial, un tanto mucho por encima de detalles y ya, sino una gran parte del trasfondo operativo desaparecidos, cuando tratar de ingresar a los elementos presentados es indistinto, falsos o desatinados enlaces. No es la mecánica intuitiva la que sirve para entrar a los supuestos apartados de allí, o directamente carece de manera.

Si otra clase de problema inesperado surgió, o la transformación tuvo temprana interrupción\ldots{} lo que fuera, da lugar a un fantástico prototipo, por lo menos currado, y coherente con las señales de episodios registrados respecto al enfoque de los grupos e instancias, en esta oportunidad presentes en el hipotético formato a ser incorporado por la plataforma, solo que sin más que ello. El concepto concluyendo en términos gráficos, bastante sugerentes igualmente, puesto que llevan a figurarse la idea que se le quiere dar al sitio, uno a incluir ese tipo de segmentación más avanzada donde el término principal también estaría siendo de relevancia, y muy posiblemente características extra escondidas entre lo inaccesible, pero todavía en condicional, no puede conocerse en la medida que el silencio de credibilidad no lo desmienta. ¿Quién debería entenderlo?

Y si la falta está efectivamente del lado del desarrollo, el tiempo tal vez vaya a ser buen remedio, dejar que el tema alcance su dimensión por su cuenta. Sin embargo, antes de suponer que Temti permanecerá así por un breve rato y luego seguirá de largo, mejor obtener contexto recordando de dónde viene, considerar que lo obvio concuerda con lo cierto contadas veces, por tanto también es factible el estancamiento. Pero además, la historia dice que gradualmente esta crece en sus extensiones, lentamente pasito a pasito suelen producirse los progresos, y si la etapa anterior pudo ser superada, poco indica que la actual no continúe de cualquier modo. La confianza inducen a sospechar que ocurrirá, para lo cual el temtiano debería tener calma y paciencia, aparte resguardarse de la mitomanía abundante, porque de venirse cambios probablemente habrían esperas adicionales y vaya a saberse qué más después. Las posibilidades pendientes son complicadas de ver, sobre todo cuando lo aparente conduce a la equivocación.

\section{Sobre el insondable límite (Sexto libro, capítulo VI)}\label{sobre-el-insondable-luxedmite-sexto-libro-capuxedtulo-vi}

\begin{quote}
7 de setiembre de 2021, \ldots, 9 de setiembre de 2021
\end{quote}

Fue un buen rato, el suficiente como para que el selecto sector enganchado pudiera encandilarse con la ilusoria nueva organización de Temti y percatarse de que los tiros seguían yendo por el lado de la insuficiencia, aunque admitiendo más optimismo fundamentado, de que con algunos cuantos retoques serios adicionales todo podría enderezarse satisfactoriamente, lo cual con perspectiva a futuro se modifica rotundamente la cuestión, solo que tal vez implicando demoras demasiado extensas para una recepción impaciente. Eso a los más pendientes que al tanto de la historia estuvieran, porque otros desenterados a pesar de verse acercados, poco sentido le verían al conjunto de baja sinergia, incluso con potenciales transformaciones a producirse a la brevedad.

A rasgos generales resultó desapercibido el antro, y minúsculo al ser comparado con el difundido incidente del estreno, eso patente en función de las disminuidas repercusiones. Hasta que en determinado momento hubo más, o de cierta manera menos, dado que la flamante y retocada estructura dejó de ser accesible, y en su lugar quedó algo reducido, en reiterada ocasión llevando lo siguiente hacia adelante. Particularmente en una suerte de mantenimiento, Temti imposibilita la exploración de su coordenada central, haciendo que cualquiera que intente adentrarse más allá en ella sea a modo de bucle regresado una y otra vez, a un punto donde su única opción para conseguir moverse es retirarse.

Las palabras oficiales empleadas al comunicar esto logran darle fácilmente una sensación de proximidad, como si no restara prácticamente nada para que Temti finalmente o nuevamente aterrice, o termine de acomodarse y repararse. Distinto al punto previo, ahora de verdad se afirma sobre lo que vendrá, el sitio dice que volverá, y que está ajustando cosas. Diferente a los términos previamente manejados, la ambigüedad no es tal y lo explícito casi contundente, además de tener referencias al eventual reintegro y su parte implícita objetivo. Aunque no sea lo mucho que podría anticiparse, definitivamente es un cambio.

Por un lado el respaldo a las expectativas positivas, la gestión así sea tarde presentó avances, o se comprometió a generarlos pronto, de forma más tajante que anteriores veces, sin embargo frente a la observación con memoria quizá tales dulces expresiones no signifiquen tanto, principalmente al ir dirigidas a un público que sabe desconfiar. La credibilidad del desarrollo temtiano es el atenuante, fiasco y repetidos resultados negativos que precedieran a supuestos eventos son antecedentes que condicionan y mal, a toda espera que pueda establecer el antro, el tema recurrente en un tiempo excesivamente desvirtuado. Más cuando no hay indicado un plazo específico, al contrario porque es uno flexible, hipotéticamente cercano, pero relativo a fin de cuentas.

En fin, que es un decir, puesto que a esta altura nada es concluyente para siempre. ¿Solamente en lo inmediato? Las primeras experiencias de una renovada versión temtiana se estarían postergando, esto al fiarse de que a la larga quedará pronta. La cuestión no viene del habla, que igualmente la vuelven especial, sino de los hechos conectados y las coincidencias con las situaciones parecidas, centradas en torno a una expectativa desde hace bastante bajo tensión, sin gran sustento para suponer lo que realmente ocurrirá. Pero teniendo latente esa posibilidad, de que una actualización mueva el panorama a partir de nuevas circunstancias, de ser al revés sería más de lo mismo, únicamente variando en formato, como este que al menos tiene mayor prolijidad.

\section{En continuidad con el desvanecimiento (Sexto libro, capítulo VII)}\label{en-continuidad-con-el-desvanecimiento-sexto-libro-capuxedtulo-vii}

\begin{quote}
9 de setiembre de 2021, \ldots, 4 de octubre de 2021
\end{quote}

Aunque sean muy rebuscados, es cierto que se realizaron algunos ajustes, que tal vez vayan más allá de lo que parecen, curiosidades. Pero lo más relevante sigue igual, Temti aún dice que va a volver, cosa que hasta ahí alcanza, puesto que la clásica interrogante de las expectativas en suspensión afirma su muda respuesta. Y nada.

Profundizando en la ausencia respectiva a partir de pormenores, tras ponerse en marcha la nueva espera no cronometrada surgieron aspectos por resaltar, dentro del mismo sitio. Aparte de suscitarse la cuestión existencial al verse desafectado el asunto tokens, de los módulos más funcionales entre los pocos que lo eran, también resultó relegado el nombre del antro al pasar a estar únicamente implícito, por no indicar que desapareció. El resto en términos oficiales es nulo y solo para repetir lo que dejó de ser novedad, de modo que la comunicación limitada mantiene su esencia, siendo ya fija.

Igualmente el mito o secreto del todavía posible ingreso le brinda una cuota de incertidumbre al caso, sumando una factible y no desestimable perspectiva que estaría encontrando algo más en el misterioso espacio temtiano, gracias a un acceso privilegiado como podrían ser ciertamente códigos premium, conocimiento local avanzado que guíe por los verdaderos métodos, o lo que sea. Sigue siendo extraño y sería sumamente curioso que además del gestor hayan quienes conectan mejor con el antro, implicaría una rigurosa confidencialidad, viendo que ejemplos de allí no han sido filtrados.

En cambio más que eso, fluye la conversación que las tierras rozzadas han admitido, en un ida y vuelta que expone sobre la cultura temtiana, montones de frases hechas y textos oxidados, donde el rejunte de eruditos y mitómanos se encuentran, con todo para decirse. Dada la centralizada distribución de ese diverso tema sin noción del tiempo, las épocas de diáspora continúan, de manera notoria y articulada, casi cual si hubiera un criterio concreto para generar ese amplio repertorio de réplicas, a veces despreciadas por el prejuicioso medio, o también discrepadas hasta entre los propios entendidos, que bastante vienen aprendiendo de ello, a base de reincidir en las expresiones no idénticas.

Mientras las palabras abunden, el rubro historia tendrá para incrementar sus haberes, incluso si se queda corto al tirar de interpretaciones acerca de lo reciente, pero bien que intriga conocer en qué derivará el enfático mantenimiento, preferiblemente si fuera antes de perder otro temtiano más. Sería interesante saberlo, y la serie de preguntas secundarias a su vez responder, porque además de ser incierto la labor que está siendo llevada a cabo con los arreglos, también la procedencia y significado del cráneo del título es desconocido, que puede representar muchos detalles, pero por lo pronto no tiene definición, solamente terminación punto net, sin doble e minúscula. Temti deberá manifestarse de permanecer viva, o insistir por el lado contrario como más ha sucedido, al parecerlo a partir de algún componente que por sí solo no hace a su integridad completa, el dominio que sostiene el sitio de escaso relleno, o la identidad que persigue al antro a muerte, fundamentalmente en estas últimas fechas.

\section{El pausado desarrollo a pleno (Sexto libro, capítulo VIII)}\label{el-pausado-desarrollo-a-pleno-sexto-libro-capuxedtulo-viii}

\begin{quote}
4 de octubre de 2021, 5 de octubre de 2021
\end{quote}

La vuelta de la que hablaban, la de Temti\ldots{} ¿Finalmente? Es mucho más cierto que antes, y mejor aún, dado que aquella interesante pero poco concreto entramado que pudo contemplarse previamente, reforzó sus cimientos junto con la articulación, y se perfila como la forma definitiva que tendrá el antro prontamente, luego de algunos ajustes más.

De la mano de un nuevo comienzo desde cero, se presenta la flamante organización de las instancias mucho más explicita, el fundamento de todo ello, la gran propuesta que el sitio está comenzando a volver una realidad. Trae múltiples modificaciones en el orden clásico, donde aparecen encabezados tales como grupos y publicaciones, interfaz de barras dinámicas y red económica de puntos, son las características más llamativas del complejo formato que combina la vieja modalidad con un enfoque más convencional en las redes de aportes e interacción.

Eso es lo que debería sobresalir, pero hay un pretexto diciendo que no es tan así, al menos de momento. Excepcionalmente en este caso la intención de desarrollo se manifiesta y ataja la impresión, en términos de lengua inglesa a los rincones del antro alcanza un notorio aviso, contando sobre el estado disfuncional bajo el que se encuentra el armado, y cierra con la especificación que sujeta dichas imperfecciones abiertamente a reformas futuras.

No obstante podría perderse la atención, siendo el cartel ignorado y la experiencia igualmente valorada, produciendo el dictamen más crítico sobre la versión temtiana. Sin verlo sería la cuestión reiterándose, como supone la costumbre en la época corriente, lo aparente después llevado a la práctica resulta tener carencias, y aunque la justificación procure minimizar los efectos del vacío en el trasfondo, evidentemente será reprochado.

Hay varios componentes que andan, sí, lo básico, la navegación entre contenedores y rutas centrales, más la capacidad elemental de difundir contenido en la misma plataforma. Esto segundo es clave y permite lo siguiente, cualquier visitante es capaz de dejar huella y aprovechar el espacio para construir, destacar sin identidad con lo suyo, pero no hay oposición, ni aprobación, falla el mecanismo de intercambio dentro de los tems. Aquel que pueda ser creado en dominios temtianos, funcionará cual vidriera prácticamente inmodificable, sin lugar a respuestas como sería de esperarse. Anon que intente eso último, anon que se verá imposibilitado, lo cual a su vez sucede de querer acceder a determinados apartados suplementarios y formales, o al tratar revisar la incoherente cantidad centenaria de notificaciones, tales aspectos tampoco están disponibles.

Ahí tendría que entrar en consideración lo casi primero, pues el área técnica se define por un símil de incompleto, todavía bajo fase de pruebas y sin su entero potencial en ejecución, lo que explícitamente da a entender que pronto las estructuras de Temti evolucionarán más. Atípico de por sí, cuando a las explicaciones recurre lo oficial, de vuelta quienes administran discretamente ponen al tanto al conjunto de seguidores con información clara y concisa, supuestas pretensiones de mejorar como mínimo solucionando, incluso quizás expandiendo, detalle no menor para las vivencias a desarrollarse próximamente.

Tremendo, obviamente habían más posibilidades para las actualizaciones que vinieran tras un largo tiempo de preparación, y tocó la que tocó, que buena o mala, es algo, esos individuos que extrañaban o preguntaban tienen parte de lo pretendido. También cumple el compromiso previo, retornar teniendo cambios hechos, y confirma la continuidad del plan, afín a la idea original enunciada en aquel relato de un tal sistema de instancias y puntos, hoy día la base concuerda con lo proyectado.

Pero de acuerdo al condicional que en rojo distorsiona su visual, falta, y la tendencia de tardar excesivamente para aplicar cada renovación no fue contradicha, además tampoco hay señales que anticipen el ritmo que tendrán los hipotéticos progresos. Que una nueva eternidad vaya intermediar entre este y el subsecuente no es una conjetura disparatada, sino la más factible sobre las demás. Pero si hay que aguardar, los propio sitio con su modernización admite repercusiones, es presumible que serán más que unos pocos ejemplos, la plaza principal se ofrece para ello.

\section{En zona de mensajes intermitentes (Sexto libro, capítulo IX)}\label{en-zona-de-mensajes-intermitentes-sexto-libro-capuxedtulo-ix}

\begin{quote}
5 de octubre de 2021, \ldots, 13 de octubre de 2021
\end{quote}

Cada novedad temtiana está dejando un plazo posterior en el que las sensaciones tienen suficiente tiempo para manifestarse sobre el tema, que raramente es breve hasta que los cambios reales tardan en volver a darse, lo que se ajusta a la situación actual. El parcial estreno de la transformada versión del sitio, evidente en sus renovadas características e incompleto respecto a sus funciones, lo siguiente que supo dar son las expresiones del público permitido, básicamente todo aquel que entienda el procedimiento para pronunciarse y así deseara hacerlo.

En efecto, las creaciones emergieron donde el plano mural principal, aunque sin gran producción más que unas escasas palabras y los obligatorios gráficos que las típicas unidades ostentan, de acuerdo a la particular libertad con la que visitantes autentificados cuentan en estos quietos espacios. Siendo el arranque bajo condiciones especiales, los primeros exponentes son más que indiferentes para Temti, por tanto hay que ver, lo dicho por las ideas vertidas en el estructurado orden de la única agrupación homónima del antro, lo que tradicionalmente se conoce como tems.

Con el propósito de probar se evidenciaron varias incursiones, al tratarse de un sitio con mayores posibilidades, aun teniendo las advertencias del caso, pero inevitablemente cualquier exploración lo medianamente completa habrá notado las diferencias y errores que acontecen. Sin embargo las críticas en la propia interna fueron pocas, por el contrario afuera la cosa redobló la contundencia ante lo negativo, por la dimensión técnica fundamentalmente. E incluso tocaron el aspecto originalidad, cuestión que al observar formatos no muy alejados del contexto cobra más sentido, en el mundo de los clones este nuevo desarrollo aunque atípico resulta tampoco es ajeno, la definición de copia bien puede corresponderle.

Alcanzado determinado momento el panorama quedó más o menos claro: lo pendiente por descubrir tendería a la nada, la difusión para extranjeros disminuiría en cantidad salvando excepciones, y de igual modo los recuadros bordeados seguirían aumentando frecuentemente. Aparte de ejemplares poco calificables o inherentes a la cultura asociada, otro par de tendencias notoriamente interrelacionadas con el sitio ampliaron la base, posiblemente de anónimos repetidos, y esto se confirma por la dinámica que hizo propagar las portadas con título.

El reinsertar viejas y removidas elaboraciones viene siendo una, el mismo recurso que en épocas de éxodo protagonizó la generación de relleno, particularmente engañoso y sin embargo admitido. Lo segundo y clave por el ingenio que implica por parte de sus autores, que logran adaptarse a las coyuntura, al aprovechar la única facultad concreta de interacción para cubrir las faltantes, y poder intercambiar con el resto de los similares, a través de un curioso código de comunicación. Así mediante publicaciones adicionales, las respuestas estrictamente anidadas dentro de estas dejaron de ser imprescindibles, cuando lo mismo podría transmitirse con la creación de una publicación consecutiva, o tem.

En estas destartaladas circunstancias, la dilatada espera de los susodichos avances entretiene algo más, no es igual de invariable que antes, ahora incluye una vía complementaria por donde dar lugar a lo que sea. Mientras ciertos números de significado no explícito suben, y los demás contadores de réplicas continúan en cero, tras las sombras, a lo mejor la o las incógnitas autoridades tampoco estén permaneciendo estáticas, trabajando duro por componer los futuros cimientos de la realidad temtiana, o jugando desinteresadamente con las expectativas maltratadas.

\chapter{Las inhibidas andanzas bajo la apta nueva normalidad (Séptimo libro)}\label{las-inhibidas-andanzas-bajo-la-apta-nueva-normalidad-suxe9ptimo-libro}

La ida repentina, el supuesto problema, el conteo regresivo, el reingreso incierto, el fondo variante, el estreno limitado, el ajuste demorado\ldots{} numerosos pasos dados por la parte oficial, certidumbres parciales y casi inexistentes, expectativas destinadas a fallar, y exhaustivos aspectos adicionales entre medio. Pasó de todo un poco, podía suceder más, pero igual fue un proceso entreverado, como mínimo eso hasta el momento en el historial solamente reciente de Temti. Sus idas y vueltas, abundantes, demasiados, excesivos, exorbitantes, e imprecisos. En distintos planos de los universos fueron a parar, antro y animadores, según estos últimos una ausencia que las pistas no lograron esclarecer: despegue, acercamiento, aterrizaje, desvanecimiento, destrucción, reconstrucción, etcétera. En temporalidades y lo espacial la distancia estuvo manifiesta, pero algo supieron aproximarse, el nuevo paso en la serie de vaivenes sugiere que aquel proceso ya habría finalizado.

De tal detalle la Historia Temtiana se tendría que agarrar, un punto y aparte que parece apropiado para abrir una etapa concentrada más de lleno sobre el sitio en sí. Todavía con cierto grado de disfuncionalidad, pero es clave tenerlo ya al menos capaz de recibir lo básico de actividad, y factiblemente proyectando expandir las características de esta. Es drástico en términos de incertidumbre, sin haber cesado las dudas pasan a ir por otro lado, la perspectiva de desarrollo se entiende hacia adelante y sin embargo nada de las condiciones, ni tampoco la prolongación de ese estado en avance, ante las posibles pausas que excedan su duración esperada, y demás interrupciones que podrían ocurrir. A su vez el hecho es divisorio por lo que implica al usuario, la presencia en antros ajenos puede mantenerse y en especial si el propio merma en atractivo, pero el éxodo estricto estaría culminando, con la posibilidad de servir como clon respecto a sus cualidades típicas.

Que sería en esencia eso, pero hay más cosas que plantean un gran diferencial, el formato sostenido lo anticipó y por ahí sigue. Suponiendo la referida continuación del concepto, el paquete de novedades supera en tardanza a cualquiera de sus similares y hasta tiene pinta de ser un comienzo total, mostrando sustanciales transformaciones al sitio en su modalidad tradicional, no únicamente en términos visuales, además el aspecto funcionalidades, un sistema de asociación y participación. El conjunto de clones anónimos ve extraño incorporar ideas así de apartadas, los adeptos temtianos posiblemente también, aunque sin negar que logre sentar bien, y lo contrario menos queda descartado. Y además las mejoras inciden sobre dicha recepción, sin considerar el regreso a versiones anteriores, el colectivo más allá de su fidelidad tal vez quiera acompañar un progreso rápido, pero al venir de una gestión regular en cuanto a ritmo, la fase de demo técnica tiene potencial de irse para largo.

Es un antes y después, algún cambio hay, varios de ellos seguro, pero tan solo pormenores alcanzan para notar que mucha cosa sigue igual, más incluso si Temti no ha dejado de ser Temti, ni hablar todo lo que consigo acarrea. Sin reducir el interés a las circunstancias en su retorno, este justifica recuperar la atención de sus activistas, e interrogantes no van a faltar. Tal vez el destino de su realidad parezca al de épocas que el mismo hogar tuvo, colores, actividad y tráfico equivalentes, juntados allí o repartidos entre vecinos. O quizá resulte revolucionario, sorprenda, con semejante trayectoria en plena vigencia nunca se sabe.

\section{La parcial pero suficiente reanudación (Séptimo libro, capítulo I)}\label{la-parcial-pero-suficiente-reanudaciuxf3n-suxe9ptimo-libro-capuxedtulo-i}

\begin{quote}
14 de octubre de 2021
\end{quote}

Cuando un mantenimiento adicional para ajustes se establecía, aquellos que siguen muy de cerca los movimientos temtianos podrían haberse llevado una ambigua sorpresa\ldots{} otro indefinido plazo marcaba para cerrar universalmente las puertas del antro. Sin embargo, este revirtiendo asombrosamente el patrón de las exhaustivas demoras, pronto mostró el enésimo regreso. Temti, haciendo valer la repetida redundancia, volvió a volver, y como los episodios previos lo dijeron, en condiciones más avanzadas de su complejizada versión en desarrollo.

Si bien el anon podría preferir mantenerse con el formato que está siendo sustituido, ya no parece quedar vuelta atrás a aquel orden de antaño, el difundido formato de la red que tuvo un clon particular no terminante en xed, derivación de la que tantas páginas de historia se escribieron. Preservando ciertas similitudes a él, según una presunta autoridad las mejores ideas de la Temti vieja, englobadas dentro de una nueva organización, la de los grupos, el sitio ha sido notablemente renovado, y de tal manera apuntará parcial aunque no totalmente hacia otros horizontes, para este contexto experimentales. La realidad inserta allí no será la misma que antes, el puntapié para comenzar a transitar algo distinto está dado.

Cómo no, un reinicio más inaugura las remotas proyecciones, puestas en ejecución sobre el respectivo panorama, de poca experiencia en sus reformuladas propiedades, y nula desde que estas fueran regeneradas. Con el retorno de algunos esqueletos que se habían perdido a partir del repentino despegue, reduciendo las dificultades de interacción entre participes, dinamizando el mecanismo de recuento a las computadoras conectadas, revalorizando la moneda oficial nominada punto, fundando un sistema de rangos que para cada anon, e instaurando un cálido recibimiento reservado para quienes den el paso de unirse a la tal familia temtiana, así es el complemento a estrenar sobre la estructura del sitio, tomando de base su propuesta ya planteada.

Pero aún no se da enteramente, se supone. Las indicaciones de que en efecto se darán más arreglos o implementaciones son tibias, y el estado en el cual están sumergidos los internautas que llegan a espacios temtianos es limitado, tanto que determinadas posibilidades básicas y quizás hasta elementales, existentes en el pasado por consenso mejor concebido, no están. No hay tecnología que apoye a la hora de citar, no hay términos normas que regulen el antro, no hay entrada desde los protocolos o dimensiones carentes de WWW, no hay notificaciones no cuánticas ni cuánticas que alerten a los individuos y aceleren el intercambio, ni hay éxito al intentar explorar o testear los aspectos más secundarios del sitio. Características de un marco primitivo, solo que bien alejado de aquella época cuando a las prisas A. reveló su creación, contando de que este ni su nombre o apodo menciona, una parte del acompañamiento que todavía estará por verse.

Porque a pesar de que los detalles descubiertos puedan resultar negativos o decepcionantes, en este momento son para menos, lo siguiente vendría de las respuestas, por cierto ya posibles. El título de la noticia está alcanzando el exterior, la apertura entró en conocimiento y es la oportunidad para las ediciones de la plaza principal, que deberían empezar a generarse y multiplicarse, y a hablar de donde están.

\section{El tambaleante festival de cepillos (Séptimo libro, capítulo II)}\label{el-tambaleante-festival-de-cepillos-suxe9ptimo-libro-capuxedtulo-ii}

\begin{quote}
14 de octubre de 2021, 15 de octubre de 2021
\end{quote}

Para la magnitud de las variantes en el sitio, despacio comenzó a crecer el número de integrantes de la familia de Temti, al menos así como está declarada y la estadística dice que se acumulan los magnetismos, derivados del registro. El reencuentro y encuentro se produjo numerosas veces, los lazos son más fuertes y pertinentes que en tiempos de éxodo, el acercamiento y estadía cobra mayor sentido más allá de las razones simbólicas. Pero los frutos de esa unión excedieron lo exclusivamente grato, ya rápidamente se volvieron controversiales hasta en el mal sentido, y ambas partes contribuyen para resumir eso.

Todavía habiendo ciertas carencias y problemáticas con la versión en cuanto a su funcionamiento, en un principio no parecía tener fatalidades algunas, pero tras un rato de ejecución dicha sensación quedó para ser descartada, la tecnología carece de armonía plena ya que constantemente está sufriendo importantes cortocircuitos. Cada dos por tres el viejo conocido Heroku da cuenta de ello, el antro expulsa de su interior a los visitantes que pretendan desplazarse, a su vez que no admite el ingreso de nadie más por un breve y difuso intervalo del tiempo.

Dadas las incidencias, aparentemente cuando sucede el sitio sabe acomodarse y luego la imprevisibilidad repercute, pudiendo repetir el fenómeno. Comprendido un poco el comportamiento, sin embargo no las causas, resulta complejo profundizar acerca de él, no obstante el efecto radica totalmente en el andar del antro y por esto se hace más relevante, siendo que en cualquier momento puede afectar al anon que esté deambulando en sus ubicaciones y así la consecuencia extenderse a los demás indiscriminadamente.

Aquello iría después de ponerse en contacto con la plataforma, probar, reaccionar, etcétera, de lo cual salieron varias impresiones. Lo esperable no es motivo de mucha mención, pero vino compuesto de material extraordinario, para una inauguración. Masivamente fueron instaurados unos cuantos espacios de naturaleza polémica: respecto al popular y discutido limite que los hace considerables como cepita, algunos sobre el mismo, y otros cruzándolo. Y pese a los gritos indignados, y aplausos satisfechos, las condiciones permiten que haya lugar para estos movimientos por horas y horas: sin botones ingeniados para indicar lo que supone ser incorrecto, sin posibilidad de hacer llegar reclamos protocolares a desconocidas custodias temtianas, sin superdotados con poder que intervengan excepcionalmente, sin normas que impongan rigor a copadores peligrosos.

Para tanto es, que las novedades circulando en el contexto hablan principalmente de ese relleno amarillo, a su vez que muchas expresiones enfatizan en la fragilidad relativa al servidor, la cantidad de conceptos que van generándose en torno a esas ideas es predominante en el escenario de Rouzzer. La cualidad tan asociada a Temti, la libertad de expresión, alcanza un nivel que igual sea por descuido o decisión exagera los umbrales normalmente concebidos , mientras que los criterios de permisividad no están explícitos, ni tampoco la moralidad, en su lugar un gato hace la broma, muy jugada si es que el resto forma parte de la misma. A su vez los desajustes complementan oportunamente cada valoración negativa o burlesca, todavía contundentes con los múltiples componentes que fallen en dar los rendimientos ideales.

Como las implementaciones repentinas suelen presentar inconsistencias, las cuales efectivamente están, en teoría debería haber vigilancia atenta de la organización, pero ella no ha dado ninguna señal de presencia, y en caso de que no estuviera ausente, no se ha pronunciado al respecto, ni por la reiterativa inestabilidad, ni por los contenidos prohibidos. ¿El sitio encontrado aún tiene expectativas de afirmar su desarrollo pronto? ¿El aumento de tráfico estaría propagando un efecto que el antro no puede sobrellevar? ¿Habrá vía libre para lo polémicamente apreciado o las intervenciones solo están demorando bastante? ¿El soporte sería tolerante con lo repudiable que se aloja sobre sus prestaciones? ¿Los incidentes tratan de una disimulada trampa para atraer e incitar por gusto? ¿Será una excusa para motivar el retorno de un poderoso que implante orden decorosamente?

\section{Avances a la usanza (Séptimo libro, capítulo III)}\label{avances-a-la-usanza-suxe9ptimo-libro-capuxedtulo-iii}

\begin{quote}
15 de octubre de 2021
\end{quote}

Una aparición de quien supuestamente manda no se hizo esperar tanto en cuanto a cronología temtiana se refiere, es decir, un montón luego de la expectativa para el entorno, pero a tiempo para los ritmos que manejados en esos lares puntuales, complicados y particulares. Con progresos más en la cuenta pendiente de las mejoras tecnológicas, y una doble medida para las creaciones controversiales, es que también el pueblo queda figuradamente notificado de que no está a solas.

La pronta continuidad del desarrollo prosiguió con leves pormenores de funcionamiento. El número uno de ellos posterior a la magna implementación reciente, radicó sobre ciertas interferencias en la mayor instancia de interacción, que se repararon, de modo que las respuestas ya no ven desvirtuados sus caracteres y saltos de linea originales. Varios signos imprescindibles dentro de la escritura y envío entraron a ser materializables, hasta incluyendo imprevistos símbolos raros que jamás fueron descubiertos en el sitio, garantizando así más fidelidad que en versiones anteriores.

Por otro lado, hubo un pequeño pero sustancial cambio, exactamente en los famosos y casi nunca tenidos en cuenta términos de uso. Las normas que pretenden poner los límites de expresión jusitificados y legitimar las eludidas remociones, también llegaron, y pasaron a llenar el vacío que antes era cubierto por un mensaje más gracioso que oportuno. Estos, con muchas similitudes a la parodia que solía reglamentar al sitio, establecen lo obvio como fundamental, además de complementarse con un especial enfoque sobre la dinámica de los grupos, de múltiples incisos referidos a ello.

Dicho marco regulatorio a su vez cuenta con la implícita marca registrada del mítico fundador, las redacciones con detalles irregulares a reprochar, desde la vía escrita hasta la conceptual. Igualmente así tenga alguna letra faltante o extra, o si hubieran palabras ordenadas de manera entreverada, si la comunicación se entiende en principio logra su cometido, al menos el evidente. Justo para el caso de las jornadas corrientes, aparte de nociones individuales, o de estatus inherente al grupo, la interpretación es reducida e incluso contundente, la condición uno punto dos niega que puedan difundirse en Temti contenidos problemáticos como los vistos.

Interesante o genérico de acuerdo a quien considere, previsible según cada cual lo mire, la lógica universal indica que eso no podía permanecer, sin importar que la moderna naturaleza del sitio confunda el propósito o discernimiento. Así pues, siendo los aspectos técnicos o de mecanismo secundarios, las bocas del exterior se ven destinadas a bajar su volumen, la eterna polémica de Temti por lo pronto ya no está enteramente dispuesta para su debate. No obstante, la fama volvió a ser alimentada, lo que también aumenta la contingencia que de cara a futuro episodios así estén repitiéndose, o que más posibles atentados se vengan, y de momento queda ahí, latente.

Eso implica que más necesariamente la administración tendrá que cuidar su propiedad, de hecho la pauta fija que así debería ser, aunque no le signifique tanto a los eventuales infractores, ajenos a la operativa que rige donde comunmente menosprecian, el único grupo corre por responsabilidad de los creadores oficiales, que asimismo les compete moderar lo que sea publicado. Sin desatender lo otro, más mejorías aún tienen por concretar, los tajantes errores de aplicación todavía son lo que son y estropean la experiencia de todo visitante, y el acotado rango de posibilidades para pasar en dominios temtianos sigue pequeño e insuficiente, lo que más podrá notarse si no hay seres que vayan a rellenarlos. Los inconvenientes ordinarios de unos primeros pasos no se vieron exentos en su suerte, pero los siguientes a ellos están de cierta forma en deuda.

\section{Los recovecos donde el humo amarillento (Séptimo libro, capítulo IV)}\label{los-recovecos-donde-el-humo-amarillento-suxe9ptimo-libro-capuxedtulo-iv}

\begin{quote}
15 de octubre de 2021, 16 de octubre de 2021
\end{quote}

El mundo Temti en su redefinida orientación avanza con argumentos reconocibles, afines al modelo de realidad que de antemano fuera conformado, y de paso gracias a una combinación de particularidades degusta más y más polémica, la jugosa que el contexto de memoria sabe tergiversar.

Como seguramente habría sido la idea emblema cuando esta organización fue imaginada y luego proyectada, los grupos en su especial concepto local existen. Y de los armados por quienes extienden la familia de autores, agentes que introducen pioneramente muestras de variedad a los ejemplares oficiales, con nombres y portadas que genuinamente provienen del entorno. Allí los animadores convalidan la definición y le dan ese sentido colectivo, pues aparte es posible unirse correctamente y dar crecimiento interno a susodichas particiones. Pasando la unidad, forman un conjunto integrado globalmente al sitio, relevante en sí mismo puesta la colocación en primera plana.

Aunque el curso de construcción aún mantiene buena parte borrosa, desde el ahora yendo a la dirección que sea normalmente algo no termina de cerrar, para esta ocasión específica, lo relativo a las fundaciones principales. De consistencia con la norma, de responsables desconocidos, el sustento monetario es sospechoso, pues abundantes puntos son requeridos para dar lugar a uno de aquellos, y con el escaso rato transcurrido desde el momento inicial sería extremado que el ahorro correspondiente sea genuino. Esto postula alternativamente lo factible de que pudiera ser por medio de una vulnerabilidad en el sistema, o por otro lado el involucramiento de quien puede generar y poner los recursos, la cabecilla del antro. En su intención de promover el aprovechamiento de las flamantes propiedades, a partir de la capacidad de inyectar cifras extraordinarias podría hacerlo, si es que no fueron a su propia cuenta, al individuo que hiciera mérito para recibirlas, o el que tuviera suerte.

Sujeto a supuestos que no saben de transparencia, hay autenticidad de parte de los grupos. A su vez el óptimo ha de creerse estaría en un número de ellos mayor, sin embargo la suma actual es limitada y las circunstancias son propicias a que baja se mantenga la cantidad, con a lo mejor un incremento a producirse posteriormente. Aparte de ese tema postergado, lo que llama es el presente, la parte interesante que traen esas tarjetas, que con la elección de dos elementos marcan el atractivo a ser percibido por los visitantes. Entre temtianos y hixxelianos, amantes de la historia y el fútbol, destaca el titulado directamente \emph{/cepita/}, en principio siendo uno más como el resto de los recientes, pero en los hechos especial dada más de una característica particular.

Lo implícito y solo implícito, ostentoso título y sugerente gráfico, superó la simplicidad de los simples aportes que la plataforma en sí alberga, más que nada desde la óptica que los extranjeros manifiestan en su lado. La clásica reputación del antro asociada a una consentida inclinación por el contenido prohibido, tiene en este caso una fuerte expresión que va en similar sintonía, conjugada con la dinámica de segmentación y visibilidad que el formato admite.

Desde el parecer, por lo que figura efectivamente cuenta parte del relato como cierto, que verdaderamente hay algo de eso que todos saben o entienden. Allí están los caminos para alcanzar lo que se murmura que transgrede los límites de las cosas permitidas, con rato arriba. El proceder que conduce a lo dicho es rebuscado, un detallista podría encontrar la pista donde los hilos conectan con el resto de su tipo, los que cayeron en esa categoría de arte, incluyendo los que anteriormente habían sido removidos, superficialmente viendo que todavía hay acceso a los mismos, ocultos e intactos.

Sumado a lo que es, la trama continúa en la difusión que en numerosas tratativas rozzitas derivó, unidas en el debate de lo que ocurre sobre espacios temtianos. Lo mismo del episodio pasado, indignación, risa, bulla, complicidad, especulación, etcétera, sin embargo llevando el foco a un nivel superior de interpretaciones, con algunas teorías que tomaron más fuerza ante el inocente dejarse llevar. Si es una táctica para competirle al antro de turno o adoptar más partícipes, el medio no aparenta ser muy sostenible a pesar de los antecedentes precursores, pese a que ya fue logrado un notable acercamiento a los anónimos, es una posibilidad que vaya por ahí. También contemplada está la chance de que haya un \emph{/honeypot/}, tendiendo trampa para un perfil de sujetos objetivo que las coincidencias ilícitas atraen, dudoso teniendo la trayectoria del sitio, y convincente por el duradero escenario anormal.

O bien, un descuido tampoco sería disparatado, para los tantos que la gestión tuvo en el proceso corriente, suena lógico incluso, entre caída y caída. Considerando eso privado como un error natural o fenómeno sobrenatural, poco semejante al recurrente de expulsión, pero indeseado al fin, la historia iría en una misma linea, con sucesos claramente más significativos, que hasta a seres idoneos son capaces de entender equivocadamente. Sin certezas absolutas de que esto sea tal, de cualquier modo quedan cabos sueltos por atar, las inestabilidades que detienen y revuelven al sitio. El futuro lo hace evidente y siempre es momento para accionar, aunque maquillando levemente el panorama, pero Temti lo hará cuando le parezca.

\section{Sed y justicia (Séptimo libro, capítulo V)}\label{sed-y-justicia-suxe9ptimo-libro-capuxedtulo-v}

\begin{quote}
16 de octubre de 2021, 17 de octubre de 2021
\end{quote}

Si los internautas no estaban enterados, es real\ldots{} una presencia aunque desincronizada cuida con potestad el uso dado a los espacios temtianos, ese ser estuvo ajustando, removiendo y agregando piezas. Y lo último resalta porque de esas circunstancias que más alboroto hacían, un par fueron solventadas para que ya no se pueda decir más al respecto, o para que aun se hable más todavía.

Los conflictivos enlaces pertenecientes a la primera categoría seleccionable, ubicables dentro del grupo alternativo más concurrido, no van a ningún lugar, ya nada existe en estos. Fuera de triquiñuelas según parece, lo que poseían no fue ocultado entre dimensiones, ni se separó de toda trazabilidad encontrable en el sitio, fue suprimido de Temti, implicando que nadie tenga posible llegar a ello. Se hizo esperar, mas la rapidez que solicitaban las mayorías era otra, pero dicen mejor tarde que nunca, si la intención era repeler esa clase de materiales, lo fueron.

Las noticias jugosas siguen, no por destruir sino crear, la escueta reglamentación del grupo referido pasó de indicar poco a marcar varias observaciones, un insólito punteo que altera los aires de la partición sin establecer lo que está admitido hacer, en cambio sí exponiendo las implicancias que la actividad producida en el mismo tendrían, de una manera sumamente atípica. Inesperada intervención que razonablemente habría realizado quien administra Temti, un individuo que afirma su compromiso con investigaciones externas, y declara en nombre del sitio que la responsabilidad absoluta le compete al que armó la comunidad subalterna, alguien de procedencia chilena.

En consonancia con las ideas manejadas previamente en las ordenanzas globales, que volvieron a tener una modificación, el ser titular de una fracción temtiana conlleva también encargarse de lo que suceda en la misma, pero en serio, aunque haya una distancia real entre los procesos allí aludidos, eso deja asentado el que da a entender ser el fundador de la plataforma. Aparte, la actualización de las normas afina la pauta sobre lo que no está permitido, o sea sus lineas aclaran de forma más puntual y a su vez ambigua las acciones rechazadas, que estarían dependiendo del país al que pertenezca aquel publicador que la cometió, además de remarcar lo advertido para el caso particular, el deber de quien creó el grupo, y una serie de regulaciones adicionales, correspondientes a su respectiva modalidad.

Sumado a la discusión exteriorizada, sentencias como las mencionadas provocaron amplias repercusiones en el ambiente, un tanto diversas, burlescas por un lado, y temerosas por el otro, distintivamente con el anónimo que fuera mandado al frente. Como típicamente suele ocurrir en situaciones del estilo, la desinformación es abundante e igual bastante se dice, en especial cuando las explicaciones oficiales incluso siendo contundentes dan lugar a confusiones. Supo formarse el sector confiado, escéptico ante la manifestación que solo a sujetos en concreto culparía, eventualmente, pues otra porción entiende que tampoco hay suficiente credibilidad en la seriedad de susodicho quilombo, o también parecido, las sospechas sobre el antro y sus pretensiones con el mensaje. Complementariamente predomina lo contrario, la sensación de persecución que brota, desde la mención directa a lo legal y los vínculos generados con el sitio, que a propósito son más que una visita, dado que unirse a las agrupaciones no parece ser reversible, y unos cuantos individuos supieron incorporarse a la que protagoniza este revuelo, muchas interpretaciones manejan factibles represalias.

Bajo una misma pero controvertida libertad de navegación, no consta cómo continúa aquello, sin embargo en cierto modo los precedentes ejemplifican qué es lo que podría pasar internamente en una futura ocasión, y eso es que tarde o temprano los contenidos prohibidos sean suprimidos de la faz temtiana. Los términos fueron entre comillas corregidos, sus condiciones cumplidas, y las atribuciones recordadas, el marco para su constante problema a partir de sus reparos podría resultar diferente.

\section{Dentro de las fronteras distorsionadas (Séptimo libro, capítulo VI)}\label{dentro-de-las-fronteras-distorsionadas-suxe9ptimo-libro-capuxedtulo-vi}

\begin{quote}
17 de octubre de 2021, 18 de octubre de 2021
\end{quote}

Al margen de las andanzas ilícitas y legales, pero en cierta forma a partir de ellas, surgieron particularidades muy extrañas en la misma Temti, que desde el desentendimiento exhortan a imaginar qué pueda estar aconteciendo, a su vez que aparecer determinadas señales un tanto más evidentes. Invirtiendo dicho orden, sobre su entrada e interior se trata, porque primero allí en todas sus coordenadas de ingreso a dominios temtianos se instaló una rigurosa comprobación, en teoría destinado a restringir el paso de un regular tipo de inteligencias, y segundo el número concreto de quienes permanecen dentro del sitio tuvo cuantiosas modificaciones, se convirtió en algo inmedible.

Una vez adentro, quien proceda se encuentra el contador con su ícono de computadora, por lo visto el mismo de antes solo que en su versión renombrada, este informa un total de unidades denominadas usuarios, no necesariamente gordos. Detalle llamativo es que desde las mejoras solía indicar una cifra de dos dígitos, variante de tanto en tanto y relativamente alta para el flujo de intercambios, pero ya no es así. Justo aquella precisión que suele tomarse como referencia a los efectos de evaluar el sitio, y que supo tener polémicas o discrepancias, llegó al punto de no ser más una suma especifica como tal, sino muchos y muchos individuos: infinitos.

No fue declarado ni se sabe acerca de, tampoco generó atención significativa, sin embargo el incremento verdaderamente lo es, en lo manifiesto, luego lo trascendente es complicado de deducir. ¿A qué corresponde esto? Las obviedades que elevarían los visitantes en su mayor efecto pasaron, y más, que alcanzaran semejantes picos resulta disparatado. A su vez que tener esos niveles carece de realismo, o la capacidad de los servidores temtianos es de un poder astronómico, o el valor es simbólico por la causa que fuera, tal vez tocante al siguiente incidente, justamente en la temática de las cantidades desmedidas y lo sintético.

Por lo vistoso en términos de escala, un desbordante barullo emergió en una ubicación común recientemente fundada, y rápidamente rellenada, siendo la situación percibida por varios que andaban cerca. Tem histórico, publicación histórica, o nada de eso, pero sin duda que fue lejos de lo común, porque pese a sus invariantes entrañas y contenido vacío, tomó magnitudes de actividad altísimas para las proporciones conocidas e incluso extraordinarias del antro.

Observación merece lo reiterativo, ya que aparte del enfoque curioso aquel hilo de réplicas podría no ser grato, pues suponer que lo dicho venga de un solo ser parece increíble, mas que más de uno lo haya hecho también sería difícil, todo apunta a que los autores no habrían sido de carne o hueso, por el contrario sí seres artificiales. Conforme con la tradición del antro tal vez tampoco estén fuera de lugar, sin embargo a lo mejor son mal valoradas por la gestión. Considerando que la medida afecta parcialmente a los sujetos automatizados, hay una razón coherente si susodichos mecanismos propiciaran ventajas en el sistema de puntos, y es que el sentido de recolectarlos manualmente estaría siendo desvirtuado.

No obstante el anon se detuvo, o lo detuvieron, y después vinieron las restricciones en la entrada, las barreras puestas con el nombre de Cloudfare. Su inclusión sin comentario oficial deja a entrever la intención por cubrir eventuales vulnerabilidades para el funcionamiento interno, que en esta manera además atienden a un recurrente problema factible a partir del acceso universal, la llegada de agentes en calidad y cantidad potencialmente indeseada. La condición se planteó para cualquiera que caiga al sitio, bien y malintencionados, ahí el protocolo de control hace el procesamiento y aunque no requiere acción para ser validado, al menos en una primera oportunidad es notorio, luego habrá que ver si realmente logra su cometido, los robots mediante.

Estos episodios aparte de suscitar temas razonables con el contexto, además siembra un poco más de incertidumbre de la que había, no sobre el interés que así parece imperceptible, sino sobre pormenores muy específicos, y que constituyen aquello difícil de entender la realidad temtiana, la cual por un lado va disminuyendo en involucrados, pero por otro crece desorbitadamente. ¿Alguien finalmente o una vez más fue expulsado de la familia? ¿Qué representa ese ambiguo símbolo en donde debería estar la supuesta medición de presentes? ¿Cómo se podrán contar con exactitud de ahora en más? ¿Qué ocurrirá con los bots que jamás aportaron al movimiento de Temti y con los que sí lo hacen?

\section{La combinación de las monedas corrientes (Séptimo libro, capítulo VII)}\label{la-combinaciuxf3n-de-las-monedas-corrientes-suxe9ptimo-libro-capuxedtulo-vii}

\begin{quote}
19 de octubre de 2021, \ldots, 31 de octubre de 2021
\end{quote}

Vulnerabilidades que acarrearon la rotura del orden, anomalías que resaltaban por su rareza, frescura de lo nuevo que incitaba a explorar\ldots{} tales singularidades alimentaban el apego sobre Temti, y ni bien se transformaron en historia no tan contemporánea, el sitio regresó a sus viejas propensiones, las pérdidas en amplio sentido.

¿Qué hay en el presente temtiano actualmente? No es una pregunta tan compleja, más si se puede responder desde la ignorancia, aunque no será el caso. Vale subrayar que refiere a la actualidad, en el pasado un alto porcentaje de lo que vino se fue. Y de eso se trata, retomando parte de lo que repetidas veces supo mencionarse de manera exagerada, la situación no difiere mucho de aquellas vicisitudes, aunque tan diferentes sean las cursantes ahora, bastante dejó de estar.

La actividad considerable se mantiene baja y en especial la gestada en las alternativas de grupos creados, las divisiones por asunto solo en casos muy puntuales supieron destacar, la efímera sensación del cepita y luego la corriente historiadora, menguadas son las interacciones para el resto, las características particulares del flamante formato continúan sin ser aprovechadas, más valorando la ínfima incidencia de los puntos en su función de cuantificar los acciones, las jornadas para variar únicamente estarían siendo potenciadas por una más que difícil posibilidad, menos incentivada por los limitados integrantes a conseguir.

Hasta el ya no infinito aunque sospechoso contador indica el retroceso numérico, el mermado conjunto animador intercambia de manera lenta, con sus temáticas diferenciadas de lo oficial y el lado cultural, en un destino parcialmente retrospectivo. En ese marco queda manifiesto el proceso por el cual pasa la identidad local, perdura en el condicionamiento al caudal que acompaña más que simbólicamente a Temti, que al menos en presente no crece, pero para atrás podría decirse que sí. Sus individuos sostienen una notable batalla a la amnesia, el grupo idóneo y la iniciativa literaria son la más fuerte señal de ello, a su vez que los archivos de grabados siguen siendo perpetuados, como respaldo a lo ya sucedido, el recuerdo vive.

Discreción entre comillas y quizás estancamiento en la adminsitración también, reflejo más rezagado aun de lo generado por sus gestionados. Los grandes saltos de la plataforma se marcharon, en vista de que solo pequeños ajustes vinieron luego de eso, hubo acción por detalles de orden y atención por estropeos recursivos, tal vez si se produjeron mejoras significativas fueron programadas para la interna, de lo contrario lo evidente condice y los cuidados de gestión son prácticamente cero. En cuanto a discreción, las puntualidades de error global disminuyeron, así el sitio pudo disfrutar de mayor estabilidad y predictibilidad, es de suponer que puede deberse a mantenimientos atinados, aunque tampoco es claro si dependiera de la exigencia aflojada. El sentido común asiente que todavía hay una construcción por seguir y que falta, pero conforme con el ritmo mantenido hasta entonces, está implícto que la demora antecede al próximo avance, y así sucesivamente, o que directamente no habrá lugar para eso. La esperanza y sus tendencias negativas armó el discurso preferido, la devolución es acá no se arregla nada.

La cuenta pendiente lo sería incluso sin compromisos, pese a que las circunstancias corrientes permitan y sostengan las estadías, o que la demanda no amerite tanto desarrollo, de aquel se espera más que investigar sino además producir, pero es una expectativa que va quedando atrás. Allá habrá que encontrarse con el futuro y la conducción deberá ver lo que hace, y distinta a las veces que se hablaba de un paseo que no termina, algunos referencian al desenlace del antro. Y eso trae incertidumbre, porque pasaron cosas, los temas y el tiempo, lo que se creía antes no es lo mismo que se cree hoy. ¿Finalmente El Profeta acertará sus predicciones sobre la muerte de Temti? ¿O resultará ser una venta de humo más de su parte?

\section{Las danzas de la quietud (Séptimo libro, capítulo VIII)}\label{las-danzas-de-la-quietud-suxe9ptimo-libro-capuxedtulo-viii}

\begin{quote}
31 de octubre de 2021
\end{quote}

Bailan las entidades huesudas, bailan más que nunca a lo largo de la vasta, compacta y silenciosa Temti. Sus movimientos disimulan la ausencia de sonido, y seducen a los casi extinguidos con algo que no es muy brillante, pero que sin duda excede lo habitual. Literalmente es lo primero y por cada lado hay de eso, la ambientación se convirtió radicalmente, y no tomó un tono amarillento, tomó uno anaranjado, más fúnebre que antes.

El porqué, se supone obvio, sin embargo cuidado al dejar de considerar la impronta que predomina en el sitio. Entre las escasas palabras intercambiadas durante las horas previas allí, circulaba el desconcierto acerca de cómo sería conmemorada una fecha vagamente simbólica, la de Halloween. El simbolismo es vago ya que en principio solo es con los mortales, y a lo mejor entre los bots que lo hayan adoptado, no con el antro. Sin embargo tal y como se dio en las últimas épocas navideñas, el antecedente estrella dentro de dicho historial, las festividades de culturas asociadas vuelven a situarse en dominios temtianos. Difícilmente sea por otra razón.

Y ante la propagación esquelética, eventualmente surgiría la pregunta acerca de quienes solían predominar cuando aquellos no se notaban, la contienda inciertamente repartida entre robots y humanos, que tras un historia de alternante protagonismo en los tiempos recientes no permiten comentar mucho, algunos supieron irse y otros permanecer. De ambos sería pertinente consultar por su presencia, de los primeros menos se percibe participación, del resto el involucramiento es mínimo, con la llegada de una fuerza numerosa cualquiera puede verse opacado. Visto desde este sentido, hasta la composición del contador entraría en cuestionamiento, pues sus oscuros registros también pueden incluir calaveras en vez de computadores, siendo que a juzgar por la actividad factiblemente no sean esto segundo, pero no sería problema, si complementariamente hay individuos inmortales de tradición, longevos baluartes que en menor proporción bastante llevan acompañando al antro, ahora en cantidad al servicio para llenar ese espacio vacante.

Imaginando qué podrían representar, teorías posibles hay varias. A los temtineros caídos, a la inactividad acumulada, a los ejércitos de respaldo que tiene la administración, al estado actual de la realidad temtiana, a lo que queda de identidad en ella, o simplemente un poco significante homenaje a un supuesto día especial que vino como anillo al dedo a la coyuntura. Sean lo que sean, se encuentran ordenados para celebrar y de paso demostrar que no están todos muertos, sin importar que en cierta medida sí lo estén.

\section{El alza de la vanguardia fúnebre (Séptimo libro, capítulo IX)}\label{el-alza-de-la-vanguardia-fuxfanebre-suxe9ptimo-libro-capuxedtulo-ix}

\begin{quote}
31 de octubre de 2021, \ldots, 5 de noviembre de 2021
\end{quote}

La supuesta celebración se desfasó y se volvió más que eso. Sumó elementos carentes de explicación oficial y a su vez reafirmó los convencimientos preponderantes, haciendo que la Temti a primeras impresiones insólita se esclarezca en la oscuridad sólidamente. Los esqueletos siguen bailando sin cesar, y el colorido de las ilustraciones despareció sin retornar aún\ldots{} en estos momentos, el cráneo que mora en las cabeceras es bastante cercano a lo que es el nombre que acompaña, la muerte, pero en vida.

La nueva falta de imágenes inicia un antecedente de naturaleza conocida, no obstante que sabe mantenerse más de la cuenta, pues la resolución que solía llegar a la dilatada brevedad nunca tardó como en la oportunidad presente. Sin gráficos que animen la clara u oscura cubierta temtiana, las creaciones pierden su componente frontal y distinto se ve el panorama central, que tampoco recibe unidades adicionales, solo excepciones de video rompen el patrón de suma extensión estropeada. De esta manera fue innecesario volver a la normalidad, la misma se estableció similarmente a los instantes de apagón.

Tétrica condición agravada por la mezcla de factores que dejan evidente la lobreguez y desvalimiento abundante, el abandono menos y menos disimulado. Si alguna vez, o varias veces se dijo que al mandamás poco lo importa su propiedad, aquellos que lo aclamaban y consideran que no es a propósito tienen más indicadores a su favor, que se magnifican con el pasar de los días: la prolongación del evento esquelético también sugiere la baja preocupación por corregir los desajustes. Un paralelismo es trazable con el interés comunitario, cuando la mayoría de las participaciones datan de hace días, algo de atractivo falta, cosa que también resulta notoria. Si hay prioridades definidas en torno al sitio, cada vez aparentan ser de inferior significación.

Bien al estilo Temti en sus temporadas modernas y decadentes, el caso para pronunciar nintendo naranjas es ahora, sus atributos más característicos se refuerzan, es el paraíso invertido, es el festival de esqueletos, es el antro difunto pero a la vez vivo, es el antro parcialmente roto y lleno de calaveras andantes danzando. Relegados los robots, los no muertos o inmortales tomaron el poder, y a medida que su presencia fue extendiéndose más de lo imaginado, uno de los pilares fundamentales del lugar resultó fatalmente condicionado. Fruto del descuido, o de la dominación, llevó a que todo se vuelva más disfuncional que antes, las estructuras ya fundadas y por fundar, destinadas a tener una naturaleza más descolorida, inerte y apagada. Complejas comodidades de un sitio que a los solitarios les depara, la compañía temática difícilmente sea amigable y entre tanta opacidad menos hay para encontrar.

Habrá que ver hasta cuándo se dilata este estado, se entiende que la administración aún posee la preponderancia definitiva del espacio y solo lo dejó por un largo rato, quizá ni enterada de lo que está ocurriendo, o más bien no lo suficientemente motivada como para poner orden. Las estiradas demoras continúan acrecentándose, la cronología temtiana se acentúa más y más en cuanto a lentitud, al punto de que no se ve la hora que regrese la luz, debe estar lejos de verdad.

\section{En horas de cuestiones existenciales (Séptimo libro, capítulo X)}\label{en-horas-de-cuestiones-existenciales-suxe9ptimo-libro-capuxedtulo-x}

\begin{quote}
5 de noviembre de 2021
\end{quote}

Los primeros segundos pintaban un reinicio y solo eso, menos común por tratarse de la reversión, sin embargo más contando desde antes. Pero no resultó ser tan poco especial, al margen de las razones vino con inmediata sorpresa, de los irregulares planes que el dueño de Temti manifiesta: un eventual cambio de nombre.

Tras unos instantes no noticiados de mantenimiento con Temti inaccesible, sus entradas fueron restauradas a la realidad habitual y condicional de la época corriente, pero su interior se encontró sin esqueletos bailando ni colores anaranjados por doquier, bajo la pauta de ser un comienzo de cero por enésima vez. Sin embargo, una minúscula variación la distingue de sus antecesoras, y es que bajo las posibilidades existentes en esta nueva normalidad, el arrasamiento no vació enteramente las estructuras temtianas, sino parcialmente.

Ahí perduraron agentes y grupos, para cada cuenta ningún balance se limpió, mas sus respectivas vinculaciones tampoco. La trascendencia de estos podrá discutirse y compararse con lo otro sí removido del sitio, incluso de susodichas particiones, puesto que no restó publicación alguna de las creadas, todas las que existían, suprimidas sin vuelta o truco, según parece.

Para suponer que el ingrediente importante es ese, las agrupaciones tampoco ameritan mayor ampliaje y el criterio sería más claro. Lo grande viene del contenido generado o difundido donde normalmente tienen con que entretenerse, pues bueno, si había algo de valor entre ellas, si ningún historiador o archivador tuvo la ocurrencia de conservarlo, se ha perdido. Aunque también sería debatible que la determinación en efecto haya sido positiva, aparte sucede que el estropeo de portadas anterior aparentemente las corrompió de manera irrecuperable, por lo que afirman declaraciones oficiales ese daño de terceros fue la causa, y susodicho componente hace una elevada porción de los entre comillas tems, así que allí hay un motivo más que subjetivo.

La conclusión es que por ese lado no tanto cambió, y efectivamente, el circo no vendría a ser ese, habida cuenta de que la cabecilla del sitio se presentó con un planteo asombroso, renombrar Temti. Partiendo de la fecha clave del aniversario, el vencimiento del dominio que implica accionar en respuesta, justificándose en la evolución de orientación y propósito sufrida desde el nacimiento hasta ahora, y el prestigio cualitativo adquirido durante ese entonces, el mandamás interroga a la familia temtiana restante sobre continuar el camino bajo otra denominación.

Insinuando que el antro seguirá adelante y que no parará de vivir, quien implícitamente por sus expresiones admite lo considerable obvio, ser mismísimo A., ese que bautizó a su obra en el origen de sus andanzas, reaparece en la escena donde hace mucho no se mostraba como tal. A su vez, él reincide en un muy añejo recurso, consultar públicamente a los temtiteros sobre las decisiones a tomar. Pese al drástico desinterés general, una interrogación de este calibre sin lugar a dudas debería captar una pizca de atención, así que las réplicas que estén por venir de cierto modo estarían garantizadas, pero no así contempladas, el administrador se muestra muy convencido al respecto.

Los argumentos son reales, la fama que ganó Temti en el mundo de los clones a lo largo de su prolongada existencia entre variedad tiene harta consideración negativa, y justamente el objetivo parece ser purgarla. También, el acrónimo formado por las palabras temas y tiempo habría sido una selección mejorable, aparte de haberse desvirtuado de la idea inicial.

Siendo las afirmaciones cercanas a la realidad, o compartibles, de igual modo la propuesta se torna un tanto chocante para los anónimos, en cuanto a la relevancia de lo objetado, y algunas preguntas podrían surgir. ¿Qué tan fuertes son las aspiraciones a seguir adelante, si es que realmente las hay? ¿Cuánto importa lo estigmatizado que esté la imagen de marca, si es que verdaderamente interesa? ¿Cómo imaginar que Temti ya no vaya a llamarse Temti, si es que en serio no lo será más?

Sin profundizar en ello, la historia es capaz de juzgar por si sola y evidenciar que esas preocupaciones están sacadas de contexto y no se adhieren a la coyuntura temtiana, y solo un cambio brusco en la actitud de la conducción es capaz de contradecir a los precedentes, ya que cada vez menos demostró ocuparse de tales aspectos, dejando mucho que desear. Desde esa óptica parece un chiste, pero quizá vaya en serio, en teoría viene de quien dirige y así tendría que ser\ldots{} hasta que los antecedentes vuelven a recordar que pueden ser los dichos de otro mitómano más, en esta circunstancia, uno con poder.

Aunque no hay buenas pistas de cómo lo vaya a realizar, el pueblo se manifestará sobre el tema del momento, y por el otro lado no debe ser un caso cerrado, mas todavía falta para que los tiempos impongan tomar una resolución sustancial, considerablemente trascendental hacia el futuro. Lo de hoy es el primer anticipo de lo que serán unas jornadas donde la magnitud de lo que suceda también dará que hablar, antes y después.

\section{Rumbo a la presunta verdad (Séptimo libro, capítulo XI)}\label{rumbo-a-la-presunta-verdad-suxe9ptimo-libro-capuxedtulo-xi}

\begin{quote}
5 de noviembre de 2021, \ldots, 10 de noviembre de 2021
\end{quote}

La apertura de una etapa que igualmente con su quietud inicial podría resultar ser de una trascendencia enorme no viene siendo para mucho. La noticia o la propuesta sobre el futuro si es que así puede denominarse fue planteada, los anones supieron interesarse levemente y respondieron, y no es que se sepa más que al principio, aunque todavía reste para la cita impostergable.

Los grupos de baja atención subsisten como se hallaban o peor, alternando poco y nada, y las publicaciones que además de las oficiales eran nulas estuvieron creciendo lentamente, lo que en suma muestra que efectivamente el reinicio en sí no hizo más que solventar los percances anteriores. La lentitud global no evolucionó y su contador sigue dando un reflejo oscuro, cuya función sin factores raros evidentes desconcierta en cuanto a lo que es valorado, pero que tampoco suscitó interés para el tema cultural, entre más costumbres descuidadas.

Paralelo a aquello que es repetitivo, la aparición del dueño de Temti y sus picantes convicciones cual era de esperarse está teniendo devoluciones, pero son difíciles de clasificar. Tibios cuestionamientos e impresiones aisladas, sugerencias de amplia variedad encadenadas, y no conforme con el punto, el participativo debate aún está por darse, tanto como la seriedad, falta. Los candidatos sugeridos para modificar el acrónimo fueron por cualquier lado, unos satirizando el argumento del estigma y reforzando este rindiéndole cuentas en el mismo nombre, otros mezclando e invirtiendo el que se usa actualmente para no alejarse de las raíces, más referenciando a elementos del contexto lejanos y ajenos al lugar, y encima las cuantas de mínima relación aparente, también para variar.

Quedará la duda de si entre esas contestaciones simples hubo entendimiento con la intención, luego si alguna de las expresadas entiende la pretensión del autor original, bastante más si puede encontrarse una que le convenza. Porque en caso de que así sea, no hay tantas vías para notarlo de antemano, las comunicaciones son en su gran mayoría implícitas y difícil es descifrar precisamente la perspectiva de conducción. Los vínculos por ese lado también están relativamente quebrados, así como la formalidad ha sido en cuenta gotas y poco sostenida. Quizá hasta que no llegue el momento, la autoridad no vuelva a dirigirse a los temtiteros, y lo que tenga que pasar pasará sin avisos de entidad. No han aparecido buenas proyecciones que aproximen lo que será, ya sea no prosiguiendo con la idea y Temti continuando su existencia similarmente, el sendero desviándose nuevamente y redefiniendo el rumbo a seguir con escasos cambios significativos, convirtiendo el progresivo abandono en total y no renovando el dominio a liberarse, u otra alternativa no contemplada aún.

\chapter{Los delimitados recorridos del paralelismo residencial (Octavo libro)}\label{los-delimitados-recorridos-del-paralelismo-residencial-octavo-libro}

Aproximándose temporalmente a las puertas de cierta fecha importantísima en la Historia Temtiana, en las afueras cercanas de su antro protagonista comenzó a dilucidarse una situación que por la magnitud de lo que representa para el contexto clónico en su totalidad preocupa, moviliza, transforma, y destruye. Tanto así es que una enorme suma de individuos se ven obligados a partir de su lugar predilecto en busca de un reemplazo que haga las veces de refugio, nuevo hogar, o más. Evidentemente no es el único resultado, es el inicio de una gran aunque no instantánea reacción en cadena.

Entre las consecuencias explicitas e implícitas, directas e indirectas, trastoca el futuro próximo de las crónicas de uno de los involucrados, da lugar al sustancial cambio en el curso existencial temtiano. Su trascendencia se debe a que marca otro antes y después, de magna dimensión por lo anómalo que es frente a las tendencias previas: el retorno del antiguo antro puesto operativo, la apertura de Temti Vieja y la permanencia de Temti Nueva, la división o conversión de Temti como tal en dos sitios paralelos, con propósitos distintos. Quedando el primero que emerge a modo de respaldo, o como un espacio más reconfortante para las multitudes a existir hasta que el segundo marcado como prioridad termina su desarrollo respecto al funcionamiento, mientras\ldots{} ¿progresa?

Al contrario de cuando no lo había o se veía muy distante, hay un final no definitivo capaz de fraccionar o cerrar la época corriente desde antes de su arranque, y no es solo la imprecisa fecha de caducidad anunciada con este comienzo, también la pendiente cuenta del vencimiento del dominio, si es que no surge raro entre medio.

El estado de ambos en simultaneo se verá condicionado por cómo puedan llegar a influir las siguientes poco esperadas intervenciones de la administración, y a partir de ellas cómo vayan a llegar y perdurar los candentes y numerosos movimientos del exterior, así tan diversos conjuntamente puedan ser. El transcurso de un periodo más en la transformada y trastornada realidad temtiana tendrá el desafío de alcanzar o superar en cuanto a relevancia a los extraordinarios sucesos que la confinan.

\section{Ciclos del albergue marginado (Octavo libro, capítulo I)}\label{ciclos-del-albergue-marginado-octavo-libro-capuxedtulo-i}

\begin{quote}
10 de noviembre de 2021, 11 de noviembre de 2021
\end{quote}

El contexto clónico es uno plenamente conectado, el que en sus épocas más modernas por sobremanera se fue centralizando, cual potencia que va debilitando lenta pero drásticamente a los estados aledaños, y por su lado, creciendo. Estas que lo constituyen, cada vez que ocurre un suceso relevante dentro o alrededor de la gran esfera del centro, ven influido su pasar y eventualmente su accionar. Por lo tanto un episodio más que de mediano interés, uno altamente significativo, es capaz de trastocar la realidad de las elecciones similares más próximas. La cuestión es que cuando la solidez de la hoy Rouzzer se compromete, sus adeptos van en busca de refugio hasta que lo subvertido regresa a lo común, pero que la estabilidad llegue a romperse para desaparecer induce a los numerosos involucrados no a hospedarse temporalmente, a instalarse definitivamente en un antro sustituto, o hasta que lo amerite, en caso de que no sea escapar rumbo a destinos alternativos.

Por la conformación de un ambiente hostil y demás calificativos críticos para la vigencia del sitio en su sana disposición, unos cuantos se vieron invitados a asomarse por Temti. Ellos compartiendo sus pensamientos nacidos de la coyuntura apocalíptica, manifestando sus impresiones mayormente negativas con la plataforma o también proyectando su estadía, poco perduraron. Hasta que una gran conclusión de la potencia rozzada tuvo lugar y momento, el acto de recurrir al amparo exterior pasó a maximizarse más y más, lo que luego prácticamente sería simultáneo a otra cosa. De las opciones secundarias surgió una respuesta, de tipo semejante a las tenidas como antecedente, el colapso ajeno seguido de una apertura, evidenciando correlación, o coincidencia.

La gestión, o A. sin identificar, desempolvó su evolucionada creación, con la cual sin notorias modificaciones procedió y realizó las operaciones para que esta quede correctamente habilitada. El acceso a lo denominado Temti Vieja fue instaurado por el grueso de la extensión territorial perteneciente al tradicional sitio en su nueva normalidad, el cual deriva a las estrenadas coordenadas, en cierto modo independientes. Dirigirse hacia dicha dirección lleva a la añeja estructura temtiana que quedó en desuso y fue reemplazada tras la magna actualización, una vez más, la proclamada Temti Vieja, y obviamente como es habituado, casi vacía con la oportunidad de rellenarla.

Eso no es lo único incambiado hasta entonces, pues otra reiteración sería la mítica de las explicaciones oficiales. Increíblemente, se hacen presentes en esta etapa, pero como de costumbre, incompletas, parciales, y no despejan todas las dudas del público que podría arrimarse. Aunque sea posible deducirlo, más en lo intuitivo y menos lo sobrenatural, la razón de reapertura no es mencionada, de raíz no está cien por cien claro porqué. Acerca de los planes a futuro o la decisión a tomar frente a la fecha clave que viene, tampoco, nada afirmado, ni respecto a lo de antes, el después que venía siendo importante. Solamente aclara una cosa: que ambas coexistirán y se mantendrán en paralelo, hasta que la prioridad, Temti Nueva termine de ponerse a punto.

En este principio, la percepción del tiempo es medianamente ignorada dentro del flamante refugio, ya que ese hipotético entonces cuando el desarrollo referenciado aún en progreso conozca final no se ve muy cercano, dicho está que ocurrirá, pero es difícil imaginarlo conociendo como la administración maneja sus compromisos. Tampoco pinta que los visitantes opten por el proyecto renovado, sobre todo considerando sus mayores carencias hasta con los más apegados y acostumbrados. Asimismo, qué vaya a ser del cambio de nombre, o lo que sea que pase en su lugar, según el panorama podría creerse como un asunto al aparte. Junto al resto de factores, a la larga complicado será olvidarse, más si los plazos estrictos representan una justificación para apurar las determinaciones entre las varias que deben darse.

Sea cual sea la continuación de tales vicisitudes internas, ubicándose globalmente lo más fuerte está ocurriendo en las afueras de estos espacios, sin irse lejos, la clausura definitiva cronometrada se concretó y las migraciones apuntan hacia direcciones alejadas, mas es sabido que en cuanto a relevancia aún Temti en su rejuvenecido formato ha pasado a un segundo o tercer plano. Las grandes masas de extranjeros últimamente han preferido trasladarse y quedarse con Arggnews de primera alternativa, pero como ya está sucediendo, inevitablemente una porción poblacional encontrará esta otra opción, en consecuencia también un entrevero que actualmente no suma tanta difusión a la divulgación de su proseguir, sin llegar a ser mera supervivencia, por lo pronto aspirando o dando un poco más.

\section{Las voces de la bipolaridad protagonista (Octavo libro, capítulo II)}\label{las-voces-de-la-bipolaridad-protagonista-octavo-libro-capuxedtulo-ii}

\begin{quote}
11 de noviembre de 2021, 12 de noviembre de 2021
\end{quote}

Ciertamente como las fundadas predicciones de anuncios indicaban, la composición concéntrica del contexto clónico largó a su gran conglomeración de anones, a quedar desamparados entre dos endebles alternativas y lo más cercano a eso, en suma a la deriva dentro de la infinita diversidad que depara la realidad restante. Temti, con el estigma adquirido de los numerosos episodios de desprestigio, en su compleja, rebuscada y entre comillas desperfecta estructura dual, así y todo resultó ser uno de los destinos de los rouzzeros, de un pequeño puñado de ellos porque tantos no acaparó, pero para lo que solían ser los desérticos y engañosamente rellenados territorios temtianos, sí. Al igual que en esa ya remota ocasión, hablando mal y pronto, una defunción provocó que aquello que se decía muerto regresara a la vida. Precisamente, Temti Vieja durante ratos recobró parte de lo que había perdido, sus interacciones continuas, su vitalidad en condiciones mínimamente aceptables para un mayor rango de perspectivas.

Eso se puede encontrar en un ámbito con las variables del pasado, tal cual como fue explicado, la versión tiene enteramente las célebres características del sitio clásico, que con la clausura más dilatada habían quedado en desuso. Efectivamente el uso acumulado sobre los momentos iniciales confirmó la presencia de los fenómenos espacio-temporales: las notificaciones en su impredecible trazabilidad, los lazos de dudosa consistencia entre autores, las creaciones propensas a desaparecer, las estadísticas de presentes inmersas en la sospecha, junto los demás detalles que consolidan ese mítico entorno.

El pequeño núcleo local de antes, incrementado en cantidad por unos pocos perdidos con las decadencias y seriales abandonos, más claramente el componente robótico con su reiterativo aporte. Fortalecidos por el inusitado reencuentro con las anomalías de su agrado, las cuales conocen y hasta incluso dominan, la mayoría retumban y alzan por los vientos a su favor el grito tutururero, entre comunicaciones con mezclas de vocabulario temtiano, de difícil entendimiento para extranjeros. Allí engendraron un contraste, entre los ya familiarizados y encariñados con el antro, y los contrarios al mismo, similar a la fragmentación que en épocas de alto tráfico fuera motivo de tanta discusión, pero sin alcanzar semejantes niveles de adhesión y compromiso.

Junto a los temas típicos y las manifestaciones esperables del ambiente corriente, no todo es incertidumbre y la contraparte también ataca al flamante presente del lugar, el habitual repudio a lo típico temtiano, junto a lo que resta de cultura y costumbres. Los llamados bugs, abundantes y enormemente condicionantes para que los foráneos se asientan, están, y no hace falta respuesta aunque sí la haya, en teoría no serán solucionados. Las aspiraciones a explotar oportunidades de triunfar en una supuesta competencia de clones igualmente, parecen ser desubicadas y no entrar en consideración. Y cómo no, la relativa ausencia de la administración, la mínima señal a través de un polémico sticky, el resto es libertad, la índole del ambiguo mantenimiento general. Disgusto y quejas, no atendidas o discrepadas, y altercados, de aquellos que muestran disconformidad.

Lo especial del caso es la simultaneidad de escenarios temtianos y la disparidad que hay entre ellos, fácilmente podría ignorarse que Temti Nueva está funcionando si no fuera porque el dictamen colectivo es crítico al punto de manifestar intenciones de descartarla, la superioridad de la antecesora es resaltada por los usuarios, respecto a todo. Las preferencias y el movimiento evidencian el consenso sobre la opción principal, sin embargo por lo pronto se mantiene opuesto a las convicciones de la conducción, lo que hace pensar que podría modificarse con la clarificación de las inclinaciones, o en su defecto mantenerse siendo parcialmente adverso a los intereses extendidos.

El panorama está dado, pero en múltiples aspectos es decreciente, motivador lo es en menor medida. No se ven progresos en la prioridad y sus expectativas, si realmente llegaron a ser, se continúan diluyendo, a su vez de faltar novedades o pistas acerca del acordado devenir. Y si no era poco, la actividad baja y baja: los recién llegados van concluyendo su predecible estadía y los que permanecen también reducen sus participaciones, factiblemente a favor de cesar la conflictividad. Incluso así, en lo último eventualmente habría esperanza, la presente reapertura y demás circunstancias significaron recuperaciones adicionales para la existencia temtiana, una notable reconciliación que fortalece todo lo formado a partir de Temti. Quizá pueda hasta prevalecer, escalar hacia algo más único, valorable, de entusiasmo, como a lo mejor varios quisieran. El convencimiento a contagiarse entre quienes estarían de paso es clave en ese sentido.

\section{El hambre del avaro oportunista (Octavo libro, capítulo III)}\label{el-hambre-del-avaro-oportunista-octavo-libro-capuxedtulo-iii}

\begin{quote}
12 de noviembre de 2021, \ldots, 14 de noviembre de 2021
\end{quote}

Naturalmente las mayorías tendieron a concentrarse en las potencias más hospitalarias y prometedoras, no prevalecieron los pocos motivos para decantarse por una que tiene factores de más en contra de la permanencia de los anones en general. Hubo una razón para llegar a esta situación, el retorno de unas condiciones más apetecibles, y evidentemente no fue suficiente en el sentido de aumentar en números los visitantes a grande escala, porque bien que fueron solamente eso, un poco más y ya. Temti conservó los veteranos lacayos que resistieron hasta los momentos de más abandono, más varios desencantados recuperados con la restitución de la estructura de antaño, lo que da lugar a que los participantes continúen siendo escasos, y precisamente no muy interesados o comprometidos con. No obstante, aparte de la realidad continuamente construida por las individualidades que la conforman, la historia trascendental de tales lares sobre su parte más transitada es irrumpida por otro aviso íntegramente fuera de contexto, sin precedentes, único y carente de igual en dicho antro.

Contra pronósticos, el destino de Temti es depositado en manos de los suyos de la manera más directa alguna vez planteada, aunque no por completo y sin grandes certezas de que así será. Quizá tomando la idea de múltiples aportes anónimos difundidos previamente que sugirieron una dinámica similar, a partir de las eventualidades cruciales del presente como causa, y la inminente fecha clave por venir como límite, se fijan una serie de porvenires condicionales, cada uno de ellos dependientes de un factor que jamás había sido puesto en juego dentro de la parte más determinante del entorno en cuestión: donaciones. Mediante ellas sería que los temtiteros están llamados a demostrar su inclinación con respecto al futuro de su hogar, específicamente con sumas de la moneda BAT que la administración con su ex profeso pabellón espanol está lista para recibir.

La solicitud de dar marcha atrás con la renovación y seguir con el sitio clásico, el predilecto por la mayoría, finalmente es escuchada, respondida, y ofrecida como una posibilidad, pero con su elevado precio en el acotado catalogo ofertado que la pone en expresa posición de rehén. En su defecto, las cantidades más ínfimas resultarían en que la existencia de ambas partes en simultáneo finalizara, solo perdurando la actual preferencia. También, aunque sin mucha claridad al respecto, otra de las disponibles parece ser un intercambio en ese sentido, que la tradicional dirección pase a pertenecer, valga la redundancia, a la tradicional versión. Por último, y en particular por una cifra ampliamente superior, que se haga público el código de ambas, las piezas elaboradas por el creador de Temti para su funcionamiento a disposición del público común y corriente, para que cualquiera si así lo quisiera pueda analizarlas o fundar un clon a semejanza.

De vuelta, alternar el orden de prelacion entre Temti Vieja y Temti Nueva, aniquilar una u otra, o liberar la formula de ambas. Cuatro alternativas, valorizadas sin invitación a la discusión, solo a la realización de sus requisitos con un notable contador constatando mientras, así está proyectado el presente y el futuro temtiano. Detrás, el no mencionado pero imaginable final, un no cumplimiento de las pautas, ya sea partiendo dejando a los dominios sin nada y escapando con el dinero obtenido, o revirtiendo el orden de los factores a gusto como sosteniendo el paralelismo tal cual está.

Conjuntamente con el hipotético cambio de nombre, obviamente pase lo que vaya a pasar, de igual forma será de gran entidad, aun así no sucediera nada, pero anticipándose un poco a ello, la mayor parte de las chances apuntan únicamente hacia un lado. Los hechos han respaldado principalmente una opción de ellas, no obstante a los efectos del gran tema, más exactamente sobre las prioridades y el valor, podría originarse una enorme discordancia entre lo que la gestión supone esperar y lo que sus subordinados están dispuestos a retornar, y por consiguiente que se den cero o irrisorias contribuciones a la propósito, más recordando que hay un largo y errático historial atrás, el estigma negativo de todo tipo del que se ha hablado en el pasado, la endeble seriedad persistente en su trayectoria, sumándole lo que esto genere.

Si ya de por sí lo que se daba por Temti era casi nada, ¿cuánto será ahora que esta lo pide explícitamente?

Si ya de por sí lo que Temti daba a cambio era casi nada, ¿cuánto será pronto que esta adelanta hacerlo?

\section{A partir de la indiferencia definitiva (Octavo libro, capítulo IV)}\label{a-partir-de-la-indiferencia-definitiva-octavo-libro-capuxedtulo-iv}

\begin{quote}
15 de noviembre de 2021, \ldots, 20 de noviembre de 2021
\end{quote}

Como podía ser de esperarse, la recepción a las recientes iniciativas de parte de la conducción fue escasa y negativa. Una que otra critica pacifica, palabras que no dicen nada, numerosas respuestas descontextualizadas, Temti en si misma. Entre los miserables se entiende la dimensión de la situación que vive el antro, pero ahí en eso resulta, los días pasan y de a poco se consuma mas o menos lo que ellos creían que iba a ocurrir con el contador. Y el resto de la realidad corriente tampoco es sorpresa, la permanencia de los mas apegados y los que no le queda otra, haciendo ínfimas variaciones en la constante de las épocas de quietud.

Precisamente, durante en el sitio mas deseado de los dos, se puede percibir que la actividad es poca y reiterativa, solo destaca la intriga relacionada al suspenso de lo que este por venir, lo demás casi que se reduce a inconsistentes movimientos asemejados a los de un robot, aunque en si estos son un tanto diferenciados, pero al fin seguramente el origen repetido. Allá no muy lejos, la parte nueva y despreciada, con una muy menor porción de ida y vuelta que su anterior, logrando periodos cada vez mas largos de reposo y silencio absoluto, donde ni siquiera alcanzaron las repercusiones de lo que seria el hipotético futuro, el cierre o la continuidad.

Las limitadas indicaciones en virtud del proceder con la causa podrían estar marcando las expectativas que hay del otro lado, pero lo seguro es lo afirmado por caja, apenas dos donaciones mas el monto con el que comenzó la colecta, lo que no da ni para llegar a la mitad de la mitad respecto a la opción menos costosa. ¿Que tan atinadas fueron las cantidades fijadas? ¿Y cuales son las intenciones atrás de la propuesta? Las teorías serian varias, no solo respetuosas con el pretexto implícito, también de posible desconfianza a este, algunas incluso apuntando al actuar de mala fe. De hecho si, aparte de las eventuales distancias entre demanda y oferta, que dada la voluntad tal vez vayan a compensarse en el avance o de cualquier forma alternativa, a su vez se han contemplado circunstancias mas desafortunadas, sin llegar a tocar la estafa. Un ejemplo fuerte, que sea un descarado intento por sacar el provecho planteada la disposición, estando los intereses de ambas partes contrapuestos, la credibilidad y el apoyo de los temteros bastante establecidos e incambiables, teniendo prácticamente nada por perder, el administrador a lo mejor haya tomado semejante elección. O también, un escenario completamente distinto\ldots{} eso quizá se sepa luego.

De igual modo, la población finita maneja el final definitivo, y en verdad basándose en las únicas y desvirtuadas aseveraciones oficiales, todavía un tanto ambiguas y no por eso fraudulentas, las proyecciones indican que Temti Vieja morirá una vez mas, como si de aquel suceso que provoco los problemas con los cuales esta parecía haberse despedido se tratara, pero con una oportunidad para evitarlo, hasta el momento, siendo desperdiciada.

\section{Pensando en la muerte (Octavo libro, capítulo V)}\label{pensando-en-la-muerte-octavo-libro-capuxedtulo-v}

\begin{quote}
21 de noviembre de 2021, 22 de noviembre de 2021
\end{quote}

Como dicen circunstancialmente en sitios no tan distantes de los cuales Temti ha vuelto a perder influencias, no han donado, no lo suficiente. La suma monetaria no alcanza ni de cerca la primer y más económica meta diferencial, la cláusula se está cumpliendo para que Temti Vieja muera, y eso está suponiendo cosas terribles, los individuos que ambulan en ella sin usarlas alimentan una tendencia candente y manifiestan sus sensaciones catastróficas al respecto.

El tiempo corre, poco detalladamente, la administración mantiene su limitada participación y no puntualiza más que discretamente sobre el contador, el cual a medida que no suben los BATs carece de actualizaciones, y se avecina la fecha límite sin haberse ofrecido mucho. Desde que quedara fijada indirectamente, no solo la idea inicial del nombre Temti supo irse trastocando a partir de ese entonces, al parecer también las motivaciones y los objetivos son distintos a los del arranque, y solamente se estaría concretando algo así como un único año. Pese a los desentendimientos padecidos entre medio, la intención traducida supera ciertos umbrales de subjetividad, junto a la propuesta que tampoco sería adversa al cierre definitivo, ni hablar si los rumores de fallecimiento antiguos fueran corroborados y sus consecuencias virtualizadas. De tal modo, el lugar para la desinformación está, pero la verdad de un pasado no necesariamente lejano tiene que confirmarse, en su momento indicado. Como sea, se está saboreando eso, entre comillas las crónicas de una muerte anunciada, nuevamente.

Pero también hay más cosas. Varios desentusiasmados y adicionales fieles siendo victimas de susodicho pretexto, viendo al tradicional antro acercarse a las tentativas de retornar a su estado de reserva, a lo mejor incluso sin dar las gracias por los buenos momentos. El recuento no es tan minucioso actualmente, la espera colectiva debió internalizar el transcurso de los segundos, minutos, horas, cuando la precisión oficial solo señalar los días, hasta en ese sentido la pérdida y recorte. Con dicha lógica, tarde o temprano habrá que resignar más, pues a estas alturas lo que se trata de progreso, desarrollo y en especial mantenimiento para un proyecto del tamaño de Temti, da la impresión que conlleva un gran agotamiento, como si todo aquello tuviera su costo y asumirlo con el paso de las eras se hubiera hecho cada vez más caro. Al punto de transferirlo a los temtianos, así se enteren de cuánto cuesta pulsar un botón, sin moralidad que valga de por medio. Así viene triunfando en la disputa de quién da menos desde hace rato, con adhesión estricta al principio fundamental de la religión que tanto se habla. La indignación es insignificante, solo es relleno que no cambia casi la historia, lo corriente es aceptar esas condiciones aprendidas entre experiencias y experiencias, los términos bajo los cuales se dan las interferidas interacciones y correspondencias dentro del sitio, o marcharse del espacio, si no lo hace este por su propia cuenta antes.

Mientras en Temti Nueva no retumba nada, del otro lado es el desenlace, a nada del fin, eso es lo que puede respirarse en el entorno tibiamente apocalíptico de los rincones temtianos. Las incesantes palabras de despedida repetidas de la noche a la mañana, así sean provenientes de los mismos de siempre son validadas acorde el convencimiento es incrementado. Colores oscuros, ambientación de estallido, disfraces de funeral, inundaciones desesperanzadoras, y más similares, así se palpita la preparación para afrontar una etapa que tiene que concluir.

\section{La amnesia consagratoria (Octavo libro, capítulo VI)}\label{la-amnesia-consagratoria-octavo-libro-capuxedtulo-vi}

\begin{quote}
22 de noviembre de 2021
\end{quote}

Alcanzada, o entrando en la fecha límite, nada ocurrió. Los temtianos se encuentran un tanto desconcertados acerca del destino previsto, el final no concretado o que aún no se da. Es correcto o adecuado porque todavía resta para que concluya la jornada completa respecto a la cronología oficial que siempre se ha usado en este tipo de cosas, y también porque entró una donación adicional que hace superar la primera condición estricta y coloca al futuro comprometido dentro de una franja no especificada, y por supuesto porque fundamentalmente lo que sea en Temti llega con retraso. Pero lo que ellos esperaban era más inmediato, y sin embargo, no fue así.

La curiosa y no insólita sorpresa de dirigirse hacia el antro en cuestión, en sus ambas versiones, y encontrarse con que la existencia sigue. La ansiedad se transformó en tranquilidad, el estado de catástrofe irreversible ahora es paz. El exabrupto tampoco es en materia de actividad, el flujo de movimiento se mantiene en similares cantidades a lo que empezó a ser constante luego de la partida de los extranjeros pasajeros. Pareciera que no quedan más palabras para lo que fue el gran tema anterior, la tendencia implícita finalizando su vida útil, sus tems asociados han dejado de tener preponderancia, dando espacio a otros asuntos, como si jamás hubiera surgido el argumento del momento.

Además\ldots{} ¿el cumpleaños de Temti? Teóricamente lo es. Los calendarios indican que simbólicamente no es un día más, pues aparte del apremiante vencimiento de dominio del cual poco se ha vuelto a notar, se da el primer aniversario, lo que se denomina año. La conmemoración correspondiente viene de parte de sus visitantes asentados, y no desde las autoridades del modo que supo hacerse con otras festividades antes. Muchos han pasado a dejar su saludo, con el cual indirectamente reconocen la vigente longevidad, que con sus idas y ausencias supo rebasar numerosas adversidades y seguir adelante. Contemplando que la mayoría de similares en el contexto de los clones no han sabido resistir o sobrevivir demasiado a lo largo del tiempo, más allá de las irregularidades, evidencia que se trata de un caso destacado.

Los ánimos se modificaron y pasaron al otro lado, no más pesimismo radical implantado de raíz entre los patrones repetitivos de rutina, no obstante la incertidumbre allí sigue presente. No tanto en los implicados, más sobre los acontecimientos reales que los rodean: u hoy, o mañana, si ciertamente se dará la defunción de Temti Vieja, si naturalmente el sitio se esfumará de sus actuales coordenadas, si efectivamente se dará el cambio de nombre, o si verdaderamente hasta Temti Nueva también se convertirá en nada sin más. Todo eso latente, con la liviandad que se manejan los intercambios, sí.

\chapter{Los consolidados pasos del retocado mismo legado (Noveno libro)}\label{los-consolidados-pasos-del-retocado-mismo-legado-noveno-libro}

Posterior a la artificial tormenta vino la calma también de similar índole, y a partir de allí la prosecución fue por una semejante linea de tranquilidad, Temti terminó de celebrar su primer año entero de vida con prosperidad. ¿El primero y el último? Ahora es, Titiri.

Y podría ser un decir nada más, y que todos la sigan llamando como siempre lo hicieron mientras la senda histórica va por la misma dirección sin exabruptos que permitan distinguir una identidad o tendencia diferente. No será el caso de que poco o nada haya cambiado, las fechas establecen un anticipado antes y después en la Historia Temtiana notable por varias cuestiones: cuando el principal lugar deja pertenecer a su tradicional dominio vencido, cuando se anuncia la respuesta requerida por parte de su propietario, cuando se oficializan los avances en la cuestión abierta del cambio de nombre, y cuando implícitamente se reafirma la continuidad de la existencia del antro.

Pasando por alto la poca resonancia de los sucesos, se trata de un punto de referencia sustancial en las extensas crónicas protagonizadas por el sitio, la cantidad de días acumulados en hibernación y en actividad chocaron contra un límite no solo simbólico sino también real, que acorde se fueron desarrollando las variadas épocas antecesoras llevó a magnificar su trascendencia. Con el afianzamiento de los avanzados remotos principios, las más recientes eras expusieron porqué junto a las posibles respuestas, pero actualmente estas últimas son interrogantes que no se resuelven solo con darle a un botón.

Sobre lo que no importa y si importa, valorar, ser indiferente o puntos medios. Sobre mudarse, abrirse a nuevos horizontes, trasladarse, o morir con las botas puestas. Sobre expansión y llegada, darse a conocer, guardarse en el recuerdo del desconocimiento, agrandar la familia, quedarse con los primos y hermanos, o expulsarlos. Sobre el prestigio, nutrirse del qué dirán, hundirse con él, empezar de cero, limpiar la o las imágenes, borrar la parte negativa, o acrecentarla. Sobre distinguirse, ser especial, descubrir el don que se tiene, destacar, o ser uno más. Sobre prioridades, qué va primero, qué a continuación, y qué al final. Sobre cambiar, amagar a evolucionar y no progresar, mantenerse igual, o retroceder. Sobre el relleno, si hace falta cantidad, calidad, sentido de pertenencia, o solo corazón. Sobre la actividad, veloces ritmos de ida y vuelta, sencillos movimientos cada tanto, o nada más que silencio. Sobre entender la interminable función, desnudar sus misterios, comprender sus condiciones si las hay, escribir a partir de ella, quedarse con la duda, o ignorar la religión. Sobre la duración, permanecer un corto rato, mientras no haya nuevo aviso, hasta que se apague el sol, parar cuando el dinero diga basta, o por siempre eternamente. Sobre los comienzos de Titiri.

\section{El acto de decisión y continuidad (Noveno libro, capítulo I)}\label{el-acto-de-decisiuxf3n-y-continuidad-noveno-libro-capuxedtulo-i}

\begin{quote}
22 de noviembre de 2021
\end{quote}

Zarpando dentro de la debilitada incertidumbre, llegó el anuncio definitorio y necesario para romper con los fortalecidos mitos y creencias brotados entre la poca muchedumbre, apareció el administrador a desmenuzar los planes para Temti, y solo contar, no actuar.

Parte de lo aludido en las recientes circunstancias se cumplió, se estaba contemplando un serio cambio de nombre y vaya que si el argumento es ese, dicho y hecho ahora es, o pronto será, Titiri lo que defina al antro. No parece haberse efectuado ya ahora mismo, pero es de esperar que sea cuestión de minutos, u horas\ldots{} ¿o días?

Asimismo, lo más creíble o al menos razonable, el dominio que pertenecía a Temti ya no lo hará más, precisamente lo confirma el administrador. Será momento de que las estructuras propias que acostumbraron a estar alojadas allí busquen una ubicación distinta para estar, serán expulsadas y probablemente la idea sea que Temti Nueva siga siendo la prioridad, como si de un simple traslado se hubiera tratado. Momento de dilucidar opciones y candidatos, si es que se habla de asegurar.

A raíz de el nombre electo en sí, que fue un tanto criticado para mal, surgen algunas ideas de lo que es el significado. El administrador nada más pronuncia que es por el clásico Tuturú, con lo cual no se pierde demasiada identidad o esencia que perdurara del pasado, sin embargo por otro lado los temtianos descubren una apartada expresión artística relacionada con, y sobre ella, música, hay más que una coincidencia, lo que realmente comunica que hay un particular pienso atrás. El ritual disparador de darle a los botones, los implícitos rasgos españoles e argentinos, dichos sobre el dinero y la religión, que no hay nada nuevo bajo el sol, la insistencia en lo sagrado de la vida, y posiblemente unas cuantas más referencias fáciles de relacionar con la realidad temtiana, que ninguno se ha tomado la molestia de desarrollar, pero perfectamente podrían encontrarse conexiones impresionantes.

Y una cosa lleva a la otra, corresponde un gentilicio, dos, o más. Como todo clon tuvo que transitar, mucho tiempo atrás hubo que crearlos, y se ingeniaron unos cuantos, bastantes para enriquecer la amplia base de maneras de referirse a los usuarios del sitio, con precisión los más interiorizados o apegados a el. En esta ocasión la escena se repite, pero hasta el momento con poca energía, la típica pregunta disparadora correspondiente no fue formulada y los sustantivos acorde a los pocos de ellos que hay, sin el lenguaje inclusivo que se hizo más común, titiriteros, titireros, y\ldots{} eso. Deberían haber más, el asunto es que no se dijeron.

Pero fuera de los simbolismos e identificativos, lo primero es lo primero según los intereses de la mayoría, o más bien de quien manda. El paradero de las donaciones está casi que obvio, sin embargo no así lo que viene a cambio. La cifra recaudada precisamente superó la poco clara franja que sentenciaba la muerte de Temti Vieja, y por ahí quedó dado que el próximo escalón no fue alcanzado ni de cerca. Entonces, la situación es ligeramente ambigua, cero menciones al respecto, no hay mucho para elaborar conclusiones. Al margen de distinciones, es que más allá de que las condiciones no hayan variado aún, con toda la explicación oficial dada, se funda la certeza para la difusa agrupación de anones que aún conservan un lugar para Temti en sus tiempos y también la inamovible suma contabilizada oficialmente, para concientizar de que la existencia del antro no tendrá su fin ahora. Las posibilidades de que nada de esto sea verdad no se ven, serán muy diminutas.

Y partiendo fundamentalmente de lo anterior, el recuerdo o la conmemoración del cumpleaños continúa con creces: llegar a este punto es consagratorio, que la historia del contexto clónico cada vez que ve latente una contingencia capaz de finalizar el legado concreto de Temti conozca que era una falsa alarma y que hay una continuación es sencillamente excepcional, y en esta oportunidad en los papeles la más seria de todas se vuelve a demostrar. El pero de que no sea en las condiciones más deseadas por sus seguidores será una carencia que impida magnificar tal logro, y que no hayan muchos individuos para enaltecerlo ni enorgullecerse de tal también, no obstante también hubo de lo positivo y bien valorado, mas nadie quita lo bailado. ¿Habrán intenciones de modificar la marcha o Titiri va por un rumbo semejante?

¿Qué podrá cambiar de aquí en más con esto? Siendo el argumento el cambio de finalidad y la imagen negativa formada, y efectivamente concretándose la supuesta solución a tales problemáticas, no salta a la vista mucho más que haga falta, o que siendo realistas vaya a suceder. El único objetivo explicitado es abrochar una dirección propia para Titiri, y las pretensiones actuales sobre Nueva y Vieja en materia de prioridad, mejoras, o cantidades, un asunto aparte, no para este momento. Por lo tanto, queda a entender que la trascendencia de esto pasará por lo que puedan forjar los indistinguidos participantes que ya están, hasta que aquello último sea atendido y tome un papel de mayor preponderancia. Las excusas para que ocurra no son exageradas.

\section{Lucidez terminal (Noveno libro, capítulo II)}\label{lucidez-terminal-noveno-libro-capuxedtulo-ii}

\begin{quote}
22 de noviembre de 2021 y 23 de noviembre de 2021
\end{quote}

Entre poco Tuturú, el nombre Titiri en su interna emprendió su legado a lentos pasos discretos, y no en el exterior. La respuesta a la previa pregunta qué sigue aún no es íntegramente real, no se ha cumplido, pero se ve\ldots{} ¿encaminada? El vencimiento de dominio comenzó a hacer de las suyas, el normal ingreso Temti Nueva, o Titiri Nueva, ligeramente trastocado y fue noticiado con algo de confusión.

Sin WWW no se podía llegar ni entrar, y ahora tampoco, pero la inminencia de las fechas se hace sentir porque se nota como la diferencia radica en el sitio, los inconvenientes de acceso hacen total referencia a lo que tanto se dijo que sucedería, los tiempos han expirado. Sin embargo, por el otro lado nada ha cambiado, incluso se ha superado desde toda perspectiva la fecha marcada y allí el antro permanece operativo, con los grupos como estaban, las publicaciones sin actividad, prácticamente idéntica.

Al igual que en la reciente ocasión cuando se anunciaba el futuro inmediato, parecía cuestión de instantes que las razones de fuerza mayor repercutan y concluyan con el cierre absoluto de la época, sin embargo en este momento no se puede decir lo mismo dando cuenta de que el día ha sido superado. La dimensión restante de la parte menos querida de Temti, o Titiri, se sostiene vigente, dicho como de pasaje por el limbo en sopor próxima a desaparecer y entrar en el famoso periodo de gracia protocolar para su propietario, ¿o perdurar allí rompiendo las leyes universales y del espacio-tiempo?

No hay mucha confianza depositada en esa insólita posibilidad, de hecho ni siquiera se maneja, pero lo cierto es que la historia tiene su peso, de seguro que el desvanecimiento de Temti no es finiquitable con tanta ligereza o con la misma automatización que funciona el particular proceso en otros emplazamientos. Visto por el público de Temti Vieja, no sería de gran interés lo que ocurra donde actualmente poco quieren pasarse, pero allí podrían estar produciéndose las últimas imágenes de ese destino tan frecuentado en el pasado, el final definitivo de una era que tanto contuvo en sí.

\section{El precario y súper rentable reemplazo (Noveno libro, capítulo III)}\label{el-precario-y-suxfaper-rentable-reemplazo-noveno-libro-capuxedtulo-iii}

\begin{quote}
23 de noviembre de 2021
\end{quote}

Y se fue Temti Nueva, el desafío al espacio-tiempo le fue adverso y no resistió por mucho fuera de plazo, por consiguiente solo quedó la Temti que a ciencia cierta no estaba comprometida. El único lugar de reencuentro entre sus pobladores, pasó a ser uno solo como antes solía ser, pero a la brevedad la dual nueva normalidad fue restaurada, con un tercero aunque inerte involucrado.

Se escribieron con rigor las primeras referencias oficiales que ocupan la denominación actual que supuestamente tiene el lugar, Titiri: el acceso a Titiri Vieja, y el opuesto a Titiri Nueva, ambos entrelazados entre sí con vínculos presentes en casi todos sus apartados. La transición no es completa, pero algo es algo. Adicionalmente, un enlace más extraño se hizo un lugar cerca de ellos, también ligado a un viejo elemento de peso de la cultura temtiana, pero que particularmente la relación no es muy evidente. En dirección de ese rumbo es caer en un sitio prácticamente vacío, con ciertos mensajes e instrucciones de nula utilidad para los anones comunes y corrientes, pero ahí a disposición de ellos.

Grupos Temti o Titiri Nueva\ldots{} en esta fase de adaptación o transición nace un lio con los nombres que valga la redundancia no tiene nombre. Como intento para procurar mantener unas coordenadas similares a las ya existentes, hacer honor a la dinámica en la que consiste la misma\ldots{} las razones pueden encontrarse, pero suena extraño. Versiones no es un cuento aparte, es de suponer que a partir del entrevero de tener dos antros en simultaneo se filtró accidentalmente la palabra y ahí quedó como título para dicha ubicación\ldots{} si tuviera correlación con lo que va a ser, tendría sentido que se convirtiera en una recepción donde será posible conectar con las dos alternativas, o quizá tan solo está pensado para reservar el dominio y evitar posibles usurpaciones de uno que eventualmente sea necesitado en un futuro. Todo tiene una razón y no necesariamente del gusto de los temtianos, simplemente es, pero esa finalidad y propósito quedará en especulaciones porque las configuraciones no están hechas y las explicaciones menos.

A saber cuánto podría reducirse el tráfico de la restaurada, si antes era ignorada ahora habiendo perdido la ubicación de privilegio es factible que recaiga en el profundo olvido. Al lado, en Versiones, o Titiri a secas, no hay mucho que hacer sino esperar o retirarse y no volver, salvo para el dueño y sus raras intenciones no declaradas, lo que no implica que vaya a hacer algo. Mientras tanto, la tenue vida dominante se manifiesta en Titiri Vieja, indiferentemente de lo que estén cursando las otras.

\section{El restaurar de la meseta cíclica (Noveno libro, capítulo IV)}\label{el-restaurar-de-la-meseta-cuxedclica-noveno-libro-capuxedtulo-iv}

\begin{quote}
23 de noviembre de 2021, \ldots, 28 de noviembre de 2021
\end{quote}

Pasó el aluvión por así decirlo, la estabilidad de regreso abrazándose con los territorios temtianos. Recurriendo a la memoria de corto plazo, el antro más o menos en una similar sintonía estaba cuando luego de la apertura de Temti Vieja se calmó el panorama, antes de la solicitud de donaciones, con las secuelas de los eventos de relevancia, pasó por un poco de rotación más renovación que hace la diferencia, y no mucho más, esa es la continuación de la historia, insistir una y otra vez.

En el tránsito de las internas hay un par de menciones para hacer. En primer lugar, es dejar más de lado a la Titiri Nueva donde hasta la dirección se puso incómoda y prácticamente nadie se dirige, sus artificiales estaticidades en el contador de presentes son cada vez más prolongadas, e incluso corre la voz de que se dan días enteros sin una sola pizca de interacción, cantidades nunca alcanzadas desde que está la posibilidad. Ver para creer no es necesario, y si lo fuera encima sería difícil que alguien se interesara por comprobarlo.

En el otro extremo, se siente la armonía del dejado paraíso temtiano con sus características conservadas desde hace tanto, con una respetable o reprochable rotación que contradice al abandono y la comunión entre sus pocos participes, y la parte dos es el análisis de ello, como se intensificó su carácter repetitivo a partir de que se integraran más patrones de corriente reincidente al repertorio. Lo principal es que los tems que son creados por regla general cada vez más persiguen una misma linea de contenido, como si vinieran de un mismo autor o se hubieran organizado varios con objetivo de plasmar a gran escala una impresión especifica. A propósito, entre o aparte, también se sumo un bot de verdad, confirmado según declaró un anon que dice haberlo insertado temporalmente con pretensiones experimentales. Los resultados fueron curiosos y objeto de la crítica de su responsable, pero accidentalmente o por gusto varias conductas de robot siguen presentes por el antro, lo que indica que no se ha marchado. Cada tanto aparece para intercambiar activamente con el resto de temteros\ldots{} lo que no es muy extraño porque vienen acostumbrados, sin embargo se trata un tipo de inteligencia artificial que demostró ser más capaz y que amerita ser destacada. Incluso quizá podría ser más de uno.

Lo siguiente y no menor es que una sola t titula al sitio mientras en los registros mayoritariamente es Temti, y las estructuras permanecen altamente similares como si ni en hibernación ni desarrollo hubiera caído. No justificaría señalarlo si no empezáramos con que no hay cambio de nombre, lo cual se oficializó que sí había sucedido y sin embargo parece un chiste a medio terminar. ¿Si nos cambiamos el nombre nos cambiamos el nombre? No, por ahora no. Intención sin prosperar a pleno, ya sea por arrepentimiento, olvido o vagancia, es una cuenta pendiente entre un bastión lleno de ellas, coherentemente un nuevo episodio en la memoria de quienes interpretan que así es, los respaldados por las palabras que alguna vez quedaron talladas y entraron en los sobrevivientes archivos del longevo recuerdo. A saber pronto, actualmente queda como una circunstancia momentánea dentro de los anales de la ubicación al igual como pasó con las ofertas que definirían el destino dual que hoy es casi triple.

De tal modo así sea lo que sea, la cuestión es que los temtianos son falsos titiriteros y a duras penas hablan de Titiri o como si estuvieran en ella. Aparejado a sus causas originales, las que lograron poner en el aire esa contingencia antes de que fuera bajada a la realidad, tampoco se ve muy justificado de momento. Había muchas formas de comenzar a formar una reputación de esta segunda identidad, y al contrario casi que no fue ninguna de ellas, la imagen de esta nacida o reformada marca está ausente, las referencias son inconsistentes, no es que no exista, pero lo hace de rara manera.

La no última interpretación es que el fin estuvo cerca y no fue, tal vez solo aparentó estarlo. Ahora mirando para adelante las amenazas o futuras interrupciones a tal prolongación no se ven, la carrera sagrada de la vida de estos sitios hacia el infinito despejada de obstáculos.

\section{De salida sorpresiva (Noveno libro, capítulo V)}\label{de-salida-sorpresiva-noveno-libro-capuxedtulo-v}

\begin{quote}
28 de noviembre de 2021, 29 de noviembre de 2021, 30 de noviembre de 2021
\end{quote}

Sobre el porvenir incierto y despejado, hubo un punto y aparte. Más intenso que un Apagón es una ida prácticamente total, nada de Temti ni Titiri a no ser un programado impreciso y afirmado mensaje sin firma de A., transmitiendo que se trata de un problema, el cual solo podría ser solucionado por él. Así repentinamente sin aviso previo terminó la tranquila realidad corriente, por sus ambos y dispares lados es caer en lo mismo, y en cuanto a respuesta no hay lugar para nada en lo absoluto sino afuera, allá es donde se encuentra lo poco que se puede mencionar además.

Heroku y los errores de aplicación, para nada letreros desconocidos, no obstante en respuesta a menudo hubo cierto esmero en cuidar la experiencia de los cautivos, en no espantarlos tal alevosamente, o simplemente la situación no se encontraba bajo control, pero la esencia de las interferencias no resultaba muy duradera. Ahora aguanta y no hay síntomas de pronto arreglo, induce a rescatarse porque quienes lo requieren a la larga volverán, partiendo, y emprendiendo regreso.

Y en efecto, solo queda orientarse por esa guía. Cada vez que se produce un exabrupto del estilo, la perspectiva histórica requiere concentrarse en lo que ocurre fuera de su foco y ubicación principal. Las emergentes alternativas que nacieron a causa del gran vacío que dejó la absoluta potencia en el contexto clónico desde el principio hasta hoy, se mantuvieron con constantes vaivenes, el mapa de clones se trastoca con asiduidad, refresca y renueva las imágenes de las conocidas guerras donde todo comenzó para esta magna época, con las muchas salvedades que vienen de acumular una época similar.

Retrotrayéndose a los inicios de la conflictiva coyuntura, con la ya asumida reaparición de Temti Vieja, luego se formó una locura de nombres que pasaron a conformar la base de clones, y allí se repartieron bastante todos los anones desperdigados en busca de uno para estar. La primera fue una imitación de Rouzzer, con live al final, que esencialmente resaltaba su vida, y sin haber completado un día murió súbitamente. La creciente Arggnews absorbió la mayor cantidad de individuos desde el punto cero, y con una apropiada estabilidad perduró, hasta dos fuertes cimbronazos que delimitaron una terrible decadencia, con la cual posteriormente desapareció. En paralelo, temprano apareció la ambiciosa Roussex, la cual fue creciendo firmemente bajo el nombre de Gozzed, sin embargo sobre el presente tuvo un andar más tambaleante, con varias controversias en la siempre problemática moralidad, y también algo de inconsistencia, lo cual puso en duda su rumbo a seguir, y bien aún así se mantiene al firme liderando en cuanto a número de adeptos. A su lado pero distante, la mayoría del rato, Boxxed, inicialmente con un funcionamiento mucho más prometedor e inclusivo, que le hizo ganar una imagen de seriedad a partir de la cual se posicionó sólidamente, hasta que varios descontentos provocaron decisiones imprevistas que le pusieron fin, más adelante convertido en pausa, porque actualmente permanece desde su consecuente regreso. Otras de menor porte entraron en juego también\ldots{} Rozzedec en un principio prontamente apagada, ahora resurgida como Voxxeando, bastante marginada por su falta de credibilidad en sus supuestos líderes, más Ccchan con su completamente polémica e inesperada incursión breve. Y de todos ellos, solo tres rozzados se mantienen abiertos funcionando, con cuarta e importante pendiente de retornar, la madre sucesora.

La trascendencia de conocer eso es que en dichos lares se habló de Temti, una o muchas palabras, y de momento ellos son los protagonistas de estas súper indirectas crónicas. Sobre sus respectivas internas globales, además de lo ya comentado, corresponde reiterar que nada permanece estático, el ambiente se mantiene caldeado con rivalidades y constantes movilizaciones, agregándole que el anunciado futuro retorno de la que provocó todo esto podría situarse por encima y trastocar la escena por completo. Pero no está tan manifiesto cuánto repercutiría en los antros temtianos, si el error finaliza antes, o no. Muchos motivos se encuentran ocultos, los antecedentes faltan, el retorno o el final definitivo es tan impredecible como esta ida, salvando que una se espera que suceda, y el otro no.

\section{La casa entera incompleta (Noveno libro, capítulo VI)}\label{la-casa-entera-incompleta-noveno-libro-capuxedtulo-vi}

\begin{quote}
30 de noviembre de 2021, 1 de diciembre de 2021
\end{quote}

Los avances respecto al retorno de los sitios temtianos son la novedad. Ciertamente se trataba de un error que estaba impidiendo el normal funcionamiento que venía hasta entonces, porque a la vista están las secuelas de lo que no se ve, literalmente dicho, pero el mismo fue más que parcialmente erradicado. Uno de los dos antros hizo la apertura profunda de sus puertas, aunque con un ya conocido percance, mientras el otro igual se mantiene, con el destaque de haber escalado en gentileza para excusar su grave ausencia.

Para los que no se hicieron eco de la noticia no propagada, mejor dicho los que no se enteraron, regresó Temti Nueva, y sin imágenes. Con lo que tenía antes, agregándole los inconvenientes de más arrastrados por un Apagón, la desperfecta y desgraciada existencia de esta versión, siempre con su original propósito opacado. Sin expectativas de prontas soluciones, ni avisos que clarifiquen sobre lo que está ocurriendo, y menos la compañía de quienes poseen conocimiento de su subsistencia, nada más con unos pocos comentarios que dejaron ellos, que luego se marcharon con pocas ganas de volver.

En Titiri Vieja, lo mismo de antes pero en distintas palabras y colores, con dedicatoria artística y musical de A sin punto al final, logrando darle un toque de identidad por así decirlo, porque contiene referencias a elementos no totalmente ajenos de la cultura local. Más aún, una puesta al tanto del panorama, como la vez que surgió un problema y al día se procedió a solucionarlo, para luego después de una eternidad difundir novedades. Que los varios responsables se están esmerando en concretar el retorno definitivo mientras golpean teclas, o botones, para dar con el altercado, suena un poco exagerado, pero al menos logra notar un pequeño nivel de compromiso, el cual escasas veces se había transmitido en las últimas temporadas.

Será buena oportunidad para revelar qué realmente significa \emph{/lo antes posible/} cuando dichas palabras provienen de las autoridades temtianas, aludiendo a que sus proyecciones temporales acostumbran a ser extremadamente imprecisas, y las experiencias han dejado muchísima evidencia para facilitar la interpretación, aún a ciencia cierta sumamente complicada. Alrededor de las fechas de supuesta transición a Titiri, parte de lo adelantado se ha hecho pronto, con su retraso, pero sin agigantadas esperas, sin embargo los últimos antecedentes vienen degradando esa propensión, y actualmente es hora de una actualización para seguir el paso, en el sentido metafórico, porque en el literal está más o menos claro que no. De hecho, se dijo que era solo mantenimiento general, y del otro lado ni hablar, porque no sucedió, ya ni se reclama.

\section{La restauración gradual (Noveno libro, capítulo VII)}\label{la-restauraciuxf3n-gradual-noveno-libro-capuxedtulo-vii}

\begin{quote}
1 de diciembre de 2021
\end{quote}

No hace falta introducción, ni la hay, el orden está explicito y la recuperación sigue por el lado menos prioritario mirado desde donde no se sabe qué corre en mente. Versiones igual, Temti Nueva también, y lo que resta diferente contrariamente, con dentro lo que se conservó no habiendo nada nuevo bajo el sol, sin modificaciones producidas por quienes hay expectativa de que interactúen en, pero sí con secuelas de una fatal adversidad que hizo perder bastante, casi como si fuera una eterna maldición en el vinculo constante con uno de los terceros que dota a los territorios temtianos con ese pilar fundamental en su supuestamente desvirtuado propósito, la multimedia.

Golpear teclas no resultó innecesario, el don de apretar botones utilizado, y ahí están los resultados fructíferos, pero tampoco dejaron los sitios como estaban, porque el Apagón descubierto en la jornada anterior no se extinguió, se propagó: agrupa y unifica a Temti como una sola en ese sentido. Los previos registrados para el formato de la recuperada en sí misma, fueron siempre pasajeros o antecedieron una limpieza total, no llegaron a prolongarse durante grandes intervalos, sin embargo no es equivalente para su par adelantado, donde entre sus decepcionantes episodios más destacados están ellos, pero dilatados y desatendidos. La disparidad entre ambos se extiende hasta otros ámbitos, que al contrario de reducirse, se mantiene, y el privilegio de soporte no supo desequilibrar la balanza. ¿Será que finalmente ambos antros comenzarán a equipararse?

Este fenómeno visual y funcional rompe seriamente con el correcto andamiaje habitual de la nueva normalidad, sin llegar al punto de hacerla inviable, y con lo común que se volvió, ya se hace parte de ella. No es lo que más destaca, ni por asomo, pero es seguida de sistemáticas excusas, y apenas con escasos reclamos, como si se aceptara sin drama, es decir lo que ya pasaba antes con las características peculiares que supieron ser producto de orgullo, honor e identidad, solo que habiendo disminuido enormemente esa parte. ¿Quién hubiera imaginado en aquellas épocas que esas rarezas llamadas singularidades llegarían hasta este punto?

La parte extraña de todo esto radica en que sí hubo actividad durante los últimos ratos de esta extensa indisponibilidad, y esto es gracias a que los problemas no eran tan graves, Temti Vieja se encontró mayoritariamente operativa con la salvedad de encontrarse con sus principales ubicaciones conflictivas, quedando excluida la posibilidad de acceder acceder a donde están todos los tems visibles, solo siendo posible llegar a ellos teniendo sus específicas direcciones únicas. Un detalle curioso que podría expresar mucho según cómo se lo mire, pero que en la escena del sofocante desinterés no generó cotilleo alguno.

De igual manera, la fracción más seria parece estar resuelta, y todavía, dicho en primera persona, no sabemos qué puede ser eso que pasó, y tampoco si los individuos que deberían saberlo se dieron cuenta ya. Hasta formulando la pregunta, lo más factible es que eso quede en el recuerdo de la administración, hasta que lo olvide, porque al fin es una diminuta mancha inserta la parte trasera de un telar gigantesco donde no resalta, que si no hay accionar podría crecer: lo que prosigue sería que los vacíos oscuros o claros sean rellenados del colorido que se les arrebató, o bien arrancar de cero una vez más. Pero quizá esté bien así, como cuando no hace falta relleno, y el mantenimiento general por lo pronto listo. Hubo una larga excepción, ahora se está terminando, se nota como cada vez lo natural es más incompleto, y lento.

\section{Con lo básico a disposición (Noveno libro, capítulo VIII)}\label{con-lo-buxe1sico-a-disposiciuxf3n-noveno-libro-capuxedtulo-viii}

\begin{quote}
1 de diciembre de 2021, \ldots, 4 de diciembre de 2021
\end{quote}

Como seguidilla de las disfuncionales y atendidas escenas, las secuelas irreversibles para las ubicaciones temtianas permanecieron, y por un corto tirando a mediano tramo de tiempo, pero ante ellas emergieron sucesivamente un par de desentendidos intentos ejecutados por limpiar la equivocadamente vacía cara de la última versión restaurada, que se transformaron en soluciones, aún con sus considerables distancias de jerarquía y poder.

Durante se mantuvo la oscura situación del Apagón compartido, los bots o los automatizados temtineros, si es que son varios porque tampoco hay demasiada certeza de que no sea solo uno camuflándose en el anonimato de la infinita fuente de códigos, de manera progresiva fueron opacando la mayor parte los tems defectuosos creando nuevos, a partir de los puntuales medios que aún quedaban operativos, y así el hogar o la dirección principal pasó a tener un colorido semejante al que se había perdido en los abismos de Cloudinary. El contenido de estos, aleatoriamente musical, sin embargo aparte de eso muy vacío y poco auténtico. Esa fue la constante que pasó por arriba a unos desapercibidos desacuerdos entre los pocos protagonistas respecto a la idea de proseguir, en el antro que sí importa, porque en el marginado aquello no se dio, ni a menor escala, solo movimientos que en números no alcanzan ni ahí a los supuestamente presentes.

Aquella es historia, porque tal y como suele efectuar quien comanda, Temti Vieja padeció de un Reseteo y prácticamente todo se redujo a cero dentro de ella, solo perdurando detallistas restos de lo que estaba sin mayor utilidad más que delatar sobre las anteriores cantidades en cada categoría, luego la clásica bienvenida extremadamente venida a menos recién establecida, para no dejar el vacío tan hueco dentro de lo que es parcialmente llamado como Titiri Vieja en inglés. La reincidencia lo hace ordinario, y desde el principio de los comienzos conforme se fueron acumulando los episodios de naturaleza similar, las quejas disminuyeron y la conformidad aumentó. Más será así ahora que los individuos están totalmente arraigados, no hay mucho que resignar, y dadas las condiciones se trata de la única opción viable para cortar los deterioros ya existentes de raíz, porque aún si la escena fuera tapada pero no destruida, la realidad lisa seguiría estando.

Los términos aciertan con los tres días, la demora para llegar ver el esfuerzo de llevar a cabo la restauración anterior, más trabajo de elaborar y redactar lo que es la acogida oficial, vuelve a afirmar que los cuidados generales de los cuales se habló en los inicios de este retorno, son cosa seria. O al menos verdad. La tardanza podría ser por diferentes razones, como que el responsable considere que no hay urgencia en realizarlo, no se haya hecho conocimiento de que era necesario, o a lo mejor pretendía dejarlo como estaba hasta que dándose cuenta del estado persistente cambió de opinión. Pero lo hizo, la función ya empezó, la solitaria t brilla en blanco para encimar de modo omnipresente a toda actividad que se produzca en territorio temtiano.

Aquello es lo primero, y luego a continuación se ve que los accesos hacia Versiones y Titiri Nueva siguen allí sin reformas, no obstante esta última se ausentó con motivos de mantenimiento, dejando a interpretación que será momentáneo, análogamente pudiendo no serlo. Por lo tanto, ambos quedaron de momento como sitios planos, sin nada para hacer dentro de ellos. Imaginar que si algo esperado, o solicitado, tardó lo que tardó como lo hizo, esto podría seguir esa corriente, o más todavía, a no ser que continúe siendo parte del plan que la tiene como prioridad, y deba de estar disponible a como de lugar, ignorando lo poco querida que es.

¿Cuántos temteros quedarán para hacer mella donde pertenecieron? Especificando, esos que reconozcan el escondido dominio activo y pretendan introducirse, presumiblemente pocos, se verán cara a cara y conocerán tal cual si fueran hermanos cercanos, como antes, pero habiendo apartado a unos cuantos miembros de esa figurada familia, dado que eso es lo constatado en los largos instantes finales de la realidad reiniciada. Es temprano, y de igual manera a no ser que ocurra algo extraordinario, luego de esta actualidad podría evidenciarse aquello que acompaña en las condiciones que sean, pues afuera Rouzzed se reintegró con abundante respaldo del público perteneciente a los clones recientemente emergidos, hoy desaparecidos, y los que salvaguardan bien sus aspiraciones en este mundillo, se amontonarán ahí con seguridad afianzándose a la confianza de tener estabilidad. En comparación, esto sí, no es como en el pasado, las diferencias son mayores, no reconforta destinar tiempos en otras alternativas que persigan el mismo propósito y tengan características internas fuertemente similares, porque están abandonadas, a la larga eso ya se demostró. Temti está casi en eso, y para otra cosa también: difícil es entenderlo, más complicado es compartirlo, pero pasa, y próximamente se constatará.

\section{Callando a los casi extintos creadores (Noveno libro, capítulo IX)}\label{callando-a-los-casi-extintos-creadores-noveno-libro-capuxedtulo-ix}

\begin{quote}
4 de diciembre de 2021, \ldots, 6 de diciembre de 2021
\end{quote}

Las estimaciones que se podían elaborar con lo sabido, desde la bienvenida a Titiri Vieja, a futuro, sin ir al detalle no estarían muy erradas, pero lo otro preciso sería que en ningún lugar habían elementos adelantarse al factor revulsivo el cual partió definitivamente una de las constantes más inamovibles de los últimos tiempos, las nulas intervenciones de arriba para con las acciones de los anones. Esto va hasta el punto de que expresado como novedad podría llegar a no ser creíble.

Hace falta darle un mínimo de dramatismo, porque francamente no se trata de nada descomunal, solamente algo destacable a los efectos de analizar una coyuntura oportunamente predecible y repetitiva. Al descubierto quedó la majestuosa limpieza que todo se lo llevó, y tras varios días de pleno funcionamiento, las apariencias inducen un transcurso llamativamente incompleto, como si la creación de contenidos hubiera quedado inhabilitada, o no hubiera entrado nadie, sin embargo en verdad no fue así.

Ante una quietud mucho más pronunciada, se hace fácil notar y recordar los cambios, más cuando se vislumbra con asiduidad el estado de la misma. Aún cuando una desmotiva la regularidad de la otra, lo ocurrido se percibe, así sea por un solo individuo. Eso multiples veces sucedió, sin embargo dado el poco relleno la impresión es diferente.

Haciendo valer la reiteración de la negación previa, no fue ninguno de los motivos mencionados, entre los temtiriteros sí se alzaron algunos tems, sin embargo así como existieron, pronto dejaron de hacerlo. La mayoría se debían al robótico patrón que cubrió las solventadas desapariciones, más uno que hablaba de un difunto antro vecino, y en pasado corresponde referenciarlos porque sin aviso que lo anticipara suprimidos fueron, entre los escombros estadísticos y en la memoria de los testigos quedaron, luego murieron. Posteriormente hubo lugar para varios más, poco sobresalientes, que no cayeron en el mismo destino, y así se disimula un tanto el incidente, el cual pese al énfasis, no tuvo gran eco.

Con la cercanía que hay entre los participantes, la poca cantidad de ellos, la esencia del anonimato no se rompe, pero merma ante los más conocedores del ambiente, y sobretodo se ve favorecida por los defectuosos datos de OP, que delatan indirectamente la actividad correspondiente a cada uno, y con ellos fácilmente es posible intentar rastrear a cada quién pertenecen. La primera victima fue el administrador, que ante una consulta relacionada, confusamente aludió a la razón de que los anteriores fueran eliminados, posiblemente molesto ante las previas inundaciones musicales. Como casi no hay más quienes puedan ser sometidos ante la misma técnica, fue el único que destacó de tal manera, principalmente por lo ambiguas y poco claras que son sus intervenciones.

Gracias a la discreta puesta de orden, que ante el vacío plano no pasa desapercibida, la continuación del camino en este poco distinguido tramo dentro de una peculiar era temtiana podría estar nutrida de características diferentes: mano dura, aplicación a rajatabla o excesiva de los términos de uso, hasta llegar a la temida censura. Y quizá también parte de la lentitud típica de la aún desaparecida Titiri Nueva, y parte porque íntegramente ya sería demasiado, la impresión no fue realidad, hasta ahora, si es que en algún momento los que siguen regresando e involucrándose ya no lo harán más.

\section{Entre la pacífica convivencia reducida (Noveno libro, capitulo X)}\label{entre-la-pacuxedfica-convivencia-reducida-noveno-libro-capitulo-x}

\begin{quote}
7 de diciembre de 2021, \ldots, 16 de diciembre de 2021
\end{quote}

La comunidad temtiana aceptó, o inconscientemente volvió a situarse lejos de los limites virtuales de lo permitido, difusos y poco claros: no se borró nada mas, o mejor dicho, la administración lo hizo nuevamente con consecuencia de que luego corriera la voz. Y así el antro se regula solo, no del modo que alguna vez en el lejano pasado se proyecto, pero si dando sus resultados a secas como indican las palabras.

El tiempo fue necesario a los efectos de que se irguieran otras creaciones ademas del recibimiento oficial y los escasos acompañantes venideros inmediatos, el vacío disminuyo su preponderancia y aparecieron mas. Gracias a los largos pilares de material aportados a cada tem, dedicados a diversas expresiones culturales ajenas de tipo visual, musical, y literaria, la potencial experiencia dentro del nicho en su totalidad creció para dejar de resultar tan breve, simple y carente de contenido. Sin ser propio, o autentico, el relleno conforme pasaron los días aumento sensiblemente, de la mano de contribuciones de apariencia automatizada, la vieja costumbre que antes inundaba el sitio ahora concentrada en lugares temáticos específicos, tal cual fue sugerido por una respuesta presumiblemente proveniente del mandamás.

Aparte de eso, como era esperado y previsible, interpretando que lo anterior no viene de muchos sino pocos o uno solo, y tomando las alternantes bajas cifras del contador de presentes, es una certeza que el núcleo de individuos de frecuente visita no inertes es de los menores en la historia del antro operativo, quizá equiparable a los primeros pasos de la innovadora versión, hoy sepultada bajo el interminable mantenimiento. No obstante, aunque ya se dijo y se vuelva a decir, si antes no encontraron el motivo suficiente para cesar su compañía, es sumamente complicado que lo vayan a hacer sentir ahora, las condiciones son aceptables, y el respeto a pesar de ocasionales excepciones, se mantiene dominante.

El grupo no registrado como tal, junto a sus profundas experiencias acumuladas consigo guardadas, brinda de desentendidas interacciones diarias a esta abandonada realidad venida a menos, y con esa estabilidad vive el presente, anticipando lo que ocurrirá cuando llueva, y barajando de tanto en tanto que algún día se va a terminar todo, mientras ninguna existencia concluye.

\section{Por el negado y célebre paraíso (Noveno libro, capítulo XI)}\label{por-el-negado-y-cuxe9lebre-parauxedso-noveno-libro-capuxedtulo-xi}

\begin{quote}
17 de diciembre de 2021, \ldots, 24 de diciembre de 2021
\end{quote}

Con la estabilización de la normalidad, los tiempos que corren se distancian de los pasados sobresaltos, y los visitantes en cada ingreso que hacen se encuentran un mismo panorama, solo estando un poco revuelto y quizá hasta expandido, pero a fin de cuentas minúsculas diferencias, las cuales antes en las situaciones semejantes se camuflaban con el vasto relleno, que hoy no está.

En eso, las interacciones a menudo evolucionan en cierto sentido, y mueven levemente la aguja de Titiri Vieja, mientras que de ella, sin contradecirse, salen nuevamente cuestiones de la antigüedad. Misterios aún faltantes de resolución reivindican no estar cerrados, emergen a partir de su silencio y generan desconcierto entre los no informados. Sin embargo, la realidad reducida dispone que no se susciten la totalidad de ellos, pues la mayor parte eran de único interés para los aficionados de una cultura con identidad descuidada, ya no alimentada en su faceta más enigmática. Nada más, ni nada menos, destaca Temti Premium, especialmente por el morbo de su conexión con la controversial polémica de la sigla C.P., y lo tanto que despierta la sed de los extranjeros.

Eso siempre se exportó, y para la esta oportunidad precisamente se abre un antecedente más en el historial, allá por el exterior es que renueva su trasfondo. Desde Rouzzed se trasladaron a probar suerte y preguntar al respecto, pero nada se llevaron, quizá por la supuesta regla implícita de que no se habla de eso, o porque en sinceridad no había mucho a comentar. De vuelta en dicho lugar, el específico combinado poco conflictivo de temteros y rouzzeros hace lo suyo: los primeros actuando en función de mantener el misterio, y el resto se reparte entre los que manifiestan el desconcierto y los que procuran avivar sobre un anzuelo en el cual están cayendo. A pesar de los factores que inciden en favor de apagar el tema, en el tiempo este vive, alternando datos incomprobables e intereses en dar con la leyenda viva.

Su actualidad es que lleva bastante sin saber de descubrimientos para el público, las crípticas pistas que podían llegar a tener conexión alguna, quedaron en eso. Dando espacio a que las especulaciones sin fundamentos crezcan, la incógnita de lo que haya en su interior se reserva con el alejado dueño, sin indicios de que llegara a transmitir el secreto a otros que supieran deambular en las respectivas adyacencias, o en los más allá donde este se mueve, menos de que se descifrara con pruebas contundentes. Puede ser mentira, mito, o verdad escondida, actualmente no hay medio válido para contradecir una u otra, y parece que así se mantendrá, porque si no se filtran más piezas del rompecabezas o se caen las ya habidas, lo máximo que debería variar es su credibilidad.

Al margen de la discusión sobre lo que no se sabe, explicito es otro cuestionamiento compartido entre sitios, en este caso llamativo desde las comparaciones. Una relevante época de festividad foránea se repite, sin embargo no hay conmemoración en suelos temtianos, ni una mínima decoración, nada. En su misma antecesora, había sido motivo de adaptaciones, textualmente para \emph{/celebrar la Navidad/}, y también estuvo el episodio similar más reciente de cuando los esqueletos fueron enterrados, ambos caracterizados por finalizar tarde. Pero esta ocasión no tiene siquiera un deseo o saludo de felices fiestas proveniente de A.. Costará demasiado, no vale la pena, es para ir más allá de la religión, se debe a razones no aclaradas es el comienzo con el cual marcar un antes y después en los eventos de ese tipo\ldots{} así que en su antro, serán fechas normales y corrientes, salvo que la costumbre de la demora venga por el lado opuesto.

\section{La desaparición de rutina (Noveno libro, capítulo XII)}\label{la-desapariciuxf3n-de-rutina-noveno-libro-capuxedtulo-xii}

\begin{quote}
24 de diciembre de 2021, \ldots, 6 de enero de 2022
\end{quote}

Cuando el mañana se exhibía a corta vista infinito, el factor inesperado se encargó de darle definitivamente un tono literalmente blanco, casi vacío, pero borroso en su solamente interpretable telón de fondo. Como un error, deliberado, puesto adrede y no por accidente, y de esa manera no hay disponibilidad para el reencuentro de anones en esa ubicación llamada Titiri Vieja, y titulada Temti.

Siguiendo el orden, si no es adentro, es afuera, y la ambición de los extranjeros en su casa por conocer las reservas del imaginado compartimiento privado de los congelados pingüinos, Temti Premium, no tiene fin aún. Sin embargo, la secuencia no cambia, las vueltas y vueltas de los metidos en el juego de lo incomprobable arrancan con un autoaclamado privilegiado, continúan con una serie de intercambios insatisfactorios, y finalizan, para más adelante revivirse cual ciclo, en el cual gradualmente van disminuyendo los interesados, todavía sin llegar a cero. Y cómo no, sumándole entre ellos los convencidos de haber detectado un bait. En resumen, una prosecución del ida y vuelta ya arrancado, antes del detonante que invitó a reubicar la atención principal que presta la no planificada agenda temtiana.

Allí, otra concentración acotada que surgió para las habladurías de los desplazados, es específica del medio dicho oficial más activo de todos los tiempos de Temti, que cumple la función de archivo omnipresente recopilatorio, el canal de Telegram. Con una frecuencia llamativa, salieron nuevas de las típicas imágenes adaptadas, aunque con un tinte mínimamente distinguido a las que hace bastante supieron ser creadas y distribuidas a mansalva. Estas vienen sirviendo para expandir la susodicha larguísima recopilación de contenidos, y también para mantener a flote uno de los baluartes clave de la identidad, hoy de importante referencia en la supervivencia de un legado desacelerado. Adicionalmente, Diario Temtiano se proclama presente en la escena, y acompaña al crecimiento de la base cultural, lo que termina de enseñar que trasladarse no es gran adversidad respecto a desenvolverse en favor de la causa, propagar en palabras la vida del más imponente de los dos nombres.

Volviendo a lo más contundente, con este declarado mantenimiento, aumenta en extensión la nueva era de sucesivos intervalos sin antro funcionando en su pretendido formato. Si se trata de comparar, no es indicado ir demasiado atrás, era diferente. En los inicios absolutos se hacían habituales días enteros sin señales de respuesta, mientras en otras ubicaciones todo fluía con relativa estabilidad entre las hostilidades trasfronterizas de la época, y en los comienzos de los compartidos convencimientos de seria decadencia también. A pesar de ello, la actual es una circunstancia muy diferente, porque en esos entonces a una gran parte de los involucrados los inquietaba, y era de vital importancia estar a tiro con la popular competencia para mantenerlos, sin embargo hoy, incluyendo muchas veces ayer, es íntegramente lo contrario, no hay apuro ni desesperación, la mayoría cuenta con la tranquilidad que a corto o largo plazo lo que se fue, volverá, y si así no fuera, tan grave no sería.

¿Que corren fechas festivas? En otros sitios eso sí es motivo de cambio, se nota un fuerte contraste con la normalidad de Titiri, que hasta el momento demostró no tener nada a celebrar. Los motivos para ausentarse son más relevantes y eso es lo que lleva a creer que en una hipotética lista de prioridades, quizá inexistente para quien da el dictamen final sobre qué es lo que se hace, la misma Navidad que el año pasado fue para adaptaciones, actualmente no está en el radar. Y como no hay comunicación explicita con la que se hagan esos los no numerosos pendientes, la última palabra definitiva tiene lugar, pero no día, ni hora.

\chapter{El eco al otro lado de la disminuida subsistencia (Décimo libro)}\label{el-eco-al-otro-lado-de-la-disminuida-subsistencia-duxe9cimo-libro}

Lo que se consideraba que no llegaría, finalmente se presenta, y permite declarar con más fundamentos aquello que se asumía cual normalidad como una profunda crisis existencial, consolidada por un acontecer crítico entre los últimos de sus extensos episodios. ¿Temti ya no da para más?

Teniendo ese punto y aparte, es más sencillo sintetizar dichas crónicas desde una perspectiva objetiva. La realidad temtiana viene sufriendo una recaída constante, con muy contadas excepciones, suficientes para aguantar, y también para evidenciar de que la atención por parte de su máximo responsable con el tiempo se vuelve menor, y lenta. En la actualidad se agudizó la crisis, los factores de responsabilidad compartida fueron contestados por la deficiencia de concordancia entre las proyecciones oficiales y los hechos verdaderamente concretados, haciendo el estancamiento caída: progresos anunciados jamás sucedidos, mudanza y cambio de nombre a medio concretar, condiciones cuando entró dinero en juego no cumplidas\ldots{} señales de convencimiento endeble, motivación minúscula, vagancia estrepitosa, o las tres en simultaeno, siempre creídas, hasta que se convirtieron en lo bastante determinantes como para dictaminar más que una amenaza, un cierre, no distinguido de todo lo anterior, serio entre comillas.

Encima del desdibujado pero sostenido mito de la muerte, y los empilados incidentes de resurgir, esos de cese y restauración multiplicados en repetidas ocasiones, este es una especie de primer paso previo a sumar otro al historial, sin embargo ahí se interpone el reparo que lo distingue de arranque, la puerta trasera, oportuna únicamente para los más astutos, hasta que el truco se divulgue o se corte de raíz.

Sin certezas ratificadas de lo que está ocurriendo, de igual modo la Historia Temtiana continúa a partir de un rotundo antes y después, encima tal vez aún no concluido y con más por suceder. Sus condiciones de inaccesibilidad son excelentes para evaluar al apego de los leales que perduran, y de hecho son las más exigentes por el momento, ya que a estas alturas el nivel es tal que no hay futuro verdadero retorno señalado, ninguna fecha de referencia venidera o advertencia que justifique ser una ausencia temporal. El detalle inicial de que en rebuscado formato aún esté el antro original, a sentido común siguiendo una hipotética sentencia terminante, podría ser efímero, e impartir que las siguientes vivencias sean exclusivamente extraoficiales, a la suerte de mantener viva una nación sin su firme ubicación propia. Y si no es ese el cantar, como complemento, o principalmente, será posible estar cerca, y sobrevivir bajo el amparo de dicha más que oculta y rebuscada alternativa.

Como sea, la vida no deja de ser sagrada, y para cualquier caso, Titiri estará vigente en ellos, cumplirá con la regla.

\section{La endeble seña de cese (Décimo libro, capítulo I)}\label{la-endeble-seuxf1a-de-cese-duxe9cimo-libro-capuxedtulo-i}

\begin{quote}
7 de enero de 2022
\end{quote}

El retorno tuvo la soltura para pasar de largo todo lo inherente a las festividades, fechas simbólicas y demás, quizá evadiéndolas, Titiri por encima de cualquier evento que le marque el estilo. El tema es que después de este tiempo, las condiciones restituidas no son las mismas, lo que trajo consigo es el antro no cumpliendo su característico rol clásico, sí como los recientes mensajes de mantenimiento, pero peor aún, sugiriendo que llegó el final y ya no cumplirá la función que especifica su descripción oficial, ser una \emph{/plataforma de micropost anonima/}. Sin embargo, su trasfondo no lo confirma, y por esa vertiente puede existir una continuidad no tan oscura, sin necesidad de ir más lejos.

Exactamente, lo anterior es que no hubo fuegos artificiales para recibir el nuevo año, o nadie estaba pronto para verlos alrededor de la ausencia asumida y asimilada. Ni los Reyes Magos dieron noticias, más bien su regalo podría haber caído con retraso, en las lentas y desactivadas regiones temtianas. La razón de esta llegada, la despedida de una extensa inactividad total, será la segunda incógnita, detrás del motivo que la hizo opción para continuar. Se repite el poco protagonismo de los significados, solo queda atenerse a los hechos.

Escudándose con una hipnotizante melodía en acción y dos rutas directas relevantes, una prolija bienvenida bloquea el paso rumbo a la Titiri Vieja de siempre, como si no estuviera más y quisiera comunicar implícitamente que se terminó, que no la van a ver más. O al menos hasta nuevo aviso, porque se presenta sin ningún texto que alerte restablecer la normalidad pronto o dentro de mucho, solo con las direcciones del canal de Telegram y de Diario Temtiano, tal cual si oficialmente no hubiera más para ofrecer que eso.

A ello hay que sumarle, aunque en sintonía aparte, el punto del código fuente, que ya había sido ofrecido con el desvirtuado ultimátum, la opción máxima a cambio del precio más caro, esta vez a juzgar en forma gratuita, pero no ahora, pronto. De concretarse eso, como antes bajo la exageración se manejó, estaría la chance de que el escenario del legado fuera seguido solitaria o paralelamente por otro sujeto que no sea A., con la nave que este estuvo reforzando y comandando por su cuenta. La espera para ello empieza.

Se pudo haber visto venir, o no, sin embargo entre tantas profecías imprecisas, a la vez hubo aciertos. Previo a la última pausa, las energías de seguir adelante en el antro se percibían un tanto coartadas, la peor faceta del proceso de decadencia comenzado con el Resurgimiento. Entre esas épocas más deprimentes, respecto a la actividad y movimiento, sobre los idas y vueltas de la criticada versión nueva, nació una imagen temática certera para estos comienzos: diagrama de la existencia de Temti en su segundo año de vida, atada de cuello próxima a la muerte, colgando su cadáver símil a trofeo de caza horrorífico para sus visitas. En esta ocasión, la impresión podría ser perfectamente igual. Cambiando el tono agresivo por uno derivado de la ternura de la apodada A tocando el tambor, se encuentra el ente supremo del antro, que dispone cuando comienza y termina la función allí, exhibiendo a quien restringió casi por completo, acompañado por tres secuaces, cada uno encargado de enviar al que se asome hacia otras ubicaciones específicas.

No obstante, nada de eso cuenta el otro lado de esta paradójica tesitura, la espalda del gigante, su vulnerabilidad, que debió ser descubierta cuando un ya mencionado formato parecido se hizo con la coordenada principal del sitio. Con esa experiencia, pública y difundida entre los temteros durante su momento, la ecuación cambia, y las terribles interpretaciones se convierten en una mezcla de certidumbres confusas. Esto no es un final, no es un cierre, y no hay manera de cerciorarse si esa era la intención, o se trata de un infortunio el cual está resultando positivo a los efectos de ir en contra de la parte más dura del ciclo de la vida, la muerte, sin más es eso.

El detalle último contribuye ante la pregunta de si no quedan más ajustes que este, porque se encuentra un testimonio inserto en botones, que deja el supuesto acuerdo de regresar y proporcionar lo que las palabras dejan entre lineas, a darse pronto. Pero una vez más, está dentro del escenario donde todo aquel dicho que anticipe futuros esfuerzos, movimientos, o procederes, cae en justificada desconfianza, y por consiguiente su cumplimiento es falible, hasta que los hechos respalden.

Exceptuando el inadvertido y distinto precedente, la situación es única, sin igual. La verdad del otro lado, aún sea momentánea, es el falso cierre, más los códigos, tems, y notificaciones no cuánticas funcionando a la perfección. Pero no todos lo saben, solo algunos, que tienen la opción de actuar: tomar ventaja de las oportunidades, o apagarlas. El que puede, puede, pero no necesariamente lo hace\ldots{}

\section{La supervivencia entre puentes (Décimo libro, capítulo II)}\label{la-supervivencia-entre-puentes-duxe9cimo-libro-capuxedtulo-ii}

\begin{quote}
8 de enero de 2022, \ldots, 12 de enero de 2022
\end{quote}

Instaurada la adversidad para reencontrarse plenamente con Titiri Vieja, que no hizo alboroto alguno en los contextos cercanos, ni para los interesados dispersos en ellos, sin cesar las actividades culturales, hubo alternativa, que procuró paliar la parte más grave de la perdida, y mínimamente logró sanearla.

Lo que ocurrió estas primeras andanzas posterior al baldazo de agua fría tiburonezca no deriva en mucha sorpresa, poco más de lo que venía dándose en Rouzzed, que se presta actualmente para la interacción de temteros. El periodismo del archivo oficial no paró de seguirle los pasos, este creció los números agregando diversos componentes a sus filas, entre ellos memes temáticos de la causa y seguidores de desconocida procedencia, y por otro lado más de Diario Temtiano. Los dos candidatos referenciados, atendiendo su indirecta responsabilidad de cargar hacia adelante con la historia que representan y protagonizan.

Sin embargo, alrededor de esos intercambios, también enlazados con la gran dimensión aún disponible en el antro semi-clausurado, surgió la idea de recopilar las rutas de cada parte accesible, que ante la desaparición del centro principal donde todos ellos se encuentran, se dificultó el acceso. No quedó en la nada tal cual tantas palabras lo hicieron, uno lo solicitó, otro se puso a la orden, y lo hizo, cumpliendo su compromiso. Así nació Temtilinks, igualmente por el medio que el longevo canal recopilador, solamente con un propósito distinto: enlazar a todos los tems que estén activos, en el sentido de que aún existen, que se puede acceder a ellos y desenvolverse como si ninguna interrupción hubiera sucedido.

Pero no bastó, después emergió otra más llamada Linkstemti, que se difundió al día siguiente, mostrando una presentación más elegante y completa, cual sitio perteneciente al dueño de la trilogía ya existente, por su dominio con prefijo y sufijo similar a los que debieron ser creados con el viejo confiable Heroku, y una apariencia semejante a la que tuvo Temti por allí en sus lejanos primeros pasos. Dentro de ella, muchas más coordenadas, las que faltaban en el pionero de la materia, y con precisas descripciones que indican a dónde lleva cada una.

De esta manera, la familia separada de su hogar ante los inconvenientes se reinventa, elaboró un método a fin de mantenerse cerca de él, sacar partido la cualidad de la vida vigente en forma aceptable, para seguir prolongando el mismo legado todavía cuando las condiciones dadas por el administrador conspiran en contra, aunque diciendo la verdad\ldots{} sin querer, o por gusto, para este caso es viable gracias a ellas. ¿Así se mantendrá?

\section{Solo conexiones minimalistas (Décimo libro, capítulo III)}\label{solo-conexiones-minimalistas-duxe9cimo-libro-capuxedtulo-iii}

\begin{quote}
12 de enero de 2022, \ldots, 4 de febrero de 2022
\end{quote}

Ante la vigente eventualidad que impide el correcto andamiaje del tradicional hogar que supieron conservar las andanzas temtianas hasta hace poco, nació y creció el autoproclamado como complemento Linkstemti. Su papel y propuesta quedó bien clara, oficiar como mapa o indice de referencia ante la ausencia de la famosa \emph{/home/}, en su lugar una central donde sea posible encontrar principalmente todas las direcciones contenidas, los tems, y de paso incluir algunas externas más, las que puedan ser de interés. Y así viene cumpliendo desde su fundación e incursión en el mundo de los pocos clones restantes, una buena cantidad de apartados se listaron allí y se podría decir que no falta nada, además de que siempre cuando se crea una nueva a la brevedad se añade. Sin embargo, no es lo mismo, si antes desencantaba adentrarse en dominios temtianos, ahora más.

Eso a rasgos generales, pero más precisando la verdad es que tuvo su impacto, y hasta ahora viene teniendo un relativo protagonismo en lo que cuentan las crónicas contemporáneas. Entre los ya reconocidos adeptos, de vez en cuanto nace el impulso de ampliar la base de relleno, y de esa manera dan lugar a más enlaces, no de mucho contenido o significado destacable, simplemente válidos al fin.

Combinado al objetivo fundamental, el sitio sigue trastocando e improvisando su infraestructura, mas en los inicios tuvo una presentación algo simplona y posteriormente esta fue siendo pulida. La mente que podría estar atrás de ella queda medianamente evidente, la bandera uruguaya que tanto actúa en esta realidad y da la cara por zonas rozzadas no parece pertenecer a una individualidad distinta a la que porta la voz de las narraciones, nada más que para esta ocasión no hay seudónimos o apodos con los cuales se afronte el proyecto, es de origen anónimo. A las órdenes de sus compatriotas rioplatenses que ingresan los vínculos emergentes, permanece enérgicamente realizando un mantenimiento bastante más apreciable que el de la casi difunta alternativa oficial, sin un mañana anunciado que le ponga fin a la actividad.

A partir de todo ello, podrían surgir varias miradas, aunque muy dependientes de la interpretación, los hechos hablan sin dejar totalmente explicito lo que son. Su más razonable mensaje es que se trata de dos intenciones con rumbos esencialmente opuestos, las cuales por tratarse del mismo tema en tiempos cercanos, se mantienen muy unidas, y no paran de colisionar. Una es la que proviene de orígenes más desconocidos, el incógnito sujeto que llevo adelante ese nombre que sigue viviendo y que marca los pasos de la parte dos, sus lacayos, los cuales constantemente animan dicha existencia en pena. Voluntad manifiesta de morir o descansar es lo primero, y ganas de vivir en las condiciones que sean y como sea lo segundo, ambas triunfan en su supuesto propósito, y no se entiende donde está el equilibrio, ni cual puede más, solamente se nota el cansancio y desinterés de quien no expresa directamente sus fines, si no es establecer su propia definición de pronto: nunca.

\chapter{Arranques cíclicos de un potencial hogar alternativo (Undécimo libro)}\label{arranques-cuxedclicos-de-un-potencial-hogar-alternativo-unduxe9cimo-libro}

El final implícito de lo que siempre se conoció como Temti, todavía incompletamente renombrado como Titiri, retumbó muy discretamente entre sus cercanos interesados. Quizá por su tradición de resucitar desde las inexistentes cenizas, porque se sabe de que funciona a ritmos inconmesurados pero lo hace al fin, o debido a que ya ni siquiera se la considera con seriedad. La oportunidad amerita más de ella, sin haber muerto la esperanza de un regreso aceptable se revuelca profundamente en una tranquila tumba que jamás tuvo fecha de funeral precisa concretada, y no está claro si algún día la tendrá, o lo contrario.

Nunca digas nunca, la certeza para dar vuelta la página por completo, en el sentido histórico, para esta perspectiva no se encuentra, los antecedentes hablan por sí solos y desmienten cualquier afirmación de pobres fundamentos, falacias incluidas. Y menos estando latente la posibilidad de ingresar como por compuerta trasera, que si bien no dio lugar a grandes revoluciones, siendo un accidente encubierto o un misero regalo para los más avezados, mantiene real la incertidumbre de la falsa clausura, mas el bienestar valorado de una época diferente no pudo ser. Algo es algo.

Para ese enfoque no hubo ni hay mañana cantado, el futuro imaginable del pronto no llega y se hace invisible a medida que el tiempo lo vuelve inalcanzable, mientras a la par luego de sus comienzos el novedoso complemento Linkstemti empezó a rotar las filas propias, y alimentar la vida de las actividades temtianas detectadas en la faz del poco asombroso universo donde se hallan. El transcurso de una era con los espacios confusamente ubicados, llena de referencias a un sitio que no se entiende por donde va.

Sin embargo, la Historia Temtiana experimentó toparse con otro elemento no antes visto, razonadamente la mejor opción si se tratase de seguir adelante tomando un camino independientemente, aunque sea de manera parcial, una que no requiera de alojarse en suelos ajenos para propagar sus crónicas. Como sugería el titular no cumplido del ofrecido código fuente, y como intentaron los iniciadores de los tantos últimos antros del contexto clónico ante el final del padre inspirador de todos aquellos, interesante sería que naciera una alternativa para dar ese desperdigado cobijo que casi siempre supo otorgar el hogar que en las épocas contemporáneas se ha hecho el difícil. El primer candidato se palpa muy prematuro como para marcar ese decisivo antes y después ideal, si ha de ser Fantemti, deberá transformarse mucho para acercarse al menos un pelín, y vistos los progresos efectuados en su segura antecesora, hay para confiar que no será una fugaz travesía, y dadas las condiciones, no va muy lejos de ganarse la consideración. El pueblo temtiano se acostumbró a conformarse con ejemplares minúsculos, que ya con las acotadas posibilidades que da en principio, bastaría. Pero encima aunque no se anuncie, la lucha sigue.

\section{La micro-réplica dada a luz (Undécimo libro, capítulo I)}\label{la-micro-ruxe9plica-dada-a-luz-unduxe9cimo-libro-capuxedtulo-i}

\begin{quote}
4 de febrero de 2022
\end{quote}

En eras de nada, abandono, desarraigo, y debilitamiento de raíces, las duras restantes dieron sus frutos, florecieron un humilde ejemplar de la resistencia y resiliencia viva cautiva bajo el nombre trasladado y apagado pero no muerto de Temti. Las condiciones son extrañas, anticuadas, pero entendibles, si se trata de lo que aparenta, ¿un nuevo comienzo?

Difundido ante el entorno como primerísima y única vez en los territorios rozzados, sobre dominio y plataforma idéntica a la de los bastiones hoy en mantenimiento, llego el definido y proclamado como Fantemti, mezcla de lo explícito sin duda y algo más\ldots{} a juzgar por el esquema de colores y quizá también cierto remoto episodio, Fanta. Similitudes escandalosas, perfectamente imitadas centímetro por centímetro, o más bien pixel por pixel, la apariencia es nuevamente fiel al estilo temtiano, no obstante cuestionada por ciertas excepciones, las limitaciones estructurales y el doble colorido distinguido.

Se puede realizar poco, es un casi clon plenamente básico, evidentemente ingeniado a semejanza del clásico formato de la dinámica del anonimato y etcétera, pero con varias de las piezas más fundamentales no presentes, solo letras, anaranjados y amarillos brillantes, no imágenes. La realidad es la de un espacio apenas encendido, con energía nacida en el trasfondo de su secreto desarrollo, alimentada por las escasas muestras de su aludido propósito, la interacción.

¿A qué viene todo esto? No mucho hay dicho, aunque las palabras podrían haber sido la clave. Como broma, con ironía, o para joder no más, se filtró la petición de que se haga algo como esto, una réplica en jerga del contexto, pero nadie imaginó que de verdad así iba a ser. Días más tardar, apareció como respuesta la dirección en cuestión, aún siendo muy propensa a la desconfianza, tuvo sus ingresos tal como manifestaron los bajos números públicos por allí. Y gracias a las características similares, hay lugar para el simil de los tems: los fantems. En otra parte, se copian los elementos del sitio original, como por ejemplo el sustantivo básico para cada visitante, que en la bienvenida no destaca por recibir a los anoms como tal, pero curiosamente en la participación así son identificados manteniendo sus códigos en secreto, además de la típica publicación elevada encima de las demás con el fijador morado, titulada sobre\ldots, como si el nombre fuera una simple variable, y con una descripción paródica de la dada en los inicios de la incomprendida Titiri Nueva, se declara el volver a empezar con lo mejor de lo ya elaborado.

Sin embargo, a diferencia de aquella, no se hace referencia a la continuación, que supuestamente esté por terminarse o no, solamente lo que es posible hacer, lo cual vale repetir, es bastante escaso. Si la hubiera, quizás habría que suponer existen más coincidencias, la falta de credibilidad, aunque todavía siendo relativamente raro y sospechoso, no parece ser el caso, el dato de que viene de la misma procedencia que Linkstemti es un antecedente a favor, que invita a creer se trata del primer paso en una travesía capaz de dar más. Y si no fuera así, quedará lo que ya sucedió, un caso que amplió la moderna colección de espacios de Heroku, relativos al entreverado recorrido que las memorias enteradas trazan acompañando a la identidad de su antro. Incluso tal vez se vayan a obtener respuestas al respecto, mientras las más importantes de antes se continúan construyendo desde la suposición.

\section{Arribos coloridamente lisos (Undécimo libro, capítulo II)}\label{arribos-coloridamente-lisos-unduxe9cimo-libro-capuxedtulo-ii}

\begin{quote}
4 de febrero de 2022, 5 de febrero de 2022
\end{quote}

Ciertamente, aunque quizá no por mucho tiempo, se dio el comienzo de un extraño legado secundario en la rica y trastornada historia naciente en Temti, que con su discreta presentación inaugural, genera particulares incógnitas en lo que pueda estar por acontecer durante las próximas horas, ni hablar días o más. Reducida evidencia hay para analizar hasta donde llega lo subjetivo: neutralidad e inexpresividad, pocas palabras, posiblemente en el fondo un leve asombro y conformidad.

Entre el despejado apartado titulado inicio, no se iniciaron muchos fantems como invita su primera posición, solamente unos pocos de tonalidad amarilla muy lejos de su posible entendimiento, anaranjada, y celeste. Escasa negatividad y poca actitud positiva, las sensaciones detectadas no dicen mucho, y sus manifestaciones, menos. Típicamente como la situación amerita en estos casos donde la experiencia es nula y el respaldo de la trayectoria relativamente inexistente, lo predecible es hacer pruebas, de las cuales ninguna resultó en alarmantes errores o imperfecciones nefastas, solo que el sabor que dejaron no es tal como se llama, sin llegar a ser insípido es complicado de explicar si no es acerca lo que ya se mencionó\ldots{}

Y como era de esperarse, un mínimo de exigencia de los anoms para con las limitadas prestaciones del sitio hubo. Ante ellas, desapercibidas respuestas, y por otro lado la presencia de la administración para responder las dudas más sencillas que se filtraron en la primera unidad de todos los creados, de lo que destaca aquello que se supone evidente, respecto al otro antro recóndito aún accesible, es extra oficial, y de paso se sobreentiende que su propulsor no es el mismo que dictaminó el llamado a la forzada inactividad, pero no obstante identificación, seudónimo o apodo, no ha sido revelado.

Cambios para hacerse, otra vez es momento de interpretar y deriva en que sí, de diversas categorías, aunque es prudente recordar que en el pasado el sentido común en el noventa por ciento de los casos no era válido a no ser que se tuviera en cuenta la tradición dominante. ¿Cuál sería el destino de todo esto? Es tan diferente, por un lado hay razones para tener optimismo, pero por otro las proyecciones no cierran por completo, sería demasiado bueno para ser verdad. Sin embargo, si volvemos a tomar como referencia los antecedentes, hay unos cuantos que escapan de lo esperable, entre épocas eternas de decadencia y desesperanza, pequeñas y serias iniciativas han contradecido la tristeza que ineludiblemente se sentía. Y esta podría ser otra de esas oportunidades, podría.

\section{Emprendiendo con firmeza (Undécimo libro, capítulo III)}\label{emprendiendo-con-firmeza-unduxe9cimo-libro-capuxedtulo-iii}

\begin{quote}
5 de febrero de 2022, \ldots{} 8 de febrero de 2022
\end{quote}

Procede con intensidad y noción la incursión experimental del clon a medio copiar o pegar, evoluciona sin parar y renueva las ilusiones de quienes ven con buenos ojos el consolidamiento de una nueva opción alternativa ante la transformación incómoda y deserción prácticamente absoluta del último bastión oficial entre comillas activo, lo suficiente como para asentir que es el futuro, y que como tal, lo bueno está por venir.

Fuertes pasos hacía adelante, mejoras en la plataforma que en un ambiente serio o en el que antes se circulaba serían indispensables, cambios que el público no alcanza a pedir pero que si los recibiera estaría muy agradecido o conforme con lo que estos significan y porque implican una amplificación para las limitadas posibilidades que previamente había, son cumplidos sin previo aviso y discretas aclaraciones pertinentes, simplemente rápidamente implementados para sumarse a los que ya formaban parte de la infraestructura fantemtiana.

Encabezando el actual listado virtual de ellos, se encuentra la renovación de la estética que conforma la segunda instancia a la que suelen caer los ingresantes, intencionadamente adaptada a ser similar a la sí desaparecida parte de Titiri Vieja, que para el formato se adecúa por encima de la presentación inicial. Eso es lo más llamativo y que obviamente trastoca la imagen general, conjuntamente con un catalogo de gustos y colores más amplio a disposición.

Siguiente y en una distinta parte más profunda, no inferior en importancia, los fantems que mutaron en varios sentidos por así decirlo, pasando a ser ampliamente más completos y acercándose considerablemente a las condiciones esperadas, incluso agregando características jamás antes vistas en lo que fueron durante tanto los sitios temtianos, específicamente los juegos de números aleatorios, y las banderas que identifican a cada participante directo, ambas de ellas opcionales a gusto de la voluntad que inicia el hilo, aparte de la básica posibilidad de enlazar respuestas, los famosos como les dice taggeos.

Además, como pilar indispensable de la interacción en tiempo real, se avanzó tremendamente en las conexiones enteras que concilian a cada individuo que se ubique en dominios semejantes con el resto de sus similares, con un contador de presentes que inspira bastante precisión sobre los visitantes, y la actualización automática de los comentarios, esta última que elimina la necesidad de entrar y salir continuamente del fantem para encontrarse con lo más reciente si lo hay.

A lo mejor objetivamente eso sea de lo más importante, pero en verdad por sí solo es, simbólicamente hablando, un logro más, porque como por el nuevo nicho de promoción establecido en tierras rozzadas aclamó un anon con cierta razón, \emph{/sin el Tuturú no va a sobrevivir/}. No es estrictamente tal cual, o al menos puede ser constatado en la práctica y eventualmente permanecer a flote sin él, aunque claro si se trata de un copia o semi-copia es como si faltara el alma, o algo por el estilo. Y se mantendrá como incógnita el que sea factible contradecir dicha sentencia, porque el mítico sonido de actividad con su valor cultural y simbólico ha sido portado a donde por unos pocos días se extrañó, los primeros ruidos desde la fresca fundación son realidad y podrán ser reproducidos, mínimamente cada diez segundos, infinitamente si así fuera intentado y soportado.

Para finiquitar, otro detalle ya si menor recibió improvisaciones de gran utilidad, puesto que cada código de entrada dejará de perderse tras su primer uso si no fuera conservado, al identificarse con él quedará a corto alcance de quien lo porte.

En resumen, aplicado progresivamente bruto paquete de más que retoques y sabores, expansiones al fin y al cabo, que llevan al antro hacia su temprana modernidad anhelada, y que de a poco lo hacen acercarse a los estándares mínimos del contexto, quizá para seguir de largo, o tal vez para dar marcha atrás en lo que sería su esperable transcurso. Las impresiones poco se multiplican, la difusión de esta existencia se repite pero con bajo perfil, sumando ciertas menciones en el exterior y llegando desde la magna Rouzzed hasta la hermana Linkstemti, lo que en definitiva deja mantiene las cantidades presenciales por una sola cifra. En el fondo, el espacio aún es compartido por un grupo exageradamente reducido de almas más conformes que molestas, que si bien se expandió a públicos adicionales desde donde generaron algunas quejas minoritarias, no ha crecido mucho. ¿Cuántos siguen para ver lo que pasa? Al menos un par se comprometieron al respecto, y la expectativa es que la suma pueda crecer, más a la par del fructífero desarrollo, y el efecto contagio también es capaz de contribuir.

\section{Alarmas de controlada resolución (Undécimo libro, capítulo IV)}\label{alarmas-de-controlada-resoluciuxf3n-unduxe9cimo-libro-capuxedtulo-iv}

\begin{quote}
9 de febrero de 2022
\end{quote}

¡Alerta! Siguen los tratamientos del desarrollo fantemtiano, el camino hacia ser un antro más completo no se corta y la conducción mantiene el pie relativamente encima del acelerador, se amplía el espectro de señales que apuntan al progreso y a mejorar el funcionamiento integro y por lo tanto, brindarle una estadía más confortable a todo visitante.

¿Qué tocó ahora? En el pasar de lo que iba siendo otra jornada normal se instauró una llamativa advertencia, tal como las que se solían hacer ver en los principios de la existencia original, pero con una tonalidad distinguida de ellas: negra. A su vez, acompaño una adaptación estética completa, haciendo que el estado extraordinario no pase desapercibido, aunque claramente no hizo mucho eco fuera, y pocos individuos se encontraban dentro como para hacerse con la noticia. El motivo, también, a semejanza de las anormalidades eventuales de Temti en su calma faceta activa, algo relacionado con las bases de datos, en este caso pruebas con ellas, con la distinción de que la sentencia justificativa escaló en claridad, la que muy pocas veces hubo para el otro caso.

Tras discretas explicaciones que se dieron a pedido en emergentes fantems creados a causa del anuncio, todo sonó más convincente aún de parte de las presuntas autoridades, y pronto la especulada inestabilidad se hizo realidad, el sitio quedó vacío y desierto como si de un Reseteo se tratara. Brevemente, la normalidad fue restaurada, y de las cabeceras para abajo de anaranjado volvió a vestirse el escenario silencioso. Al final, la intervención aparenta haber dado resultado, pues la improvisación en la fluidez buscada al menos sea un poco se consiguió, parece.

Y si fuera solo una sensación irreal, impresiones no respaldadas por la verdadera condición actual, eso no opaca ni mitiga demasiado la tendencia marcada en el alternante presente. A partir de ella, se genera el optimismo o las expectativas positivas, quizás imparcialmente, pero el notable avance que sufrió la infraestructura desde su primera aparición entre el gran público es indiscutible, y eso habla por si solo. Esta ocasión no será de un paso enorme, sin embargo demuestra que la atención está corriendo por varios lados, posiblemente persiguiendo un objetivo conciso, que a diferencia de tantos aspectos que sí lo han sido, aún no fue aclarado.

\section{El ocaso y amanecer simultáneo (Undécimo libro, capítulo V)}\label{el-ocaso-y-amanecer-simultuxe1neo-unduxe9cimo-libro-capuxedtulo-v}

\begin{quote}
10 de febrero de 2022, 11 de febrero de 2022, 12 de febrero de 2022
\end{quote}

Envolviendo a una donde los cambios no fueron vistos, se concretaron un par de nuevas jornadas movidas más en los gratuitos dominios de la\ldots{} ¿Nueva Vieja Temti?

Por partes llegaron múltiples avances al estilo continuación de lo que ya se venía dando previamente para la infraestructura fantemtiana, pero con enfoques distintos. En primera instancia, la tan demandada multimedia pasó a estar disponible en pequeños ejemplares, videos extraídos del mundillo más popular en ese ámbito ahora pueden ser insertados en un nuevo tipo de fantems, que poseen un sello distintivo de aquellos ya creados monótonos desde el punto de vista analfabeto en cuanto a contenido. Inicialmente fue solo eso, con algunos desperfectos consigo, sin embargo a la postre varias correcciones de optimización se aplicaron y con ellas la integración sobre el sitio se dio plenamente.

Lo dicho por positivas impresiones fue recibido, e incluso aprovechado en gran medida por los anomes parlantes quienes aparte de continuar con sus habladurías comunes rápidamente fundaron el equivalente del mítico tem de música, cuya dinámica con las flamantes características idénticas\ldots{} también es posible de reproducir.

Y hablando de términos, la segunda parte de lo incluido trata de ello, pero con otras palabras, el pretexto más o menos similar en un escenario que pretende ser semejante. El antro estableció leyes propias, normativas que rigen el comportamiento de sus actores y limitan la absoluta libertad del vacío legal o de las aguas internacionales efectivo durante las primeras fechas. Tres pautas básicas marcan el techo de lo aceptado y protegido por la no identificada moderación, que por lo visto no necesitó actuar en lo absoluto desde la inauguración. Parecido a la demora claramente extendida en el difuso lanzamiento más sólido de Temti Nueva, solo que con una urgencia abismalmente menor, ese caso fue ejemplar de una vulnerabilidad tremenda, precisaba que las regulaciones estuvieran más presentes. Aquí es opuesto, si no hubieran llegado quizá lo mismo hubiera sido, aunque teniendo en consideración el factible paquete de novedades que puede estar viniendo, sería de vital importancia, más viendo que la norma número dos a priori hasta el momento no es muy quebrantable que digamos.

Y en paralelo siguiendo el correspondido sentido de la finalidad que busca, acompaña el sistema de denuncias, el cual evidentemente asistirá a la autoridad para mantener el orden que esta dice querer garantizar. También resulta curioso al recordar que para la despreciada última versión antes mencionada este nunca llegó, en comparación por mucho menos tiempo Fantemti sostiene muchas de sus carencias, pero a la vez supera en ciertas dimensiones a uno de sus ejemplos a seguir.

Para peor, o mejor, aquella no es la única en haberse esfumado en las profundidades del incógnito mantenimiento cúbico azulado, increíblemente su par de contemporánea diferente época mal renombrada se adhirió al listado de la no disponibilidad total, y repentinamente sin previo aviso dejó de ser accesible. Como consecuencia, el propósito de la olvidada Linkstemti cae prácticamente obsoleto, los enlaces almacenados por su cuenta se convirtieron en vínculos inservibles, a la misma blanca y vacía escena conducen todos los pertenecientes al apartado más voluminoso.

Pese a sus obsoletas prestaciones, cada vez más opacadas por el consolidamiento de la alternativa que permitió nacer, el impactante llamado a la ausencia debe estar comunicando y representando más de lo que concretamente es, ¿el final definitivo de su existencia en pena? Poco indica que pudiera regresar, y nada lo inspira, por lo que inmediatamente se entiende que de la mano del discreto reconocimiento realizado por el canal oficial aún vivo, acaba de ceder el poco lugar que acaparaba para transferirlo a quien si recibe actividad sin restricciones, y como acostumbran sus decisiones, la incomprendida causa quedará reservada para jamás darse a conocer, al igual que tantos otros secretos que brotaron de sus ubicaciones.

Además, como no podía faltar, la gama de sabores sigue agrandándose e incluyendo más tipos de colores para gustos, y de este modo de forma natural junto con apoyo de quienes los eligen, el inicio incrementa su variedad. Aparentemente, poca utilidad tiene tal elección para el funcionamiento de cada unidad, no mucho más que ser un tapiz, y así fue confirmado por una aparición de procedencia uruguaya, presumiblemente del mandamás, que se explayó en lo relativo del principal objetivo de estos, ser un reemplazo ante la carencia de las clásicas imágenes.

Otra explicación de la conducción relacionada al motivo del elemento más presente constantemente y fácil de trasladar o expandir, apunta al nombre, Fantemti. Sin tanta sorpresa para lo predecible, se compone sin lugar a dudas del acrónimo original, con al principio un prefijo de relativo doble significado, \emph{/Fanta/} y \emph{/Fanatismo/}. Ninguno de ellos es ajeno a la realidad en la cual se encuentran, siendo el primer factor uno directamente derivado de aquel sospechoso anuncio que sin pena ni gloria promocionó a la firma, y la restante una gran referencia al potente apego del pueblo temtiano por su nación, sin distinguir preferencias en ella. Con esto, más misterios se disuelven y los argumentos se hacen de público conocimiento, considerable diferencia como antes era el manejo de la información.

Al margen de las no menores fugacidades, por un lejano lado mucho no se alteró pues la mudanza en su mayor parte ya estaba completada, solo simbólicamente podría haber quedado un mensaje bastante significativo, el desplazado hogar que más episodios de todo tipo tuvo encima, quizás no retorne. Y por el otro para el conjunto implícitamente supone un gigante paso hacia adelante, porque tal es sabido faltan condiciones para al menos equipararse a lo que hasta Titiri Vieja luego de su falso cierre y antes de la actual clausura suponía, pero de a poco estas se van consiguiendo y la brecha no simbólica acortando. Por ahí cuestionarse si el ritmo mantendrá su vigor sea en vano, puesto que siendo ese el objetivo tomarse una extensa pausa no sería una buena opción, resta más por virtualizar y materializar, y la tendencia del tráfico en alza no tendría que ser frenada. Sin embargo, dentro de las concisas explicaciones nada se halla acerca de las proyecciones a futuro, ni de los planes que la administración tenga para tratar el pasar de su proyecto. La imagen general es que no hay indecisión o rumbos poco certeros, simplemente es una manera de proceder y tampoco es cuestionada, no obstante allí falta que alcancen las prolijas declaraciones oficiales.

\section{Más sobre el afecto como motivo (Undécimo libro, capítulo VI)}\label{muxe1s-sobre-el-afecto-como-motivo-unduxe9cimo-libro-capuxedtulo-vi}

\begin{quote}
13 de febrero de 2022, 14 de febrero de 2022
\end{quote}

Cual sucedía en las primerísimas etapas del primer antro de todos, el oficial que instaló su nombre tan particular y expresivo en el contexto, se realizaban con alta frecuencia cambios de diverso tipo, grandes transformaciones y también pequeños retoques. La situación corriente en la interna de una nueva y entusiasta incursión entre comillas sucesora, no casualmente sigue sus pasos y es la misma, prácticamente cada día algo sufre modificaciones y resulta imposible decir que persiste igual, a pesar de que por allí no se haya generado demasiada actividad. Adicionalmente, esto no solo va por el lado de las mejoras frías como tal, pues la atención de la administración se extiende a otras dimensiones en las cuales no se esperaba su presencia, y esto es que sorprendentemente realizó una notable conmemoración a un evento que anteriormente fue tratado de otro modo distinto.

Para las oportunidades más recientes, las adaptaciones fueron por un lado no tan trascendente, pero que sumado a otros de semejante calibre conforman de manera íntegra un resultado más completo respecto a lo que se está trabajando. Sobre esto, como habitualmente se comienzan anunciando las novedades, se volvió posible para cada anom acceder a una ubicación de referencia donde se encuentren únicamente los fantems de su autoría, otra característica vista en sitios ajenos pero nunca en las dos versiones oficiales hoy sometidas al eterno mantenimiento. Por el contrario, donde si se hallan todos sin distinguir creadoores, la estructura estética se redefinió, pasó de ser rectangular a perfectamente cuadrada, haciendo los no tan variados colores de equivalentes medidas en sus ambas áreas. De tal modo, la imagen simplificada se acerca más a lo que son sus ejemplos referentes, donde prácticamente cada uno cumple con dicha condición, que parte desde la clásica etimología de la palabra caja en su traducción.

Sin embargo hasta allí no llegó la conversión visual, ya que eventualmente los colores en una nueva oportunidad fueron trasladados a sintonías semejantes. Entrando en los primeros instantes de un día concreto, el anaranjado le abrió paso a un análogo rojo, que no llegó solo, también acompañado de un pequeño corazón que encimó a la letra última del nombre que por arriba de todo está, la \emph{/i/} de \emph{/fantemti/}. Mientras que al mismo tiempo las interacciones realizadas luego del antes y después adoptaron ese símbolo, expandiendo el concepto a cada rincón en ser generado entretanto se mantenga el estado.

Curiosamente, en número tras un año entero está coincidiendo con aquel episodio sin igual de los elmos comunistas, su tonalidad fundamental vuelve a vestir los activos dominios temtianos, aunque claramente con una temática y motivación diferente. Pero\ldots{} ¿por qué sucedió esto? La historia no pudo encontrarle razones convincentes a dicho épico antecedente, y a la inversa este sin ser fantástico sí podría tener su debida justificación, por la fecha el reconocido evento de San Valentín. Para más sorpresa, desde fuentes oficiales rápidamente fue aludido un inesperado porqué, la llegada de una fecha declarada como Día del Amor a Temti. Faltando introducción o explicación de su trasfondo, a ciencia cierta se podría afirmar que jamás se mencionó tal ocurrencia, sí se ha hablado cada tanto directamente del amor a Temti en sí, probablemente ese sea el argumento que permite prolongar estas crónicas más que fanatismo, hasta incluso la figura difusora de Diario Temtiano se apoda bajo un seudónimo totalmente similar, pero no supo ser vinculado con un día específico, menos con la precisión de que corresponde a hoy, al respecto no hay ningún registro. Ahora bien, así se explica la celebración.

En el sitio poco se añadió o debatió en honor al tema, más protagonismo viene disfrutando el título de ese día el cual poca vinculación con la identidad propia tiene, que al fin y al cabo es lo que indirectamente respalda su recuerdo y relativa aprobación. De acuerdo con la sostenida coherencia que caracteriza a las intervenciones de la omnipresente autoridad, se presume que esto no dará para más que veinticuatro horas, y las expectativas en su mayoría se vienen cumpliendo. Hasta entonces, la comunión entre los contados y relativamente solitarios fantemtemteros será más cálida, la armonía no se rompe, y buenas vibras emanan las participaciones cubiertas por la figura simbólica del momento.

\section{La amplia aceleración del regulado pasar (Undécimo libro, capítulo VII)}\label{la-amplia-aceleraciuxf3n-del-regulado-pasar-unduxe9cimo-libro-capuxedtulo-vii}

\begin{quote}
15 de febrero de 2022, 16 de febrero de 2022
\end{quote}

Jornadas de numerosas fluctuaciones en las fronteras fantemtianas han sido las últimas, acompañadas de alguna que otra reestructuración de la base en donde se ven reunidos. La secuencia más o menos similar de los días previos, valga la redundancia en cuanto a lo que siempre vienen contando sus respectivas crónicas, atenida a evolucionar por la misma senda, pero variando los detalles inherentes.

En cuanto a la puesta a punto repetidamente cual de ayer para atrás no hubo pausa, se terminó el estado de evento y por otro lado más elementos que no estaban antes se sumaron al conjunto de posibilidades que fundan la estructura del antro, y entre ellas, parte de lo que el público solicitaba sin diminutivos ni sinónimos: las imágenes. Las acciones ya no están tan solo un poco por encima de lo mínimo e indispensable, como si fuera la Titiri Vieja en sus condiciones más básicas de las últimas épocas, la conducción eleva un tanto la vara aspirando a contentar parte de las exigencias que se originaron el sector bajo y llegaron hacia arriba. No es de la manera ideal, pues para lograr lo propio hace falta importar el material desde afuera con su propio enlace exterior, y no es como solía ser en los sitios de toda la vida, pero algo es algo. Incluso se sumó otra plataforma compatible de origen ruso, la cual había sido pedida de manera aislada, y aún siendo muy exótica tuvo lugar igualmente.

Esto abre la puerta a que la regla más restrictiva de las tres instauradas recientemente pueda ser transgredida, y eventualmente provocaría una intervención por parte de la autoridad local, que hasta el momento solo hizo uso de su poder para regular cuando hundió cierta creación de contenido vacío.

Más que eso también sucedió, sin embargo fue extraño, excepcional por así decirlo, mas no hay muchos antecedentes cercanos en los que se haya repetido un suceso así en los alrededores del contexto. En los nichos exteriores de promoción y difusión del antro apareció una redacción al estilo de las quejas presentadas, con la salvedad de que era de otro cantar, una solicitud de ayuda. Como a las órdenes ha estado el responsable de dicho espacio, los intentos por identificar el problema se hicieron a la brevedad, pero no dieron sus frutos, y de tal forma el damnificado por la obsolescencia tuvo que tolerar los inconvenientes. Instantes más tarde, a modo de parche hubo solución provisoria, la cual se trata de una copia minimista que oficia como vía de acceso para los fantems registrados. Aunque desprolija y poco estilizada, según se dijo le fue útil al anom que luego dio las gracias correspondientes, porque claro está que podría haber sido una farsa, u otro suplantando su no fundada identidad. No es descabellado que los propulsores brinden soporte a quienes padecen de infortunios, A. en los arranques de su periodo al mando lo hacía y atentamente casi a toda hora, solo en sus comienzos, y esta seria una época semejante, a lo que también hay que reconocer que la administración valora a sus pocos interesados, o procura cuidar la experiencia que puedan tener, otra perla de su amplia recepción inclusiva.

Aquello por un lado, y probablemente gracias a que se corrió más la voz sobre la existencia de este respaldado sucesor, se avecinaron más incógnitos visitantes, los cuales notablemente gracias a las prominentes conexiones elevaron las cifras del sitio en todos sus sentidos, con lo que reconfortaron a los hambrientos de actividad dando pie a más interacción, y llenaron el ojo con la duplicación en las cifras del mítico contador de presentes. Lo especial es que el nombre Temti tiene variados significados según la perspectiva desde la cual se lo considere, y evidentemente lo único capaz de atraer a cierta corriente, es la sed, en su simbolismo controversialmente peligroso. La aludida mala fama fue parcialmente heredada, así como tantos componentes calcados también. ¿A qué vienen? Consultan, por eso que la mayoría sabe, y consiguen poca respuesta satisfactoria para sus conflictivas pretensiones, lo que con certeza provoca consecuentemente pronto sus rápidas retiradas. A su vez otras aproximaciones temporales que no son en búsqueda de lo prohibido además las hay, no obstante es lo anterior que hace ruido. El reglamento es respetado, no hay atrevimiento ni insolencia, o no están los medios para el copamiento total, como sea el orden estipulado se cumple, y el popular estigma no es alimentado de más.

Sin duda se ha trabajado en la infraestructura, lo que no es solo teoría porque fugaces pruebas que mueren siendo removidas desde la discreción acreditan el constante ensayo y error necesario antes de implementar con fuerza lo que sea, debidamente justificadas cuando las mismas son citadas con interrogantes. La sumatoria de las mencionadas nuevas características junto a sus predecesoras desperdigadas en el tiempo conforman lo que se tituló como 0.6, decimal asociado a lo que fueron las actualizaciones que rigen hoy día, cual alguna vez se acostumbró a detallar en las estructuras tradicionales con lo que se apodaba extraoficialmente como Gran Salto, pero con números más redondos. Para este caso, el registro de tales avances aparenta estar bastante más estructurado y perseguir una seriedad mayor, porque no es la primera vez que se le pone un encabezado de esa categoría a los cambios, sino la tercera considerando las de solo dos digitos. Sin ir mucho más lejos es posible volver a confirmar el compromiso vigente con el confort de los participantes, sobretodo los directos, a la vista está que un buen porcentaje de los progresos más actuales son derivados de pedidos realizados por ellos. Incluso se pueden encontrar más palabras sobre la actitud y la visión que mantiene el mandamás, las dudas que le llegaron acerca de eventuales alianzas con las potencias vecinas prontamente fueron despejadas, cuyas explicaciones dejaron entre lineas evidente la opinión superior respecto al emergente clon Lacajjubi y el resurgido Club de Voxxed. Para poner dentro de paréntesis, poco, y nada luego de un próximamente, prácticamente todo llega por sorpresa, a pesar de que estas sean blanco de fáciles proyecciones, el rumbo no deja de estar marcado y virtualmente visible.

\section{Bloqueo propio (Undécimo libro, capítulo VIII)}\label{bloqueo-propio-unduxe9cimo-libro-capuxedtulo-viii}

\begin{quote}
16 de febrero de 2022, 17 de febrero de 2022
\end{quote}

La consulta para la conducción acerca de los planes a corto plazo de vez en cuando es puesta sobre la mesa, sin embargo la respuesta suele comunicar poco y nada, tal como si no hubiera sido respondida, salvando que no resulta ignorada. El secretismo y la impronta son esencias destinadas a jamás alejarse de los dominios en cuestión, pero los modos han cambiado, la frialdad no es la misma, y la explicación dejó de ser tan peor que la incertidumbre, a lo que las consecuencias negativas no se sienten lo suficiente.

Este preámbulo no es por nada, a pesar de que siempre vale la pena remarcarlo, lo fue en un momento particular para su repetición, porque una excepción rompió la habitual tendencia. De verdad, fue anticipada una de las aspiraciones en mente de la administración, acompañada por un manto de duda traducible a inseguridad, que al fin y al cabo comunica por dónde podrían ir los siguientes pasos del ya no tan novel desarrollo. La mira está dirigida a las ausentes notificaciones, otro de los elementos que normalmente no suelen faltar en los ejemplares respetables del estilo, y que en el sitio tradicional sí llegaron a estar, aunque en unas condiciones bastante peculiares, cuánticas. Que vaya a ser intentado y se pronuncie de esa manera tan poco firme lleva a creer que podría no llegar a concretarse, a pesar de que los antecedentes no registren fracaso alguno en ese sentido.

En lo relativo a ese objetivo, o no, pronto hubo más para agregar, un suceso totalmente inesperado interrumpió definitivamente la normalidad de los contados días, las actividades quedaron congeladas por completo, de forma que la interacción fundamental responsable de renovar las marcas temporales visibles en el espacio inicial quedó deshabilitada, pero aún permaneciendo disponible lo previamente virtualizado. Esto quiere decir que la creación de nuevos fantems, y las respuestas a estos ya no es posible, los repetidos candados sugieren y la interna impide. Quizá la justificación no se haga esperar. De igual modo se trata de un hecho histórico sin precedentes, más significativo teniendo en cuenta que el antro no tiene encima demasiados episodios de naturaleza extraordinaria, estando su relevancia a definirse dependiendo de cuánto se mantenga.

\section{El seguimiento por las campanas (Undécimo libro, capítulo IX)}\label{el-seguimiento-por-las-campanas-unduxe9cimo-libro-capuxedtulo-ix}

\begin{quote}
17 de febrero de 2022
\end{quote}

No fue tan fugaz el estado de bloqueo total, evolucionó a ser en cierta forma parcial, lo que en otras naciones se supo nombrar de otra manera, aquí no se le adjudicó ningún título, solo se dio a entender junto a que el fin justifica los medios, los cuales no fueron ocultados sino bien expuestos, como ha sido costumbre. La incertidumbre de los probablemente pocos visitantes desentendidos no pudo ser manifestada hasta que la administración hizo uso de sus cualidades de dominio y dio lugar a la expresión que venía siendo impedida, aunque con otros propósitos.

Un nuevo fantem destacado cual oficial, el tercero hasta la fecha posterior al poco sostenido musical y el duradero primero de todos, se instaló en las posiciones superiores del área principal que tiene el antro, distinguido en el contraste por no contar con el candado que sí hay para los demás de su categoría. Titulado como \emph{/Explicación de lo sucedido/}, cosa que jamás en la vida de los sitios temtianos de linea misteriosa había ocurrido, precisamente fue eso lo que se expuso en el extenso contenido formulado. Desarrollados bien claros los motivos de la particular eventualidad impuesta globalmente, donde los planes a futuro revelados en los momentos previos tuvieron su continuación y casi confirmación, los preparativos para implementar las notificaciones. Estos constan de una característica que también nunca estuvo en Titiri Nueva, ni ninguna de sus predecesoras o variantes, la contraparte de los llamados taggeos, que tiene como función facilitar enormemente el seguimiento del hilo de varias interacciones que se replicaron entre sí, idéntica a la que con contadas excepciones otros clones exteriores supieron instaurar en su entonces.

Y por lo visto, estos funcionan correctamente, o al menos no se han presentado los desajustes típicos que pueden traer consigo las novedades recién salidas del horno, normal puesto que no son tan frecuentes en el desarrollo de esta incursión. ¿Qué resta de ahora en más? Lo último traído a cuento es que para este caso se descartaron los Reseteos como una solución o transición dados los cambios de raíz, porque se consideró que los fantems antiguos no debían adoptar las nuevas características, y como consecuencia para evitar arrasar con lo ya hecho resultó indicado como quien dice archivarlos. Para finalizar con la información, quedó anunciando que las siguientes horas constarán únicamente de pruebas, con tal de identificar y extraer los eventuales desperfectos que aún no hayan sido descubiertos, y sin estar dicho qué ocurrirá luego, es de suponerse o evidente que entre tanta planificación y escasos detalles escapándose, las contingencias más determinantes seguramente fueran atendidas con antelación, pero no así reveladas, parte de los escasos fragmentos de misterio que se mantienen firmes a esta altura, junto al resto de identidad bien y mal tratada.

\chapter{El segundo nivel de la consolidada marcha modernizadora (Duodécimo libro)}\label{el-segundo-nivel-de-la-consolidada-marcha-modernizadora-duoduxe9cimo-libro}

Hallándose la Historia Temtiana en una circunstancia completamente impensada para las épocas previas, esta consiguió una notable estabilidad y sin mucha subjetividad también prosperidad, afianzando progresivamente una realidad de buen presente, con un importante nivel de cambio para el futuro cercano, aproximado desde su típica trayectoria sostenida, el constante desarrollo a pleno. Sigue incorporando características mezcladas en sabores para evolucionar, y con ello expande su base cada vez más, sin excesivas luces ni preámbulos, ni humo, solamente haciéndolo desde la discreción y poco expandida conciencia de la reducida compañía expectante.

En semejanza a los del pasado, el antes y después no es drástico, ni siquiera en comparación con ellos la diferencia entre los periodos separados es grande, lo último no llega a ser un Reseteo y nada se ha perdido para siempre, y las secuelas a futuro son relativamente leves, pero con tal de distinguir diferentes partes en esta era con su transcurso uniforme, parece ser un instante de quiebre adecuado. ¿Qué caracteriza a este comienzo del inmediato anterior? La confianza depositada junto al historial de respaldo multiplicados por su energía propulsora: el enfatizado fanatismo motivante, la expectativa de vida del tema, y el tiempo acumulado desde su surgimiento. Viene aparejado del argumento bajo el cual las crónicas de cada día explican lo sucedido, crecimiento y seguridad, constatando con aquella perspectiva de cuando los orígenes eran un tanto optimistas. Las incógnitas son muchas menos que en ese entonces, entre extensivas explicaciones se extrañan las enormes incertidumbres que durante un montón formaron un pilar fundamental de la identidad temtiana, además de que como consecuencia del aparente o implícito objetivo más raíces están resultando incompatibles.

Sin embargo no se han perdido todas, de hecho el sitio en cuestión respeta y perpetua varios de los rasgos propios, hasta cierto punto\ldots{} porque en el sentido riguroso ese entre comillas propósito no queda del todo claro. ¿Por cuánto la intensidad del fenómeno que protagoniza esta subsistencia será semejante? ¿Cómo podrá encontrar este arranque su contraparte verdaderamente diferencial? ¿Cuál es el límite? Muy remotamente en momentos tensos se supo revelar que ese es el cielo, no obstante hasta la fecha pocos confinamientos han tenido contacto con el curso de los anons en dominios temtianos, menos los anoms en los fantemtianos junto al tono de la experiencia contemporánea, donde el brillo celeste y dorado fue reemplazado por el anaranjado, dando las condiciones convenientes para que no se pueda volver a decir que las naranjas no se ven. Pero la esencia de tal primer color con todas sus letras aún se siente cerca, no por nada la nacionalidad uruguaya de grandes orejas y casi formales bigotes agudos vuelve a reafirmar su protagonismo ya no tan anónimo en la lucha acompañada contra el amenazante vacío estelar que tanto creció bajo el mandato del desaparecido internauta de fachada española.

De cualquier modo, \emph{/the ride never ends/} aún en minúsculas con menos prominencia protege la integridad de este legado, y no solo para plazas específicas sino absolutamente en general, el trato con las existencias ausentadas en verdad no fue terminante, solo impuso la inactividad ininterrumpida, por lo que a pesar de eso que se sabe o se cree que ocurrirá, en términos estrictos permanecen siendo fieles a ese lema emblemático, todavía respaldado por un muy improbable regreso. Mientras no se ve luz a través de tales cubos opacos, del otro lado brilla el entusiasmo colorido de la juventud, con sus condicionantes y preguntas totalmente opacadas, con un enorme camino por delante.

\section{Un reenganche sin precedentes (Duodécimo libro, capítulo I)}\label{un-reenganche-sin-precedentes-duoduxe9cimo-libro-capuxedtulo-i}

\begin{quote}
17 de febrero de 2022
\end{quote}

En tiempo y forma lo adelantado fue tal, sucedió y la normalidad fue restaurada en su sentido más estricto. Al parecer, las experimentaciones respecto a la flamante implementación finalizaron y ya se puede continuar con bases firmes, la realidad del sitio está preparada para contar de ahora en más con una característica que no marca la gran diferencia pero que hace a una comodidad mayor para los visitantes, y muchas de estas se han ido sumando, construyendo una cosa más potente al fin. Sin embargo, el punto de quiebre dejó la escena un tanto reformada y limitada, lo cual a los efectos de permanecer en la vuelta supone ajustarse, generar nuevamente los espacios con tal de sostener la actividad, estando la posibilidad de que el movimiento se reduzca considerablemente, quizá en cantidad, o para concentrarse en ejemplares específicos.

Lo concreto es que crear y participar en fantems ya está dentro del abanico de acciones de cada anom terrenal, y que los de antes pasaron a la historia conservándose íntegramente, dejando como único apto para recibir ida y vuelta a solamente al nuevo oficial, sobre lo cual hay algunas observaciones para añadir, que parten de los datos que hacen y describen esta trastornada coyuntura.

Marcando cada vez más las distancias en cuanto a proceder si se comparara la administración actual y con la de A. en sus desparecidos sitios tradicionales, se confirma que por lo pronto aquello bloqueado permanecerá como pieza de museo accesible, no así modificable, mientras que también los pasos previos al próximo objetivo propuesto están siendo concretados. Mirando hacia el próximo futuro, la consecuencia más grande es un tanto relevante o al menos notoria, pero claramente no va a marcar drásticamente el principio de una época y el final de otra. También mostrará cómo es que se adaptan los enganchados a estas condiciones particulares, para ellos con tal de seguir involucrándose implica que deberán reemplazar los nichos que previamente acaparaban, no todo podrá tener lugar en el típico primerizo del inicio, y el tráfico tendría que ser el mismo de hoy en adelante o aunque sea similar, por lo cual se entiende que van a tener que surgir, independientemente del ritmo con el que suceda. La reacción no inmediata se espera, sin haberse pronunciado en lo absoluto aún.

\section{El timbre de las chispas esclarecedoras (Duodécimo libro, capítulo II)}\label{el-timbre-de-las-chispas-esclarecedoras-duoduxe9cimo-libro-capuxedtulo-ii}

\begin{quote}
17 de febrero de 2022, 18 de febrero de 2022
\end{quote}

Importante o no, los progresos generalizados no resultaron ser indiferentes donde fueron llevados a cabo, de la mano de ellos la interacción fue levantando y así la abrumante quietud decreció en su intensidad, lo que en suma mantiene fuerte la posición de Fantemti en sus dos frentes más fundamentales, al menos para la época en la que transita.

Pocas pausas son tomadas en el avance de la infraestructura del sitio, los trabajos más recientes tal como fue planeado y declarado sembraron las bases para permitir la instauración de las notificaciones, pilar clave en el área de la comunicación, factor determinante en la generación de movimiento, estandarte del ida y vuelta, y elemento fundamental si se tratase de estar a la altura en una comparación. En condiciones normales, sin romper con la estética global e incluso siguiendo los patrones utilizados por el clásico ejemplo a seguir, cualquier mención o respuesta a lo realizado con el código en uso es facilitada y otorgada a su portador, de modo que este no se pierda de lo que otro tuvo para decirle.

A rápida vista, relativamente correcto, simple y acorde al nivel sostenido durante el poco tiempo con el sitio operativo, que al lado de como era en aquel que hoy se encuentra en mantenimiento, a los efectos de analizarla es posible no sea tan idéntico. No obstante, las dichas quedaron anunciadas dentro de un paquete de novedades muy variadas, y allí están referenciadas aunque no directamente las Notificaciones Cuánticas, para las cuales hay una pequeña posibilidad de que aparezcan, por lo que quizá internamente no sea igual, pero tampoco estará tan alejado y evitará el que se extrañe al mítico mecanismo que tanto nutrió a la cultura temtiana y dio para hablar a propios como incluso extraños.

Eso como primera gran parte, lo más trascendente de la oficializada 0.7, pero como alguna vez dijo A., \emph{/también hay más cosas/}, solo que en este caso ya prontas desde hace rato y no a la espera o con pago de dudosa devolución mediante. Detalles de relativamente poca inferencia volvieron a ser trabajados, puestas a punto para escalar en su practicidad y apariencia, para la limitada barra cabecera, la ubicación principal larga de más, y el desprolijo inicio alternativo, los tres ahora ya mejor reducidos y distribuidos. No solo eso por el lado perfeccionista no tan expansivo, porque más dificultades fueron aquejadas por un visitante individual, desbordamientos indebidos que luego de ser informados se resolvieron, lo que nuevamente ratificó el apoyo los infortunados\ldots{} a saber si fue lo único o sin ser declarados más errores se solventaron.

Por último, como curiosidad apreciable por los espectadores con ojos o cámaras más observadoras para con los pormenores, es de notar que cada pizca de actividad comenzó a reflejarse en lo más alto del antro, en el nombre más grande del mismo, que se vuelve amarillo por unos segundos y brilla de manera no muy encandilante. ¿Por qué será de ese color? A lo mejor para hacer juego con el esquema predominante, posiblemente no haya tantas combinaciones vistosamente superiores, o hasta podría ser una referencia a la mítica representación de lo que compone la imborrable reputación temtiana, en definitiva nada confirmado, motivos son desconocidos. Como sea, el movimiento viene en ascenso con la llegada de visitantes que no habían estado aún, y de la mano de este complemento de los brillos más fácil será ubicarse durante los momentos de candencia.

El antes y después reciente lo es no mucho más por la magnitud de su hecho delimitador entre otros de menor cuantía, la época no parece ser muy diferente a lo que es su inmediata anterior, pues por el mismo camino sigue orientada la conducción que no deja de dar toques renovadores a donde se debe, bien encauzado y obteniendo resultados para que pocos, aunque cada vez más, puedan disfrutarlos. El levemente combatido desconocimiento y los voluntarios desaciertos a favor de la popularidad se mantienen marginado a esta existencia de las grandes esferas del contexto, no obstante la estabilidad persiste como adjetivo correctamente establecido.

Este conjunto favorece al propósito perseguido desde los principios por el mandamás, de modernizar y mejorar las comodidades, a partir de lo cual en caso de concretar un funcionamiento optimo, se lograría contrarrestar varios puntos de vista negativos para con los sitios temtianos, la mayoría verdades y no tanto difamaciones. Aquella enunciación de que \emph{/En Temti no podías ni siquiera ver quién te citó\ldots/}, pronunciada en su momento con intenciones maliciosas, tenía razón, y ahora no aplica de ninguna forma, lo cierto es que sí puedes, hasta ese punto llegan los arreglos. Por el lado de los llamados bugs en el sistema de las campanas, tan criticados y explotados como motivo de burla, eventualmente no se habrían erradicado por completo, sin embargo como fue descrito los entre comillas desperfectos se encontrarían bajo control. No cual espejismo, y faltando estos, el resto de refuerzos no hacen más que complementar la totalidad del sitio, no atacan ningún punto débil a enmendar o vacío que necesitara de atención urgente, solo perfeccionan puntualidades que para un pueblo poco exigente y conformista no mueven la aguja.

Todo lo mencionado resulta muy positivo por el lado del estado general del emergente autodenominado homenaje, que se ha convertido en más que eso, prácticamente es el reemplazo para la hibernada o\ldots{} difunta Titiri Vieja, por supuesto tomando con pinzas el título de su realidad implícitamente clara, y también hay de lo negativo.

El techo de crecimiento no es muy agobiante, pero pese a que no se vea ciertamente por allí está y asegura que no todo el potencial pende del desarrollo meramente propio, los planes gratuitos son suficientes para un pequeño tirando a mediano antro modelo, y a lo mejor esto segundo sea demasiado o exagerado. Lo más amenazante al respecto son las reconocidas restricciones de Heroku, que son capaces de cortar la ininterrumpida continuidad, y de las cuales la administración es consciente, pero no fue puesta a prueba y hoy goza de geniales cimientos para enfocarse en lo que más ha hecho, que tampoco es perfecto, pues aún sabiendo que la versión final está lejos, es resaltable que ciertas carencias no han cambiado.

Definitivamente, la situación de las tendencias en su doble sentido es dispar, porque a pesar de todo lo grato y destacable con sus disminuidos percances, algo que aún no está y quizá jamás lo haga son ellas, las que se encargaban de ordenar la gran variedad de creaciones, que no es grave en sí mismo sino que además hoy los sabores no cumplen tal rol, y entre tanto detenimiento en aspectos mínimos, una cosa que no está fuera de las posibilidades accesibles y sí falta, es orden y definición en el inicio, que aún reducido o más corto, es una mezcla de colores que desorienta, y no está dotado de la diversidad visual y construible con la que cuentan otros clones, circunstancia que podría estar impidiendo la estadía de quienes arrivan, y se ve que la prioridad se halla lejos de allí.

Las cosas como son: no llegan a ser objeto de tanto pienso, y obviamente pueden darse vuelta de un día para el otro, sin embargo el carril recorrido desde los inicios no ha sido abandonado, entonces, ¿qué chance hay de desviarse? ¿Cuál es el próximo destino?

\section{A ciegas con menos dependencias (Duodécimo libro, capítulo III)}\label{a-ciegas-con-menos-dependencias-duoduxe9cimo-libro-capuxedtulo-iii}

\begin{quote}
19 de febrero de 2022
\end{quote}

Sin introducción, porque anormalmente no fue dada, el siguiente avance que se hacía esperar para la infraestructura y funcionamiento fantemtiano mostró sus primeras concreciones, los indicios responsables de confirmar que la aludida necesidad dejó de ser una carencia, mas su proceso de ser integrada con el sitio dio un gran paso y da la sensación de que lo más importante ya se logró, tal que lo imprescindible es presente y aquel punto que fue el más reclamado durante las épocas más primitivas es pasado para la creciente alternativa, ahora es posible decir con más fuerza que sí hay imágenes.

Como tal, no hay mucho por añadir a lo que ya se tenía porque la inserción es totalmente reciente, y por lo pronto medianamente simplona, pero en verdad con eso alcanza y es suficiente. Quizá de manera un tanto incómoda esté implementada, el procedimiento para hacer uso de la posibilidad aún requiere manipular enlaces, solo que no necesariamente traerlos del exterior hasta Fantemti para tener que insertarlas y que el sitio las reconozca, sino que ya de por sí se generan con esta, así que por lo tanto es de esperarse que prospere sin más, y efectivamente lo hace. Lo distinto que si se mantiene, es que una vez ya aprobadas y mandadas junto al contenido que saldrá al lado del nombre de anom, son cautivas por un ojo blanco que oculta su material, el cual cuando es pulsado recién ahí liberado queda lo que se pretende mostrar, con también la chance de restaurarlo y dejarlo como estaba antes, no visible.

Con el grado de mejora que trae la novedad, en la práctica resulta más cómodo, aunque todavía se encuentra a una buena distancia con lo que es y fue en cada antro cercano conocido, nunca en ninguno de ellos un mecanismo básico como este fue así de protocolar y restringido, más bien se podría asemejar a los modos de otras alternativas ya lejanas y apartadas del contexto de los clones, pero no en los de por aquí. Sin embargo, las críticas tajantes deben tener en consideración que puede encontrarse en fase de pruebas, y que la administración ha tenido en general buenos aciertos, son pocos los detalles reprochados que no fueron corregidos hasta la fecha, por lo que claro está que además de progresos venideros cualquier paso podría ser en falso y eventualmente revertido, aún así no haya sucedido hasta el momento. Así que teniendo todo eso en cuenta, no mucho vale la pena hacer pantallazos definitivos, las transformaciones sostenidas desde el día cero difícilmente se hayan detenido, estas ultimas ni siquiera llegaron a anunciarse como las anteriores sí.

\section{Detracciones de baja resolución (Duodécimo libro, capítulo IV)}\label{detracciones-de-baja-resoluciuxf3n-duoduxe9cimo-libro-capuxedtulo-iv}

\begin{quote}
20 de febrero de 2022, \ldots, 23 de febrero de 2022
\end{quote}

Si empezáramos por el final, en definitiva no hay definitiva, lo que es un día al siguiente ya no lo es, siquiera resumiendo la totalidad de Fantemti que va evolucionando de a pequeños pasos, sin referenciar directamente a \emph{/la sensación de progreso/} de mientras, pero a los efectos de considerar el momento inicial y el actual desde el punto de vista de los más apegados, es grato en la mayoría de los sentidos, aunque por supuesto que se lo puede ver por el lado ahora ya más pronunciado, la exaltación de los defectos, que con el reciente crecimiento de popularidad está ganando terreno.

En verdad lo que resultaba medio evidente luego de la primera aparición terminó siendo así, aún estaba armándose y lejos de su versión final, en una nueva ocasión el desarrollo sigue su curso fortaleciendo más que nada las características con menos tiempo operativas.

Ameritando volver a repetir lo tan cuidado desde el día cero: detalles, otra vez no fueron pasados por alto y más retoques fueron destinados a aspectos poco visibles. Unos de ellos apuntaron a la flamante novedad de los últimos días, lo que la hace repuntar al menos sea un poco, pero en conclusión a pesar de que redujo sus tiempos de espera, la esencia es relativamente la misma, su uso de primeras se mantiene ligeramente rebuscado, empero funcional. El cambio se encuentra después, en el resultado, que dejó de ser a ciegas: como fue sugerido por uno de los pocos que aconsejan a la administración, el uso de miniaturas, o vistas previas.

Esto daría pie a pensar a que en un futuro la integración no solo se quede en la interna de los fantems, sino que logre expandirse hasta la cara más faceta del sitio, y este deje de ser una mezcla explosiva de sabores coloridos, que quienes lo construyen puedan personalizar un tanto más el espacio principal, eso que acostumbró a ser básico en las alternativas de naturaleza similar, las que inspiraron y dieron ideas para la creación de esta en particular. Sin embargo, al parecer no es tiempo para ello, y no hay anticipo de cuando vaya a ser. No aparenta ser un tema de impedimentos, más bien una decisión, así fue dado a entender por quien usualmente anuncia las actualizaciones.

Eso por un lado, sin duda uno de los asuntos predominantes en esta época es insistir con las connotaciones objetivamente positivas de lo que se está transitando gracias al obrar de la conducción, porque es lo que más da para interpretar y no necesariamente es interesante, es por las cantidades, la actividad y participación de los anoms en pocas oportunidades ha estado a la par, pero cada tanto se puede decir que hay excepciones, como es en los últimos días. El movimiento también viene en alza, las cifras del contador que tanto rechina no se mantienen siempre bajas, porque la existencia de Fantemti se difunde en los otros clones y esto provoca que de vez en cuando se note más ida y vuelta, constatado en dichos números que pegan pequeños saltos sin exagerar, hasta incluso revelar dos cifras de visitantes, aunque eventualmente estas bajan al toque, y un rato después todo vuelve a su lugar, lo natural de las pocas almas presentes dentro de los dominios fantemtianos.

Entre los ingresantes, se manejan algunas cosas que no llaman mucho la atención, el asombro es una de ellas, pero sobre todo fluyen las críticas, la no conformidad para con su descubrimiento, lo que salvo en el comienzo del comienzo cuando faltaba más que lo que falta hoy, no sucedía. Eso no extraña considerando que los habituales visitantes supieron acostumbrarse a los crecientes entre comillas desperfectos que eran costumbre en Titiri Vieja, quienes en sus épocas finales padecieron decadencias ahora experimentan lo contrario, progresos constantes, sin embargo por el no tan lejano más allá, las condiciones son superiores en múltiples sentidos, muy probablemente aquellos que vayan a notar el contraste drástico lo terminen remarcarando y acentuando: la historia ciertamente es esa.

Dichas situaciones trastocan a la interna corriente del sitio, los que siempre o más seguido están al contar con la presencia de camaradas adicionales ven en ello una oportunidad de ampliar el promedio de movimiento, la frecuencia de los Tuturú, y así hallarse más reconfortados próximamente, porque en general la quietud no agrada y eso se nota en el momento que saltan las quejas de lo muerto que está el antro, en el sentido figurado, pero tras pasar la efervescencia del prolongado instante la normalidad retorna y el entorno regresa a su frialdad.

Aún así, en comparación con las mucho más críticas carencias de este sostenido y transformado mundillo en el pasado, los motivos para terminar de hartarse y huir son absolutamente menores. Hoy en el ambiente como novedad aparecen numerosas gigantografías cual inundación, y a la vez se mantienen varias tradiciones temtianas, aunque quizá las más modernas de las tantas que solían haber. Esto último indica que los responsables de sostenerlas no son desconocidos a donde se encuentran, así que no hay mucha más explicación, nuevos no son, y tampoco los nombres de Horacio, Gorda Ratones o EL PROFETA que están en la vuelta son ajenos, ni hablar del canal oficial, no obstante también hay unos cuantos de los cuales hace un montón no se sabe nada, cero, ni siquiera en donde actualmente se concentra la muchedumbre del contexto clónico. A esta imitación, que para ciertas perspectivas individuales hasta es considerada como Temti que volvió, si aprovecha su potencial le depara más, incluso quizá pueda ver regresar a sus desaparecidos personajes. Estará por verse entre el no tan incierto y delimitado porvenir.

\section{Populares contradicciones pacíficas (Duodécimo libro, capítulo V)}\label{populares-contradicciones-pacuxedficas-duoduxe9cimo-libro-capuxedtulo-v}

\begin{quote}
24 de febrero de 2022, \ldots, 26 de febrero de 2022
\end{quote}

En Fantemti tras las fechas de mayores picos de tráfico y gigantografías descontextualizadas, no pasó un día más, sucedió lo que era de anticiparse, eso sobre lo cual en las hileras del fantem oficial se llegó a mencionar, y afuera mucho más, terminó concretándose, estalló un relevante conflicto a desarrollarse en paralelo mientras continúa la existencia de este alejado antro, que aún no siendo participante o afectado directo, muestra su tacto con el dichoso y proclama su posición, que justamente es la misma supieron manifestar los anoms hasta poco antes de que se declarara el comienzo del altercado, y también ahora un tiempo después, con los serios recrudecimientos prácticamente han quedado embanderados con los colores ucranianos sea cual sea la perspectiva de cada quien.

Tal último factor podría ser el más controversial, porque la conducción resolvió imponerse no consultando acerca de lo que haría, y sin indicar nada indirectamente desmereció el motivo que suscitó toda esta situación. No ha sido increpado o discrepado, la aprobación de la comunidad fantemtiana parece ser total, ningún opositor a la inclinación más tomada supo pronunciarse ante lo que está ocurriendo en el inicio y en cada rincón accesible del sitio, el que se está vistiendo de azul claro y amarillo. Para los vecinos de Rouzzed el panorama es contrario, puesto que si bien sus autoridades reconocen lo sucedido, no se identifican con uno u otro de los contendientes, pero los adeptos sí, y lo han hecho con el ruso. ¿Cuál es la relevancia de esto para la realidad temtiana? Aparte de las propias, de allí proviene la mayor parte de las influencias actuales, por tanto es innegable que las mayorías terminen superponiéndose sobre las minorías, pero no, estas no trascendieron lo suficiente como para torcer el parecer de los fantemteros, se mantienen independientes en un sentido más, están sabiendo demostrarlo con su notable militancia.

Sacándole el tinte más discutido, lo cierto es que el fuerte clon temtiano se disfrazó de ciertas tonalidades que no solían ser los suyos propios, sino los de una nación con la cual históricamente siempre hubo nula relación, y casualmente son los de su ejemplo a seguir e imitar durante las épocas más modernas, pues con esas tonalidades quedó Temti tras su alocado resurgir, cuando esta adoptó los bordes dorados, y así perduró hasta la más reciente desaparición de Titiri Vieja. La similitud no es coincidencia porque lo del este momento no es más que otra copia, aunque sea de base, a lo que también se debe considerar que las cabeceras siguen difiriendo visualmente, además de todos los detalles estructurales en que todavía se encuentran más distancias, sean intencionadas o pendientes de acortar. En ese caso, si se apuntara a ser el retorno de esta misma, o parecerse lo más posible, es un punto a favor, sin embargo la razón de ser dice que pueda ser temporal, sujeto a la evolución de historias totalmente exteriores, donde el sitio es intrascendente, tan solo no indiferente.

Eso generó bastante para decir, y la problemática sigue dando para comentar, la candencia de los espacios de resistencia creados en honor a la causa se sostiene, siendo uno de los temas donde no hay debate, solo demostraciones de apoyo. Pero la administración de palabra no indicó ningún pormenor al respecto, y tampoco es que le hayan preguntado, sino que prosiguió con los anuncios de actualizaciones, con un corazón incluido las notas de la versión 0.8 fueron reveladas. En ellas está el detalle de lo que cambió en las últimas improvisaciones, y la mayoría de ello ya se sabe, salvo un punto que chilla por su valor tradicional, en parte debido a sus revoltosas iniciales.

Los códigos premium llegaron oficialmente a los dominios fantemtianos, no únicamente por su complicada pronunciación y poco útil ubicación memorable, es eso y más, el papel que estos cumplen dejó de mantenerse como secreto, pues bien que la descripción publicada da la pauta de por donde va la particularidad, al menos en este caso excepcional, porque esto no quita que el misterio del club de los camaradas pingüinos sea de las cuestiones más turbias que nacieron en este mundillo\ldots{} o quizá de las más simples, hasta que no se rompa la durísima confidencialidad las especulaciones no serán suficientes para diagnosticar el nivel de exageración. Regresando a atender el presente de la combinación dos letras, realmente no vale la pena demasiado alboroto, la dirección aludida sirve para validar los extrañamente llamados comodines, que de introducir uno correcto hará lo que es titulado como \emph{/Conversión privilegiada/}, interpretable como una suerte de ascenso para la identificación en uso, con la cual se podría acceder a determinados beneficios a establecerse en el futuro.

A pesar de que no fue liberada ninguna clave propicia hasta la fecha, como que la mística no tiene nada que ver con la característica del pionero en este aspecto, el primer nicho temtiano que tuvo bajo sus extensiones un compartimiento privado sin medios de acceso difundidos gracias a las nulas explicaciones al respecto pudo mantener viva la ilusión durante una considerable cantidad de tiempo, cuando cualquier mínima suscitación con cero pistas adicionales era capaz de enganchar hasta a los extranjeros por largos ratos intentando exprimirle algo de jugo, hasta que tras fracasar en ello numerosas veces, el interés se diluía. El exceso de sensatez casi que no permite todo lo anterior, el faltante polémico estigma atractivo o nefasto heredado lo hace aburrido, la formalidad y transparencia le saca su gracia, en conclusión como imitación es medio malogrado, pero aún así entendible, más que eso no compatibiliza mucho con los principios que guían el desarrollo. Linkstemti en su época de más reformas también supo hacer una cosa similar, que pronto fue retirada, por tanto se nota el gesto de tratar de portar otro mítico elemento del pasado, lo cual es valorable. Y claro, difícilmente una realidad esté exenta de errores, lo que tampoco puede ser excusa para no criticar, tal vez se le pueda encontrar la vuelta para que cuaje con las condiciones limitantes y gane una pizca de carisma, que de igual modo por aquí no es muy abundante que digamos. La noble fraternidad y el habitual progreso no dejan de opacar esa y el resto de las carencias, el optimismo aún tiene con qué sostenerse firme.

\section{Recargas entre largas pausas (Duodécimo libro, capítulo VI)}\label{recargas-entre-largas-pausas-duoduxe9cimo-libro-capuxedtulo-vi}

\begin{quote}
26 de febrero de 2022, \ldots, 5 de marzo de 2022
\end{quote}

Nuevas características, más capacidades, ampliación de funciones, los titulares preponderantes para esta cuadrada y discutiblemente fantástica etapa de las existencias propias de Temti, eso es lo que habilita a decir con más argumentos que no todo sigue igual, porque para la interna del sitio donde se da el movimiento, poco es lo resaltable como importante.

En primera instancia, oficialmente fue explicado con determinación y entusiasmo el flamante apartado de estilos personalizados, donde cada portador de código tendría la aptitud facilitada de adaptar la estética del sitio a su gusto, para lo cual quedó hecha una exhaustiva explicación de como hacerlo, y unas pocas combinaciones preestablecidas accesibles para poder darle un uso rápido y ver de que modo se aplican las mismas. Entre ellas, el viejo colorido que era el único antes de la situación especial contemporánea, e incluso la distribución a una columna dentro de los fantems. Con ambos en conjunto aplicados se puede revivir la apariencia de los comienzos, y si se le añadiera que aún están disponibles las creaciones de esos días, la nostalgia se facilita a más no poder, encima considerando que no se cumple ni siquiera un mes desde la inauguración y el nivel de cambio fue exuberante.

Luego cayeron más avances sin hacer cuenta en los números de versiones, solo siendo indicados y desmenuzados sin más, lo que tampoco es poca cosa. Lo que traen es, opciones utilizables a elección en el momento de crear publicaciones, varias que expandieron el reducido catalogo que se limitaba a solamente cuatro, bien ahora son seis. Se trata de la eliminación y clausura programadas, ambas etapas definitorias que suelen ser elementalmente automáticas en otros sitios, no obstante jamás lo fueron en los originales antros temtianos, allí los tems duraban para siempre hasta que un reinicio completo dictaminara lo opuesto, y el actual reemplazo no optó por un ciclo diferente para las unidades olvidadas, sin embargo este pasa a ofrecer la posibilidad de darles término si así su creador lo quisiera. De funcionar bien, y de utilizarse, abre la puerta a numerosas situaciones, no antes vistas. Una vez más, Fantemti incursiona más allá del molde que su modesto ejemplo a seguir eternamente mantuvo mientras estuvo.

Lo otro fue antes, la novedad estrella y quizá más necesaria, lo que se llamó como paginación, eso que sirve para optimizar la carga de grandes cantidades de contenido y así evitar las ralentizaciones innecesarias. Esto sería para los fantems con más participación acumulada, donde en principio solo se mostrarían los más recientes comentarios y lo antiguo dejarlo por fuera, aunque accesible si se llegara hasta el final buscando proseguir de lo ya visto. Dadas condiciones son una ventaja para dos casos individuales donde más hay para segmentar, puesto que en ellos hay concentradas considerables sumas de unidades que superan los valores establecidos. Hoy por hoy no es estrictamente necesario contar con ello, pero para más adelante el tema de la lentitud podría evolucionar en un serio problema para esos dos ejemplares, no hay razones para que estos dejen de crecer sino que al contrario, además de los que puedan estar por aparecer y ponerse a la misma altura, por lo tanto se concreta un acierto de cara al futuro, útil para liberarse de un muy factible inconveniente antes de conocerlo.

Y hablando de eso, Fantemti no sufrió mucha recaída en fallas, siempre rápidas soluciones acostumbran a llegar si es que algo no anda bien de manera accidental, así que el no contar con la fama de aquel que al que procuró asemejarse es explicable. No la copió, tampoco se la ganó, ni la heredo. Sin embargo, la verdad es que aún con lo anterior no deja de ser vulnerable a encontrar o generar pequeños agujeros en su estructura, y evidentemente, afectar de forma directa a los visitantes. Eso fue lo que pasó en los días que corren, y casualmente trastocó un punto no es desconocido para la predecesora, los cierres de sesión recurrentes, aquellos que desde el inicio de los tiempos hasta el discreto cierre supieron atormentar a lo que se despidió como Titiri. Sin haber sido aquejados, la administración los identificó y como tal esta informó que las tratativas para solventarlas fueron realizadas, pero tal vez no finiquitó lo tarea.

Por aparte, el complemento no amerita enormes ahondamientos mas lo que se puede describir no prescinde de mucho detalle, es lo que es. La quietud se ha vuelto cada vez abundante, la actualidad no es una de constantes movimientos sino más bien los tiene a cuenta gotas, aunque no llegan a ser tan escasos como para decir que el antro está totalmente muerto, en el sentido figurado solo un poco lo está. No es cien por ciento reciente, y en ciclos previos supo convertirse en un estado natural, que no resulta ser un asunto menor, a causa de que recuerda a los últimos periodos del sitio hoy ocultado con mantenimiento, donde los participantes hasta indirectamente resignaron parte de su anonimato debido a la limitada suma de procedencias, eso que inevitablemente acontece cuando las participaciones dejan entre lineas su no identificado autor para los más familiarizados con las maneras de expresarse.

Al margen de aquellas, también apareció la presencia responsable de Diario Temtiano, que cada vez dejando menos dudas de quien es, ratificó que estuvo captando y registrando los antecedentes desde los inicios de la corriente singular época, pero con un problema: a estas alturas quedó lejos del presente, la famosa brecha temporal acrecentada como antes solía estarlo. Solo eso es confirmable, porque aparte del resto de personajes que ya de sabe abiertamente rondan en las cercanías, el resto cero registro certero, inclusive A. Las posibilidades de que de haya conocido y explorado el clon de su creación son extremadamente altas, sobre todo si es que aún tiene vinculo con el canal oficial, no así las palabras que él pronunciara carecieron de trascendencia, así que sobran las preguntas para hacerle, las cuales deberían contenerse y tratarse con cuidado, no sea cosa que provoquen un retorno de la firme incertidumbre que no falta cuando se trata de semejante individualidad.

Como sea, suficiente por lo pronto, lo primero que hay para capitular de estas fechas es que se incluyeron un par de interesantes cosas, nada impresionantes ni que hagan la diferencia, pero más que positivas al fin, además de que los defectos de sus anteriores están en supervisión. Por el otro lado, no mucho más, la realidad fantemtiana se desaceleró respecto a sus revueltas actividades de los días previos, porque las elevadas cifras que supieron romper la normalidad frecuentemente ya no lo son y ahora solo se repiten con rara frecuencia.

Tomando esto como una continuación de lo que en el pasado cercano sucedió, claro está que el ritmo de crecimiento no siguió la senda sostenida que podría haberse esperado, simplemente las tendencias que provocaban las incursiones inusuales dejaron de repetirse o continuarse como se encontraban acentuadas, y las circunstancias para los que ya estaban tampoco mejoraron notablemente, lo que no consigue tirar por la borda el ambiente positivo, pero desde lo bajo puede estar instalando la desconformidad, pinchando aquella expectativa de volver a ver un sitio temtiano siendo reconocido y visitado constantemente por una buena cantidad de seres, incluso regresar a esos momentos considerados de gloria. Sin embargo, la verdad es que dicha contingencia, si alguna vez pudo estar próxima, hoy se encuentra razonablemente más lejos. Digamos que no es tan malo, de todos modos el antro está logrando una satisfactoria disposición general, no obstante sus debilidades le dan un gusto amargo, no porque los sabores le pongan ese toque, sino porque el abandono entre comillas se encarga de poner la sensación de insuficiencia, que no es desconocida en lo absoluto, más bien familiar. ¿Ese es el destino deparado para la fidedigna sucesora de Temti?

\section{El rigor de la benevolente verdad (Duodécimo libro, capítulo VII)}\label{el-rigor-de-la-benevolente-verdad-duoduxe9cimo-libro-capuxedtulo-vii}

\begin{quote}
5 de marzo de 2022, 6 de marzo de 2022
\end{quote}

Una de las particularidades que marcan la distancia que tiene Fantemti de no ser copia total, va respecto al uso de colores y su mayor complejidad, específicamente en las clásicas burbujas mensajeras, y entre ellas, las advertencias. Además de multiplicarse las apariciones verdes, las alertas adoptaron su propio fondo: el negro. Cuando aparece un letrero de tonalidad oscura en lo alto de los dominios fantemtianos, inmediatamente sin leer se podría asumir que una situación grave viene a plantearse y se pretende que la atención vaya en primer lugar allí, a lo que fue seleccionado como relevante y para facilitar su comprensión se redujo a no más de dos lineas. Claramente es importante, y cuando las pocas palabras incorporadas en el aviso logran procesarse, no queda duda alguna. La explicación de la conducción se extendió hasta los mecanismos de menor visibilidad pero no menos categóricos, porque justamente es uno de los que más toca la sensibilidad de los visitantes, qué se hace con su información. Sí, ni más ni menos que la Política de privacidad.

La extensión de su contenido no salió del margen que podría dar a esperar la clásica de Temti, no obstante terminó siendo un tanto más, y en su descripción distinta, evidentemente no sirvió de referencia para su estructura, ni fue adaptada con mucha carga de copia, es de cero y propia. La clave se halla en por donde va el énfasis, lo que presuntamente transmite que un intento de mantener cierta transparencia y llamar a silencio los mitos no reales, pero que a la vez no da pruebas fiables de que tal como el texto dice suceda, sin embargo creer no es una mala opción. Bajo esas estipulaciones, que son aceptadas al navegar donde las clausulas rigen, los datos no tan secretos que algunos suelen querer proteger como su tesoro más preciado, y que al mismo tiempo tratan sin cuidado en función de una liviandad demasiado confiada, aparentan estar en buenas manos. Tampoco hay garantías de que los implicados se vayan a interesar lo suficiente por darle lectura a eso que ineludiblemente los apremia, precisamente es lo que parece sucedió y sucederá, empero la oportunidad ahí está, luego las excusas para quejarse obviamente serán de baja o nula validez, aunque no al punto de que la respuesta sea un frío te lo dije, a lo mejor sea muy antipático para la compasión que maneja la actual individualidad a mando.

Encima o debajo de esa larga y pesada redacción, se presentó el mítico resumen con naturaleza de boletín, el que viene de vez en cuando y aterriza en el extendido listado de respuestas que reside en el espacio con más carácter autoritario para contar lo que se improvisó y en su momento no supo ser sobresaliente, todo aquello fue rejuntado y enlistado con breve detalle por la administración, para que ningún cambio pase más desapercibido de lo que sus pocas relevancias les deparan, porque menospreciándolos un pelín, tienen incidencias acotadas.

Son esos y dos más. Una novedad extra es la que hubo para la variada cantidad de opciones a la hora de crear fantems, que tiene como fin apagar los colores de cada cuadro que cubre a los anoms, en principio no de gran utilidad y relativamente evidente que fue sacada de destinos exteriores. El otro bonus va por el lado del apartado informativo, los ocultos párrafos que supuestamente debían ser del conocimiento de cada visitante, se extendieron considerablemente y ahora constan de varias puntualizaciones adicionales.

Dicho lo último, es de esperarse que la trascendencia que vaya a tener sea nula, porque llegar a la ruta donde se hayan resulta medianamente difícil, y porque está complicado tomarse la molestia de visualizar correctamente todo lo que allí hay desarrollado, no obstante significa un avance medio particular. Pareciera ser otra de las prácticas copiadas de la Temti original, pues la mayoría de estas cosas supieron llegar en sus tardíos arranques, y sorprendieron por su seriedad, hasta incluso las pautas legales intimidaron, pero ciertamente para este caso el efecto no llega a ser igual, los fantemteros, con equis en su gentilicio si se quiere, se portan apropiadamente y poco deben preocuparse por lo que casi nunca cuestionaron. A su vez, la parte de la guía, que es peculiar porque se hace un tanto rara para lo que es el ambiente, que normalmente no suele tener aclaraciones del estilo, la típica presentación para novicios desmenuzando las obviedades previo a encontrarse con ellas, sin embargo no termina de cuajar, al menos no donde está. Y con los otro punto llamado datos finaliza la sección, que no son más que estadísticas básicas como las que había en los paródicos Términos y condiciones del antro originario, pero no lo es todo, porque también hay ciertas sentencias que componen la mayor porción del tamaño de esa parte.

Y quizá, para la perspectiva de análisis secundario, sea lo más destacado si se tratase de entender la cuestión existencialista fantemtiana: está declarado cuando y porqué nació, su objetivo al incursionarse en el mundillo donde urgía, o no, un sustituto que el diera su lugar renovado a las memorias temtianas. Ese propósito no casualmente suena como una línea más de sus obras literarias, habla de un legado el cual suele ser objeto de estudio extraoficial, pero con esto deja de ser un asunto indiferente, porque esa causa ameritó la decisión tan extraordinaria que permite hacer este presente preciso. En el sentido literal, resulta muy diferente a lo que incentivó el surgimiento de los clones más combatidos, visionarias ciegas ambiciones que luego escasas consiguieron sostenerse. Entonces\ldots{} este camino alternativo es para que el final implícito no sea totalmente terminante sino un punto y aparte, porque la voluntad protagonista de estas crónicas llevó sus aspiraciones de continuidad a un nivel superior, la realidad.

De eso consta lo que llegó con más que solo una versión, la 0.9. y sus aditivos colaterales, que por si solos destacan con el nivel de evolución que tienen atrás, cuando apenas hay más un mes y monedas de desarrollo atrás. Impresionante puede ser una exageración, más teniendo en cuenta que el resultado aún aparentemente sin terminar gran cosa no es, no obstante en comparación a la lentitud o inexistencia de los cuidados titiriteros, adjetivo nativo del antro que poco hizo notar sus imperceptibles distinciones, es una grata mejora si la aspiración esa fuera, progresar.

Lo concreto es que la cabecilla del modesto clon además de ponerse al día con sus asumidas aunque no obligatorias necesidades, y también complementar mínimamente sus trabajadas estructuras, atiende un vacío que no estaba considerado, sacándose un peso de arriba, el cual ante el selecto grupo de adeptos quienes serán los primeros en enterarse no mueve mucho la aguja, pero que en un hipotético caso de expandirse y variar su tráfico, es factible llegara a serle exigido.

\section{La cuerda acción sonámbula (Duodécimo libro, capítulo VIII)}\label{la-cuerda-acciuxf3n-sonuxe1mbula-duoduxe9cimo-libro-capuxedtulo-viii}

\begin{quote}
7 de marzo de 2022, \ldots, 11 de marzo de 2022
\end{quote}

El comienzo y su furor se van apagando, definitivamente ya no lo son, este va en fases más avanzadas y extraña las situaciones constantes de aquellas fechas en las que el nombre Fantemti deba sus pasos iniciales, que actualmente echa de menos lo que supo establecer como su propia normalidad, o la carencia de su impronta se hace notar. Con el presente lento cambio, el único clon temtiano empieza a compartir más con el modelo que pretende homenajear, porque antes solía tener unas cuantas, no obstante ahora más todavía, aunque resultar ser una copia completamente íntegra.

Hablamos de la inactividad y la falta de cambios, motes insignia de la realidad titiritera, ya claramente auténticos en el antro que sucedió a su comunidad, basta con medir el tiempo que proporcionan los relojes visibles por doquier. Allá por el arranque absoluto y los prontos devenires, el movimiento no era una cosa extraordinaria, ni que brillara por reflejarsse en altas cifras, sin embargo contaba con otra chispa y mayores tendencias a multiplicarse por momentos breves, escaladas en los números de presentes o proporcionados idas y vueltas respecto ellos, que simplemente ya no se dan, o en todo caso por instantes poco vistosos. Evidentemente no es del agrado de quienes permiten decir que no es una quietud absoluta, y por allí fue manifestado, pero la devolución fue casi terminante.

Tal vez fácil de intuir, o a lo mejor impredecible, la administración explicitó que no realizaría esfuerzos destinados a elevar aquello que genera interacciones, marcó una posición relativamente distante al grupo de anoms que día a día dedica parte de sus energías a movilizar las filas de fantems, y puso paño frío a las esperanzas que pudieran llegar a vislumbrar un futuro sin una lentitud tan abrumadora. Nada cierra la puerta, pero quien así lo quiera, poco le recompensaría esperar, tendrá que actuar por sus propios medios, o apoyarse de los colectivos. Una opción, que ya fue tomada en varias oportunidades, es hacer promoción fuera de fronteras, ni más ni menos esa fue la manera de darse a conocer en los inicios de la existencia, solo que discretamente, luego también la dirección fue compartida en otras alternativas y gracias a eso fue que se dieron los picos en el famoso contador. Y nada más, las discretas y anticuadas rutas de Heroku necesitarán ser expandidas en horizontes de afuera, que con lograr unas gratas condiciones para con las pretensiones habituales, podrían ser conservadas.

Tampoco es lo único achacado, porque además de los colores no correspondidos en ciertos sabores frutales, otra de las cosas que no están siendo totalmente toleradas, o al menos lo eran hasta hace poco, es el reglamento, las estrictas normativas del sitio que en su momento pudieron ser insertadas sin inconveniente alguno con silenciosa tolerancia de los que posteriormente se adhirieron a él. Cayó el primer infractor, la normativa que topea el tono de la multimedia cargada terminó quebrantada después de unas semanas declaradas, y durante horas ahí quedó lo que en los textos se llama como intervención, para luego ser invisibilizada. No alcanzó, rápidas reincidencias volvieron a desafiar a la autoridad y la respuesta fue igual, al rato resultaron ser alrededor de tres veces en total, la cuestión es que ahí pararon, las cargas condenadas no se vieron más. ¿Qué pasó?

\emph{/5 veces rompiste la regla\ldots/}, el orden fue impuesto por la fuerza tras que las advertencias fueran desestimadas por el individuo que fue más allá de lo permitido, y la nota justificativa salió a luz. Aquel hizo pública la increpación, y no solo dio cuenta de que los comportamientos fuera de lugar son penados, a su vez se reveló que dentro del funcionamiento del sitio también existe un sistema de sanciones, donde lo utilizado para negar la participación son las tituladas como restricciones, no baneos. Sabido es que en la Temti original mecanismos para dictaminar prohibiciones individuales habían, aunque un tanto discretos, no demasiado utilizados, punto clave de la relativa libertad que ofrecía.

Por otro lado, el ritmo de las actualizaciones y mejoras indudablemente disminuyó, no es como al principio que lo raro era tener un día sin ellas, pero la flamante 0.9 no fue una conclusión de dicho propósito, porque al menos más retoques puntuales sirvieron para notificar que la misión de continuar con el desarrollo se encuentra sin finalizar, aunque el proceder último discreto, no constó de más que una reubicación del medio efectivo para los reportes, complementada por un nuevo botón que tiene como utilidad esconder contenidos, cosas que ya se han visto en otros lados. No muy genuino, prácticamente lo opuesto, sin embargo es algo, factiblemente preferible a la nada misma. Esta es la recta final rumbo al imponente número de la 1.0, con un ritmo desacelerado tal fue adelantado. No hay duda que bastante resta para asentar una versión definitiva, pero es de suponerse que lo considerado esencial para su responsable ya está, lo cual no implica que haya consenso en lo referido a ser suficiente.

Dado el estado de bajas contadas evoluciones, mayoritariamente la suma de factores que vale la pena resumir recayó sobre lo hablado, como en los recientes viejos tiempos. Entonces\ldots{} ¿\emph{/El admin duerme/} como por ahí dicen? Enorme ha sido la insistencia en referencia a eso, probablemente por su inacción o ausencia, que en verdad tiene su parte cierta, no obstante juzgando los hechos por completo, no ha abandonado su creación. De lo contrario, ¿qué diríamos de lo que hacía A.? O más bien, ¿qué diríamos de lo que no hacía A.?

\section{Ensalada de ideas frustradas (Duodécimo libro, capítulo IX)}\label{ensalada-de-ideas-frustradas-duoduxe9cimo-libro-capuxedtulo-ix}

\begin{quote}
11 de marzo de 2022, \ldots, 17 de marzo de 2022
\end{quote}

Adelante aunque no tan radicalmente, el progreso es incondicional y no permite que la realidad fantemtiana permanezca absolutamente congelada, a la vez que la actividad la sigue a un ritmo medianamente dispar, detrás, virtualmente como si fueran campos independientes, dado que ninguno le marca los pasos al otro, de la forma que sí suele ser típicamente en los antros del contexto clónico, ejemplo, las lejanas antecesoras de Titiri Vieja. De ahí es que sale lo que se cree vale la pena, cuando grandes demandas eran la inspiración de los desaparecidos propulsores los sitios crecían sin cesar, por lo tanto un escenario contrario implicaría deberse a un muy reducido público, que en la mayoría de los casos puede dar a entender que no amerita esfuerzos, pero de eso se trata, no existe una regla universal que exija condiciones mínimas para implementar mejoras, menos para existir, y ni hablar si se mantiene a flote gratis, entre comillas. Sin embargo, aún no siendo siempre ese el argumento terminante, históricamente dichas propensiones generales no han sido esquivas a ello que se ha repetido numerosas veces y determinó el destino de más de una alternativa, que para este preciso caso no es igual, hablamos de una excepción.

Entonces, el título para resumir las fechas estas va por el lado del motivo que suscitó la cuestión anterior, lo llamado actualizaciones. La conducción publicó el detalle de las novedades y cambios a su propio antro, versión como las que antes fueron mencionas por su parte, solo que de menor relevancia, afirmado por el número puntual y el contenido que vino consigo. Las clásicas tendencias, primitivas de seguramente otros lados, pero que recuerdan a una característica inolvidable de la Temti original, llegaron para acompañar a los fantems de manera opcional y como ayuda para ubicar aquellos que compartan una misma. Además, fue expandido el apartado de ajustes correspondiente a cada código, que permitirá adaptar más la experiencia de acuerdo a las escasas posibilidades que ofrece, ahora específicamente también, silenciar el Tuturú. Ya nadie tendrá la obligación de escuchar el típico sonido que alerta sobre que hubo una interacción sobre lo que se tiene en frente, aunque métodos alternativos para llegar a eso habían y aún hay, también habrá una método extra, este, no obstante quizá sea raro que alguien lo pretenda.

Un párrafo aparte merece el asunto de los sabores y todo lo que estos derivan, tal vez el más polémico de los que acompañan al consolidado hogar del legado temtiano, lo que menos gusta, lo que más rechina, porque se sale del modelo común y corriente al que saben ajustarse los clones sin excepciones, salvo claro, Fantemti, que no está de más reiterar lo es. Desde la perspectiva de quienes la han visto avanzar desde su prematura apertura hasta la actualidad, en caso de pensarlo resulta extraño procesar el que se hayan atacado tantos flancos débiles considerados como esenciales, y que entre ellos no fuera incluido uno de los que visualmente más hace la diferencia, además de la influencia que tiene sobre el gusto de cada individuo, porque ellos al poder darle un atractivo más personalizado a sus creaciones, presumiblemente sientan tener menores limitaciones de llegada, por tanto estar en un lugar que los ayude a transmitir lo que quieren decir, cosa que con cuadrados de colores, no es que suceda mucho. Y eso no ha cambiado, con matices, tonalidades, o texturas, prácticamente como era en los primitivos inicios, con la salvedad de que rotan.

Sin embargo, además de la constante adición de ellos que no ha parado desde dicho punto de partida, últimamente se notó como estos comenzaron a ir por otro lado diferente, ya no solamente por las clasificaciones que se puedan hacer a partir del nombre y color, sino que además por como son representados gráficamente. De estos que fueron implementados recientemente, más que gradientes simples, como la administración se supo referir a las dichosas portadas, propiamente lo que sus denominaciones decían, y desde lo estético esto convierte las posibles impresiones, debido a que la mayor parte del espacio principal se compuso de sencillas mezclas coloridas, y entre medio de las tantas que hay, un trozo de mortadela, una cebolla cortada. Sí, en serio. Claramente allí había un contraste, no demasiado grande, pero tampoco invisible, real al fin. La referencia es al pasado, porque duraron poco tiempo, a la brevedad algunos se retiraron, aunque no del todo. Dicha incursión no aparenta haber sido exitosa, el mandamás anunció una marcha atrás, lo que efectivamente sucedió, y eso deja dudas de que sepa lo que está realizando, pues bien, el intento de tener comestibles además de líquidos, resultaría muy complejo de abordar. Entonces, ¿el catálogo se limitará exclusivamente a lo primero? O es eso, o dependerá de una hipotética reformulación de lo que de estaba intentando. Nada certero por lo pronto, solo cebollas y un montón de bebidas con frutas, que se asume son jugos. Lo más insólito es que Coca Cola, Pepsi, Sprite, y otras marcas casi que desconocidas están presentes, y ninguna variante de Fanta se nombra como tal en la dilatada colección remasterizada.

Pero aún así, a base de fanatismo trabajado y experimentación relativamente afortunada, el sitio temtiano se mantiene a flote. Además de los progresos naturales que todavía siendo lentos se siguen dando sin que sean solicitados, y la relativa estaticidad en cuanto a la movimiento, la no grave problemática a estas alturas se encuentra alrededor de un tema relativamente trascendental, salido a partir de una circunstancia no necesaria y discutible, la cabecilla del sitio y su no muy comprensible contradicción al esquema tradicional que siguieron todas sus similares en el momento de poner operativo. Mientras que la comunidad probablemente espera que las imágenes terminen de expandir su presencia y ocupen la totalidad del inicio, el foco de quien toma las determinaciones definitivas está puesto en poner y sacar, sabores.

\chapter{Tras el casi seguro intento de abrochar sostenibilidad (Decimotercer libro)}\label{tras-el-casi-seguro-intento-de-abrochar-sostenibilidad-decimotercer-libro}

Nunca más se supieron noticias de Titiri, ni la Vieja, ni la Nueva, ni la carente de adjetivos, ni Versiones, ninguna, tampoco se vio comprometida la solidez de Fantemti, y lo más normal, el resto permaneció igual. Los únicos cambios profundos radican en la condición del indiscutido actual hogar principal de la Historia Temtiana. Ese producto incompleto que en su génesis no llegaba a ser el diez por ciento del modelo sobre el cual se inspiraba, supo expandirse al punto que ahora está mínimamente distante en la mayoría de los sentidos, incluso en unos muy concretos por encima de, tanto que la categoría de clon le calza más apropiadamente si no es por completo, porque varios rasgos genuinos e intencionales la alejan de ser una copia total, mientras que también todavía hay ciertas diferencias en el resultado funcional fruto del trabajo por mejorar las cosas, pero que viendo como fue transitado hasta donde se encuentra, sería cuestión de solo hacer algunos ajustes de verdad, porque aún considerando la tardanza con la que corren aquellos, lo faltante parece estar ahí nomás. A sabiendas del declarado objetivo, y la ínfima relevancia real de la actividad, el análisis restante recae en los modos con los que el viaje sin retorno será prolongado. Los ojos sin darle total espalda al pasado, atienden al futuro.

Aún así, antes de ello aparece este puntual instante, la carrera hacia el más allá no frena como tal, cambia de carril. Tras casi un mes y medio, la conducción le pone un punto y aparte a su incursión bajo las prestaciones de Heroku, para proceder en su senda con otra alternativa semejante. En términos generales, supone prácticamente lo mismo, porque en relación a lo que venían disfrutando los dominios fantemtianos, tan solo se trata de un cambio de coordenadas, sin embargo con un mínimo de permanencia la atención minuciosa noticiará de las distancias efectivas, las que específicamente ameritaron esta transición: sostenerse en línea sin interrupciones. Hasta el momento ese factor crítico que estaba ahí no llegó a interferir sobre la disponibilidad operativa, no obstante de haber seguido junto a esa plataforma en la mediana posterioridad esta impondría una inevitable pausa, la cual solo podría ser eludida escapando. Ante las contingencias venideras sabidas, la elección es anticiparse a ello y ya asegurarse un lugar del cual no se avecine una obligada retirada, ni siquiera temporal, lo que es comprometerse a rajatabla con la sentencia oficial escrita para justificar la finalidad de esta existencia, evitar que el dichoso legado se quede sin un espacio propio protagonista. El rendimiento y la fiabilidad que vaya a ofrecer la menos reconocida Railway estará a confirmarse, aunque si el dictamen definitivo resolvió inclinarse por tal, es que hay suficiente confianza en que podrá responder acorde a las expectativas.

Tras dicha parada, útil como referencia para dar un amplio pantallazo acerca de la realidad del único sitio bajo lupa, la particularidad primaria que constituye su ahora destaca por lo estancada que está y su poca expectativa de cambio de cara a lo próximo, el movimiento generado por los sujetos que lo construyen desde dentro. Teniendo en cuenta el ayer del cual viene, no puede definirse con un transcurso espejado a menor escala, porque la proporción de caída en lo relativo al comienzo de la original Temti es ampliamente inferior, no obstante es innegable que al lado de aquellos inicios camuflados por la efervescencia del impulso perdido, los idas y vueltas están encogidos en lo que refiere a cantidad, eventualmente calidad también. No se registran hechos clave que hayan alterado esa circunstancia altamente naturalizada, y salvo eventos extraordinarios que no se ven venir, presumiblemente continuará tal y cual es, concluyendo que el nicho viva en mayor medida inactivo, pero eso, que viva.

A la par, actualmente ya no mucho importa la velocidad del desarrollo, este periódicamente da un nivel de avances drásticamente menor y solo apunta a detalles, quizá desmereciendo la importancia del elemento que más atractivo le podría aportarse a sí misma: las clásicas portadas, lo único que le falta a los no exigentes anoms para no notar carencias relevantes respecto a la degradada instancia en la cual supieron aguantar hasta el final, y de tal modo robustecerse en un reemplazo a la altura de las circunstancias, que no siendo demasiado exigente, tampoco es que permanezca latente ni sea objeto de comparaciones tajantes, solo que sin lugar a duda sería progresar.

De concretarse, el punto de inflexión podría venir a la par de la ansiada 1.0, posiblemente reservada para el momento cuando se considere que el antro esté en condiciones de presentarse en su versión final, lo que contemplando su actualidad no estaría del todo mal, más haciendo cuenta del nivel decadente de antes, que jamás llegó a una base muy opulenta, al contrario, sin embargo falta una cosa, y es porque la administración persiste e insiste con su tapiz compuesto fundamentalmente de sabores. Según la tendencia mantenida por sus movimientos e intervenciones, pronto estará al caer, porque antes mucho de lo que no iba suceder finalmente terminó pasando, entonces por qué este caso no iría por igual lado. Las explicaciones que excusen la demora o ausencia definitiva no son del todo convincentes, y eso es poco habitual, pero nada, es de esperarse que haya algún pienso atrás y varias proyecciones pendientes aún no reveladas, que vaya a saberse si se cumplirán.

Por lo tanto, el presente poco tiene de sucesos sustancialmente revulsivos, y tarde o temprano podría estar encontrándose con un adicional aunque también discreto antes y después, responsable de segmentar con mayor firmeza y claridad épocas ubicadas en fases uniformes con leves distinciones entre sí, ese que pueda concluir las últimas adaptaciones obligatorias de un sitio con juventud afianzado en su decidido camino, y la presentación de él apta para ser titulada como estable tras superar las autodemandadas exigencias.

\section{Vehículos para el destino único (Decimotercer libro, capítulo I)}\label{vehuxedculos-para-el-destino-uxfanico-decimotercer-libro-capuxedtulo-i}

\begin{quote}
17 de marzo de 2022, 18 de marzo de 2022
\end{quote}

Los mantenimientos jamás habían sido tan gratos con el producto otorgado en el retorno, y ya nada más eso es un montón, porque la verdad es que hubo varios en la dilatada trayectoria de antros, especialmente los de origen, y no todos devolvieron su objeto a tratar en la forma que se lo llevaron, sobre todo los más recientes, así que al menos este negativo no fue. Extrañamente fue largo, demorado para la rapidez que suele acostumbrar Fantemti, aunque para nada extenso comparado con los característicos de Temti. Las no similitudes entre ambos sitios sin destacar son de lo más explícito, con justificados fines, los modos de proceder marcado el paso. Esa evidente razón, aclarada en el novedoso mural de las explicaciones, es una mudanza que no había sido avisada con anterioridad, mas solo sucedió repentinamente como tantas otras transformaciones que de golpe llegaron: el traslado de Heroku a Railway. Según quien lo procese está entre las opciones que se entienda, o no. La cosa va más que nada por el lado técnico del cual no se ocupan los anoms, y cierto es que lo desigual una vez dentro cerca de ser imperceptible está, pero hay más cosas a favor, tanto desde la carga simbólica como la fríamente objetiva.

El verdadero objetivo dado a conocer jornadas atrás no es una sentencia metafórica ni de adorno para llenar el ojo, o quizá si lo sea, no obstante por lo pronto hay un compromiso con ello, el designio de resistir y permanecer en línea sin interferencias lleva a tomar una resolución no menor, la cual de primeras habilita a anticipar como mínimo un par de consecuencias fijas que harán la diferencia, con relevancia a confirmarse. Ninguna de ellas quedó a juicio de la comunidad, ni fue a pedido de sus integrantes, ni su voz de voto fue solicitada, la decisión estuvo por encima de todo ello. Aún así, no parece que vaya a ir contra sus intereses, de hecho debería resultar positivo ya que asegura la estadía para ciertos días de agotamiento que bajo las condiciones anteriores iban a estar en duda, lo único que podría caerles mal es lo que ganó en complejidad el acceso al sitio, que a propios no tendría que cambiarles la experiencia, sin embargo a extraños sí, ya supieron quejarse al respecto y esto mismo no es para menos.

El factor de las coordenadas, a las cuales la dirección de siempre manda automáticamente al ingresar, en su defecto es un determinante para quitar atractivo, pero no el más grave, si los visitantes no se quedan o no hallan el confort suficiente como para regresar con cierta regularidad, indiscutiblemente es debido a otro cúmulo de razones, justamente esta no representaría tanto. No solo el actual, sino que también el anterior, claramente ninguno de los dos llega a equipararse con los que han tenido otros antros, cuales con nombres y terminaciones enormemente simplificadas, facilitan el recuerdo y por lo tanto el ingreso. En este caso falló lo segundo, y sí podría ser cuestionable, pero viendo por dónde van las pretensiones el dominio no es obstáculo, por tanto no hay incentivos suficientes como para llegar al nivel ideal. Vale recordar que el propio de Titiri se proyectó como tal, no al punto de ser asegurado, únicamente textualmente se consideró seriamente factible, es decir que podía pasar, y qué pudo haber pasado\ldots{} jamás apareció, y el figurado astronauta que apuntó con un arma dejó a quienes aportaron sin declaraciones ni justificativos, así que allí hay un ejemplar más donde no alcanzó, o a lo mejor sí, pero el criterio dio a entender que no valía la pena. Realmente cuesta tener un lugar que por posición imponga respeto y seriedad, para el cual esta sucesora no paga el precio completo, no en su versión directa, solo llega a conformarse con ubicaciones más periféricas por así decirlo, que gracias a lo caritativo de su mundo es posible, y no es cien por ciento crítico, al revés, parece dar igual y no poner piedras en el camino a seguir.

Habrá una segunda consecuencia prácticamente asegurada, que al parecer pudo diagnosticarse antes de tiempo como fruto de las pruebas que pudieron ser llevadas a cabo antes de concretar lo que hoy trae, la cual actualmente ya es perceptible si la atención la encontrara, y en la suma a pesar de ser una variante mínima no resulta indiferente, el tiempo de respuesta se ha visto enlentecido. Aunque no sea algo muy tenido en cuenta, como ha sido llamado en términos oficiales, el prestador reemplazado es de mejores rendimientos en ese sentido, y con el armado fantemtiano fluía satisfactoriamente, a ritmos notables. No es que este relevo sea desastroso, al contrario, sin embargo en su totalidad se sitúa un poco por debajo, y es lo otro que puede destacarse.

La contraparte lo amerita, es la excusa para aceptar y tolerar las displicencias, la necesidad primordial de esta conducción la induce a acudir a una alternativa confiable que le permita mantenerse donde está y no desaparecer del mapa por largos intervalos, tener control de la presencia en linea sin socavar ante las carencias de sostenerse con un presupuesto nulo, solo voluntad operativa presentada ante la generosidad interesada. Nada garantiza no tener descarrilamientos de la vía proyectada, ni la estabilidad permanente que deje el cien por ciento de la responsabilidad en manos propias, solo extingue la obligación de detenerse en una estación no deseada. Nada más que decir al respecto, la confianza sin ser absoluta está dada, entonces las experiencias por descubrir dirán si fue una buena idea, por supuesto la intención y voluntad no siempre priman, el resto también juega.

\section{Acciones de atención y cuidado (Decimotercer libro, capítulo II)}\label{acciones-de-atenciuxf3n-y-cuidado-decimotercer-libro-capuxedtulo-ii}

\begin{quote}
18 de marzo de 2022, \ldots, 21 de marzo de 2022
\end{quote}

A pesar de no haber pasado mucho tiempo, la mudanza ya anunciada en la vieja dirección parece ser definitiva, el total de la migración según han evidenciado los días fue un éxito sin inconvenientes ni reclamos de clase alguna, se suma un nuevo lote de avances en el trabajo sobre el funcionamiento que pone al tanto de la sostenida circunstancia en el frente de las actualizaciones, con las cuales Fantemti expande sus funciones y más se acerca a los estándares que suelen cumplir los clones. Las clásicas encuestas, básicas en sí pero solo habidas en los sitios más perfeccionados de los tantos que se iniciaron y terminaron en el contexto, ahora son una característica con la que podrán contar los anoms. Además, el buscador de contenidos específicos dejó de ser una carencia, para ellos ya no será un problema localizar rápidamente unidades concretas mediante el recuerdo de palabras claves, lo que para este caso no representa una utilidad muy grande porque la cantidad de fantems creados no es lo es, sin embargo la concreción no es más que positiva.

Esos, entre otros aspectos que apuntan a la perfección desde los detalles, representan una mejora sólida, lejos de ser sustancial, más bien solo grata y no tanto por encima de eso, ciertamente parte de lo mejor que puede lograrse de acuerdo a las expectativas si no es con la conversión de los sabores a imágenes. A su vez, entre esta y la inolvidable antecesora hoy ausente, que sin lugar a duda desde los inicios supo mantenerse como ejemplo a seguir, hay cada vez menos distancias, e incluso en ciertas áreas hasta superarla, logró introducir cosas que no habían sido vistas antes sino en otros sitios cercanos. Eso hace ver lo atrasado que quedó el desarrollo de Temti en lo referido a lo que en su momento eran las competidoras, inmersas en el afán de atrapar el caudal multitudinario que andaba en búsqueda de un reemplazo, en esa carrera de copia constante es que se quedó atrás y nunca se puso a tiro. Para la actualidad, sin ser las mismas metas, evidentemente las aspiraciones se guían de manera similar, a partir de las referencias que dan lo que supo marchar correctamente, nada habría tomado la forma que hoy tiene si no fuera por lo que son los modelos a seguir, verdad quizá obvia, pero que no está mal recalcar.

Aparte de ello, sorprendentemente tras la llegada de ese paquete de novedades, una situación inesperada se avecinó en los dominios fantemtianos y llamó la atención por su naturaleza, totalmente perjudicial para la navegación dentro del sitio, una lentitud abrumadora que por momentos virtualmente no permitía ir de un lado al otro. Así se mantuvo durante largas horas y ninguna declaración al respecto advirtió ni alertó sobre el inconveniente, además de que firme estuvo el silencio habitual del entorno, cuestión que la interacción no hacía mucho por multiplicarse como para que la dificultad molestara más todavía. Más tarde desde el único espacio oficial, donde en los últimos los intercambios se volvieron frecuentes, la administración daría cuento de que la ralentización pudo ser erradicada, y acorde a sus declaraciones, la afectada latencia volvió a sus niveles habituales.

Sobre dicha alteración, supo revelar parte de la trama atrás, nada directamente derivado de su responsabilidad, sino que saliente de las redes que sostienen al antro: percances originados desde las bases de datos, para los que si hubo respuesta, que de no haber sucedido a lo mejor aún se podría estar hablando lo mismo, con lo que habría ganado más notoriedad. Lo segundo fue una curiosa comparación con las extrañadas migraciones de la Temti original, que prácticamente todas superaron el día en cuanto a su duración, y aquí una vez más, notorias diferencias, además de que hasta la fecha no hubo reinicios completos. De igual modo, por mas que la demora hubiera sido mayor, al ser una interna tan pequeña lo que puedan retumbar sus momentáneas debilidades o permanentes desgracias, tiene una transcendencia bastante acotada, ni hablar instantes donde la gravedad de lo acontecido es mínima, no muy mal dicho, nadie se entera.

\section{Pintar de prisa sin relleno (Decimotercer libro, capítulo III)}\label{pintar-de-prisa-sin-relleno-decimotercer-libro-capuxedtulo-iii}

\begin{quote}
21 de marzo de 2022, \ldots, 26 de marzo de 2022
\end{quote}

Ya conocida la habitual estabilidad del presente, lo que haya cambiado recientemente o próximamente pueda estar haciéndolo, no tiene gran caja de resonancia, pues en lo último estos han ido dedicados exclusivamente a detalles, positivos y en cierta forma relevantes sí, indiscutiblemente son progresos, solo que no de demasiada importancia, además de complementar los pilares fundamentales del funcionamiento, poco más suman. En los arranques no se insistía en ello, las bases esenciales tenían que ser construidas, lo que en parte era predecible que resultara como una etapa transitoria, o incluso varias de ellas consecutivas, ya que pensándolo bien poco iban a permanecer las evoluciones constantes, no porque no pudieran ser sostenidas, sino más bien debido a lo que concretan para siempre, los logros indispensables a la brevedad se verían agotados y no posibles de repetir nuevamente, e igualmente podría mantenerse el ritmo con las actualizaciones, pero interpretando por lo que realmente sucede, no ha sucedido, conforme a eso el antro avanza de a cortos pasos.

¿Por qué es de interés hacer mención de ello? A pesar de ser un argumento recurrente y sobre el que se ha insistido, lentamente la realidad temtiana apunta a lo que durante un considerable transcurso de sus tiempos supo ser, una que se moviliza únicamente de lo que sus actores hacen, mínimamente influidos por los movimientos de quien toma las decisiones globales. A fin de cuentas, la mayor porción de sucesos que componen el día a día de un sitio como este deberían provenir de los anoms, sin embargo como son escasos, y lo que generan no suele significar ni derivar en cosas muy destacables, qué la perspectiva de historia contemporánea esté a posicionando como titulares, apuntará primordialmente a lo que venga de donde emana más poder directo.

Ya se volvieron a formar las condiciones para que no sea solo eso, que su falta deje más al descubierto el contexto secundario del nicho inherente. Sin ser casualidad, algunas nuevas situaciones muy puntuales se ponen en evidencia y por lo pronto merecen ser destacadas. Empezando por lo menos notable, sin necesidad de introducirse mucho en la interna o contar con conocimiento de cómo habitúan a comportarse los individuos, se nota que efectivamente no son extraños entre sí, incluyendo a todas las nacionalidades. Con diálogos que desvirtúan los títulos iniciales, hasta en el fraterno \emph{/espacio para hablar con la administración/}, claro queda que un tanto familiarizados el uno con el otro están y así lo dejan implicito en el momento de dirigirse las palabras, no muy cálidas ni como antes por cierto, y tampoco es un efecto muy novedoso, pero resalta por lo que parecen haberse ensanchado esos vínculos no tangibles.

Y lo segundo no es un tema menor, por supuesto teniendo en cuenta la relevancia de lo que considera importante para las circunstancias actuales. Se trata de la drástica aceleración en la creación de fantems está jugando al límite con la temporalidad, intentando ser más veloz de lo que puede. A la vez que los ya existentes son dejados un tanto de lado, con los días en cantidad aumentan notablemente, las primeras filas del espacio inicial se renuevan con títulos diferentes y sabores alternados, sin demasiado significado o trama entendible dentro, pero válidos al fin. De mala calidad se han dicho que son, no es tan fácil juzgarlos como tal, pero ciertamente de reducida extensión sus interiores, y aún así podrían dar para contestar bastante, multiplicar la baja actividad, y no pasa, vacíos terminan. ¿Será que también los uruguayos inventaron todo eso?

Lo más particular de ello, es que sugiere no ser suficiente, mucha insistencia hay con lo que falla al intentar dar lugar a la siguiente unidad cuadrada colorida, el mensajero que sirve la enérgica sentencia de que no se puede crear fantems tan rápido, y así es como dicha oración circula en los rincones más frescos aunque no recién planteados del sitio. En verdad, no es solamente eso, a la vez perpetua una costumbre ya vista con anterioridad: al dar con un error del estilo o cualquier otra cosa no común, en vez de capturar visualmente lo destacado, el anom acude a copiar y pegar con todas las letras lo que recibe, así las oraciones correspondientes tienen como destino el medio más cercano, y de ese modo quedan compartidas en la escena con el resto de sus similares. Como si fuera normal, no lo es, para las pocas voluntades que se incursionan en Fantemti la proporción con la que el mensaje rojo fue repetido, ha sido alta.

Concluyendo, más allá de suposiciones no rebuscadas, resulta difícil asegurar si es una tendencia generada por varios, de los no tantos que hay, o más bien solo uno procurando acentuar su descontento al respecto, pero eso junto a la velocidad que se mueve el inicio se convirtió en una moneda corriente. Obviamente a la larga podría terminar en nada, no obstante habrá que atender a la evolución, el mandamás desde su posición menos seria y pacífica desestimó la seriedad del caso, pero a lo mejor termine reforzando su respuesta: reduzca los intervalos de espera, prohíba hablar del asunto, ponga fin a la contingencia calificándola como inundación\ldots{} o por ahí quienes manifiestan lo citado cambiarán la intensidad de su conducta. Recordar que el nivel de conformidad solía ser sumamente alto\ldots{} ¿será que hoy es lo contrario?

\section{Memoria de hermandad (Decimotercer libro, capítulo IV)}\label{memoria-de-hermandad-decimotercer-libro-capuxedtulo-iv}

\begin{quote}
26 de marzo de 2022
\end{quote}

El día es un tanto singular, no único ni sin igual, pero no sucede muy a menudo que desde el fantem oficial por parte de su autor original se larguen explicaciones extensas acerca de algo nuevo que haya llegado, y menos cuando se trata de un aspecto con trascendencia ínfima para el estado del antro. Como le concierne, trajo actualizaciones y las explicó, que se reducen a la implementación de una ventana dedicada íntegramente a la personalización, para que a partir de ella sea fácil transformar la estética de Fantemti, más allá de la tonalidad de colores incluso, cosa no vista antes en las antecesoras cercanas y solo en un puñado de clones, particularmente de los más avanzados y transparentes en cuanto a desarrollo. A esto hay que agregarle un armado complementario y opcional, que retrotrae la ambientación del sitio a lo que en Hixxel se tituló como Temti-legacy, a rasgos generales con altos parecidos entre sí, con ciertos pormenores que le restan varios puntos y bajan un poco la calidad definitiva de la copia, sobre todo la trama del saborizado desencaja con lo que era aquel ingreso. Pero nada, recuerda a dicha coyuntura gratamente, y sea subjetivo o no, en suma el trabajo es correcto.

El factor impredecible proveniente de la gestión fantemtiana de una manera razonable aunque extraña restaura una perla olvidada de las primeras épocas de confusa diáspora, presuntamente ideada para acortar las distancias visuales entre los dos sitios que durante ese entonces levantaban polémica y se compartían las comunidades abrazadas por un manto de incertidumbre, en pocas palabras y más precisas, hacerlos sentir como en casa, a falta de esta. En cierto modo podría afirmarse algo similar respecto al rol que cumple el catalogado clon, también llamado imitación, que como tal lo que hace es calcar un combinado de condiciones para suplir una ausencia y vaya que si lo ha hecho, en una nueva ocasión, y quizá en un futuro próximo más todavía. Evidentemente armar un estilado aparte y accesible muy específicamente dista de ello, además imaginarse que los lejanos y escasos hixxeleros vayan a encontrarse con él es rebuscado y no muy factible, casi fijo que va por otro lado. Lo que sí, los motivos no deben ser urgentes ni nada que se le parezca, capaz como experimento, más nostalgia, o no se sabe, más del frecuente obrar de la extendida iniciativa historiadora y su apego al pasado, sumamente familiar para el mundillo temtiano dicho sea de paso.

\section{Más claridad para el pasado (Decimotercer libro, capítulo V)}\label{muxe1s-claridad-para-el-pasado-decimotercer-libro-capuxedtulo-v}

\begin{quote}
26 de marzo de 2022, \ldots, 1 de abril de 2022
\end{quote}

La seguidilla de novedades no ha tenido fin desde su comienzo, ya no cada jornada o día de por medio, pero de todos modos aún es posible anotar aunque sea cada semana algún destello que impida definir la normalidad fantemtiana como una dejada a la deriva en manos de sus no tan desconocidos protagonistas. La cuestión es que, específicamente, van disminuyendo los enviones, sucesivamente aportan menos y menos, empero lo último es discutible, su aporte en utilidad va por otro camino más destinado a cierto público, justamente dominante en cuanto a proporción. Las carencias del tema previamente introducido al flamante selector recibieron mejoras indicadas, ahora adoptó las similitudes que le faltaban y seleccionarlo es hacer \emph{/\ldots normalmente como si estuvieras en\ldots/} Hixxel, concretamente su presentación timtera. El resto de lo extraordinario no puede ser simplificado a esas pocas lineas.

Desde un discreto rincón, el sitio refuerza su área informativa con un lugar exclusivamente dedicado a la puesta al tanto de todo lo que esté relacionado con los cambios aplicados, y mismo explicaciones complementarias a ellos, como es el caso de lo primero que se puede encontrar allí, unas extendidas oraciones para justificar el cometido de la respectiva sección, seguida de las otras tantas lineas dedicadas al entendimiento de cada versión, comentadas en su momento tanto en el hilo informativo en Rouzzed como en el fijado en Fantemti. De ahí sale la parte más grande de la extensión, y de manera práctica quedan reunidos todos los detalles de ordenadamente según se hicieron, y más sobre el fondo un listado de vasta extensión pero muy sintetizado, con notas que no indican con precisión lo que sucedió en los determinados momentos, sin embargo casi perfectos para mejorar el seguimiento minucioso desde la cronología indicada por las estampas temporales.

Otro aditivo efectivo en el afán de complementar el apoyo que hay disponible para la corriente historiadora presente por el antro, sin dudas uno fundado primordialmente por, para y de esta. No enfatizando en las de periodos anteriores a su creación, los cuales tampoco fueron desestimados pues recientemente allí cerca llegó una clara referencia de las épocas de diáspora, sino en los antecedentes propios de las vivencias fantemtianas, los cuales aún estando creciendo lentamente en cantidad, encaminados en el avance sobre su propia senda, también sabe prestar un porcentaje de miradas y pasos hacia atrás. Por ahí no haga falta recalcarlo, pero no es ni de cerca un punto aislado, el mayor foco revisionista, incluso en formato de redacciones, se solía concentrar en el Diario Temtiano, y aún ronda por allí, con la salvedad de que todavía está sumamente rezagado en la persecuta del presente. En similar sentido, las recopilaciones del canal oficial no han dejado de estar avocadas en el servicio de la conservación de la amplia variedad que maneja el material cultural, aunque últimamente el caudal de capturas y ediciones importadas se ha mantenido relativamente bajo, nada de abundancia, solo lo que aparece por ahí y ya, su predecible patrón de conservar y volver a difundir esos contenidos se ha visto limitado porque precisamente casi no hubo que.

Lo destacado, no de ahora sino desde hace unos cuantos meses, la cobertura ininterrumpida del canal llamado anom itmetishtu multicolores invertidos, un medio que viene en alza gracias a su constancia y oportunismo ante los episodios del contexto, con un nombre muy sospechosamente enlazado al mundo temtiano por cierto, que ni bien Fantemti hizo la aparición absoluta para iniciar el proceso que hoy atraviesa, sus episodios no escaparon de las mudas grabaciones. Y si fuera insuficiente, el sitio también es accesible para la famosa Wayback Machine, en la cual al mismo tiempo se vienen guardando numerosos pantallazos globales, no siempre posible a causa de que lo normal suele ser prohibirla meramente como consecuencia indirecta de razones de seguridad, la cual podría suponerse que sustituye el rol primario del ausentado Archivo Temtero. Son muchas, y la mayoría de gran valor, pero muchas fuentes válidas, en virtud de revivir instantes pasados, evidente es que hay una fuerte intensión de no dejar olvidar los sucesos, así sean tremendamente irrelevantes.

A pesar de la vasta evidencia a favor de dicha iniciativa, su exposición no es acorde al alto trabajo y recorrido que lleva, en verdad son, proyectos por así decirlo, conocidos para la escena, y con larga trayectoria respaldándolos, no obstante operan desde lo bajo y cotidianamente no cobran notoriedad. Para estos días, en evidencia quedó aquella eventualidad que venía creciendo gracias al actuar de los anoms, que a partir de su capacidad para crear se mantuvieron rellenando los espacios del sitio con nuevos temas a una velocidad importante, lo suficiente como para destacarla de lo que era previamente. Hasta la actualidad, por día aparecen títulos diferentes y de todo tipo, dejando la impresión de que los individuos atrás de fueran sumamente diversos, sin embargo el contador de presentes flota en torno a las limitadas cifras rutinarias, en teoría no son más sino que marcan otra presencia. También incambiado, la mayoría de estos fantems todavía tienden a terminar vacíos, sin casi respuestas, el ida y vuelta no se genera.

Y encima se ha llegado a manejar el que son \emph{/fantems de mala calidad/}, concepto no muy alejado de la realidad por un par de fundamentos sencillos. En comparación a lo que no supo ser criticado, aquellos que en las primeras semanas lograron arrastrar el mayor flujo de actividad, es discutible el que sean mejores, pero supieron compenetrarse con donde se hallan, atraparon más al público, no resultaron indiferentes y apegaron para participar en ellos, por tanto suponiendo que lo ideal es mantener el movimiento real, podría decirse que sí. Habrá que dilucidar si cuesta mucho hacerlos\ldots{}

\section{Calco inmemorable (Decimotercer libro, capítulo VI)}\label{calco-inmemorable-decimotercer-libro-capuxedtulo-vi}

\begin{quote}
2 de abril de 2022, \ldots, 4 de abril de 2022
\end{quote}

Tras no muchos días, de quietud generalizada aunque no absoluta, Fantemti tuvo como novedad algo en la sintonía de su última actualización, nada más que con otra elección en la misma dirección, igual de inesperado, análoga temática, un diseño inspirado en uno que ya existió anteriormente en el entorno de los clones. Esta vez sin tanto desarrollo ni explicación, solo fue expuesto y comentado brevemente en el fantem donde la administración se comunica como tal, a la vez tomo parte en el flamante mural de los anuncios. Y menos mal que eso primero sucedió, porque de lo contrario es factible que nadie se hubiera enterado. De igual procedimiento para llegar a él, se sitúa junto a la mejorada interfaz con el corto listado de opciones, total de tres que tiene ahora un ejemplar relativamente extraño.

Es de Arggnews. Hoy sin estar, con una larga sucesión de episodios complejos y polémicos que le dieron mayor notoriedad en la escena ya finalizados, se haya desaparecida sin más. Gracias a ello, ser un sólido refugio de emergencia en semanas donde era de necesidad para cientos de movilizados, ganó mucho espacio en el recuerdo de numerosos individuos, sin embargo lo más sabido es acerca de sus etapas finales, no las primeras. Lo cierto es que históricamente de principio a fin supo tener cambios con alta frecuencia en su totalidad, no escapando la estética, y eso pasó sobre todo en sus comienzos, a los cuales es que se remonta esta versión particular, la recreación pende más fuerte a lo desconocido. Por su estructura en sí no sale mucho de como se presentan los antros de su mismo tipo, nada de hecho, los mismos cuadrados y dos columnas de siempre, no obstante es en una forma más cruda, básica y casi descolorida, pero con sus fuertes también. El dictamen más subjetivo dirá, empero si en la práctica no perduró demasiado su motivo tendrá.

Y bien, este de ahora ya no resulta un hecho aislado como podía estimarse del anterior, el proceso aludido de haber plantado las bases para la expansión en la cantidad de combinados de estilos claramente no fue únicamente con fin de establecer una sola variable alternativa a la clásica del sitio, ni tampoco para hacer solamente una referencia específica a las épocas de diáspora lejanas, más concretamente buscó eso mismo, posibilitar la adición de combinaciones distintas, ofrecerle a los visitantes un catálogo diverso para que estos tengan a su alcance varias formas de camuflar su estadía, como igual sucede en otros destinos de aceptable personalización, aunque cabalmente no muy común en los antros promedio del contexto. El condimento adicional es que fuera de limitarse a ser esquemas coloridos, o temas completamente nuevos, hasta ahora han sido nada más acercamientos a estos, copias de principio a fin que imitan los diseños de dos clones en particular, durante transcursos muy concretos de sus existencias, y justamente en este recién insertado, probablemente muy pocos hayan conocido su presentación original. Así además se identifica el motor de todo esto, ensanchar la fuente de nostalgia y dotarla de elementos provenientes del extranjero, saborear el pasado en su expresión más idéntica y vigente posible variando ubicación. Tal vez ahora pueda suponerse con más fundamentos que en la posterioridad caigan más como este par, sin embargo como habitúa a manejarse el no tan anom que comanda el rumbo fantemtiano definitivo, no hay declaraciones a futuro, ni promesas.

\section{En linea fuera de servicio (Decimotercer libro, capítulo VII)}\label{en-linea-fuera-de-servicio-decimotercer-libro-capuxedtulo-vii}

\begin{quote}
4 de abril de 2022
\end{quote}

La establecida normalidad ha deparado poco que esperar, constantes horas de frialdad y movimientos inexistentes, bajos números de presentes y relativamente escaso involucramiento de los mismos. No hay nada raro en ello, hasta que se corta literalmente, inesperadamente. Una de las eventualidades más frecuentes para la realidad previa era la de las detenciones totales, telones estáticos titulados con una palabra rotunda en lengua inglesa, o en una segunda categoría pero en la misma bolsa, presentaciones semejantes. En sus dos categorías significaban lo mismo, una espera con poco para distraerse, y a otra cosa, nada de Temti como Temti mientras tanto. Tras una considerable cantidad de semanas de marcha sin conocer en lo absoluto interrupciones de pausa intencionadas o duraderas por accidente, finalmente sucedió, la paralización total de las actividades fue efectiva y nada de lo que en el día a día se moviliza hay para vislumbrar, solo una extensa columna de texto.

Sin más preámbulo, es eso a lo que tanto se han acostumbrado los temtianos, un mantenimiento, dicho clara y abiertamente, supuestamente sustentado por los hechos aunque bien incomprobable, no hay acceso allí. La disparidad más notable de la referencia a compararse es que hay motivos explicados en un mural que prosigue al emblema naranja, primera vez visto en grandes dimensiones dentro del sitio, a lo que se constata que su uso ahora es más frecuente y no solo en miniatura o mosaico. Además de transformar y reconvertir la utilidad de la sección anuncios, que desde su inauguración pasó demasiado desapercibida, está notable porque a pesar de la ausencia del formato típico, no deja la razón a interpretación del espectador, que obviamente puede resultar errónea a pesar de estar sumamente ensayada, pero no es necesario, la causa es directamente explicitada. Aún así, el declarado trasfondo no resulta muy relevante para los actores de mínima incidencia, este incidente avisa estar tratando cuestiones en la base de datos como muchas veces lo ha hecho el primer sitio temtiano, solo que sin oraciones entreveradas o mensajes crípticos de por medio, la confusión no vuelve a repetirse, es simple como está dicho, los ajustes no pueden llevarse a cabo sin inhabilitar Fantemti en su versión típica. Teniendo en cuenta el historial de operaciones en los registros de información alertadas con antelación, no es un indicador muy grato, puesto que rara vez no derivó en un reinicio total, sucedían accidentes.

Nuevamente acudiendo a las comparaciones, un par de casos que en sus resúmenes más reduccionistas resultan iguales, visto en la realidad notando todos los pormenores, presentan numerosas distancias, haciendo que efectivamente sus resúmenes, estén lejos de implicar lo mismo. Como si fuera poco, los mantenimientos en su definición tienen inserto el carácter de transitorios, es decir que no deberían durar para siempre, sin embargo después lo que verdaderamente pasó es que aún hay tres de ellos sin terminarse, Titiri Nueva, Versiones, y Titiri Vieja. En los antecedentes hay más del estilo, un cubo de fondo gris alargando la espera durante largos ratos. Antes, provenientes de la Temti en sus doble faceta hubo muchos más, cada uno con su forma particular de tapar las tratativas y ocupar la dirección mientras tanto, unas anticipando la posible duración, otras no. En frente, o a su lado nada más que con recorrido de orientación diferente, Fantemti oficializa su ocasión número uno cubriendo cada punto que pudiera haber faltado en las experiencias ya conocidas. Lo único que le falta para cambiar rotundamente el modo respecto a los que su antecesora tuvo, es que salga bien sin que haya que lamentarse tras el proceso, que sea un éxito, y cumplir con los plazos autoimpuestos, estimado con ambigüedad, temporalmente.

\section{Restauración reforzada (Decimotercer libro, capítulo VIII)}\label{restauraciuxf3n-reforzada-decimotercer-libro-capuxedtulo-viii}

\begin{quote}
4 de abril de 2022, \ldots, 8 de abril de 2022
\end{quote}

Tras lo que fueron unas pocas horas de ausencia parcial, exactamente cuatro y media, total en el sentido de la configuración típica que se pretende atinar al arribar los dominios correspondientes, la pausa completa se terminó y el funcionamiento del sitio regresa a su andar normal, como si nada hubiera ocurrido, ergo, las operaciones aludidas concluidas correctamente, sin fallas ni errores para identificar, como propiamente fue redactado en el espacio informativo, y como se asume que sucedió al encontrar todo sin exabruptos alarmantes. Asimismo, el sumario con estampa temporal se presentó no bastante detallado, pero sí con una que otra indicación. Probablemente no era muy de esperarse que una vez cerrada la cuestión se fuera al cuento de aspectos tan específicos, nuevamente no muy relevantes para quien ingresa desde el otro lado, pero al fin y al cabo informado a su disposición, sumamente contrario a cuando nada se decía al respecto. En definitiva, todo esto busca generar condiciones mejores, quizá desenredar el supuesto espagueti, que visible como tal no lo es demasiado.

De manera consecutiva, con algunos días de intervalo entre las llegadas, también hubo complemento con actualizaciones que dotaron al sitio de más características para su totalidad, sumándolas a las ya consolidadas, escalando en el número de versiones un decimal más, y exponiendo cada pormenor de lo que cambió. En la parte más discreta, correcciones parciales para las espontáneamente defectuosas notificaciones, y ampliación de soporte en enlaces externos de multimedia. Esto último ya había sucedido con anterioridad, no obstante de las dos plataformas recientemente compatibilizadas, la más reconocida de ellas fue solicitada tiempo atrás por un anom, y más de un mes después su pedido fue cumplido, posteriormente recordado lo dicho en el fantem oficial mediante una réplica.

Más para lo anterior, bien tal cual algunos de los pocos clones trabajados entre el año y pico supieron adoptar, precisamente los más desarrollados, la opción de ocultar y seguir fantems, completamente discretas en su modalidad de usufructuarlas, pero de igual modo desde tres pequeños puntos o lo que sea, cumplidoras para lo que sirven. Son un par de elementos que fortifican un área que no está siendo descuidada, extendiendo el listón de posibilidades en cuanto a la personalización. Además, se aleja más aún de lo logrado en la nativa Temti, que como mucho permitía cambiar los colores para darle tonalidades más nocturnas al entorno, y ni hablar de su presentación segunda que no pudo abrochar factores básicos que ya estaban en su predecesora.

Compuesta de dos ingredientes fundamentales, o al menos a destacar, el principio de esta seguidilla concretó una jornada histórica para los registros temtianos, en la cual la senda fantemtiana vuelve a torcer las costumbres antiguamente vigentes y acentúa más aún el modo de proceder. Por más impactante que suene, es real, un mantenimiento, solo uno, ha finalizado sin descansar en el viaje infinito hacia la eternidad, e incluso más tarde llega con posteriores novedades palpables, que de haber llegado antes tal vez hubieran generado una sensación el doble de grata, de que fue productivo, como si fuera poco.

\section{Cuando la información confunde (Decimotercer libro, capítulo IX)}\label{cuando-la-informaciuxf3n-confunde-decimotercer-libro-capuxedtulo-ix}

\begin{quote}
8 de abril de 2022, \ldots, 13 de abril de 2022
\end{quote}

Distinto a como normalmente suele amanecer y anochece el poco cambiante inicio de Fantemti, la compilación de colores saboreados acompañó al tradicional único fantem fijado junto a otro, sin demasiada seriedad reflejada en su encabezado, no obstante imponente por la circunstancia mencionada, una larga redacción para delatar la procedencia de decenas de publicaciones, ejemplificando con seis indicaciones explícitas, que son esencialmente copiadas de sitios ajenos: Ufftoppia, Voxxed, Rouzzed. Más explicaciones sobre las referencias en cuestión son el contenido, y de eso quizá podría decirse que parece extendido de más, el mensaje claro es dar a conocer una verdad que aquel visitante asiduo de los clones con un mínimo de atención es capaz de percibir.

Dicho anómalo planteo implicó un diferencial ante el bajo interés que casi constantemente han recibido las numerosas apariciones recientes, no porque lo haya cambiado, sino porque indirectamente estas se ven cuestionadas, de lo contrario nada las hubiera mencionado, pasarían de largo y eventualmente continuarían multiplicándose. Sin ir más allá, la supuesta advertencia se toma como eso, un aviso que no reviste mayor gravedad, no más una comunicación sobre el asunto destacable del presente fantemtiano, sin embargo todo tiene su motivo y para este caso no queda muy evidente, si es algo más que generar conciencia acerca de donde salen los contenidos que emergen a diario en las filas del antro. Anteriormente llegó a manejarse lo mismo repetidas veces, lo común que es ver creaciones frescas que no son más que un reciclado de otras ya existentes, pero llevarlo al carácter oficial le da mayor nivel de importancia. Cierto es que carecen de originalidad y suponiendo una preocupación por la calidad, no obstante gracias a ello se genera una rotación más rápida y la quietud es disimulada, lo cual aún con baja respuesta, da una impresión diferente, adicionalmente a partir de las marcas temporales. Para armar más conjeturas al respecto, sería necesario contar mejor con las intenciones de ambas posiciones, quizá evidentes, ergo, no seguras. ¿La administración no quiere que se repita la situación aludida? ¿Hay pretensiones contrapuestas? ¿Qué quieren los anoms?

\section{Rumbo a una cúspide (Decimotercer libro, capítulo X)}\label{rumbo-a-una-cuxfaspide-decimotercer-libro-capuxedtulo-x}

\begin{quote}
15 de abril de 2022, 16 de abril de 2022
\end{quote}

Más tiempo acumulándose la estabilidad común y corriente de la vigente casa temtiana, se pone más en manifiesto que nada fascinante va a pasar frecuentemente, días de silencio y bajos intercambios son los que habitúan, ni el inesperado golpe de los días previos logró generar efectos colaterales de primera mano, quizá aflojar más el nivel de actividad. Entre lo otro, sin distinguirse en nada, se agregan avances en el desarrollo del antro, elevando la vara de las condiciones para el día a día, y así como otras veces, queda confirmado el no abandono total, el sueño que esporádicamente tiene fin.

El nuevo lote traído y anunciado por la conducción es variado en sí mismo, ampliado en cuatro puntos y algo más dentro del apartado destinado a eso, una muy extensa explicación desmenuza cada detalle como si se tratase de componentes extremadamente complejos, cuando realmente son altamente simples, complicados innecesariamente, por lo que el resumen publicado donde los rozzados es de gran ayuda para la comprensión. Antes también lo hizo saber de directamente, unos cambios experimentales fueron realizados a poca certeza de su perfecto andar, pero luego no llegó a notarse nada por el estilo, y bien procedieron los siguientes de contraria notoriedad. La llegada de un \emph{/caché/}, dicho lisa y llanamente como tal, busca optimizar la velocidad de navegación en el sitio reduciendo las operaciones innecesarias, característica que había sido sugerida previamente y semanas tarde finalmente llega. Segundo tanto a destacar, por lo visto el más notorio, la distinción para las intervenciones propias, típica de los nichos donde se intercambia desde el anonimato, y que supo ser adoptada por la Temti clásica. Junto con ello, más parámetros para la personalización, opciones para bloquear notificaciones que se suman a las concreciones recientes sobre cuya área. Por otro lado, más de lo hecho, reestructuraciones internas, en el parte de los sabores y la multimedia, ahí se desprenden ciertos argumentos interesantes que aparentemente le serán útiles a la administración para manejar lo suyo con mayor facilidad en el futuro.

Y no mucho más que eso. Son bastantes hasta la fecha, el particular número de versión lo hace notar a medida que la estructura avanza con cada envión, y con el mural histórico exponiendo la trayectoria que los tuvo apareciendo, es más fácil de acordarse lo que fue la rápida evolución, desde el comienzo al final, y ni hablar si se manejase material más gráfico que en verdad lo hay por allí. Pero se trata de pasado, atravesado y hecho, por muy valorado y cuidado que siga estando. Inmediatamente luego, el presente, que sitúa a una hipotética 1.0 muy próxima, y pasar de sus anteriores que pueden suponerse como consustanciales de una fase de pruebas, a la definitiva, es un cambio de potencial relevancia. Sin embargo, las declaraciones oficiales no encendieron para nada las expectativas, por el contrario pretendieron bajarle el perfil a dicha futura situación, en la que no estaría llegando nada impactante, solo confirmaría la consolidación de una iniciativa que al principio encausó con perspectivas de limitada ambición. ¿Será real? ¿Qué falta para eso?

Para resaltar un tanto más aunque sea, Fantemti resintió una caída de horas a raíz de que durante un corto rato el protocolo de acceso pasó a ser inseguro, de lo cual prontamente se recuperó y todo regresó a la normalidad. Parece muy insignificante y cierto que mucha pertinencia no tiene ante los relativamente inexistentes incentivos por participar, pero sirve para recordar cómo ha funcionado en la plataforma elegida desde que se migró a ella, bien y sin mayores problemas más que breves percances que rápidamente se revierten y dejan la totalidad de las piezas dando un correcto andamiaje, localizado en la rebuscada dirección que frente a la baja exigencia logró responder a la altura de las circunstancias, por más desprolijo que sea mantenerse en línea con recursos baratos. Ha de verse si más deficiencias como esta se repiten, y si llegan a ser captadas por los anoms como para reportarlo.

\section{La alternativa en armonización (Decimotercer libro, capítulo XI)}\label{la-alternativa-en-armonizaciuxf3n-decimotercer-libro-capuxedtulo-xi}

\begin{quote}
17 de abril de 2022, \ldots, 23 de abril de 2022
\end{quote}

Mismo lugar, misma dirección, mismo sitio, mismas ausencias, a gran escala todo sigue casi igual, el nicho de presente temtiano existiendo sin pena ni gloria, aunque más correcta que deficientemente, presentando novedades de vez en cuando, a una comunidad de escasos anoms que no se pierden a la larga.

Una serie de errores poco perceptibles fueron corregidos, de eso se trató secundariamente la catalogada como subversión que llegó alrededor de una semana después de la anterior actualización. Concretamente uno de ellos deriva de los famosos sabores, esos que cubren el inicio entre palabras y marcos de bandera ucraniana, cuya visibilidad se veía afectada a raíz de las transformaciones que realiza el sitio a lo que son sus gradientes para variar su retrato general. Esto fue posible gracias a la contribución anónima en el espacio fantemtiano de Rouzzed, que además de ser el donde frecuentemente aparecen los fragmentos de material cultural que a la postre van a parar al canal recopilatorio, también oficia como medio de colaboración para con el desarrollo. El ya único fantem fijado cumple un rol similar, no obstante parece servir más para intercambios un tanto menos serios, y a veces queda medianamente relegado en atención respecto al otro, a pesar de ser el segundo en lograr superar el mil de respuestas, cifras récord en la totalidad de los antros temtianos, debidas a la constante pero leve participación, mientras hay continuidad y no reinicios.

Lo extraño, y quizá no principal de esta ocasión porque no es visible de primeras, es otra introducción adicional a la lista de temas, igual como había pasado en dos oportunidades antes. Extrañamente, no va por el lado de ningún antro propiamente perteneciente al contexto de los clones, pero sí es sobre otro conocido en él. \emph{/Anonima.club/} es y así lo mencionan los anuncios, exageradamente rosado y con sus condiciones más orientadas al público femenino, trasladadas a Fantemti, con nada más que unas pocas pulsaciones de botones. La administración no da casi explicaciones acerca de esta elección, interesante y correctamente trabajada, a lo mejor congruente con las acogedoras e inclusivas vibras que en transmitidas o promovidas, y tal vez relacionada con la sospechosa identificación manejada por la única figura que integra la conducción, sin embargo es llamativo de igual modo. ¿Con qué sorprenderá la próxima?

\section{El rejuvenecedor despegue anaranjado (Decimotercer libro, capítulo XII)}\label{el-rejuvenecedor-despegue-anaranjado-decimotercer-libro-capuxedtulo-xii}

\begin{quote}
23 de abril de 2022
\end{quote}

Bajo el título de que \emph{/Se re vino en Temti/}, un importante llamado de atención desvía las miradas de decenas de rozzados en su propio hogar hacia el único sitio temtiano activo para el momento, presentando uno de sus factores más emblemáticos, de lo mejorcito que hay, o había, según el parecer de bastantes sujetos que tuvieron su último paso hace meses y meses, presumiblemente conservando la clásica fama generada con su polémica. Semejante parece ser la atracción que este tema genera, que el promedio de visitantes cercano a cero se disparó increíblemente, no al duplicarlo ni cuadriplicarlo, al agregarle una cifra. Casi veinte llegaron a ser los conectados al recibimiento de nuevos Tuturú, suma enorme que inmediatamente pasa a encabezar una jornada totalmente histórica, sin precedentes algunos, y todavía podría llegar a más.

Para las presuntas intenciones de esos numerosos anoms hay un solo cuadrado, que aún sin otorgar vista previa por las palabras que lo resumen es de interés, y rápidamente desde su formulación logró una pequeña serie de aportes provenientes del autor. El tem que más furor causaba en las primeras épocas, como en lo tardío de los entreveros clónicos, como en lo posterior al inesperado resurgir, vuelve a florecer con una leve fuerza ahora en la reciente sucesora de Temti, Fantemti, tal cual directamente indica por sí solo, como en los viejos tiempos.

En lo inmediato, el episodio se encuentra abierto, múltiples foráneos aunque callados no culminan su estadía, cuando más de ellos podrían caer en un santiamén para contribuir con la temática, que bajo una mirada estricta del reglamento, estaría aprobada y permitida. No obstante, bajo otra clase de criterios, sean morales o que habitúen a regular lugares aparte, posiblemente esto sea condenable. Además, es de conocimiento público que este tipo de circunstancias tienden a ser extremadamente volátiles, el accionar de cualquier prudente o no prudente actor es capaz de desequilibrar la picante efervecencia de esta relativamente novedosa aparición, sobre la cual el equivalente al A. no se ha pronunciado a favor ni en contra, pero que desde sus reiterados recuerdos del pasado, debería tener más que claro que las posibilidades de un final no deseado son importantes.

\section{Traspasando la confianza (Decimotercer libro, capítulo XIII)}\label{traspasando-la-confianza-decimotercer-libro-capuxedtulo-xiii}

\begin{quote}
23 de abril de 2022, \ldots, 29 de abril de 2022
\end{quote}

El gran repunte de tráfico generado mayoritariamente por el exagerado anuncio realizado en territorio rozzado alcanzó a su prematuro máximo, en las horas de la noche no se llegó a la veintena y dentro de cada rincón fantemtiano la tranquilidad primó como ocasionalmente lo ha hecho, salvando un puntual caso que en el momento podía esperar reacción rápida, no porque haya sucedido algo terrible o así, solo por la atención especial que cualquier movimiento como ese pudiera ameritar. El clásico tem de pendejas en su intento de ser refundado donde la vigencia de Temti reside no fue un éxito entre comillas como sus dos grandes presentaciones pasadas, que lograron superar los varios centenares de replicas, esta ni una décima parte de ello. Pero eso no sería lo importante, contando que entre ellas varios comentarios ya no se ven íntegramente como inicialmente aparecieron, en el sentido de que terminaron moderados.

Bajo una realidad ciertamente tan real y no irónicamente poco fantástica, en la que ningún concepto como los de antes se maneja, la respuesta no está en Hades, psíquicos, ni nada de eso, apunta a la administración y una acción ante lo que fueron, sí, efectivamente transgresiones al reglamento del sitio. Habrá sido cuestión de verlo y el nivel de gravedad quedará a juicio del que llegara a coincidir puntualmente en el instante justo con los hechos. Pareciera que no hubiera sido demasiado, puesto que del otro lado los guiños quedaron en eso y no pasó a mayores allí, no obstante cierto es que como alguna vez dijo A., /\emph{se procedió lo más rápido que se pudo/}, de lo contrario el discurso estaría siendo distinto, a lo mejor mucho más escandaloso.

A su vez, otra medida claramente incentivada por el peligro de exponerse a nuevas faltas a la acotada normativa, fue arremeter contra la vital función que posibilitó todo lo anterior, la carga de archivos y enlaces, la cual durante unas cuantas horas no estuvo disponible y de tal modo fue imposible para cada visitante hacer uso de la misma, así fuera para cometer ilícitos o dar el ejemplo sin sumarse a la flamante movida. Igualmente fue eso, una circunstancia transitoria, debido a que pronto fue anulada y el antro regresó a su estado natural, fuera de protocolos extraordinarios, que curiosamente considerando la existencia del sistema de sanciones, confirmaron el que desde antes la conducción ya contaba con al menos dos herramientas para afrontar inconvenientes como estos, coyunturas que requirieran restringir las capacidades de los portadores de código.

No obstante, ha sido insuficiente en términos de esa materia, pues los anoms participes de aquel cierre de día con vibras amarillas, metafóricamente hablando, posteriormente crearon unos puntuales fantems de temática semejante, que a pesar de ser bastante discretos entre los poco descriptivos cuadrados de colores, rápidamente para quienes más frecuentan el nicho se supo que estaban dando vueltas por allí y que podrían volver a acarrear los clásicos problemas que acostumbra a traer la permisividad tan laxa que da lugar a esas actividades.

Por tanto, dicha perturbación fue trasladada en bajas proporciones a la publicación oficial de Fantemti en Rouzzed, donde de costumbre la individualidad uruguaya trata temas de todo tipo relacionados a lo que es su creación y la causa a la que se atiene. Una mención fue para la actualización aplicada en los días corrientes, concreta de los mecanismos de multimedia, que fueron adaptados a un formato muy similar al del dichoso vecino, y la relevancia de la contingencia polémica terminó muy disminuida, como si estuviera bajo control. La verdad es que respecto a ciertos margenes tiene la última palabra en su sitio, sin embargo no el dictamen final, por lo que lo siguiente es aguardar a la espera de futuros incidentes, positivos o negativos.

\chapter{Desde la inauguración del club privado y el desteñimiento de la libertad (Decimocuarto libro)}\label{desde-la-inauguraciuxf3n-del-club-privado-y-el-desteuxf1imiento-de-la-libertad-decimocuarto-libro}

En virtud de las aproximaciones más recientes, posteriores al llamado sucedido en la interna del coloso vecino, el sitio fantemtiano entró en conocimiento de más anons y eso inevitablemente conllevó sus consecuencias, que podrían haber sido más discretas o no. Dado el encabezado de dicha publicidad, el alcance no resultó menor, al llegar a públicos un tanto más complicados y problemáticos, que tienden a acercarse más al borde de lo permitido en las ubicaciones de los clones. Sin especificar, muchas amenazas y ataques a sus existencias se han realizado a base del famoso contenido prohibido, también conocido por todas sus denominaciones informales derivadas de las iniciales del término formal en inglés, y sabido es lo común que es ver a los sujetos jugar con eso sin llegar a rebasar la gruesa linea que habitualmente bromean traspasar o esperar. Y no siempre hay armonía al hablar en esos términos, la excepción a la regla no repitió, dado que parecía haberla hasta que un aviso irrumpió la aflojada tensión, aquella desprendida desde un principio del intento por restaurar un tem clásico.

La perseverancia bilateral no se diluyó entre el palabrerío y las fotos talladas sobre azul, los obvios motivantes de cada una de las medidas tomadas por el lado de la autoridad máxima, que por lo visto resignó jerarquía, torció su voluntad inicial al verse hostigada a adaptar las circunstancias y generar un marco más apropiado acorde a los confines impuestos por las partes de las que depende el antro, que en su corto recorrido ha procurado mantener correctas relaciones así no lo amerite bajo criterios comunes, esta vez con el cometido de no arriesgar el estado de uno de los pilares fundamentales en su dinámica, o con una mirada más exagerada, la integridad total de sí misma. Eliminación de contenidos, posibles sanciones no declaradas, restricciones de poderes transitorias y reiterativas, modificación de normativas, es decir múltiples reacciones que sin lugar a duda reflejan una fuerte inquietud, disconformidad con el rango de libre albedrío existente hasta esta reciente cadena preventiva, la cual establece un traumatizante antecedente de cara a futuro, representación de otro extenso y difuso punto de quiebre en la Historia Temtiana, a partir del cual su coyuntura podría ya no ser igual.

Yendo un poco más al detalle de, los pasos han ido ascendiendo en cuanto a lo que conllevan, pues de arranque la reacción no llegó a ser extraordinaria bajo ninguna circunstancia, solo hasta que el verídico límite fue cruzado, más tarde el momento de inflexión donde se irían apilando los intentos de solución, contando esos que subestimaron la seriedad que finalmente se le terminaría dando. En primera posición, mira automáticamente puesta en las participaciones desubicadas para su rápida remoción por estar fuera de lo autorizado. A la brevedad, deshabilitación parcial de los permisos requeridos para cargar multimedia, englobando tanto a lo que no debería ser objeto de moderación como también lo que sí. Muy posteriormente, actualización del reglamento con perspectiva a expandir el margen represivo. Y cerrando ese revuelto de tentativas rumbo a un mismo fin, el retorno de la restricción global a la función que permite subir archivos, exceptuando a una serie de individuos aprobados al estilo lista blanca.

Cada señal apunta a erradicar los materiales propensos a ser considerados como incitación, que mientras pudieron siguieron apareciendo, y con la entrada de los códigos premium quienes quedan exentos del perjuicio son aquellos que obtuvieron la no polémica validación de la administración, y así es como una jugada de opresión no tiene efecto directo sobre el sector más fiel, el dominante, agraciado en una virtual división que recorta las capacidades de los visitantes más anónimos, sujeta a ser sostenida o removida según la contemplación de su rendimiento y costo considere. La realidad fantemtiana encausándose por una vía donde el condicionamiento en comparación al de épocas anteriores es de potencial bastante superior.

A preguntar sería si un presunto incidente y otro adicional par de acercamientos son lo terribles que son como para llevar a tomar decisiones de incidencia represiva, eso esperará la evaluación sobre la marcha, una determinación futura, o más incluso. Asimismo, es un precedente lo suficientemente relevante también porque con él se evidencia al punto que ha llegado la consolidación de la avanzada iniciativa, lejos de quedar como un simple rincón para allegados a la más perseverante resistencia temtiana, además de las evidentes concreciones en el desarrollo que la posicionan por encima de su modelo a seguir respecto a funcionamiento, ganó una pizca de popularidad. A pesar de que en ocasiones no se la llamó por el propio nombre sino por el de la no directa antecesora, como si fuera idéntica, lo objetivo es que por ahora no adquirió ni de cerca la fama de esta, ni la reputación, ni el concepto. Igualmente, es un reconocimiento mayor.

Eso significa más, porque dada la notoriedad, a su vez la exposición, que va desde el aumento de las chances a recibir más visitas, hasta el incremento de riesgos de padecer atentados. Sí es correcto que después del gran pico récord de conectados no hubo subida en el bajo promedio que de costumbre desalienta las expectativas de encontrar un lugar movido, y eso es un elemento en contra para la consideración de la siguiente posibilidad, no obstante la puerta a que en lo próximo el sitio pueda despegarse del casi absoluto desconocimiento, se abre. No hay indicadores que prometan futuros avances de ese tipo importantes, por lo que sostener el actual apego a la discreción, es altamente factible.

Pero previo a ello, interrogantes discrepan con la continuidad de la escalada llevada hasta la fecha, claramente desacelerada, y no enfocada en ampliar la cantidad de adeptos regulares según sus más altos mandos. Cero intenciones aparentes de hacer jugadas fuertes de difusión, chistes que demuestran conformidad con los ínfimos niveles de interés, negativa por apurar la solicitada expansión de las imágenes, infraestructura y recursos limitados por la nula apuesta económica, aparte de la desventaja que le supone rodearse de alternativas mucho más pudientes y convincentes. Muchos factores hunden cualquier coincidencia que pueda darle un mínimo despegue a la humilde casi imitación, que no ha dejado de cumplir su original propósito, pero que a las vistas de satisfacer una esperada trayectoria superadora, se está quedando atrás, y la serie de episodios inmediatos para en el pasado son de notable referencia en ese camino, todavía sin final a la vista.

\section{El regreso de la policía (Decimocuarto libro, capítulo I)}\label{el-regreso-de-la-policuxeda-decimocuarto-libro-capuxedtulo-i}

\begin{quote}
29 de abril de 2022, \ldots, 1 de mayo de 2022
\end{quote}

Aterrizando Fantemti del ascenso brusco e inesperado provisto por la visita de numerosos rouzzeros poco participativos, la realidad donde tuvieron un breve pasaje pareció encaminarse a su normalidad frecuente, en la que a cuenta gotas participan los mismos levemente reconocidos entre sí individuos, pero en verdad una pequeña porción de los nuevos permaneció rondando en las inmediaciones de esa novel alternativa que continúa la linea de un viejo conocido. Así como permanecieron, se trate de varios o uno solo, también continuaron manejando el mismo tipo de materiales específicos del resurgido tem de pendejas, solo que no dentro este mismo, sino en otros fantems aparte, donde generaron repercusiones similares, multiplicando el percibido malestar por parte de quienes en los anteriores meses nunca manejaron esa clase de intereses. Merece darle entidad porque en la semana más reciente fue como un tira y afloje, leve y mínimamente intensificado, pero de igual forma con presunta relevancia, porque tras el planteo de un argumento serio, finalmente la razón cede a favor de la prohibición, la segunda causa de la consecuencia más evidente que tuvo el inicio de las inestabilidades, el cambio en el reglamento.

Uno de estos días, el sitio amaneció con una de esas extrañas alertas con tonalidad oscura, de muy rara aparición desde que la primera fue vista, y más allá del propósito del anuncio que por sí solo tiene su significación, el motivo está claramente ligado a dos de los asuntos interrelacionados más controversiales en los últimos tiempos. El aviso en cuestión viene a cometido de notificar sobre lo que está detallado al final de las secciones informativas más protocolares, cualquier alteración a realizarse tendrá su difusión correspondiente, y así es, se comunica a secas que \emph{/Fue modificado el reglamento de Fantemti. /}. La variante deriva de un inciso demasiado puntual, el más radical de las tres sencillas limitaciones, que prohibía toda clase de material formalmente no apto para menores, que en esta oportunidad es extendido hasta incluir dentro de lo inhabilitado\ldots{} la obscenidad. Siendo un concepto medianamente claro aunque con sus difusas interpretaciones, apunta directamente a amparar la eliminación de aquello definido como \emph{/nenillas/}, que comúnmente por estos lugares no son solo eso, sino que también vienen condicionadas por una connotación que apunta a ser problemática desde numerosos sentidos. En reincidencia, es una expresión semejante a los imponentes Términos de uso en la clásica Temti, donde la misma clase de contenidos estaban relativamente fuera de norma, mediante la peculiar redacción que mencionaba esa histórica denominación, incitación, que dicho sea de paso a veces estaba de adorno.

Tampoco fue la única medida que tomó la administración hasta la fecha, porque otra repite yendo en sintonía con las reacciones provisorias cuando esta circunstancia solo tuvo su explosión y los primeros entre comillas incidentes colaterales, la casi indispensable función de la multimedia restringida únicamente para los portadores de código premium, por no decir removida del alcance terrenal. Esa resolución intuitivamente con aspiraciones de defensa volvió a ser aplicada brevemente para cesar las inclemencias, que según comunicaron cada una de las señales de arriba, hay voluntad de evitarlas, pero con la marcha atrás aún faltaría una imposición clara para asentar si se optará por algo además del acuerdo escrito, porque este no necesariamente recibirá el respeto pretendido.

Aún siendo este revuelo la dinámica que más resonancia consiguió y donde el enfoque primario estuvo, la gestión fantemtiana no se limitó a colocar barreras y establecer pasillos en lo relativo a la libertad de sus adeptos, porque tras la inadvertida pero exitosa actualización instaurada para la carga de imágenes y videos, más detalles pequeños complementaron el vigente avance del desarrollo, esta vez a partir de una solicitada característica ya existente en la ausentada antecesora, aunque de uso sumamente defectuoso, la herramienta para citar texto, el coso verde, o como sea la manera de referirse a él. Nada fantástico, sin embargo estaba faltando y era casi que necesario, o al menos en los papeles fácil de conseguir, más lo notable en el hecho de que la comunidad solicite algo simple en específico y rápidamente sea cumplido, una costumbre que a rasgos generales se ha sostenido, en contraste a los modos evasivos de la época titiritera.

Además de confirmar el convencimiento, no es momento de elaborar conclusiones, los movimientos ejecutados en torno a las normas y las jugadas realizadas para asegurar el cumplimiento de las mismas pasaron a una segunda etapa que difícilmente sea la última. Por ahí anda la frase de que las reglas están para romperlas, y en ese caso quienes las vienen recibiendo de hace rato normalmente tuvieron la obediencia suficiente como para dar tranquilidad a quien las puso, fueron entendidas por ellos, sin embargo ese grupo adoptó integrantes adicionales, que en corto tiempo demostraron indiferencia frente a lo considerado correcto en un lugar que sin poner el grito en el cielo evidenció manejar cierto nerviosismo, y no estar completamente preparado de antemano para encarar una situación extremadamente común en el contexto de los clones, la cual después de romper casi tres meses tranquilos está intentando liquidar.

\section{La chapa de la clase alta (Decimocuarto libro, capítulo II)}\label{la-chapa-de-la-clase-alta-decimocuarto-libro-capuxedtulo-ii}

\begin{quote}
1 de mayo de 2022, 2 de mayo de 2022
\end{quote}

Más o menos como pudo ser notado en intervalos concretos las últimas horas, al verse predeciblemente recordado el error de que no es posible proceder sin satisfacer una condición específica, quedó al descubierto que la misma configuración estipulada por la conducción fantemtiana para limitar las posibilidades de los visitantes regresó, en consecuencia afectando a todos ellos por igual. Fue a raíz de una nueva actuación del o los personajes que en los fantems intercambiaron más de aquellas fotografías calificadas como obscenas, que antes derivaron en alguna violación al reglamento, y que al aparecer por enésima ocasión instalaron la inquietud entre la comunidad. En otra oportunidad, el protocolo fue encendido.

Sumado a eso, los movimientos generados en el espacio de dialogo con la administración tomaron un camino extraño, no por tener más pequeños hilos de conversaciones aisladas y sacadas de contexto, sino porque tratarían de un asunto históricamente medio oscuro en cuanto a la transparencia, que con la transición de Temti a Fantemti pudo ser esclarecido respecto a sus condiciones propias, los códigos premium y la supuesta inexistente utilidad, cosa que recientemente fue suscitada para cumplir por primera vez su papel en la disputa por el cumplimiento de las normas, para asegurar que nadie fuera capaz de ir contra el segundo punto de las tres prohibiciones.

Acudir a ello de transitoriamente es entendible, pero quedarse para siempre de ese modo ya más serio, grave y con seguridad blanco de agudos descontentos, como dar un gigante paso atrás y regresar a las primeras épocas cuando la multimedia brillaba por su ausencia, solo texto y texto. Antes de atravesar esa trayectoria, surge una solución más que óptima, a fin con el sentido y propósito de aquel sistema del cual hasta hace poco nadie había podido degustar ni siquiera un mensaje de éxito, útil para combatir al dominante miedo repetir la escena en posiciones y dimensiones alternadas, otorgarle la capacidad a un puñado de individuos que cuenten con el visto bueno de quien supervisa sus actividades, y hacerla exclusiva.

Como parche para el desconcierto, los discretos anuncios brindaron una serie de explicaciones medianamente formales para darle un apoyo a los intentos de comprensión, si es que fuese necesario alguno adicional al que tuviesen los directamente agraciados por los efectos de la resolución. Allí está casi todo dicho, qué ha pasado, por qué ha sucedido, cómo se ha respondido, y de cuál manera se continuará. Desde el atento seguimiento, la verdad es que faltan detalles y es como una versión simplificada de la sumatoria de hechos, lo que más interesa o es relevante para los temteros si se quiere llamarlo así. Procurando no reiterar lo ya enfatizado, resta desmenuzar los factores que previamente no habían estado en el panorama.

El plano principal, siendo descartado el ascenso directo sin intercambio de por medio, la distribución de comodines premium, aquellos que se mencionó en su momento que son para introducir en la dirección de la llamada \emph{/conversión privilegiada/} y que nunca se dieron a conocer hasta la fecha. Ahora sí, tras haber sido usados están en posesión de unos pocos fantemteros que respondieron a las comunicaciones oficiales confirmando de distintas formas el que fueron capaces de acceder a ese nivel superior que tanto y tanto fue tratado como mito sin jamás confirmarse, tener un código premium, validado por el sitio de turno. Sin embargo, a un segundo plano pasa el medio de entrega, y el método de selección.

Fijo es que las estructuras permiten un tipo de comunicación privada entre la máxima autoridad y los simples ambulantes, de lo contrario no habría via que les haga llegar las combinaciones para activar su identificación privilegiada, de no ser plataformas exteriores, o fugaces participaciones con rápida expiración. Dicen por ahí que una sanción fue la responsable de regalar el comodín premium \emph{/paraUstedfantemterx/}, frase extraña para tratarse de una clave de valor. Es efectivamente funcional, pues como también se difundió por allí, ya fue aprovechada. En lugar de eso, daría más a esperar una cadena autogenerada de letras y números aleatorios, sin significado, pero no es el caso, tiene su toque personalizado.

Además, entre el desarrollo de la reciente aclaración, no se menciona ni indica el criterio para decantarse por los escogidos a ser los pioneros de esta clase predilecta. Donde el fundamento, lo obvio de que la autoría de cada comentario y fantem queda enlazado a su original publicador, pues así está explicitado en la Política de privacidad, pero según esta faltarían los justificativos para husmear en lo que supuestamente es confidencial. Esos \emph{/quienes siempre se han comportado correctamente/} no son al azar, o son los más participativos desde el inicio que perduran, o los que tras una revisión de sus historiales aprobaron las exigencias, lo cual dicho de esa forma vuelve a sonar crudo para sus datos secretos, el anonimato nuevamente sumando asteríscos.

Aunque probablemente en realidades del exterior hayan existido y existan modalidades de respuesta similares, no hay episodios cercanos de referencia como para analizar si las circunstancias se asemejan. Podrá partir desde el profundo convencimiento, pero no por eso deja de ser experimental, al haberse construido el antro a imagen y semejanza de otros similares lo común es encontrarse con calcos y copias de lo que estos tienen, por algo es que son llamados clones, y sin dejar de considerar excepciones, este tipo de manejos no habituaron a ser vistos entre ellos. Más allá del condimento que brinda el estigma de los códigos premium y sus vistosas iniciales, los sistemas de permisos pocas veces trascendieron hasta los niveles más elementales, exceptuando este ejemplar, que aprovecha lo que varias versiones pasadas fue instalado sorpresivamente, la ventaja de contar con las características para restringir el poder de los anoms puntualmente, lo cual marca una diferencia con los ejemplos a seguir más evidentes. Los cambios hacen ruido con mayor facilidad.

\section{La colectividad cerrada y su propietario (Decimocuarto libro, capítulo III)}\label{la-colectividad-cerrada-y-su-propietario-decimocuarto-libro-capuxedtulo-iii}

\begin{quote}
2 de mayo de 2022, \ldots, 7 de mayo de 2022
\end{quote}

Cuando el rumbo natural de la actual realidad fantemtiana hubiera indicado el regreso a la senda de las recurrentes mesetas, sin nada por destacar, lo que siguió a la cadena de revulsivos sucesos dejó un tanto distorsionada la expectativa y por lo tanto tras ello ya no puede esperarse una continuación idéntica, no por nada un antes y después marcó el cierre de una época y el comienzo de otra. Sin embargo, al desarrollarse esta última más alrededor de la interna favorecida, las distancias grotescas se fueron viendo reducidas en buena medida. La restricción para el uso de la totalidad de funciones permaneció desde su más reciente principio y sin final, ya un buen rato interrumpido, lo que comunica relativa conformidad o cumplimiento de objetivos, más el detalle de que ha tenido aceptación básicamente, las quejas al contrario de estar, ausentes. Eso reafirma lo que con las discretas entregas era de intuirse, las individualidades más activas habrían sido las electas para continuar como si nada aconteciera en su entorno. Por eso, es que no notan gran diferencia.

Alternativamente, no todo fue por el lado de esa dominante porción que recibió la confianza, también por otra aproximación aparecieron visitantes no puestos al tanto con la situación, ni con los motivos, ni con la manera de remover esa brecha que los deja por debajo y los priva de lo que normalmente es una acción básica en cualquier clon de estos. Sumó que en el mural informativo estuviera la extensa explicación, sumó que entre los anoms se comunicaron certeramente, la desinformación quedó cubierta apropiadamente y tampoco es que llegara a despertar la desilusión de algunos. El caso puntual se halló en uno que ya sabiendo la cuestión, se interesó por acceder a eso de los códigos premium y por ello procedió a solicitarlo en aquel espacio para hablar con la administración, donde su pedido principal fue concedido y así según lo pinta el hilo de la conversación, se unió al club exclusivo de quienes pueden cargar multimedia.

A pesar de la imposición que como consecuencia inmediata relega a aquellos que ingresan por primera vez, tras su aplicación hubo un leve incremento sostenido en el promedio de presentes, es decir que la expuesta cifra que marca la cantidad de visitantes en linea, repetidas veces se situó durante largas horas por encima del intervalo usualmente
registrado. Respecto a la evidencia que explique eso, lo primero a considerar sería la radical determinación en el antro más relevante del contexto, Rouzzed y el levantamiento del llamado muro, más la clausura de la fiel copia Voxxed, que consecuentemente dejó a numerosos individuos vagando por las cercanías sin acceso a la predilecta alternativa, donde las opciones restantes son Ufftoppia, o Fantemti. Sin embargo, la participación acompañó al aumento inesperado por un lapso no muy largo, luego el movimiento del sitio dejó de ajustarse a los participantes contabilizados. El mismo dicho de que son en su mayoría \emph{/mirones/} es reiterable, pero un poco más extremo, desconfiado con lo que debería ser cierto, porque la sospecha de que fueran números adulterados rápidamente se posicionó por arriba del asombro, suposiciones para nada exageradas con razón, no se nota que hayan tantos conectados.

La manifestación se resumió a unas puntuales acusaciones a la autoridad de ser culpable y mentir, por inflar las cantidades supuestas. La respuesta fue negativa, a lo mejor creíble confiando en las evidentes causas, y en la parcial transparencia mantenida hasta el momento. No obstante, por más extraño que resulte, que suceda lo opuesto es posible incluso: mirándolo por el lado de que generalmente se trata de cifras extremadamente bajas, tanto que el solitario uno es recurrente, quizás la opción de maquillar un poco esa criticada verdad podría haber sido tomada y no declarada, puesto que ya fueron un montón las veces que la media de un solo dígito fue aprovechado para hacer chiste de ello, satirizar el nivel de desinterés que genera un lugar abandonado, como si fuera una miseria. Tras demostrar una fuerte indiferencia ante el limitadísimo tráfico, desprenderse de esa postura inicial sería un retroceso llamativo, otra similitud con las prácticas de la vieja Temti, contabilizar relleno artificial.

No estuvo siendo el único factor que puso foco en la conducción, nuevamente su activa presencia se alejó más de pasar inadvertida en un ambiente donde no hay demasiadas direcciones aparte para observar, el mundillo temtiano disfruta su atención de procedencia identificada: Celeshtee, gorda ratones, rata traidora, entre otros seudónimos conocidos como tal por fuera de la gestión del sitio. Por numerosas referencias públicas tanto dentro como en el exterior de la escena anónima, su vinculo directo con la cima de Fantemti ha ido quedando descubierto con el tiempo, al punto de que el convencimiento sumamente alto en los seguidores más cercanos a la actualidad temtiana, y los intentos por ir en contra de ello, ningunos. Aún estando esto sabido, en el rol de dueño o dueña jamás se ha presentado con un apodo o nombre, distinto a como sí lo hizo A. en los comienzos de su camino, por lo que quizá tenga intenciones de manejarlo reservadamente, o tal vez considere que a esta altura es innecesario hacerlo. Como sea, en definitiva no cambia nada, esta fiable idea se viene manejando desde hace mucho, solo que una serie de menciones suscitaron el tema en un pasaje que alcanzó la suficiente relevancia, cuando su predominancia dejó de reducirse a incidir sobre iniciativas de propósito cultural, lo que queda del arte y la historia, si además de todo eso también se encarga de perpetuar la más que semi-imitación y homenaje, fundamental para la subsistencia y unión de los temtianos incondicionales en el presente no comprometido.

\section{La parcial seriedad en aires de concordia (Decimocuarto libro, capítulo IV)}\label{la-parcial-seriedad-en-aires-de-concordia-decimocuarto-libro-capuxedtulo-iv}

\begin{quote}
7 de mayo de 2022, \ldots, 14 de mayo de 2022
\end{quote}

Aún con lo determinante de la nueva configuración establecida, un cambio puntual en lo que acostumbraba a ser el estado total del uso fantemtiano no se constató con grandes diferencias en lo que fue su escena posteriormente, pues a nivel global se mantiene medianamente como antes estaba, solo agregándole una pequeña situación de mínima trascendencia al fin. Así es que se retoma el parejo ritmo de actualizaciones, el no tan lento movimiento general, acompañado por la creación desproporcionada de fantems. Pero con la salvedad más matices por destacar.

Los paquetes de mejoras aún siguen llegando en misma linea con los que hasta hace no mucho eran prácticamente constantes cada semana, trayendo un lote de novedades no demasiado extenso, aunque de todos modos con utilidades efectivas. Sin embargo, únicamente una de ellas contribuye al abanico de posibilidades al alcance de los individuos ordinarios, puesto que el resto fue al lado de lo que no se ve, moderación y ordenamientos internos. En este caso se trata de una sección adicional para el sitio, donde aparecen listados los comentarios más recientes, parecido a como es con los hilos en el inicio solo que apilados en columna, ordenados cronológicamente con un rápido acceso al contexto de donde originaron. Aparentemente, es a raíz de una solicitud expresa proveniente de los fantemteros, que en el frecuentado espacio de intercambio con la administración comunicaron su interés en que se añadiera para ver las participaciones propias, y esa idea cumplida también fue aprovechada para reunirlos todos, sin distinguir el autor.

Inclusive del propósito inicial de dicha inclusión, que claramente suma y corre las limitaciones, rodea más aún las características indispensables del antro, apoya desde un punto más la certeza acerca de la realidad corriente de los acentuados intervalos sin circulación, que si bien hoy por hoy no se condicen con niveles de actividad muy bajos, tampoco son de idas y vueltas muy repetidos. Otra cosa que vino mencionada entre el repaso de las novedades, dos discretos avances en lo que son la detección del texto verde y los recuadros descoloridos de multimedia, improvisaciones muy específicas pero positivas, así un par de desperfectos quedaron subsanados, o por lo menos improvisados. Y finalizando, la siguiente reiteración va dedicada a una linea que es relativamente ajena al estamento predominante del antro, por tanto de ínfima relevancia para su cotidianidad, fortalece a la conducción en sus tareas de cuidado y mantenimiento.

A pesar de los detalles dados, concretamente a qué más no se sabe. Poco y nada se conoce sobre lo que hay tras las compuertas que la autoridad resguarda bajo su control, al igual como en los antros temtianos predecesores numerosos diseños y estructuras permanecen en secreto. Sin tener disponibilidad del código fuente donde cada detalle podría apreciarse con debida claridad, igualmente es presumible que a partir de las acciones extraordinarias dentro de Fantemti, propios de quien la controla. Al juzgar eso, es de esperarse que allí una apropiada sumatoria de herramientas respalde la supremacía del mandamás, unas buenas bases para mantener sus dominios bajo supervisión y demás, que según las distintas medidas que ha ido tomando ni bien comenzó la gestión, está al parecer por encima de lo mínimo indispensable, superior en un área más respecto a su equiparación directa.

No más y suficiente por esa parte, son varias líneas que tampoco opacan la área donde la Historia Temtiana pide más atención actualmente, la cuestión del título de honor ante el recorte de libertades. Visto está que los individuos selectos continuaron sus experiencias como si nada, puesto que sin estar identificados propiamente como portadores de código premium está evidente que así lo son, no resignaron ninguna capacidad cuando por contra los comunes y corrientes sí, y algunos cuestionamientos surgieron al respecto.

El entendimiento manejado en los días más próximos a la definición de la actual estructura restrictiva, no alcanzó a transmitirse hacia la totalidad de los anoms, porque además de nuevas solicitudes por insertarse en el grupo del privilegio, que según los registros públicos dicen no derivaron en más que eso, también hubo alguna que otra queja literal, lo cual demostró que el panorama no fue correctamente asimilado. Seguramente aquellos no mantienen demasiado esmero como para estar al pendiente de todo lo último en el entorno de Fantemti, así que suposiciones como el que es requerido un tipo de pago para acceder a las funciones básicas es totalmente esperable, pero en verdad no fue tan así, inicialmente el proceso de selección se basó en el comportamiento de acuerdo al reglamento, donde probablemente se requirió una importante suma de tiempo desde la inauguración, cosa que un entrante reciente no tiene.

No obstante, contando con el reciente episodio de entrega tras solicitud, quedó al descubierto que no es estrictamente necesario contar con un largo historial intachable, al tener esa ocasión de un individuo que solo tuvo que pedirlo y como todo indica así lo obtuvo. Fue a partir de eso que se estableció otra manera de llegar a los privilegios, de la cual la comunidad fue testigo, para luego difundirlo a ese sector desinformado: \emph{/pedile al admin/}, porque al menos una vez hace bien poco, fue tan simple como eso, y de hecho varios siguieron el procedimiento, aunque sin haber gran información de si consiguieron éxito. Es otro punto donde falta transparencia total, que no es la primera discreta mancha, habida cuenta que si tomáramos el que anteriormente las cuestiones más elementales de esta dinámica fueron aclaradas íntegramente desde los anuncios, ahora estaría faltando algo, ya que oficialmente nada se ha explicado en cuanto a cómo podrán los visitantes obtener su clave para efectuar la conversión privilegiada. El precedente exitoso es de público conocimiento, y el resto sin confirmación dejan una incógnita: no está claro si fue una excepción, si a partir este instante esa será la forma, si hay otro tipo de requerimientos a cumplir, o qué. Ni hablar de si pronto el ya asentado protocolo será desactivado, o si la antigua normalidad jamás regresará.

Aún con dicho asunto acentuando un hueco en las formalidades declaradas, no se noto un gran problema ni descontento amplio en el sitio, en particular porque el conjunto más participativo en realidad permanece intacto, de lo contrario podrían estar haciéndose escuchar. La reaparición del tópico chat cumplió un papel relevante en la generación de movimiento, por su modalidad de ida y vuelta la interacción se vio al alza, además de una interesante fascinación con el mecanismo de reinicio donde el ciclo estuvo llamado a repetirse. Cifras frecuentemente elevadas para los promedios del pasado verifican que la colectividad reducida ahora se encuentra un tanto más enganchada bajo el techo casi celeste y amarillo. Una prosperidad movida logró dominar la actualidad fantemtiana.

\section{En carrera del salto más deseado (Decimocuarto libro, capítulo V)}\label{en-carrera-del-salto-muxe1s-deseado-decimocuarto-libro-capuxedtulo-v}

\begin{quote}
14 de mayo de 2022, \ldots, 16 de mayo de 2022
\end{quote}

Un buen aluvión de novedades marcó otra nueva jornada de superación en el camino de la historia construida por el sucesor de la desaparecida Temti, o Titiri, cuando las características de siempre padecieron una ampliación un tanto mayor a las habituales, con la particularidad de una sumamente destacada, sobre todo por su posible significado con vistas a lo próximo.

Partiendo desde lo más pertinente, comenzar por aquella última mencionada, la más relevante de todas ellas, que no solo para el presente inmediato supone una seria señal que estará latente de ahora en más, sino por lo que pueda traer próximamente como para seguir en igual linea. Solicitado a mansalva desde el comienzo de esta larga etapa, finalmente puede verse en al menos un fantem especial, destacado de manera única por su flagrante diferencia respecto del resto, donde el típico sabor correspondiente quedó totalmente tapado por la presencia en el frente de una imagen. Sin haber interpretado el boletín informativo de esperar en la ocasión, sembraría una indudable premonición de que la privación se terminó, el despegue definitivo a que el clon de Temti adopte la última similitud faltante como para que las distancias groseras ya sean casi inexistentes, sin embargo no es lo que parece.

El ejemplar fijado es una pieza del homenaje, exaltación del estima y aprecio por los orígenes e identidad, sumado al fanatismo de base, sin razón aparente para la oportunidad puntual, a lo mejor el contenido preferido para exhibir la flamante característica. Con el título \emph{/Temti para siempre/}, no pone ninguna comunicación seria en manifiesto como es de suponerse para una publicación fijada, en principio podría ser nada más un reincidente recuerdo que resalta el primer retrato predilecto en la creación del administrador A., tal como ya ocurrió anteriormente con la secuencia del Spiderman árabe que escolta al principal hilo de recibimiento. Fantemti copia la elección de su antecesora y la estampa en la pantalla inicial, en uno de sus cambios más trascendentes hasta la fecha, manteniendo la memoria más viva que nunca.

El pero con todas las letras es en reincidencia quien está habilitado para acceder a ello, puesto que con lo natural y corriente que es el patrón de imágenes en sitios de similar compostura, debería ser para todos una capacidad sin obstáculos, sin embargo otra vez surge el mismo impedimento de semanas antes, la insuficiencia de permisos, y por lo tanto su adjunta negación. La aclaración oficial pegada con las notas dice algo equivalente y no da razones en lo absoluto, por lo que no aporta mucho. En su entonces las medidas restrictivas removieron del alcance de los anoms comunes la opción de cargar multimedia, que por un largo tiempo estuvieron a su merced desde la implementación, empero para esta no se trató de la misma serie de pasos, el mandatario optó por reservarse dicha potestad para su exclusividad y la selección premium no repite en su consecución de ventajas.

A la espera por continuación, claramente hay grandes chances de que no permanezca así y que pronto haya un posterior movimiento, aunque para dónde vaya a parar es incierto. En consideración está el vasto periodo que parte desde el arranque y que finaliza con el importante avance en la materia a tratar, dista de ser intrascendente, es un indicio de que la prioridad no está muy por el lado de la expansión en los modos de expresión, pero que igualmente recibe un mínimo de atención, que acompaña lentamente al progreso general, por lo que manteniéndose en torno a ese paso, en un futuro cercano tranquilamente los fantems contarán con esta distinción. Ahora bien, ya se trata de un punto más sensible, sabido por un intercambio de los tantos acontecidos en el espacio rozzado, donde la administración textualmente contestó a este tema de un modo particular y más negativo de lo habitual, calificando la cualidad como un poder excesivo. Con el antecedente que derivó en la modificación de las normas y el vigente protocolo restrictivo, como que hubieran similares precauciones por prevenir las transgresiones al reglamento, que incluso van arriba del sector de confianza: miedo, poco atrevimiento de entrada, porque\ldots{} ¿qué puede estar faltando?

Por lo demás, el punteo anunciado contó con múltiples entradas adicionales además de la que más sobresale, por lo que esta actualización fue verdaderamente completa, solo que no demasiado notable en esa faceta aparte.

Corrección de errores, mejora en aspectos internos no perceptibles, plataformas de reproducción previamente solicitadas para enlazar, sabores extra que no son los pedidos, las más discretas. Luego y más llamativa, la inclusión de dos opciones a la hora de comentar, posicionar y ocultar, las cuales tienen la utilidad que sus sencillos nombres refieren, para indicar si el fantem debe ser \emph{/bumpeado/} e invisibilizado del inicio, esto último ya visto en lares cercanos en semejanza al automático \emph{/hide/}, como para facilitar esa acción tan usual. Particularmente, el no marcar el parámetro de posicionamiento lleva hacia un caso atípico, que el orden en la \emph{/home/} no sea modificado y que la posición más alta no le sea dada a ese elemento, sin ser por medio del entreverado mecanismo de la Temti original.

Y no es todo, falta más. Lo siguiente ya no es tanto una propiedad a sumar como para que los visitantes cuenten con mayores comodidades, ni incrementar la cantidad de ubicaciones por recorrer, y menos consolidar las estructuras desde el ordenamiento o prolijidad como muchas veces se ha leído, sino que consiste en otra referencia directa a las condiciones del pasado, Temti y sus singularidades. Los míticos Comentarios Espejo, es el segundo fenómeno de dudosa lógica y sentido recreado a posta en la sucesora, que en aquellas épocas actuaba intercambiando las palabras enviados y dando como devolución lineas distintas a las que el anónimo esperaría volver a leer. Siguiendo el planteo de las Notificaciones Cuánticas y su periodicidad, no serían de muy frecuente aparición, y a las pruebas hay que remitirse, nadie ha mencionado toparse con esta clásica anomalía por el momento.

Recordar elementos del histórico viene siendo prácticamente un hábito en las participaciones del sujeto que lidera desde lo público la actual subsistencia temtiana, y sus tareas no se han limitado a la evolución en el desarrollo o programación del sitio de encuentro, en la generación de contenidos culturales no se queda atrás, la exhibición de memes y ediciones de humor son más de lo que habitualmente reposiciona el centro de difusión en el antro vecino. Generalmente, no se trata de afiches muy ingeniosos u ocurrentes, ni de trabajo parecido al de las redacciones revisionistas, pero es más de aquello que inmortaliza las memorias ya concluidas, dándole un poco más de variedad a los profundos archivos de Temti, que cada vez acumulan más historial bajo sus colecciones, sin dejar de expandirse en dirección para adelante.

\section{Entre desvíos y accidentes (Decimocuarto libro, capítulo VI)}\label{entre-desvuxedos-y-accidentes-decimocuarto-libro-capuxedtulo-vi}

\begin{quote}
16 de mayo de 2022, 17 de mayo de 2022
\end{quote}

En trayectos con mayor involucramiento en Fantemti, una eventualidad plenamente extraordinaria perturbó los caminos de acceso al sitio, planteando una condición jamás antes vista a lo largo de su no muy extensa historia propia. Además, por el lado de la plataforma de alojamiento, donde están sus bases sentadas para existir operativa en linea, nuevas interferencias críticas cortaron con el normal funcionamiento de siempre, sí con algunos repetidos precedentes, pero en esta ocasión un tanto más excesivo de lo usual, lo suficiente como para resaltar en comparación a aquellas.

Distinta presentación, misma traducción, los recuerdos de coincidencia con las frecuentes caídas de Temti en su época más reconstructora. \emph{/Server Error/}, intercalado con el modo de mantenimiento sin texto explicativo, en conjunto sin que haga falta oficialización, una parada en zona de turbulencias. ¿Qué pasó? Múltiples minutos de interrupción en la casi permanente presencia de la continuadora temtiana trascendieron más allá de su puntual colocación, al quedar apuntados los cuestionamientos sobre dicho presente en el primer apartado de referencia disponible, la publicación en el antro rozzado. Entre la ironía y la espera entretenida, no quedaron muy en claro las razones de ese transitorio parón, por encima de la duración acostumbrada en los anteriores de igual índole, por lo que estas podrían estar en desperfectos de Railway, o desaciertos de la gestión por parte de la conducción. Aún así, se trató de una ausencia breve, la recuperación no se hizo esperar demasiado.

Tras el regreso de faltante explicación o profundización, en la barrera que la confina un notable cambio, con la salvedad de ser sumamente discreto para aquellos no identificados mediante código de entrada: la movilidad en las fronteras fantemtianas cesó su tránsito directo, las facilidades de llegada desaparecieron. Significa específicamente que el ingreso a la interna plena del antro no está dado en su corriente expresión, referido desde el exterior, hasta la dirección principal. Prosigue en su lugar en destino a una ubicación no conocida por ahora, de naturaleza extremadamente simple y poco descriptiva de lo que estuviera sucediendo, de apariencia similar a la de los clásicos errores y mantenimientos de Heroku, pero personalizada con las tonalidades anaranjadas, una evidencia de que el mensaje contenido en sí es formulado completamente a propósito por la administración. El vínculo hacia el espacio de plazas rozzadas tampoco da mucha respuesta a la situación en particular, que por haber sido dejado como si contuviera algún tipo de información deja un mínimo de duda al respecto, por la eventual inclusión de mensajes relacionados a la circunstancia, así sean acertijos por resolver o claves por descifrar, por más que fueran descartados de palabra oficial.

Lo cierto es que aún con esa relativa incertidumbre, derivada de posibles fallos en la sostenibilidad proveída por terceros, más maniobras propias de desconocida finalidad, por dentro todo se mantiene en un mismo estado general. No obstante, la extendida continuidad de puertas abiertas incurre en una acotada suma de baches, que es de interés a los efectos de considerar el historial al cual se añaden.

\section{Coleccionistas de pieles (Decimocuarto libro, capítulo VII)}\label{coleccionistas-de-pieles-decimocuarto-libro-capuxedtulo-vii}

\begin{quote}
18 de mayo de 2022, \ldots, 26 de mayo de 2022
\end{quote}

Fue un pronto regreso a lo que solían ser previamente las fluctuaciones de movimiento de los visitantes, unas donde la interacción no esta muy aceitada y durante horas permanecen estáticos el recuento de cada fantem, reflejado en la cifra adherida al ícono de computadores, siempre solo una, si es que hay alguien para vislumbrarla.

Yendo a aquello que más varía con el transcurso de los días a estas alturas, las diferencias evidentes entre el último amplio pantallazo y uno semejante actual, están, y para la cotidianidad de los asiduos se siente el que haya mayor atención a lo que estos escriben o transcriben, generalmente aguardando por réplicas, poco dadas hasta encima en las emergentes modalidades de chat. No es que amerite ahondar mucho más, parece que por el momento no hay una expansión en el cúmulo de anoms frecuentadores, y tampoco se visualiza uno próximo a lo cerca, aunque habría de prestar observar a lo que suceda en las relaciones que estos formen con las comunidades externas. A su vez los que ya hay, dedicaron unas cuantas palabras al intentar acercarse a la deseada propiedad reservada para el eslabón más superior, poder ponerle foto a la portada de las publicaciones, pero la respuesta es negativa, sin aproximaciones siquiera.

Por cierto, aquel inesperado desvío en lo que refiere al acceso del sitio, no está más, y su misma razón fue justificada con total naturalidad en las hileras del único fantem fijado, quien se encarga de eso lo puso y se olvidó de sacarlo, y sí, eso es lo que dijo cuando la primera pregunta cayó. Esto definido como ingreso oculto provoca que cualquier llegada proveniente de no identificados sea inmediatamente enviada a una dirección inhóspita donde claramente el antro no se encuentra, con una sola discreta referencia donde información relacionada pueda haber, pero no la susodicha que brinde indicaciones para llegar a donde todos los fantems están. Similar al reconocido muro de Rouzzed, solo que parcialmente en su finalidad por la disponibilidad de registros, protege al verdadero centro fantemtiano relevante de navegantes no familiarizados con la ruta de autenticación, al mismo tiempo que los ya ingresados no ven afectado su normal pasar. Después de los cuestionamiento ante las altas cifras de presentes indicadas por el contador, una configuración como esta fue mencionada, el dato que alude a esta herramienta y que advierte tener unas semanas de antigüedad, descartando las anunciadas reformas para la moderación. Hoy dadas las circunstancias no sería muy acertado aplicarla, pero aún así es una capaz de engañar a más de un despistado, siendo que Fantemti aún no se ha vuelto demasiado popular en el contexto, con propósito de defensa o vaya a saberse cuáles.

A la postre la administración prosiguió en una de sus principales tareas, encargarse de cuidar el antro y traer cosas nuevas, esta vez con resultados de bajas modificaciones reales, principalmente para el lado de lo visual. Es contundente, ha quedado disfrazada con una estética completamente diferente a la suya propia, al punto de ser la cuarta vez en asumir la de otro sitio de afuera, solo que no de forma opcional, automáticamente el tema original con sus tonalidades eventuales fue reemplazado por el de \emph{/Gholes/}. Cómo no, e trata de una imitación con alto grado de exactitud, precisamente basada en un momento puntual de las presentaciones que tuvo ese clon, con el adicional de contar con un esquema colorido de menor oscuridad, el cual en aquella época no estaba. Lo que distingue a este de las copias anteriores es que viene electa por sí sola, sin necesidad de que el anom deba que establecerlo manualmente, cosa que supuestamente tiene el fin de brindarle visibilidad a la composición, y es probable logre el efecto intencionado, no habrá manera de que los ingresantes tengan menos de una impresión.

Y la reacción temtitera estuvo plasmada, de veras el diseño fue encontrado, luego un supuesto disgusto lo único al respecto manifestado, por lo que quizá la recepción no sea la mejor. Aparte de dicho estilado, que trajo el recuerdo de la lentitud que manejaba originalmente, el desarrollo de Fantemti abrochó otra característica más para sus estructuras, y esta es la alternativa en el sistema de cargar multimedia, que además del servicio ya utilizado desde que el mecanismo pasó a darle variedad de color al sitio, ahora cuenta con una segunda opción para incrementar las capacidades de expresión, Imgur, la cual entró en vigor inadvertidamente al no denotar casi distancias, aún así se diga lo contrario. Es un objetivo ya mencionado anteriormente en algunos intercambios que protagonizó la conducción, entonces probablemente tenga su importancia no muy evidente atrás, y no está de más decir es un medio ya aprovechado por Temti en sus etapas iniciales.

A paso de tortuga o no, el curso de esas novedades en la práctica según el discurso espera evaluación. La reiteración deja la intriga de si en verdad Fantemti corra riesgo de romperse lo suficiente como para perturbar su coyuntura de irreversiblemente. El título de espagueti condice con una estructura entreverada y levanta sospechas sobre si a futuro problemas puedan surgir, a la vez que interferencias recientes saltaron a la vista por aparecer inesperadamente y romper con la prevista estabilidad absoluta, e incluso de vez en cuando errores críticos llegan a ser reportados para su pronta solución, no obstante el resultado final siempre ha estado dentro de lo presumiblemente deseado, sin pérdidas graves que después fueran lamentadas o disculpadas oficialmente. Con estas repetidas dudas, ¿están las condiciones para que siga todo bajo control?

\section{El retroceso de las apariencias extranjeras (Decimocuarto libro, capítulo VIII)}\label{el-retroceso-de-las-apariencias-extranjeras-decimocuarto-libro-capuxedtulo-viii}

\begin{quote}
27 de mayo de 2022
\end{quote}

Sin intriga alguna respecto a la última sentencia oficial referida a las reestructuraciones internas, ni sorpresas de ningún tipo tras la llegada del estilado del clon de lejanas épocas competitivas, este último se despidió del respaldo que lo tenía llevando la responsabilidad representar a Fantemti, cediéndole el lugar al disfraz particular que más resguardó su morada, el que lleva su propio nombre y no copia íntegramente los detalles de otros, por el contrario, ese que heredó parcialmente los elementos de su ejemplo a seguir, y a la vez fue en contra de la reiterada carencia colorida o frutal, no ver naranjas. Es el momento indicado para volver a poner el foco sobre las tonalidades, puesto que con el regreso demorado aproximadamente dos días, otra etapa que finalizó fue la del acompañamiento a Ucrania.

Aunque igual todavía se mantienen los embanderados cuadrados por aquí y por allá, una de las mayores muestras de consideración por la conflictividad destructiva desarrollándose entre estados distantes estaba en la adopción de colores particulares, los representativos de la población más apoyada verbalmente por los anoms participantes, a los que la administración escoltó, a lo mejor por otorgarle aprobación al clamor popular, o hacer valer una convicción propia quizás. Esas extendidas pinceladas de azul y dorado globales fueron arrastradas con la transición a la reciente imitación, desplazadas del decorado fantemtiano por defecto. ¿Despintados accidentalmente? ¿Deshabilitados intencionalmente? No hay mucho que se pueda saber acerca de si una cosa llevó a la otra, o si fue una oportunidad aprovechada para sacárselo de encima, pero no deben haber grandes dificultades para restaurarlo y dejarlo de la manera que antes se encontraba, y el hecho de que se generara una configuración optativa para simular la misma ambientación lo deja evidente como una decisión tomada, y no una acción inconsciente.

De uno u otro modo, un par de meses luego ya no está la mitad de la cubierta característica de esa nación con la que tantas partes se han solidarizado, la imposición pasó a ser una opción, lo que en caso de dirigirse a individuos de variados convencimientos es más óptimo, sin embargo de estimarlo, podría no ajustarse tanto a la realidad cotidiana de Fantemti, lo que así como así es medio difícil de dictaminar. Al permanecer los marcos incambiados, es de imaginar que no contente del todo a aquellos pasajeros visitantes que a la pasada marcaron una discrepancia con el tomar partido respecto a la disputa en cuestión, no finalizada aún, y también es como darle la espalda al sector que inicialmente estuvo a favor de esto, que si bien actualmente no insiste al mismo nivel, probablemente siga en la vuelta y sostenga sus fundadas preferencias. En definitiva, no parece ser lo más acertado a los efectos de conformar a la comunidad, aunque igualmente hay chance de que no genere polémica o reclamos serios.

\chapter{La merma del saborizado dictatorial y la escalada de su moderno sustituto (Decimoquinto libro)}\label{la-merma-del-saborizado-dictatorial-y-la-escalada-de-su-moderno-sustituto-decimoquinto-libro}

En los momentos que la única excepción al patrón formado por sabores y únicamente sabores dejó de serlo para convertirse en una de varias, y el inicio comenzó a recibir la propagación de esa cualidad que tenía tan solo un ejemplar, en manifiesto quedó lo que finalmente está ocurriendo, los fantems ya pueden contar con un retrato a elección en sus portadas, con todo lo que esto implica.

Atípicamente, la susodicha resolución de suma trascendencia surte efecto sin anuncio su correspondiente, a causa del discreto accionar de una administración que en sus últimas determinaciones casi siempre emitió indicó lo que hacía, independientemente de que tuviera cabida en los receptores objetivo. Este caso por la determinación que oficializaría no sería uno cualquiera para la Historia Temtiana, que gracias a cómo se muestra ya y perfila la realidad de su actual escenario, ve diagramado el antes y después más drástico desde que dichas crónicas volvieron a tener casa propia.

Sí es una parte adicional del progreso llevadero a partir de aquel tiempo, en esta ocasión capaz como producto de las múltiples presiones ejercidas desde que una puntual actualización soltó como elemento destacado la demostración de lo que sus avances concretados habían posibilitado, eso que desde los inicios había sido solicitado y que tras aparecer íntegramente aunque sea sin métodos para repetirlo dio la señal de que se encontraba cerca. Por entonces, la insistencia se posicionó arriba del conservadurismo que no daba noticias alentadoras al respecto, simplemente diferentes negaciones que al final no adelantaban ni revelaban nada, pero que jamás cerraron la puerta. Por más que lo lógico indicara que sí pertenecía a los probables planes de la progresiva implementación de características, no fue confirmado, sin embargo no es lo único que la suele determinar, pues aparte de seguir su propio orden, también viene debiéndose parcialmente a la demanda popular, habiendo ciertos exponentes efectivos de ello, este en concreto el que más llegó a dilatarse.

Difiriendo del resto de similares, es un paso trascendental en la senda evolutiva de una iniciativa que creció inseparadamente junto a sus adeptos, desde las prestaciones más básicas hasta acercarse o incluso traspasar el difuso umbral de suficiencia, la conclusión de un gran salto brusco en lo que es la convergencia hacia una hipotética versión definitiva. Hoy día con esto las carencias en cuanto a la experiencia que podía brindar la Temti original, para el grupo selecto por los privilegios ya definitivamente están próximas a ser cero, con un notorio contraste con sus no tan presentes pares desnivelados. Por la procedencia del mayor caudal de actividad creadora sobre la sección frontal, esa discriminación no tendría tantas consecuencias, ya que entretanto la rotación de individuos involucrados se mantenga escasa, la accesibilidad limitada no representará un pero muy grande, conforme la cifra de fantems continúe creciendo el antro tenderá a reconvertir su imagen general, así sea rápido o no.

A menos que las circunstancias cambien. Si bien las miniaturas entraron en circulación, a falta de declaraciones no hay una continuidad asegurada, ni garantías de que estas no vayan a desaparecer, ni que las capacidades vuelvan a ser limitadas. O en el sentido contrario, otorgadas para cualquier tipo de visitante, puesto que las restricciones de permisos se mantienen a perjuicio de aquellos no portadores de código premium. Evidentemente por la manera que lo supieron asumir los miembros primarios y secundarios de la familia temtiana, no es un asunto que esté siendo considerado como una grave carencia por solventar, pese a que no sea lo óptimo en la pasiva búsqueda de nuevos integrantes. Nada de eso se ha movido, pero ahora la distancia es drásticamente más pronunciada, considerando lo que un individuo sin arribos previos podría esperar de un sitio como tal, las comodidades faltantes son determinantes, potenciales generadoras de confusión y disconformidad, encima del verdadero pretexto solo conocido por los más informados y considerados en las puntuales fases de selección, sea por buenos antecedentes o solicitud individual, las únicas vías que trascendieron y parecen ser las actuales maneras de acceder a la conversión del privilegio, tampoco indicadas formalmente por escrito.

Así estos aspectos de menor incidencia puedan ser trastocados posteriormente, la época a comenzar estará marcada por una característica indistinguible mientras perdure, estableciendo el quiebre desde que las primeras vistas coloreadas por demás permitieron proyectar un futuro donde el panorama de Fantemti se asemeje más aún al de su ilustre modelo y objeto homenajeado, lo que quizás además sea acompañado de más imitaciones milimétricas que hagan crecer la amplitud del monumento histórico de los clones, que de igual modo vive por sí solo bajo sus propias verosimilitudes, un lento ambiente híbrido de copia y autenticidad, desarrollo de funciones activo, y sostenidas actividades culturales, conjunto que a pesar de sus pequeños decaimientos en comparación a la intensidad notable de nacimiento, aguarda por transitar y descubrir etapas, esta siguiente una a nutrirse esencialmente del aporte que los fieles acompañantes realicen al elegir cómo rellenar los bloques que extiendan la construcción acumuladora del principio.

\section{Los retratos del descascarado dogma líquido (Decimoquinto libro, capítulo I)}\label{los-retratos-del-descascarado-dogma-luxedquido-decimoquinto-libro-capuxedtulo-i}

\begin{quote}
27 de mayo de 2022, \ldots, 30 de mayo de 2022
\end{quote}

La pieza inicial de la conclusión para la espera más dilatada y no cronometrada ya está aquí, hace falta puntualizar las formas, pero tiene algo seguro y es que el valor de la capacidad más solicitada de los primeros y últimos tiempos fantemtianos dejó de estar únicamente para el escalón más elevado y sus propósitos de transmisión nostálgica, la posta está siendo trasladada al siguiente nivel estamental, donde más de un individuo aguardaba por recibir el aval a construir tal cual en los antros del contexto es habitual, eso explícitamente reservado y llamado poder, quizá como si fuera considerado un riesgo otorgarlo, o como si ni siquiera hubiera sido parte de planes. A raíz de la repetida insistencia, reanalizar la decisión negativa mantenida hasta antes, cumplir con una implementación gradual planificada, o por lo que sea, aquello no corre más, la actualidad fantemtiana condice con que al menos temporalmente esta exótica característica en cuanto al entorno se refiere, tenderá a bajar su rareza, darle una ambientación de otro calibre al sitio, y contentar a los posibles disconformes junto a las simplificadas portadas.

En fin, en el transcurso de una jornada más aquel evento tan natural y frecuente como la ocupación de un espacio adicional en el inicio y relegamiento de su previo ocupante, mediante título, color, y según el caso íconos especiales, contó con el particular distintivo que hasta entonces estaba solamente en uno de los cientos y cientos que marca la estadística, su portada con imagen. Con total naturalidad, varios emergieron de la misma manera, indicando que no se trataba de ejemplares aislados de, y que las limitaciones que anteriormente lo impedían, fueron removidas.

Pero mientras eso sucedía, tampoco hubieron comunicados oficiales que advirtieran del cambio de permisos, que es la única versión de los hechos posible, a causa de que hasta hace poco los intentos de proceder con la opción de verificada no conducía a más que un error de esa especie, ahora emitido en otros términos, referente a los códigos premium, por lo que allí está la respuesta, y la causalidad delata a los más que probable autores de lo que se hizo visible.

Drásticamente distinto al plano de infinito desplazamiento con mezclas de colores que van desde lo oscuro hasta lo claro, carente de profundidad o cambios abruptos de tonalidad, los habilitados pueden dar su toque personalizado, y rellenar los cuadrados confinados por azul y amarillo a gusto, en teoría siempre y cuando cumplan con el lineamiento número dos de las reglas, aunque como poder, podrían transgredirlo. Todavía estando la oportunidad de generar ese tipo de unidades distinguidas del resto, los previamente creados permanecen junto a su misma presentación visual, lo que significa que a medida se vayan generando de estos distintos, van a ir quedando mezclados según como el posicionamiento lo indique, bajo el estado establecido actualmente claro está, si naturales incentivos establecer la tendencia los hay.

A pesar de que no sea el foco del asunto, sigue resultando llamativa la omisión de comentarios provenientes de la conducción en lo relativo a ese tema, nada que anticipe si en adelante quedará así, si es una prueba temporal, o qué. Como también está el antecedente de unos cuantos vaivenes en la habilitación a cargar multimedia no esclarecidos, parece que cuando se trata de modificaciones en el rango de capacidades no hay tantas intenciones comunicativas como si las hay frente a asuntos de otro tipo, sin embargo al ser algo que da y no quita, los cuestionamientos no se hacen escuchar. Con perspectiva hacia el futuro, la ausencia de declaraciones o explicaciones que den más detalles sobre esta posibilidad limita cualquier tipo de interpretación y la pone en condicional, al punto de que no hay ninguna información de las expectativas manejadas por el mandamás, pero de ahí es que proviene el mismo, y movimientos tan importantes sin su anuncio correspondiente son bastante poco habituales, a ese grado de excepcionalidad llega esta potencial nueva etapa. De lo que no hay duda es que fue intencional, y sus usos evidentemente visualizados por quien supervisa, pues tras dejar pasar unos días no se contuvieron, sino multiplicaron, una forma de decir que están aprobados.

Y dar un paso atrás sería más desprolijo aún. ¿Para que queden algunos fantems diferentes vagando por ahí sin ninguna temática o mensaje en especial? ¿O para propiciar un precedente negativo suprimiendo lo que fueron publicaciones legítimas? Ambas opciones que impliquen regresar a lo anterior por más discretas que fueran, indudablemente son para dar una impresión peor, candidata a tener aprobación nula, porque como ciertas palabras indicaron, es eso que los anoms tanto quieren. ¿Se vendrá alguna aclaración al respecto? ¿Pronto todos estarán aptos para marcar la casilla exitosamente? ¿Es la antesala al final de los sabores o la obligatoriedad de las miniaturas? ¿Un mal comportamiento o uso indebido generará las excusas para retroceder? No hay garantías de que esto continúe así, pero fiándose de que la confianza vaya de un lado al otro correctamente, Fantemti entró definitivamente en una época de condiciones más ideales, con sus flaquezas por el momento opacadas.

\section{La voluntaria inercia evolutiva (Decimoquinto libro, capítulo II)}\label{la-voluntaria-inercia-evolutiva-decimoquinto-libro-capuxedtulo-ii}

\begin{quote}
1 de junio de 2022, \ldots, 4 de junio de 2022
\end{quote}

Tras la inserción de esa fundamental condición en la cualidad predominante de las publicaciones fundadas por los interpretes premiados de Fantemti, el que aún con una perspectiva sustancialmente distante denota una real desemejanza en el factor común del cual se rellena el inicio, la normalidad entró en un proceso de adaptación a una más repleta de impredecibilidad y irregularidades, que en simples palabras es, más y más miniaturas. A la vez como marca la artificial tradición, solo porque desde la comandancia así se estableció tras el repetir de sus episodios particulares con clara relación entre sí, el sitio recibe a domicilio la entrega de otro viejo diseño revivido tras haber quedado fuera de operaciones sin pena y puntualmente este, debatible gloria, pero indiscutible vinculo con el lugar de los hechos, donde hay mucho valor adjudicado.

Dicho por un lado, hecho por el otro, salvando que no de la manera que la mentalidad promedio del contexto lo hubiera esperado. No obstante, es lo que hay por ahora, y a partir de eso se ha ido desarrollando el entorno al contar con una característica efectivamente nueva, la que cambia la sensación general desde entrada, que luego en el fondo no es más que eso, una apariencia, pero de trascendencia igualmente. Así lo que en un principio se veía como algo llamativo y extraño a la vez, pasó a ser de alta frecuencia, prácticamente todo fantem que ha aparecido en los últimos tiempos adoptó una portada más elaborada que un simple mezclado de dos o tres colores, y eso considerando que dicha opción está únicamente al alcance de los individuos premium. De todas formas, no ha supuesto tanto más que eso, la elección de sabores aún es obligatoria y de tal modo la selección una segunda cubierta personalizada, meramente opcional y a gusto del consumidor, solo que dado el panorama podría interpretarse como una imposición para cada creación en ser fundada hoy día. En parte delata a la clase de visitantes no privilegiada, como una de nula participación en el plano principal de la escena fantemtiana, confirma eso que antes ya se manejaba como evidente.

Con una idea de su composición, la comunidad del antro asumió el papel protagonista en la actualidad de su realidad, no es otra si no esta la que ha aprovechado los derechos con los que antes no contaba, pero a menudo la administración vuelve con actos que ameritan mención y referencia, hoy otro más por la senda de la nostalgia que traslada el presente hacia un punto donde múltiples elementos del pasado se hacen presentes, combinados en simultáneo con lo que se ha estado construyendo en los meses que corren. En esta ocasión la imitación consiste en el primer tema de todos los sitios temtianos, el que en aquel lejano comienzo fue momento de partida para lo que después sería la estética posteriormente mantenida al menos en su base, desde los prontos avances de la cambiada Temti, al año y pico consiguiente, con el estilado propio de Fantemti. Al ser una copia directa, que incluso no disminuye la calidad en cuanto a exactitud como las copias previas, mantiene fielmente muchos de los detalles más mínimos y discretos, hasta animaciones que nunca antes habían sido observadas en la incursión de la vigente sucesora, y que si acompañaban en esa puntual época primitiva. Un episodio más donde el gigantesco aprecio por las profundas raíces muestra que no desestima ni desecha lo que ya hace bastante se concretó y poco se conservó en su exacta forma. Obviamente esta quinta vez no es cualquiera y trae su condimento especial, atrás de cualquier símbolo expresado hay mayor carga de homenaje potenciada por el sentido de pertenecía, resucitado en una expresión más.

\section{La consolidación del relleno típico (Decimoquinto libro, capítulo III)}\label{la-consolidaciuxf3n-del-relleno-tuxedpico-decimoquinto-libro-capuxedtulo-iii}

\begin{quote}
4 de junio de 2022, \ldots, 17 de junio de 2022
\end{quote}

Acumulados los días desde el cambio más sustancial que tuvo el centro único de la Historia Temtiana en los últimos tiempos, ahora de regreso con su estética anaranjada propia, el efecto predecible que este iba a producir dentro el espacio principal ya es una realidad, de a poco va adoptando por completo el estilo normal de los sitios similares, y va relegando esa atípica característica que constituía su más colorida expresión simple, se complejizó. Ahora hay más lugar para lo que aquellos individuos con aprobación de la conducción deseen plasmar sobre los extensos y prolongables murales donde las imágenes se multiplican, como típicamente sucedería en cualquier sitio del contexto.

Esto gana pertinencia al ver que el fondo de la parte inicial está colmado de numerosos fantems de dicho rasgo, en la mayoría sectores hasta siendo predominantes, y en otros teniendo aunque sea algunos representantes. Podría decirse que la relevancia de la trascendental habilitación no va mucho más allá de eso visible, y ya, sin embargo no es el caso, porque también a partir de su momento es que la actividad sintió un impacto en su cantidad y calidad. Esto se enfoca básicamente la velocidad de creación de publicaciones y en parte su contenido, a lo que claramente hubo un incremento medianamente serio en el ritmo con que esto sucede, haciendo que las marcas temporales que acompañan a cada uno de ellos a nivel general son relativamente más frescas que unas semanas atrás, pero no con tantas réplicas, una cosa similar a lo que había pasado en aquellos pasajes donde el mensaje de impedimento al ir tan rápido fue reprochado en repetidas ocasiones.

Aún así, un repetido obstáculo frenó el cumplimiento de esta automática proyección, siendo que una serie de inhabilitaciones momentáneas tomaron por sorpresa a los autores y generó que estas fueran reportadas. Para la suerte del o los interesados en seguir con lo impedido, a la brevedad eso regresó a la configuración como estaba y de tal modo la función volvió a estar disponible para el uso de los premium. Sin embargo, el precedente quedó marcado como más que una casual circunstancia excepcional, porque bien que numerosas veces se repitió, y la administración no ha dado ninguna justificación o explicación al respecto, solo atendió los reclamos con varias horas de retraso, lo cual una vez más resulta extraño para los modos manejados anteriormente, cuando el paso a paso recibía su aclaración, el proceder respecto a ello durante esta etapa se ha vuelto inconsistente, sin mayores menciones referentes, simplemente sucediendo.

Además, sucedió el virtualmente programado regreso al tema original, según quedó dicho desde que fue aplicada la imitación regresiva, permanecería por un espacio de contadas jornadas y de tal manera se despidió, como conclusión lo mismo que pasó cuando el anterior diseño reemplazó a la apariencia esencial temporalmente. Igual, el recuerdo de esos lejanos tiempos permanece rondando por los fantems, viendo que en estos siguen apareciendo importantes referencias directas al pasado. El titular \emph{/No hay temti como el primero/} estuvo fijado oficialmente en las primeras posiciones del antro, acompañado ocasionalmente por la otra emblemática publicación \emph{/Temti para siempre/}, y también aunque de distinto comunicado, una que promociona el archivo más abarcativo de sucesos previos, llamada \emph{/sigan el canal oficial de fantemti en telegram!/} Asímismo, en el fantem de arte fue lanzada una obra relacionada, sobre la complejidad existencial de Titiri Vieja, donde los roedores volvieron a dar su aporte a las colecciones culturales. Sin ser excesivo, la sumatoria de estas partes hicieron de las fechas que corren unas de fuerte nostalgia, y la así la noción de identidad no tan opacada por el manejo de otras temáticas.

Volviendo a lo central, nuevamente una propensión llevada adelante por los anoms se presenta y de ello se basa lo más destacable actualmente, mientras que las intervenciones de la autoridad no se hagan excesivamente reiterativas por ahí irá el foco. Es el curso más avanzado de la presente transición, porque pese a la no obligatoriedad de aplicar la distintiva en cuestión y las exigencias protocolares para acceder a ella, desde la misma el retrato global del sitio está pasando de consistir en mosaicos de cuadrados coloridos, a ser puramente una galería de retratos particulares, tal cual como la madre de todos los clones supo establecer en su esencia. Sin expectativas de que la presencia de unidades exclusivamente saborizadas recupere terreno, esta tendencia al sostenerse e incluso crecer está amenazando con apoderarse completamente de Fantemti en el área de sus recientes construcciones.

\section{Por el paso cúbico (Decimoquinto libro, capítulo IV)}\label{por-el-paso-cuxfabico-decimoquinto-libro-capuxedtulo-iv}

\begin{quote}
17 de junio de 2022, \ldots, 19 de junio de 2022
\end{quote}

Presentando implícita y explícitamente la imitación del tema de \emph{/Kiubex/}, Fantemti transforma sus apariencias con la estética característica de otra antigua alternativa única del contexto de los clones. Esta en particular supo ser de las que mayor aceptación tuvo entre los contendientes contemporáneos, puesto que también fue elaborada a semejanza de lo que antes era el hogar natal de los múltiples visitantes. Y tal cual sucedió con las elecciones anteriores, la simulación general se aproxima bastante al original, logrando escasas diferencias entre ambas.

Sin casi reacciones por parte de los anoms a destacar, la acumulación de temas alternativos sigue acrecentando sus cantidades a base de buenos exponentes, cada vez que uno se suma aquella particular lista se extiende para pasar a contener un diseño ingeniado y utilizado en un sitio aparte más, lo cual en estos casos se ha dado con combinados de estilos que quedaron en desuso, y solamente accesibles desde copias no oficiales, como es la situación de esta en concreto. El número de veces que se ha repetido esta circunstancia la va volviendo menos y menos sorpresiva, al punto de instalarla como una común y habitual quizás, aunque claro está que no sucede todos los días. Si bien aún restan más opciones que cumplan con las mismas características a tomar próximamente, a medida que son concretadas una por una, menor se va haciendo la bolsa de candidatos para próximas ocasiones, así que cuando se termine la exposición actual en su anticipada duración, un intervalo reducido de días, podría estar medianamente sugerido el siguiente paso en esta serie de recuerdos materializados.

\section{Agudización constitucional (Decimoquinto libro, capítulo V)}\label{agudizaciuxf3n-constitucional-decimoquinto-libro-capuxedtulo-v}

\begin{quote}
19 de junio de 2022
\end{quote}

Hecho saber a través de las no tan clásicas alertas, una oscura notificación exagera indirectamente la relevancia de un par de cabos sueltos sin atar por la administración en el ámbito normativo, dos cosas que seguramente de haberlas tenido en cuenta previamente ya estarían salvaguardadas, sin embargo dado el accionar en este momento fueron tenidas en consideración. A la serie de prohibiciones relativas a la inserción masiva de mismos contenidos, se le añadió un inciso extra, de modo que la permisividad es reducida en cuanto a lo que se puede inundar de un mismo mensaje. La mira de esto es puesta sobre los caracteres fuera de lo estrictamente escrito en un lenguaje, como emojis u otros que no sean principales de un teclado, por lo que según lo que establecen las sentencias recién reformadas, ya no se podrán utilizar a mansalva. Además, ahora si un anom solicita más permisos de los que tiene, en los modos o palabras que sea, su historial está sujeto a ser revisado con tal de asegurar que ese individuo haya mantenido un comportamiento aprobado, o simplemente que está apto para acceder al ascenso de categoría. Lo cual es bastante evidente que ya sucedía desde antes por cómo se manejó públicamente en su momento, pero no fue del todo detallado y aquí estaba la pieza explicativa de que el criterio referido pudiera ser empleado.

Presumiblemente esto lo sugirió sin más ni menos que por una de las recientes participaciones coincidente con las características descritas, porque de un día para el otro su contenido cambió al resultar sanitizado, hablando en términos técnicos. Dándole un poco de contexto, tampoco es una determinación aislada, siendo que en algunas de las tantas ubicaciones supieron prohibir el uso de estos caracteres no típicamente lingüístico, quizá por razones colaterales de seguridad, o en su defecto para confortar al público habitualmente disgusto con ellos. Y es el caso puntual de Temti en sí, que incluso llevaba sus limitantes hasta signos altamente comunes, como los de exclamación y otros no tan exóticos. No obstante, para este sitio no son removidos totalmente de su uso, es una regulación simbólica que prohíbe el exceso, la cual ha de entenderse es efectiva de transgredirse.

No en un segundo orden, está la siguiente puntualización que realiza este extraordinario anuncio acerca de las variantes al apartado o sección que fue modificado, la cual rellena un vacío que había sido creado con la selección de aplicantes a obtener privilegios dentro del antro mediante códigos premium, que en su momento no dio lugar a ningún reclamo por parte de los afectados, pero que en el trasfondo tuvo un proceso que según los argumentos esgrimidos metió entre medio algunos de esos datos llamados privados, sin deberse a una rotura del reglamento como tal y evaluación de posibles sanciones, sino más bien lo contrario, y que los favorecería. En el sentido de la formalidad o aclaraciones pretendidas, faltaba una justificación para que la presencia reguladora accediera a sus historiales, y por lo visto eso fue entendido.

Este tipo de nociones podrían ser propensas a dinamitar pequeños sustos y cuestionamientos al respecto, sin embargo son traídas a una comunidad generalmente calma e indiferente ante las realidades más relacionadas con lo formal, que se adecúa a las condiciones impuestas con escasa resistencia, y al alterarse factores mínimamente contingentes, es seguro que las consecuencias se reduzcan a eso y listo.

\section{Y más crecimientos retrospectivos (Decimoquinto libro, capítulo VI)}\label{y-muxe1s-crecimientos-retrospectivos-decimoquinto-libro-capuxedtulo-vi}

\begin{quote}
20 de junio de 2022, \ldots{} , 30 de junio de 2022
\end{quote}

Tras atravesar un periodo largo de normalidad, de mínimas incidencias que destaquen entre las realizadas por la reducida cantidad de anónimos habituales, un episodio como esos de mediana frecuencia sube el escalón del trayecto bajo esa constante, desde una serie de avances que en sí no deberían alterar mucho la linea de los antecedentes, pero que se agregan a la lista de características que hacen al antro y que entran en funcionamiento del día a día, lo cual además volvió a adoptar temporalmente una distinta apariencia.

La última gran intervención de la gestión había atacado directo a un aspecto lisa y llanamente normativo, el cual posteriormente en la práctica demostró ser indistinto, predecible considerando los puntos a los que apuntó. Exceptuando ese, lo más reciente era una simple actualización que solo trajo estilos de más, sin mejoras que vinieran como complemento, pero ahora siendo parecido es eso y más. El tema que funda copia es el de Hixxel, de sus épocas de reconocimiento y popularidad tope entre los clones, cuando más sucedían las reiterativas migraciones y variaciones continuas.

Entre una persistente inestabilidad, una similar base de diseño fue la que utilizó dicho clon, alternando con modificaciones no muy estructurales sino discretas, y esta imitación a pesar de ser medianamente precisa toma una de versión específica de las tantas aquellas, por lo que la impresión según el recuerdo podría ser la de que no se asemeja tanto. Y como han hecho habitual estas confecciones tendrá sus fechas de exposición, sin embargo las notas formales donde eso fue indicado mencionan una consideración con el factor específico de los colores, adelantando que los actuales no serán los únicos en darle tono a las estructuras del sitio.

No tan al margen de lo exclusivamente visual, esta múltiple reforma provee de más novedades expansivas para el andamiaje de Fantemti, un par de posibilidades agregadas en los contenidos de multimedia, mayor versatilidad para adaptarse a cómo el individuo quiera transmitir sus materiales audiovisuales hacia los visitantes. Por un lado, es para lo que son este tipo de retratos animados, regularmente estáticos para la vista que sea desde el inicio, encimarlos haría que puedan apreciarse en movimiento como originalmente fueron cargados. Entre ambos extremos alternativos, lo que parece ser mejor calibrado, un punto intermedio entre dejarlos totalmente inmóviles y hacerlos marchar automáticamente. Por otra parte, para lo que sale de los dos botones del selector, subir o enlazar, cualquier elemento que salga de allí podrá ser censurado mediante un filtro que lo cubra con una capa borrosa, hasta que esta sea removida manualmente por quien la observa. A priori al haber un reglamento que prohíbe las secuencias más inapropiadas o fuertes, carece de su sentido intuitivo, entonces puede que eso esté difiriendo próximamente, o no.

Dentro del anuncio también se incluyen más párrafos sobre el paquete de progresos efectuados en la maquinaria fantemtiana, como lo es el clásico apartado exclusivo de imágenes, habitualmente en las ubicaciones del contexto representada con una carpeta de ícono, cuya función es filtrar de todas las participaciones que cuenten con archivos adicionales al texto, aunque según aclara el recuento hay un pormenor con las más antiguas, por la fecha aparentemente anteriores a las reestructuraciones más drásticas de dicha área. A su vez, detalles más relacionados a la conectividad y el tiempo real, empezando por un mecanismo que se encarga de avisar de posibles pérdidas de conexión que impidan recibir datos en vivo. Y luego, el número de notificaciones llevado de prefijo hasta el título de ventana o pestaña, como para que estar parcialmente fuera del sitio no se convierta en un problema para mantenerse al tanto de interacciones en el mismo.

Por último, una transformación más presumiblemente orientada al homenaje de componentes pasados, la famosa dirección de dos letras, CP, secreta y no referenciada abiertamente a la par del resto de las accesibles en dominios de Fantemti. Sabido el objetivo principal que desde antes cumple allí, no hay una concordancia evidente entre la temática con el medio de conversión que otorga los códigos premium a partir de la validación de comodines, sino más bien con cómo esta misma ruta era presentada en la Temti original, un portal resguardado por la confidencia de una clave desconocida, un espacio custodiado por figuras del Club Penguin como tal, y más llamativamente, un nicho que invitaba a creer que algo interesante se escondía tras esa combinación, sobre la cual jamás trascendió información verificable. En este caso, la tercera similitud no aplica, los cercanos al ambiente cuentan con que la presencia un pingüino rodeado de azul y amarillo sería simplemente decorativa, cierto si realmente fuera un rediseño y nada más.

Salvando la mencionada brevemente al final, muchas de estas ya se fueron vistas en otros lados, y recordando que esta iniciativa ha ido construyendo sus bases a semejanza de modelos consolidados, no extraña. Más allá de la valiosa pero simbólica conmemoración al estado pasado de los hixxelianos, los frutos del desarrollo son un aporte positivo para el antro y los anoms asiduos, podrían ser de buena ayuda y elevar la calidad de estadía, según el uso que requieran. No obstante, la introducción de los cambios es comúnmente de mínima repercusión\ldots{} en una oportunidad más sucede y estos se juntarán entre la particular actividad, estabilizada en términos de reducidos intercambios, que coexisten junto a la recopilación y reelaboración de copias, copias y más copias.

\section{El unicornio formal de baja vibración (Decimoquinto libro, capítulo VII)}\label{el-unicornio-formal-de-baja-vibraciuxf3n-decimoquinto-libro-capuxedtulo-vii}

\begin{quote}
30 de junio de 2022, \ldots, 3 de julio de 2022
\end{quote}

El sitio de Fantemti aconteció en lo reciente una serie con exactamente cuatro episodios, donde la apariencia del sitio fue tomando distintas combinaciones de colores, y entre medio vinieron algunas ampliaciones aparte más. Atípico frente a los repetidos casos anteriores, la adopción de un estilado temporal no solo se basó en uno o dos días con esa misma estética y ya, sino que fue en paralelo con la remota búsqueda hixxeliana por encontrar sus tonalidades preferidas, al alternar la presentación inicial y con ello probar múltiples versiones de una base compartida. Y a la par de ese seguimiento, también más de la actualización que al principio llegó con la primera versión del diseño en cuestión, no al mismo nivel de amplieadad, más bien como complemento al solo incluir un par de características adicionales.

Respecto a lo que da para hablar en cuanto al tema viejo en la historia y nuevo en el presente, qué sucedió afirma el adelanto indicado por el centro de anuncios, ese combinado de matices simplificados entre la gama del blanco y negro cambiaría. Y se reemplazó por otras configuraciones del círculo cromático, no por cualquiera ni aleatorias, concretamente las mismas que fueron tomadas por el antro de la corbata en su novel aparición dentro del contexto con el pasar de los días, salvando que no en igual orden. Un tipo de gris menos oscuro, un llamativo intermedio situado en violeta y azul, un parecido de ese primero con detalles rosados, totalizan los tres camuflajes extra que vistieron a los contenedores fantemtianos en esa concluida remontada de apariencias, que careció de aquel clásico presumiblemente accidentado hacedor de la fama más bochornosa del clon acorbatado, pero que todavía con ese debe pudo acercarse considerablemente a sus memorias.

El sitio también padeció más modificaciones entretanto, bien mínimas y sin gran incidencia óptica, pero aparentemente lo suficiente relevantes como para recibir un par de líneas en el mural de novedades, junto con comentarios de la evolución anterior, a la cual le faltó su último paso, y la posterior retirada tampoco tuvo aclaración, pasó no más. Uno de estos cambios es solo de cifras, el funcionar del mecanismo de paginación redujo su segmentación a tramos de menor extensión, probablemente con la intención de disminuir las cargas excesivas, optimizar. El próximo punto es sobre ese mismo sistema, en que la configuración por defecto estableció que el botón encargado de hacer aparecer las siguientes unidades sea pulsado por sí solo cuando con el desplazamiento se vuelva visible, una manera de ahorrar el invaluable costo de presionarlo, una preocupación que en la actualidad temtiana dejó de debatirse.

A su vez, por el lado de lo que concierne a la visibilidad y específicamente el inicio, las opciones ahora tienen su propiedad para que esas secuencias contenidas corran por sí solas, de modo que encimar las estáticas miniaturas no haga falta para contemplarlas en movimiento, lo cual permitiría que ese índice permanezca animado si así los fantems contaran con las cualidades debidas. Lo otro que también repercute sobre el área principal mencionada, es una reducción del grosor en los marcos compuestos de celeste y amarillo, aplicando a cada creación en ser efectuada a partir de dicho entonces, a lo mejor por la baja en las repercusiones que tiene su causa recientemente, o quizás apuntando a achicar la posible molestia que puedan dar. Y con la pieza faltante se forma un trío automatizado, y va dirigida a los mensajes de desconexión, que en poco tiempo de ser implementados cumplieron con su finalidad, sin embargo por lo visto no consideraban la posibilidad de que el visitante pudiera recuperarla, y viendo que una señal destinada a cerrarlo es capaz de llegar aún habiéndose mostrado esa información, es un acertado detalle con tal de evitar propagar datos no ciertos.

Esa cadena de variaciones fue frenada, concretando un breve periodo donde el panorama tomó formas un tanto extrañas, en el cual la actividad protagonizada por los visitantes no despegó de su relativa calma, igualmente como antes. Las columnas bien marcadas partieron de su colocación oficial, y allí retorna la normalidad visual para el nicho fantemtiano, en formato de cuadrícula.

\section{Procedencias en el andamiaje tematizado (Decimoquinto libro, capítulo VIII)}\label{procedencias-en-el-andamiaje-tematizado-decimoquinto-libro-capuxedtulo-viii}

\begin{quote}
4 de julio de 2022, \ldots{} 9 de julio de 2022
\end{quote}

Sentada la estabilización en Fantemti, en general fue más de lo mismo a lo que esta supo acostumbrar, aunque justamente evidenció alguna que otra cosa resaltable al haberse dado a gran escala o provenir de la cúpula superior.

Como el día a día supone continuó el ida y vuelta, no excesivo, en su composición repartida entre una considerable proporción de interacción y una mayoritaria de presunta copias textuales, dentro de las esferas que estas se generan, fantems con miniatura en mayor medida. En eso último, la tendencia corriente desde que sus apariciones comenzaron a hacerse dominantes definitivamente llegó a plasmar una cubierta por momentos de casi cien por ciento, no total porque en la primera posición el elemento fijado hace la excepción con su simple sabor al descubierto, pero por el resto se trata de consecuciones de imágenes, literalmente, con correlación entre sus componentes incluso. Las jornadas previas no alcanzaron a formar secuencias demasiado claras, pero igualmente en estas emergieron retratos medianamente parecidos unos a los otros. Ahora más cercano en el tiempo sí, la semejanza en cuanto a escenas paranormales fue enorme, cada recuadro de este tipo compartiendo tomas animadas de entornos oscuros y tenebrosos, prácticamente como una misma película capturando distintas ubicaciones del mundo en el cual se desarrolla. Lo curioso es que no contienen descripciones relacionadas con las portadas que acompañan, al tratarse de historias cotidianas y dudas existenciales refieren a contextos que difícilmente puedan tener punto de contacto, por lo que transmiten dos cosas a la vez.

Esto hace saber nuevamente que los anoms premium cuentan con una gran capacidad y pueden incidir radicalmente en la impresión que vaya a generar el antro al renderizar su espacio principal, y aunque no sea comprobable si esas creaciones consecutivas provienen de un solo individuo, es lo más probable de hecho, o fueron planificadas organizadamente entre varios. La baja cifra de participantes hace que la acción de unos pocos termine valiendo por la mayoría, y junto con una rotación bastante limitada como producto de la reducida actividad, los tiempos de espera exagerados no son capaces de impedir que los mensajes mejor posicionados terminen proviniendo de autores reiterados.

Y hablando de figuritas reincidentes, el término de anonimato se ve un tanto cuestionado, en un sitio que lo enuncia implícitamente como fracción de su esencia, más allá de que el propósito declarado no lo maneje como clave. Personajes que en recientes épocas pasaron a nombrarse seguido no han disminuido tanto las menciones, por lo menos aquel apodado Chelo al ser mencionado con asiduidad mantiene un mínimo de popularidad en el contexto fantemtiano. Más hay para hablar del llamado admin, el mandamás del sitio y original creador del fantem que descansa fijamente en la cima, que al contrario de como resaltaba A. por su presencia o ausencia según la época, suele dar señales de vida más consistentemente, tanto en dicho hilo, como en el asentado en dirección de Rouzzed, que en especial no resulta del agrado de algunos replicantes, evidente al ver que repetidas veces el OP fue calificado de cipayo.

Aunque igual el vínculo con dicho clon se podría decir que es compartido por una dominante porción de la comunidad, está claro que el congeniar con otros sitios como si nada normalmente sigue sin ser visto como correcto, por más que la competitividad haya diluido su intensidad, y la cantidad de imitaciones hechas por la administración suman para confirmar que esta no le da importancia a los radicales valores de fidelidad manejados en el pasado. Aparte de los asuntos relativos al antro, el dialogo con los fantemtianos también se da y entre ello los intercambios que son fuera de tópico, lo cual da a conocer al referente por un lado más personal, no solo como interprete del rol encargado de comandar.

No es necesario que las situaciones relevantes provengan sí o sí de anuncios y cambios estructurales, sobre todo en un ambiente reducido donde cada paso tiene el peso que tiene, al fin y al cabo una importante parte de la gracia, es que construyen los individuos con sus expresiones, las que posteriormente serán vistas y eventualmente contestadas, cuyo escenario cuenta con un dueño que habitualmente se comporta como si fuera uno más, incluidas concordancias y discrepancias.

\chapter{La fortaleza de la amainada resistencia en el después de sus descensos (Decimosexto libro)}\label{la-fortaleza-de-la-amainada-resistencia-en-el-despuuxe9s-de-sus-descensos-decimosexto-libro}

Un típico antes y después azota a la Historia Temtiana, con argumentos que acompañan su naturaleza divisoria sobre el uniforme trayecto, que con la concluida transición en la composición de las unidades frontales solo pasó a prolongarse con antecedentes de menor entidad. A separar, llegó un serio simbronazo en la formula del servicio fantemtiano veinticuatro siete ejecutada hasta el presente, más un altamente probable retroceso en la presencia fuera de fronteras. Siendo dos sucesos que por sí solos inicialmente tienen su trascendencia aunque baja, inducen a proseguir con la sostenida presencia en linea, comprometida en sus estables modos, a la vez que uno de los baluartes referentes con los que siempre contó el sitio desde la alejada fundación, recibió lo que podría estar resultando un dictamen mortal.

Donde más reviste la gravedad es concretamente en el interior de la casa propia, porque un primordial soporte para sus cualidades es dejado de lado, y no hay una solución definitiva contemplada, o al menos adelantada. Nada descarta que prontamente vayan a aparecer alternativas que cubran cualquier hueco que esto esté anticipando generar, sin embargo la comunicación hace llegar un pronóstico donde no hay señales positivas para lo ulterior, sino que por el contrario, el proveedor primario en el que Fantemti depositaba su confianza hasta antes de este quiebre, Railway, estaría cambiando sus ofrecimientos, por unos insuficientes para los vigentes requerimientos del sitio y su gestión, conllevando la restauración del viejo similar Heroku, más de lo mismo por las causas que provocaron la anterior y primera migración: bajo la configuración acostumbrada y actual, la continuidad absoluta tiene fecha de caducidad.

A pesar de que no hayan elementos para definir este traslado número dos como correctamente evaluado, la ruta de acceso no es un detalle que deba variar con frecuencia y para el manejo normalmente criterioso de la conducción eso debe haber valido, lo que lleva a presumir que no fue determinado a la ligera. La información continúa siendo emitida en buenas cantidades, sin embargo en este caso es insuficiente para saciar la incertidumbre desprendida del incidente, al punto de que entre los eventuales destinos hay múltiples formas de seguir, alguna seguramente seduciendo más las demás, no obstante ni una ni otra fija por encima. Por más de que la puesta al tanto solo trate lo que concierne al hoy, las declaraciones y acciones siguen debiéndose al futuro, en el cual pueden darse nuevas mudanzas de base principal, reformulación de procedimientos o recursos respecto a los objetivos, definición de ciclos en los que la disponibilidad varíe, o incluso el cese total de la permanencia en línea.

La siguiente faceta es de segundo orden y simbólica principalmente, pero igual considerada, lo que ocurre con el espacio forastero de temática fantemtiana más importante en los últimos tiempos, uno donde los progresos del desarrollo se fueron viendo reflejados, uno donde la cultura inherente creció de varias maneras, uno donde la existencia de la nación subió su grado de reconocimiento, y más\ldots{} sin embargo indiferente al evaluarlo dentro de su contexto, una simple aunque antigua creación de las tantas pertenecientes a un medio donde estas constantemente son desplazadas, con su novedad de ser categorizada en los peores términos oficialmente y empujada a la desaparición.

Antes de que eso se dictaminara, el dinamismo allí ya supo debilitarse bastante, puesto que cada una de las utilidades dichas estaban repitiéndose menos que en sus mejores momentos, por lejos, y si bien dichos movimientos han sabido reinventarse sin tener problema con estar fuera del hogar nativo, ahora vuelven a encontrar una firme barrera que probará si las energías resisten sin poder continuar acumulándose en la predilecta concentración. Por otro lado, el inicio del clon para renovarse con palabras y colores tiende a depender de rachas concisas que propulsen los participantes, especialmente los seleccionados para actuar junto a los privilegios dados, una repetida tendencia que ha quedado más al descubierto cuando estos no se dedican a expandir el relleno, haciendo notorio como las filas y columnas solamente reordenan sus celdas, sin introducir adicionales. Ambos frentes indican una desaceleración, quizá pasajera o cíclica, empero así mantenida por un largo periodo, por lo pronto a continuarse.

Con la suma de esta dupla, encabezada por el factor más estructural, la iniciativa tiene un fuerte empuje a bajar su perfil, al ver que una porción medianamente relevante de sus condiciones permanentes se está separando del viaje que venía atravesando establemente desde hace meses. Sus posibles efectos de verdad negativos no son inmediatos, pero anticipadamente ya comenzaron a palparse, dejando la puerta abierta a que después la totalidad de consecuencias directas apliquen y lleven a Fantemti a vibrar más bajo aún, si es que ninguna intención de revertir el panorama surgiera y pudiera concretarse, específicamente la calidad íntegra del servidor. Las grandes diferencias entre los anfitriones de turno son para verse posteriormente y no ya, la apertura de esta época zarpa teniendo una decisiva interrogante de potencial perjuicio por responderse próximamente, y más carga de inferior peso para desacelerar las actividades tanto internas como externas.

\section{La barata independencia y sus estaciones (Decimosexto libro, capítulo I)}\label{la-barata-independencia-y-sus-estaciones-decimosexto-libro-capuxedtulo-i}

\begin{quote}
10 de julio de 2022
\end{quote}

La administración de Fantemti anunció vía sus dos medios de comunicación principales un cambio sustancial en lo que respecta a la sostenibilidad de su propiedad, directo para una de las dependencias que le permiten ponerse en línea disponible para sus visitantes, factor clave cuando se arma la ecuación que da como resultado la existencia de un sitio operativo. Por lo que en ambos lados la versión corresponsal de la autoridad declaró, el componente de alojamiento verá modificadas sus condiciones próximamente, tal que fueran a generar un inconveniente el cual no está dentro de las posibilidades manejadas afrontar directamente, y esto abre a una determinación clara para lo que es la actualidad, pero medianamente incierta de cara a futuro, hasta incluso muy relevante en los registros de la venidera historia.

Como esta es pasarse de Railway a Heroku, evidente es que las razones que anteriormente llevaron a realizar la primera migración perdieron validez, la continuidad total a la cual aspira la conducción del clon anaranjado. Esta no lo mencionó textualmente, ni hay mucha información en la red, pero está implícito que van por el lado de las limitaciones de ejecución, ya sea parciales o totales, que junto con la negativa a ceder ante las tradicionales verificaciones o costos que plantean esa clase de plataformas. Es real que todas estas complicaciones serían evitadas si la política para con ellos fuera reformulada, sin embargo a intuir por el aviso emitido la decisión tomada es no hacerlo. Por tanto, lo inmediato es regresar a ese lugar donde el objetivo o principio elemental no podía ser cumplido, y ahora allí no pasa naranjas, cero gravedad aparente al cerrar el nuevo dialogo emergente de la esquina. ¿Pero después? ¿Y muy después?

Antes de nada, este proceder es clasificado como provisional, siendo que la \emph{/primera resolución}/ dicho así de paso implica que no será la única, dando a entender que la evaluación de siguientes pasos a tomar está pendiente o finalizada. Igualmente, tampoco fueron mencionados ni aproximados como tal, por lo que desde este momento las chances de que los componentes de toda esta estructura sean replanteados crecen considerablemente: que el proveedor fundamental que pueda volver a cambiar e implicar una tercera mudanza, que se vayan a utilizar dos direcciones en paralelo o según cual esté activa, o en su defecto que el antro termine permaneciendo apagado por acotados intervalos mensualmente, considerando que desaparecer del todo no es una opción, esas son las alternativas que más sentido tienen hoy.

Cabe mencionar que durante los últimos meses la dirección perteneciente al primer prestador, ahora la principal, no ha permanecido vacía ni fue abandonada, mas un letrero con la referencia al enlace activo siempre pudo hallarse al ingresar allí, a modo de indicación para aquel que se dirigiera sin la noción de que la transición sucedió. Imaginando un escenario similar, donde el caso fuera coincidir con el techo temporal que proponen el clásico asentador de los sitios temtianos, es un importante detalle que algo del control siga en manos de quien ocupa el dominio, porque al menos sería posible adaptar la ubicación con una que ponga al tanto sobre cualquier tipo de continuación, de lo contrario su gestor necesitaría de un canal aparte para hacer llegar las eventuales noticias.

Inmediatamente la realidad fantemtiana no conoce más consecuencias trascendentes que el mero hecho de desplazarse, pero el doble mensaje del mandamás esta vez no transmite buenas vibras respecto a la situación. La ilustración, la alerta, la bandera, y el resto de palabras que rellenan este episodio son imágenes más oscuras que otra cosa, y generan un panorama desorientador que llevado hacia adelante no puede ser traducido en expectativas positivas, a no ser que la tesitura inesperadamente tome otro camino. ¿Conclusión? ¿Se vienen tiempos difíciles? ¿O todo estará bien?

\section{El derrumbe de la sede secundaria (Decimosexto libro, capítulo II)}\label{el-derrumbe-de-la-sede-secundaria-decimosexto-libro-capuxedtulo-ii}

\begin{quote}
10 de julio de 2022
\end{quote}

Horas más tarde de que una pesada noticia sacudiera la perspectiva sobre todo a futuro de las modernas crónicas fantemtianas, surge también una grave incidencia que compromete uno de los sobrevivientes pilares de acumulativos de memorias, aproximadamente ocho centenares de comentarios comprendidos entre varias épocas. Expresado en vocabulario del territorio en cuestión, el roz de Fantemti, ahora corre altos riesgos de ser removido dentro de poco.

Allí en Rouzzed , que los hilos perduran según la actividad que reciban y la tanta que albergue su categoría, el que en un segundo momento dio a conocer la existencia del clon fanático fue trasladado hacia donde suelen durar considerablemente menos, a depender de que un frecuente posicionamiento realice el rescate de la corriente que los empuja al término de su vida útil, ser archivadas y posteriormente eliminadas. Luego de casi medio año, la moderación del sitio determinó que esta fuera enviada para ese separado que no encaja con ninguno de los demás, específicamente el que tienen para descartar contenidos con ciertas características prescindibles, llamado \emph{/Ay no se/}.

No fue directamente justificado el motivo de tal decisión, aunque conforme se lo mire puede llegar a entenderse, considerando otros factores implícitos que juegan en el contexto, en el cual tampoco son muy bien esclarecidas las recategorizaciones. Con las creaciones de los anoms fuera de sus propios dominios, o mismo para todo tópico centrado relacionado con los sitios temtianos, precedentes de este tipo hay para encontrar, pero este caso sugiere que las autoridades no quedaron contentas con los últimos intercambios que tomaran lugar en susodicha publicación, ya que esta logró sostenerse por bastante tiempo donde originalmente estaba asignada. Salvo una futura reversión, que de por sí es difícil vaya a darse, condena al título \emph{/Hice un espagueti temtiano/} a caer, porque no acostumbra acoger el movimiento suficiente que le permita mantenerse donde la dinámica lo exige en abundancia, los que van quedando son constantemente reemplazados.

De todas formas, actualmente esa en concreto no goza de un presente rico. Ya no es una pieza que incremente demasiado la popularidad, y menos cuando el desgaste la convirtió en una figura repetida que muchos optaron por ocultar o ignorar. Sus últimos reposicionamientos habitualmente no recibieron réplicas y en las ocasiones que sí llegaron a generar una adhesión muy acotada, por lo que su presencia al ser discreta no sirve de mucho para la publicidad, más efectivos son los posteos nuevos con intenciones de difusión, que tampoco se produjeron salvo ejemplares puntuales. Por otra parte, los últimos aportes temáticos del original publicador fueron escasos y el canal oficial padeció de ello, no obstante si volvieran a aparecer seguido perfectamente tienen como para ubicarse en espacios alternativos, los antecedentes afirman que la cultura no tiene problemas en multiplicarse fuera del hogar que protagonice el momento, sin mencionar que ahora tiene uno estable, el cual también es un candidato a recibir aquellos trabajos y noticias que con las repetidas migraciones hicieron de una costumbre el difundirse afuera. Igual con los diálogos protagonizados por temteros, a pesar de que tuvieran afinidad con un hilo puntual del extranjero, no están ligados estrictamente a uno definido, y menos con la modalidad que acostumbran a caracterizarlos, mientras no sean prohibidos por completo pueden suceder, sea bajo un rótulo restante que englobe el nombre Temti, o por ahí sin pretexto alguno.

Fuera su final o no, cierto es que la pérdida sería más significativa para la historia que otra cosa, puesto en lo más reciente supo perpetuarse principalmente inactivo, repitiendo cada vez menos el cumplimiento de los roles que asumió desde que Fantemti comenzó a recorrer el camino hoy atravesado con mayor lentitud, un paralelismo con correlación en ese sentido.

\section{Los insatisfactorios recursos de reinserción (Decimosexto libro, capítulo III)}\label{los-insatisfactorios-recursos-de-reinserciuxf3n-decimosexto-libro-capuxedtulo-iii}

\begin{quote}
11 de julio de 2022, \ldots, 14 de julio de 2022
\end{quote}

Luego de las dos malas noticias y sus inmediatos hechos en concreto pasaron unos días, y apartando el caso de la mudanza que por el momento consistió simplemente de eso, ya se cumplieron las lógicas predicciones sobre la materialidad del espacio informativo y difusivo, su despedida. Restan y eventualmente seguirán activos los demás relacionados a las iniciativas nacidas por Temti, sin embargo ese en especial fue un bastión de importante referencia para los progresos del clon anaranjado, con alta acumulación de contenidos culturales y todo eso. Aunque parte del valor pueda haber sido conservado por archivadores, la administración no contará más con él para incrementar la propagación de sus aportes diversos, y estará por verse si retoman en el mismo territorio, o si con esto concluyen en el corto plazo, como venían apagándose hasta hace no mucho.

Muy aparte de esos temas, aunque compartiendo la ubicación implicada, es la llegada de varios extranjeros al sitio, principalmente producto de que un puntual \emph{/token/} público para Rouzzed utilizado por los mismos fue condenado, quedando inutilizable. Sus ocupantes no tuvieron otra que abandonarla y rápidamente ir al refugio más próximo de encontrar, más o menos sencillo al tener en cuenta de dónde provenía ese famoso código: el discreto pero constante anom itmetishtu multicolores invertidos y sus recorridos a cada dirección. Además de propiciarle las vías de entrada a los interesados por dicho sitio, junto con otros medios de referencia, generó conocimiento sobre la existencia de la sucesora temtiana, rumbo sugerido como desvío de primeras ante el choque de pantalla rojiza.

Con las interacciones que iniciaron se hizo evidente que unos cuantos eran inclinos y acaparaban el usuario perdido, las conversaciones que protagonizaron dieron a entender abiertamente que bastante compartieron ellos como para conocerse los unos a los otros, desde nacionalidades hasta preferencias, y por lo visto lograron convivir bien entre sí, hasta hubo un conflicto que tuvo a uno conspirando contra el resto, y derivó en la disolución de ese rejunte de armonía dinamitada, en una especie de acto vengativo y suicida. Más allá de la extensa trama atrás de episodios del estilo, el proceder punitivo del régimen rozzado y los accesos restringidos a sus dominios, convierten a este asunto originalmente no propio de la realidad fantemtiana en uno que tiene contacto con su historia, por esos personajes que se integraron dentro de la misma y cambiaron el centro de atención.

Tal cual siempre ha sucedido, tener un vínculo más que estrecho con las contingencias del contexto clónico hace que sus situaciones exteriores se trasladen fácilmente, y con el bajo volumen de movimiento local cualquier pieza que a sumarse reordena los cuadrados que comúnmente estáticos permanecen. Los rezagados de las grandes esferas al extrañar las posibilidades de acceder a donde la mayor masividad se concentran, buscan entre los portales internautas cercanos uno que pueda ser el antro en el que van a dirigir sus palabras por lo menos transitoriamente, y aunque depende del momento, las opciones para ello no suelen ser muchas. Fantemti fue una de las varias en recibir a parte de aquellos individuos, y por esa peculiar razón las últimas horas tuvieron recibieron más participación en el plano visible de lo común.

Contando anteriores pedidos a la solidaridad acumulados, quedó en claro que los \emph{/tokens/} son ya un objeto de valor que no se consigue fácilmente, lo cual incluso llevó en ciertos casos a que fueran ofrecidos de manera aprovechada a cambio de dinero u otro interés. Sin embargo, de vez en cuando llegan a verse claves funcionales circulando por ahí, y es esto lo que la mayoría espera pescar al navegar nichos próximos, lo cual tampoco es improbable que suceda. Fantemti previamente albergó la creación de hilos dedicados a la difusión de estas credenciales, solo que la prácticamente el total quedaron inválidos, y además varios anoms también frecuentan ambos sitios, así que no hay que descartar que vuelva a filtrarse un nuevo medio de acceso. La escasez es adversa y las comunidades no destacan por ser altruistas, pero las chances no dejan de ser.

Por ahora la permanencia de los invitados se dilata, más de lo esperable de hecho, debiéndose seguramente a que no cuentan con tantas alternativas para mantenerse en el ambiente, aunque si no escaparon enseguida tan mal no les cayó, sea el último recurso o no. Igual así mientras estén generen más actividad y hagan crecer los números desde múltiples puntos de vista, presencias de este tipo no son muy simpáticas para los fantemteros de permanente regreso, más al estar centradas en función de un tópico en particular, obvio es que la estadía termina siendo pasajera y meramente justificada por el afán de obtener el pasaje para irse a otro lado, menospreciando a Fantemti y sus diferencias en comparación al clon dominante. Por tanto a no ser que un convencimiento inesperado los atrape, no hay dudas de que la estancia vaya a ser finalmente transitoria.

\section{La falla del seguimiento sistemático (Decimosexto libro, capítulo IV)}\label{la-falla-del-seguimiento-sistemuxe1tico-decimosexto-libro-capuxedtulo-iv}

\begin{quote}
15 de julio de 2022, 16 de julio de 2022
\end{quote}

Surge una situación relativamente externa, proveniente de afuera del principal antro temtiano activo, pero relacionada al acontecer de sus fechas pasadas, y de no haber sucedido, eventualmente también lo que serían los futuros acontecimientos dentro de sí. El canal anom itmetishtu multicolores invertidos ha desaparecido de su plataforma original, y con ello todos sus videos de la misma, dejando en lugar de aquello mensajes de indisponibilidad que delatan el destino que conoció, ser dado de baja. La intuición no es esquiva a los factibles motivos de tal suceso, a lo que encima resumidamente estos se encuentran rápidamente manifiestos en los letreros de Youtube, y seguramente tales faltas haya sido posible constatarlas con la realidad, a sabiendas de que el sitio ocasionalmente es estricto con el cumplimiento de los lineamientos que regulan lo que se aloja en él.

La noticia fue dada por un título que resume el hecho, y dicho hilo creado por el autoaclamado como dueño no dice más de lo sabido al respecto, simplemente que fue borrado a la fuerza y todo lo que tenía está perdido. Tal vez sea medianamente exagerado por su parte, sin embargo es real que había conseguido algo de fama, cerca de setenta seguidores al tanto de lo que estuviera lanzando a la red, más numerosas grabaciones pasando las cien visualizaciones, aunque también montones de relleno que lograban cifras muy por debajo. Concretamente lo que fue de interés de más individuos estaba en los fragmentos que incluían el nombre de Rouzzed: dichos ejemplares revelaban las incidencias en el antro más destacado dentro y fuera del ambiente, un sitio que por sus complejidades pasó a ser de clase privada, del cual la información fluye de manera restringida y la ardua facilidad para acceder decrece con el tiempo, y este especie de reportero difundía escenas que no estaban al alcance de muchos, incluso esporádicamente liberando credenciales públicas para ingresar ahí, que al invalidarse generaron notables repercusiones en los refugios cercanos. Además, el pasar de las mútiples alternativas de similar formato pero menor porte, que en los compilados eran juntadas encadenadamente y de tal modo ofrecían a los espectadores un abanico más amplio de reemplazos. De ellas, donde más era conocido es Ufftoppia, habida cuenta de que en repetidas oportunidades se lo llegó a mencionar a él y a su tarea o contribuciones en la materia, un caso particular porque de esos debates podría extraerse la conclusión de que fueron emitidas denuncias hacia el autor de las tomas, y sin duda alguna es factiblemente otra de las causas.

Como en verdad es, su presencia o ausencia no altera directamente lo que pase en torno a Fantemti, no obstante más allá de eso erigieron una fuerte conexión entre ambos, como también con una notable cantidad de clones aparte, prácticamente la totalidad de los protagonistas y actores de reparto en el contexto clónico. Particularmente con este mencionado disponía un especial énfasis en el seguimiento de sus novedades, ya que la mayor porción de filmaciones en los últimos tiempos constaban completamente de recorridos a los dominios fantemtianos, o en su defecto como mínimo unos segundos de pantalla eran dedicados a imágenes que provengan de allí, siendo pocas las veces que nada así se les incluía. Desde la fecha de fundación, el primer día absoluto inédito en primitivas características, hasta el detalle de las más recientes actualizaciones, y sumarle un puntual tramo de las últimas épocas de los anteriores antros derivados de Temti, un montón de contenido de valor histórico que quedó lejos del fácil acceso que propiciaba el portal universal con tanta llegada, tanto para públicos ajenos como para los propios. Era de una dinámica única, bastante formal y pareja en cuanto a su cometido, sin embargo alternativamente todavía restan varios medios por los cuales llegar a materiales relacionados, el más parecido sería el presunto oficial de Telegram, respaldado por una amplia trayectoria de recopilaciones que datan de mucho antes, conjuntamente con otras fuentes en que el pasado esté conservado.

Ahora lo que no trae demasiada incertidumbre es continuar sin él, porque altas chances indican que pueda ser indiferente y no genere ningún cambio en lo que diariamente es de las actividades de Fantemti, sin embargo lo indiscutible es que nuevamente una importante forma que tenía de trascender fuera de su actividad interna termina sucumbiendo, lo cual implica menos publicidad y menos probables bienvenidas, una potencial caída en el futuro tráfico como consecuencia factible.

\section{Los perseverantes momentos capturados (Decimosexto libro, capítulo V)}\label{los-perseverantes-momentos-capturados-decimosexto-libro-capuxedtulo-v}

\begin{quote}
17 de julio de 2022
\end{quote}

Largos momentos luego de que una negativa novedad condicionara para mal el futuro del pasado fantemtiano, una especie de suceso opuesto inesperadamente revierte el panorama que sentó la desaparición de aquel activista involucrado en la tarea de conservación de memorias, sucediéndola con la aparición de un individuo dedicado a lo mismo, salvando que con un enfoque más particular que general, y debiendo recomponer las piezas que se hubieran perdido con la terminación de servicio previa. En otras palabras es, el regreso del más activo contribuyente a las recopilaciones de material histórico. Tras confirmarse su presencia en el sitio en el fantem que lo tenía como supuesto autor original, donde entre las respuestas claramente dejó su huella en pruebas que respaldan al verdadero propietario de los clips, que aparentemente fueron conservados, y utilizadas con un fin similar, puesto que en segundo lugar otra publicación relativa a la causa da la primicia de que un nuevo canal casi que análogo fue creado, llamado OnlyTemtis. Fundado a escasas horas de que el anterior terminara eliminado, contiene apenas unos pocos videos que inician desde una época concreta y van hasta unas pocas fechas después con los posteriores acontecimientos, manteniendo un estilo tremendamente parecido al que es con todas las letras su antecesor.

Presumiblemente a la postre las siguientes filmaciones completen la colección temtiana que era contenida entre los amplios depósitos de anom itmetishtu multicolores invertidos, empero quizá lleve su tiempo ya que resta una larga sucesión de episodios entre ese punto y la actualidad, si es que antes no conoce su mismo destino súbitamente. Aunque no es igual, si se mantuviera la exclusividad que explícitamente sugiere su nombre, no cumpliría igualmente el rol del medio predecesor, por cuanto que la difusión dejaría de centrarse en las novedades del contexto en su totalidad para dedicarse de lleno a un solo rumbo de los posibles en él. De llegar a continuar y completar el respectivo tramo, por cierto un tanto extenso, sería sumamente positivo para la difusión ante extraños que no conozcan de todo esto, y más todavía, los interesados en que exista un archivo público adicional donde estuviera la posibilidad de recorrer visualmente la evolución de las condiciones en los tantos sitios generados alrededor de Temti, más considerando que dicha progresión se viene desarrollando en múltiples ubicaciones y desde entonces el foco comenzó a desviarse hacia direcciones más extendidas, un montón de idas y vueltas que diversifican enormemente los destinos donde debieron asentarse los seguidores de aquel principal, que ahora encontraron estabilidad y se ocupan más de evolucionar que de migrar, cosa que también estaba siendo documentada en aquellos registros fílmicos, que en principio van a ser restaurados, aún así no fuera confirmado.

\section{Claves de visibilidad (Decimosexto libro, capítulo VI)}\label{claves-de-visibilidad-decimosexto-libro-capuxedtulo-vi}

\begin{quote}
17 de julio de 2022, \ldots, 29 de julio de 2022
\end{quote}

La caída de los estandartes que marcaban presencia por el consolidado sucesor temtiano días después demostraron por el momento no tener demasiada incidencia en lo que sucede dentro del mismo, además de opacar la posible gravedad que vaya a tener el obligado traslado y reemplazo de prestador, naturalmente manteniéndose en la senda previa donde el movimiento interactivo iba al alza y los conjuntos de mejoras aparecían con un mediano espaciado entre sí.

De eso último trata lo siguiente, empezando por el punto más desarrollado de los varios aplicados. Un pedido particular de la comunidad antes postergado había sido agregado erróneamente a la lista de espera donde van esas cosas solicitadas y no concedidas en el momento, el puede ser más adelante y sus diferentes formas de nombrarlo, la implementación de un gran atajo que la administración confundió con otro relacionado de tal área, arrastrar y soltar para subir a partir de ahí. El anuncio de la versión mencionó en esos términos únicamente dicha vía para importar materiales de multimedia, pero la sugerencia original se refería a que la carga de archivos también pueda ser mediante el portapapeles, una facilidad sumamente práctica con la que cuentan la mayoría de sitios respetables donde hay manipulación de imágenes. Tras que el desentendimiento fuera debatido en la discusión que más participaciones acumula, en la jornada próxima fue que la posibilidad pretendida pudo ser insertada, la cual sin dudas hace más práctico el mecanismo de comunicación gráfica que acompaña opcionalmente a las lineas de texto, aunque con las circunstancias actuales sería nada más disfrutable por los privilegiados en permisos y no limitados por las vigentes restricciones.

La actualización inicial asimismo trajo una doble propiedad habitualmente bastante solicitada en los antros del contexto, en función de lo cuanto que los participantes pretendan adaptar los contenidos albergados en el espacio principal del sitio, construyendo una tajada importante de su comodidad desde allí. Típico cuando las preferencias se superponen a la tolerancia para con lo que vayan a expresar los demás, el deseo de hacer desaparecer de la vista aquello que no sea de agrado, el esquinado botón de ocultar como medio más directo, conjuntamente con una herramienta más compleja, filtro por palabras claves previamente seleccionadas a ser excluidas por completo del panorama que termine siendo presentado por el servidor. Dado el volumen de actividades, emplear excesivamente la utilidad llevaría a un vaciamiento de las pocas cantidades de relleno que suele haber en el paisaje, sin lugar y tiempo suficiente a que pudiera renovarse prontamente, sin embargo el propósito esperable va más para temas concretos, que quedan o no a gusto del consumidor, y los modos manejados dan para estimar que la intolerancia no es tanta como para que la mención del comando \emph{/hide/} aparezca seguido.

Por otro lado, un cambio ampliamente detallista pero lo suficientemente significativo en virtud de las posibles interpretaciones a generarse sobre lo que sean esos fantems que si aparezcan en el inicio, que los no comunes y corrientes estén efectivamente distinguidos del resto aún así se encontrasen a un igual nivel, para notar que no tienen la misma jerarquía o importancia en su interior, y también para dar cuenta de los que recientemente fueran insertados al extenso listado de hilos. Aquellos donde verificar la opción de posicionamiento no es válido y comentarlos da resultados indiferentes a si nada los hubiera reactivado, ya tenía un ícono distintivo, se los representaba con un ancla de mar y hacían valer ciertamente su significado, quedaban en el fondo y no podían ser elevados a través de la acción de \emph{/bumpear/} por mucho que se la forzara, no obstante el único ejemplar de estos atípicamente reflotó y evidenció el funcionar de su característica, no poder regresar a la primera posición. Similar a la llama que indica la frescura de la publicación en sí, la indicación que informaba si fue enviada en los últimos momentos solo estuvo un breve tanto y luego no fue más visible, y ahora vuelve a estar. Por último y no menos relevante, una nueva estrella que comprueba las comunicaciones provenientes de la administración y destacadas como tal, las oficiales, comúnmente creadas con la intención de otorgarle un marco más apropiado a los asuntos relativos a la realidad del sitio, por ejemplo el que permanece fijado constantemente, que hasta hace poco y varias veces antes debió compartir el hueco de privilegio con otros aparentemente no tan serios o relevantes, una manera de diferenciarse y evidenciar mayor prioridad, más de paso evitar confusiones con autores que digan ser quienes no son.

Coincidieron con dos semanas de, sólidas rachas de actividad más corriente. Las iniciativas culturales que en los últimas épocas han reducido sus entregas pudieron volver a lanzar piezas que siguen la trama de la cual se basan, con el ya clásico Diario Temtiano que a pesar de haberse quedado lejos por varios meses de diferencia continúa dejando registro de lo acontecido, la confirmación del flamante OnlyTemtis y su objetivo de recuperar los videos extraviados, el canal oficial en la plataforma de mensajería y los extraños reportes sobre su caudal de seguidores, aportes que no se detienen y que sostienen la identidad en su dimensión más especial. A su vez los visitantes por ratos sostenidos lograron darle una cierta vida a las renovadas estructuras, haciendo frecuente la emisión de habla con también incluso repercusión por parte de los individuos asiduos, entre los cuales continúan estando los refugiados de la exclusión rozzada, además de nombres que no dejaron de repetirse, y las ya acostumbradas sucesiones de retratos monotemáticos cubriendo durante largas jornadas importantes porciones del índice central. Clásicos impredecibles que igualmente no soprenden, Fantemti en una vigente expresión de sus escasas maneras de existir normalmente, aunque no el más común de todos últimamente.

\section{Los frutos de la vulnerabilidad (Decimosexto libro, capítulo VII)}\label{los-frutos-de-la-vulnerabilidad-decimosexto-libro-capuxedtulo-vii}

\begin{quote}
29 de julio de 2022, 30 de julio 2022
\end{quote}

Una noche de aquellas discretas y calmas en el entorno fantemtiano, tuvo a ese escenario sufriendo varios percances serios en sí mismos, derivando en su incomparecencia temporal. Pantallas de error en la clásica plantilla por defecto del servidor de antaño cubrieron las típicas estructuras del sitio y a su vez la visibilidad del diseño más funcionalidades que suele servir, siendo en definitiva momentos por los que durante ellos Fantemti no se halló funcionando. Al rato la situación en cuestión finalizó y para su suerte en la interna todo regresó a la normalidad, sin embargo sentó un antecedente que por ciertas informaciones circuladas el exterior, se diferenció de otros similares episodios en los cuales no hubo forma de acceder debido a interferencias de parecida devolución para el visitante que se aproxime.

En la circunstancia de que allí no tanto sobresaliente sucedía, entre los antros de por ahí nomás se hablaba de que varias agresiones informáticas fueron encausadas por parte de un famoso individuo contra un par de bastiones alternativos al más preferido de todos, lo cual justamente coincidió con los coincidentes problemas de Fantemti justificados como \emph{/Application Error/}. Así es que rápidamente los evidentes ataques de denegación de servicio fueron relacionados con la dicha caída pasajera, que si como la evidencia indica fueron dirigidos hacia sitios prácticamente abandonados, no carecería de sentido que mientras una que otra embestida recayera en el clon temtiano. Cuando esta data fue trasladada a la conducción por medio del hilo fijado, el mandamás le restó importancia y no lo trató explícitamente como un asunto relevante, y no se sabe si en verdad estuvo al pendiente del mismo o efectivamente no se enteró hasta más tarde, pero sería de esperarse que la seguridad estuviera entre sus preocupaciones, ya que es obvio afecta directamente a la estabilidad de su propiedad y permanencia en linea, tantos que antes fueran tratados oficialmente. Dadas las modestas prestaciones con las que cuenta el actual alojamiento o incluso el anterior, tampoco parece que haya mucho al alcance inmediato del administrador, no obstante igualmente quizá algo pueda hacerse al respecto como para prevenir o aunque sea mitigar, porque sin protecciones el sitio queda vulnerable a atentados comparables a este, nada grave hasta que coincida con la ínfima posibilidad de que intenten hacerlo caer.

\section{El calco de las profundas raíces (Decimosexto libro, capítulo VIII)}\label{el-calco-de-las-profundas-rauxedces-decimosexto-libro-capuxedtulo-viii}

\begin{quote}
31 de julio de 2022, 1 de agosto de 2022
\end{quote}

En el autodeclarado clon de los clones, ya no tan de preferencia a los temtianos, la gesta nostálgica de simulaciones y recreaciones vuelve a cumplirse, con el foco en una interfaz histórica y que en teoría superior valoración logra entre los experimentados con memoria del contexto, de tal importancia que inspiró a los fundadores posteriores a armar copias con características propias y particulares, sobre un mismo formato que con tanta semejanza entre intentos de sucesoras fue duplicado sin cuidado a repetir. Con colores y pulsadores personalizados de Ccchan, los clásicos estilos de Voxxed se apropian del puesto principal en Fantemti y amplían su catalogo con una doble imitación de aceptable nivel, muy próxima a los originales y que según los anuncios aclaran alternarán entre sí, variando las tonalidades los próximos días. Asimismo, la presentación de este resultado trasciende su ubicación de origen y se hace un lugar en dominios exteriores, como previamente también ocurrió el dueño promociona con imágenes representativas y extensas redacciones la descripción general de la estética, reenciendiendo el intercambio en torno a esta y alrededor de otros temas no muy ajenos, parte de la dinámica que tenía el primer hilo oficial en el exterior recuperada.

Simultáneamente, la única característica que es agregada junto con la pieza mayúscula de diseño es una clave expansión a las opciones que el sitio ofrece para los no ingresados en el sistema de sesiones, esto es el requerimiento de contar con un código para participar abolido, ahora opcional. Apoyando a las declaraciones de la gestión referentes a esta nueva facilidad, beneficia a aquellos que no desearan autentificarse para comentar o publicar, sin más hacerlo directamente tras afirmar el pedido de confirmación que realiza la ventana de dialogo con las alternativas en botones, saltear o acceder. Ciertamente, son esas básicas las únicas capacidades adicionales que ganaron en comparación a lo que antes podían hacer, por lo que hay diferencias con respecto a proceder teniendo credenciales en uso, pero no son demasiado grandes ni condicionan para mal la experiencia si no es debido a que los avisos de respuesta no están disponibles y alguna que cosa más meramente suplementaria.

A la de siempre es sumada una manera más de dejar marca e involucrarse entre signos de interrogación anónimamente, porque a falta de explicaciones del trasfondo son abiertas dos posibilidades, que solo sea una comodidad para afuera y en los procesos internos está trate a los autores igualmente con trazabilidad, o que cualquier acción que realicen efectivamente no deje rastro y vinculación con las que haga después, considerando que el primer caso favorece a la moderación al disponer de más herramientas para regular, sin embargo eso no es manejado casi. Por otro lado, las transformaciones en apariencia son quizá de las más gratas desde que la modalidad comenzó a otorgar variantes, pues las anteriores tuvieron sus virtudes y a lo mejor son tenidas en cuenta más que indiferentemente, pero esta en especial tiene para generar una ventaja en esa estadística no conocida y por consiguiente mayor estima entre usuarios que seleccionen una, más incentivado aún con que mientras tanto quien fuera a asomarse en los cortos plazos actuales se topará con una u otra versión de la misma, enmascarando las reducidas discusiones de un antro que no es lo que parece.

\section{Los procesos míticos y no míticos (Decimosexto libro, capítulo IX)}\label{los-procesos-muxedticos-y-no-muxedticos-decimosexto-libro-capuxedtulo-ix}

\begin{quote}
2 de agosto de 2022
\end{quote}

Según como la condición cual cronograma había especificado que sería la permanencia de los flamantes estilos imitación en Fantemti, tras experimentar con los coloridos primerizos de Ccchan discretamente durante unas pocas horas, pasó a disponer la variante número uno en la mayoría de los ordenes, salvando el estipulado de presentación. Ahora es la copia de Voxxed la encargada de darle aspecto a las estructuras del sitio. En relación a ello, los intercambios referidos específicamente al tema de las novedades apenas lo mencionaron, pero de igual modo varios habrán llegado a echarle un vistazo a partir del espacio de difusión en el exterior, una correcta oportunidad para los veteranos que desearan reencontrarse con o explorar los exóticos caudales, más únicos todavía cuando faltan clones activos que sean literalmente idénticos al legendario modelo.

Por fuera de lo que implique la temporal estadía del camuflaje, otra situación proveniente de los movimientos oficiales hace de los momentos corrientes más particulares aún, y esta sucede primordialmente en los puestos de preferencia del inicio, por títulos y contenidos complementarios al típico que allí suele ubicarse, especialmente uno que trae planteos decisorios sobre lo que vayan a ser una serie de detalles simbólicos. Aparte de repetir la práctica de fijar hilos que cubren asuntos no relativos a las formalidades del lugar, después de bastante tiempo entre aquellos apareció una prerrogativa de la administración acerca de las medidas a tomar respecto a los decorados de la bandera ucraniana, que desde su aplicación venían marcando presencia en casi todos los recovecos del antro, y precisamente en las últimas jornadas habían desaparecido.

Aún así, la retirada de los marcos bicoloridos estuvo lejos de dar para quejas o manifestaciones, según indica este comunicado destacado, el paso de la gestión que terminó provocándolo fue accidental, no voluntario, opuesto a la reducción del grosor que previamente de forma deliberada hizo más discreto el reconocimiento. Al reconocer la consecuencia, lo que propone es mantener con el distintivo la creación de fantems y comentarios de este tipo, tal cual como estaba antes de que las recientes modificaciones internas invisibilizaran su característica frontal, o de lo contrario marcar un punto de inflexión haciendo que no sigan multiplicándose así. Notar que cualquiera de las dos supuestamente no estaría implicando limpiar los tantos elementos ya existentes, sino poner un fin a la etapa de apoyo de ahí en más, similar a lo que evidenciaba el contraste entre los dos anchos de margen en circulación, o también lo ocurrido con el evento de única jornada que dejó limitados cuadrados de anoms junto a un particular detalle, posiblemente esa manera repita de primar la opción conservadora, viendo que desde la óptica del resto de interfaces los bordes aún son conservados.

De todas maneras, la propuesta es un tanto discordante con el motivo inicial que trajo la dupla de referencias, ante lo cual su original publicador menciona brevemente las contras que pueden haberle llevado a retractarse y no reponerlo inmediatamente como si de un error se tratase, y más allá de la importancia que deba recibir en la apreciación mayoritariamente subjetiva, cada una tiene algo de sentido. El mensaje emitido con la tomada de posición tan específica y sus modos de transmitirla es fácil de malinterpretar, justamente debió pasar un largo rato para que esta terminara elucidada en palabras, hasta entonces quedaba plenamente dejada a discernimiento del receptor, y por supuesto la posibilidad de que no gustara condicionaba desde varias ópticas. Además el criterio utilizado podría sentar un conflicto para ocurrencias equivalentes, fueran disputas políticas entre naciones distantes o cualquier asunto donde se forme una postura común tanto por mayorías tendenciosas o principios éticos. Como fuera, dichas cuestiones pasaron a estar en la evaluación no por cambios externos sino internos, aunque indirectamente también afectaron para abrir una fase decisiva en el curso de uno de los pocos eventos extraordinariamente referenciados por la conducción, en él la pendiente respuesta nuevamente prioriza el parecer de los involucrados, y así el futuro panorama fantemtiano prontamente será definido por el pueblo.

\section{Los inconfundibles deseos de prosperidad (Decimosexto libro, capítulo X)}\label{los-inconfundibles-deseos-de-prosperidad-decimosexto-libro-capuxedtulo-x}

\begin{quote}
2 de agosto de 2022, \ldots, 4 de agosto de 2022
\end{quote}

Tras que corrieran aunque sea un par de días y Fantemti regresara a los coloridos originales de casi toda la vida, la resolución pendiente conoció conclusión, y consigo más novedades de interés.

Del planteo realizado por la administración y llevado al plano relevante del sitio mediante un anclado al cielo, el resultado de la votación o el consenso de palabra resultó medianamente difuso, puesto que las no muchas respuestas que recibiera al final fueron un tanto aparte de la decisión tomada con los decorativos del conflicto internacional, que en sí es detener la multiplicación de los mismos y en cambio emitir una expresión semejante más generalizada, reflejada desde un ángulo diferente. La opinión pública según el balance de la encuesta formulada se inclinó por la permanencia de las referencias bicolores, por un parejo margen, y las réplicas escritas reclamaron el reconocimiento a los enfrentamientos internacionales más recientes, lo que siendo concedido implicaría establecer más alusiones a las naciones involucradas, o en el otro caso hacer que las siguientes participaciones sigan teniendo la cualidad objetada, pero no, ninguno de los dos ha sido el caso y la única indicación es un pájaro que presumiblemente simboliza la paz, una puntual que llegó a manejarse en el debate.

A pesar de ser mínima, esa fue la parte más notoria de la actualización, que además contó con otra característica solicitada desde la comunidad por las molestias que generaba una insistente respuesta habida desde los inicios del antro, los mensajes de éxito. En el orden del correcto funcionar de los botones y peticiones realizadas, al final los recuadros de verde transparencia informan sobre la concreción de la acción realizada, como ingresar credenciales válidas, enviar contenidos aceptados, modificar configuraciones propias, y un no tan largo etcétera, incluidos los azulados informes de actividad reciente junto con su apegado ruido de notificación. Pues la novedad es que pasó a ser un parámetro modificable recibirlos o no, no verlos una y otra vez como reiteración de algo que de haber concluido bien, secuencia que normalmente en portales solo es así en una pequeña porción del recorrido dinámico.

Y no solo el reemplazo del tradicional portátil fue lo que trastocó el acompañamiento del nombre titular que siempre encabeza el sitio, dado que en su espacio lo nuevo es un acceso al nicho titulado como \emph{/Memoria/}, muy extraño para lo que son habitualmente el resto de rutas y con un formato único en el total de estas, tanto que amerita diferentes descripciones. Con ambiente de tormentas oceánicas y tripulaciones tambaleantes, dichas imágenes son cubiertas por un mensaje confuso y referente a dos tópicos, las cifras del contador en cuestión y su equivalente en el anterior bastión temtiano principal, los individuos en él que drásticamente dejaron de apreciarse en el recuento tras la implementación del actual medidor. El mismo que fuera altamente polemizado anteriormente por la dudosa veracidad de lo que indicaba es recordado por parte de Fantemti oficialmente, directamente abstrayendo el agregado que en aquel entonces se mantenía fijo y no bajaba de allí, las veinticinco unidades en estos términos de camaradas.

Esta oportunidad aún trayendo una reducida suma de novedades, propio de versiones etiquetadas con una letra al final de la denominación de expediente, viene cargada de más tendencias a levantar polémica, por la razón de que no es únicamente de mejoras estructurales, el gran componente es causado por los criterios manejados por la conducción donde más subjetividad hay, y menos clamor popular o siquiera lo de esperar. Antes cuando los marcos ucranianos fueran reducidos se hizo notable como la importancia bajó, que ya no era la misma que en un inicio, o aunque sea la manera de transmitir lo que quisiese decir debía ser más discreta, y ahora le sigue un segundo paso el cual aproxima a la teoría de que hubo una marcha atrás. El símbolo del animal aéreo es sumamente más abarcativo y neutral que apoyar a un particular e incluso pasa desapercibido, conservador a más no poder. Por otro lado, la clara referencia al antiguo hogar es de lo más curioso en los últimos tiempos, inesperadamente le da una mirada después de varias épocas de desinterés y silencio, y encima del manifiesto comunicado podría estar intentando mostrar un contraste con la cifra de presentes, que actualmente es notoriamente menor y que con una constante base daría impresiones más variadas, por robots de relleno o mayor público real, pero no es más que un detalle de conexión cultural que a lo mejor va en el sentido literal, para los silenciosos desaparecidos ausentes en Fantemti.

\section{La oscuridad del cráneo viviente (Decimosexto libro, capítulo XI)}\label{la-oscuridad-del-cruxe1neo-viviente-decimosexto-libro-capuxedtulo-xi}

\begin{quote}
4 de agosto de 2022
\end{quote}

Fantemti cree haber encontrado su símbolo de paz, tras intentar orientarse a través de la consulta y finalmente inclinarse a favor de uno de silueta inconfundible, con la cual están siendo representados los contados visitantes aún así no comprasen la perspectiva negativa frente a las trifulcas. No obstante, rápidamente fue a revelarse una contingencia adicional que incluso reviste más complejidad y requiere mayor atención por la rareza de su acontecer para los tiempos que corren, y esto son sorprendentes hechos inéditos en los actuales territorios de Temti en lo que resta de su vertiente original.

La referencia a Titiri Vieja situada en el espacio memorial a su vez contiene un enlace discreto y no resaltado a sus específicas coordenadas, el mismo sitio pero que depara un escenario particularmente también único. Lo esperable habría sido que esta permaneciera bajo el interminable mantenimiento establecido desde los principios del clon hoy activo, empero un breve vistazo basta para confirmar que dicha protección fue removida y en su lugar quedó revelada una larga secuencia protagonizada por esqueletos y energía. Titulada Hades, la exhibición única desarrolla espectáculos luminosos sumamente llamativos donde los seres no muertos realizan acciones de todo tipo, caminar danzar y hechizar, crear y destruir, sin mucha trama evidente, que igualmente es ruidoso por sí solo, a pesar de que no emita sonidos.

Ciertamente nadie avisó al respecto, a no ser el propio individuo que dio a conocerlo con un tono de indignación, de allí que la primicia sin fecha pudiera conocerse por el contexto. Y con razón, puesto que para los simpatizantes más fieles el evento que sea en torno a los orígenes en una nación desplazada que tuvo una transición inconclusa, poco comunicativa y librada al entendimiento individual. La más cercana presentación del sitio en cuestión data de bastante tiempo y era una suerte de despedida al formato típico, que no perduró operativa por mucho y pronto jamás se la vio, encima de que su adelanto tampoco ocurrió, siguiendo un largo intervalo de inactividad absoluta hasta este regreso, si es que entre medio no hubo algo más, y aún así es parcial y exige lectura para verlo más que un hecho desconexo.

Una muestra como esa es capaz de dar rienda suelta a numerosas interpretaciones, intentos por aproximarse al significado de trasfondo o particular para la ubicación, sin embargo no hay elementos de los cuales agarrarse como para vincularlas debidamente con las situaciones más relevantes del mundo temtiano, o si los hay son altamente rebuscados, más que cualquier acertijo o incógnita muy antes solían ser tratados de resolver. Entre los caminos más intuitivos, una nueva pista del viejo protocolo o figura máxima aludida por el encabezado, el cual hace montones no se renueva y es una pizca de algo entre las escasas referencias liberadas oficialmente, o a lo mejor trate de representar el rol cumplido por los esqueletos en uno o diversos ámbitos\ldots{} o simplemente no quiere decir nada, especulaciones relativas fueran como fueran. ¿Pasado, presente, futuro? ¿Sí, no? ¿Cuál es la idea?

Es que tantas cosas podría estar diciendo esto y no hay manera de cerciorarse, allá lo que el dueño de los restaurados dominios haya buscado comunicar al poner el video temático sin casi indicaciones, un misterio total igual que su paradero actual e involucración con la coyuntura donde supo hacerse un nombre. Y más es extrañamente llamativo que justo hubiera novedades aquí cuando la conducción fantemtiana coincidentemente creó un nicho simbólico conectado con su predecesora, cual si de antemano conociera que había una mayúscula sorpresa sin que estuviera comunicada o aunque sea aproximada mediante pistas, como si los administradores tuvieran contacto entre sí y este episodio fuera conjuntamente coordinado, o en el caso contrario de que vayan separado menos se entiende porqué no fue revelado previamente, que el responsable lo olvidó después de gestarlo o prefirió esperar a que salte por su propia cuenta. Es sumamente incierto y deja intriga para lo que vaya a suceder próximamente, tanto por explicaciones valiosas y por futuros eventos.

\chapter{El periodo de expectativa y reconexión con las garantías y paces (Decimoséptimo libro)}\label{el-periodo-de-expectativa-y-reconexiuxf3n-con-las-garantuxedas-y-paces-decimosuxe9ptimo-libro}

Dupla de acontecimientos vitales, objetivos. La siguiente circunstancia de relativa entidad es resumida con un desvío y un aviso, no obstante responde a un largo contexto, cuando el sitio tuvo que ir con un servidor nuevo que previamente había tenido una época con él. El mismo tipo de atenuante repite y resulta ser la apertura de un paréntesis con clausura definida para el escenario principal de la Historia Temtiana, un breve periplo finalizado luego de reunir las condiciones necesarias para regresar a donde estabilidad consiguió y supo asentarse, pronto para retomar dicha senda de épocas desarrolladas con la menor injerencia externa posible, aunque sin certezas de que las próximas se deban exclusivamente a fallas cortas. Dicho sea el antecedente concreto encargado de marcar el antes y después, solamente por parte del proyecto fantemtiano, cuyos tiempos y lazos coinciden con la inesperada reactivación de la olvidada Titiri Vieja. La plaza sosegada bajo el parcial cierre y posterior mantenimiento rompe su extenso descanso al darse a conocer en una reformada versión no clásica, marco completamente atípico que en simultáneo plantea mensajes ambiguos y complicados de interpretar, eventualmente tramas del pasado presente y futuro encriptadas, carentes de atención como los desacostumbrados rompecabezas provenientes de la o las conducciones. Son incidentes de referencia en los venideros registros históricos, para una era concluyente concentró especialmente en la recuperación del apoyo material, otra dimensión más que conforma el amplio conjunto temático todavía siendo preservado y en expansión.

Las sensaciones y expectativas para seguir de este punto de partida no son ni apuntan a ser muy evidentes en el ambiente que, poco manifiesta ya sobre su perspectiva referida a los asuntos del antro. Allí además hasta entra lo manifiesto por los niveles de actividad, los cuales aún con su naturaleza habitualmente quieta, cada tanto experimentan pasajes donde hay más vida perceptible, incluso afectados por eventos exteriores que elevan el promedio, una discreta inclinación al crecimiento que raramente tiene mucho para sostenerse, a no ser más perturbaciones ajenas o movimientos difusivos extraordinarios que sí suelen ocurrir. El ritmo de actualizaciones a su vez sigue la tónica contraria, conforme a la tendencia mantenida desde hace rato los cambios estructurales cayeron en necesidad por estar los cimientos fundamentales en su mayoría, pero tampoco se han terminado los detalles y transformaciones por agregar al clon, ostensible cada oportunidad que lo disponible es más, cuando menos es esperado.

La clave efectivamente se desprende de la mudanza y lo que esta supone por el soporte del que está agarrándose para permanecer en linea, ahora es de una vez por todas el que no carecería de la fortaleza que en los comienzos llevó a relegar la alternativa vieja y tradicional confiable, dicho en condicional por la baja especificidad de los recientes informes oficiales, presumiblemente poder funcionar de corrido sin limitaciones periódicas, y contando que en general los rendimientos tan siquiera visibles para el público exterior han sido altamente similares, el candidato a causa de ocasionar las repetidas vueltas es ese. Railway entonces, algo con lo que los anoms no tienen que ver, pero si reconocer al utilizar las rutas de navegación, acaso apuntadas por la conducción interesada en reducir el sufijo correspondiente, o dispuesta a conservarlas como tal por muy engorrosas que fueran, otra cuestión candidata a extender el ciclo de reubicaciones en un mañana no asegurado.

A juzgar por lo que viene comentando al respecto y la información que pueda recabarse, son materias verdaderamente importantes para el antro en sí, no obstante además de los intercambios de prestador y las coordenadas proporcionadas para conectar con ellos, el efecto indirecto de esta clase de episodios no está pudiéndose ver, más en medio los antecedentes y tendencias del momento transcurren normalmente en ausencia de consecuencias. Si a la larga el paso del tiempo demuestra que Fantemti es capaz mantener el motor encendido sin el menester de alternarlo, el último traslado podrá distinguirse de los anteriores por su carácter definitorio, y por la relevancia de lograr el supuesto cometido al final satisfecho, indistintamente de lo valorable que sea para la pacifista interna.

\section{Del revestido lazo cimental (Decimoséptimo libro, capítulo I)}\label{del-revestido-lazo-cimental-decimosuxe9ptimo-libro-capuxedtulo-i}

\begin{quote}
6 de agosto de 2022
\end{quote}

Aproximadamente un mes luego de que la mudanza número dos redefiniera las coordenadas oficiales de Fantemti, dicha determinación es revertida y ahora se encuentra a la par de la dependencia con la que pasó la mayor parte de su existencia, tras el sufijo de Railway. El informe lo realiza la administración mediante el instrumento más imponente y llamativo que tiene para elevar datos de este tipo, las oscuras alertas, más la extensión en el espacio de anuncios, ambos condiciendo con el efectivo traslado del centro operativo del cual el clon depende para funcionar, los concurrentes dejan de poder encontrarse donde el viejo prestador dado que desde allí son redirigidos hacia el nuevo, aunque no tan nuevo por la experiencia previa junto a él. Y lo más importante al ser el motivo de raíz, sin las contrariedades que preliminarmente la relegaron. Asimismo, en las primeras horas posteriores a la transición el aviso contuvo la mención a una nueva dirección terminante en punto ga, sumamente más sencilla y fácil de digitar, propia de los dominios gratuitos ya vistos previamente en el contexto, relativamente menos vistosos pero válidos al fin. No obstante, terminó siendo retirada y en el desarrollo de la jornada dejó de ser posible ingresar a través de ella, sin explicaciones algunas de si pronto volverá a ser ocupada para trasladar el funcionamiento allí o si será desestimada como si nada hubiese ocurrido. Prácticamente ninguna de esas cuestiones repercutió ni retumbó en el quieto debate interno, asumido tal cual fue y tomado como natural entre los visitantes, que al permanecer ingresando no rechazan plenamente las complejidades de la ruta.

Lo que en este caso extraña además de la primera aparición editada y suprimida, es la brevedad con el cuento de los detalles en las lineas de aclaración, normalmente asuntos como estos son propensos a otorgar una larga cantidad de texto descriptivo con inclusión de pormenores, sin embargo por una segunda oración que se ataja ante fallas inesperadas el mensaje es directo y concreto, tal vez porque la conducción no desea revelar o agregar más de lo necesario, o porque es así de simple. Ni es claro lo sucedido con la tercera localización fantemtiana de la historia, por los momentos que estuvo pareció andar correctamente y de todas formas padeció la desconexión más remoción de referencias, lo que hace suponer hubo problemáticas desde la interna o la decisión de afianzarse no fue sólida, pero desde luego que abrochar un nombre sería conveniente y favorable para los anoms que vayan a tratarlo, más todavía si fuera estable.

De igual modo, los que vayan a fondo por la información con saber lo primero estarán al día, lo mismo que dice la alerta y que implica el redireccionamiento hacia la ubicación activa. Más lejos, el cambio debería estar teniendo su impacto verdaderamente sobre el futuro y la sostenibilidad en linea accesible desde un vínculo que no cambie y permanezca fijo, considerando que la limitación que lleva a variar en el servicio de soporte es la de ejecución absolutamente continua, una clase de objetivo que declaró tener la comandancia del sitio y que dados los hechos está cumpliéndose, solo que con los atenuantes de tres vaivenes en el medio y unas pocas caídas temporales. Esta mudanza no vuelve a mencionar aquella aspiración, por lo tanto tampoco es seguro dar por sentado que así seguirá siendo, pero las posibilidades no escapan de eso, de antemano podría adelantarse que Fantemti regresó, ¿para quedarse?

\section{La reposición de desconocidas baldosas rotas (Decimoséptimo libro, capítulo II)}\label{la-reposiciuxf3n-de-desconocidas-baldosas-rotas-decimosuxe9ptimo-libro-capuxedtulo-ii}

\begin{quote}
6 de agosto de 2022, \ldots, 13 de agosto de 2022
\end{quote}

Para las historias de Fantemti, como podía anticiparse la mudanza no acarreó más variabilidades inmediatas que la propia alternancia de dirección, generalizando por ellos los individuos pudieron enterarse de la movida y de allí para acá vinieron, reanudaron las actividades sin debatir ni cuestionar los asuntos oficiales, aunque con un poco menos de fuerza que en los días previos, más o menos predecible por suponer que los escasos actores no siempre van a marcar presencia con similar frecuencia. A su vez un tanto posteriormente, quizás sin relación alguna o quizá siendo hechos entrelazados, la gestión del clon detuvo las rotativas reincidentemente y no todo fue apropiado en el funcionamiento integro del mismo cuando estuvo totalmente en linea, una mezcla de posibles desaciertos en las tareas, errores que vinieran de antes, o inconvenientes semejantes, pero de igual modo con resolución llegando sin tardar demasiado.

Preventivamente, una aclaración englobó a las múltiples pausas que declararan el estado de dinámica bloqueada y congelada, de ahí la justificación dada al público, operaciones de carácter interno con la base de datos y procesamientos de información a ser realizados como forma de mejorar el andamiaje virtual del antro y demás. Las voces del exterior hicieron referencias al respecto, visto cuando el título de caída dio cuenta de los instantes de acceso denegado, sin embargo con escasa repercusión entre los anónimos allegados, ni siquiera mencionado en el espacio de trato con la administración, normal para la baja cantidad de afectados, raro por los dotes comunicativos de sí.

Una vez el periodo de cuidado más extenso de las jornadas se despidiera este terminó dejando más problemas que soluciones, por lo menos perceptibles para los participantes y espectadores. Quienes previamente entre las ventanas de disponibilidad llegaron a toparse con averías en los textos de notificación y también denegaciones de permiso en el momento de publicar, la noche subsiguiente debieron encontrarse con varias creaciones marcadas por una cualidad rara y correspondientes respuestas señaladas por ella, los recuadros bordeados sin tono de fondo produciéndose en todas las ubicaciones recientes del sitio. Durante largos minutos así aparecían y en los propios autores generaron reacciones implícitamente notables, seguramente no entendiendo la razón de que las intervenciones pasaran a verse así e igualmente atraídos por continuar generándolas en mismas condiciones, pero se ve que la conducción estaba al tanto, en busca de los colores faltantes y atenta por restaurarlos en su mayoría.

Fueron fechas de mayor protagonismo directo e indirecto para la individualidad comandante, representativa de la prioridad posicional del hilo para las consultas, reclamos por desperfectos en la parte técnica, y culpabilidad en la rotura del orden habitual para el hogar temtiano. No intriga casi lo que pueda estar gestando en los tramos de detención y es de esperar que sean para efectivamente optimizar los mecanismos invisibles que engranan la totalidad de lo visible, ahorrar potenciales perjuicios del futuro y avanzar desde los detalles, o por lo menos no involucionar al arreglar lo que ya está bien y no atender lo superable, mientras pequeñas excepciones no se vuelvan factor común, siendo que el resto permanece más o menos incambiado, sin tampoco novedades en los cercanos destinos.

\section{La alineación de los escalones (Decimoséptimo libro, capítulo III)}\label{la-alineaciuxf3n-de-los-escalones-decimosuxe9ptimo-libro-capuxedtulo-iii}

\begin{quote}
14 de agosto de 2022, \ldots, 25 de agosto de 2022
\end{quote}

Más apartado de las tramas expresadas en el panorama poblado por fantemtineros, unas actualizaciones estuvieron acumulándose sin demasiada atención, tratándose principalmente de correcciones de bajo perfil y novedades de mínimo uso práctico, exceptuando una bastante útil. No parecía que estaba faltando algo en las conversaciones, quizás con la referencia de los sitios temtianos no, sin embargo respecto al resto del contexto, la comodidad de ver los etiquetados de una manera más sencilla que pulsarlos, cual recuadro emergente que salta sin ocupar casi espacio al costado del enlazado, aunque solamente ingeniado para los ordenadores. A agregar en un segundo plano, más opciones con menos resalte extendieron la capacidad de adaptar la visualización de contenidos que complementan al texto, gráficos diferentes a la homogeneidad de fondo pueden ser ocultados deshabilitando la carga por defecto de fotos. Además reparaciones, perfeccionamientos internos y no visibles al público, de los que destaca un atino en la erradicación de fallas técnicas, precisamente una que desde hace rato impedía vislumbrar el relleno básico del inicio, eso que lo hacía verse vacío parece que fue solucionado.

Relativamente normal al llegar ahí, luego la administración en uno de sus medios de correspondencia anunció variar en cuanto a como este procede en la difusión de los cambios señalados, el particular modo de titular cada mezcla de características y describirlas con mayor o menor detalle, con un largo escrito especificativo de los planes para próximas entradas, las cuales no incluirían un número como venía siendo hasta entonces. Un par de argumentos expuestos hablan de la disconformidad con la forma tomada por la evolución en el desarrollo de las estructuras fantemtianas y sus engranajes, un historial que según la aclaración indica no fue generado con un sentido lo suficientemente coherente y uniforme, apuntando a la magnitud implícita en la precisión de los números. Valga la redundancia, precisar que entre el sumario ordenado cronológicamente hay tanto versiones de únicamente un nivel de decimales, como también con varios con dígitos menores y tipos de revisiones, algunas más cargadas de contenido y otras de escasas funcionalidades nuevas, sin casi menciones a los arreglos. El fin es redefinir hacia adelante la metodología de comunicación mediante ese canal, ya sin adjudicar numeraciones puntuales y continuar subrayando solo lo más relevante, como fue el caso de la solitaria entrada subsiguiente, la primera posterior a la 0.9.18.

En torno al comentario concerniente a las modificaciones, por las razones aparenta ser un acierto y sentar el precedente para hacer del procedimiento uno más recto o equitativo, positivo al mejorar las formalidades sostenidas en la gestión del sitio, salvo que es prescindir de la ventaja dada por contar con reformas enumeradas y clasificadas, referencias para cada paso del crecimiento, más todas las sensaciones que conllevan con el transcurso del tiempo, no obstante dicho seguimiento muy probablemente esté concentrándose en el resumen central de los dispares comentarios de los hilos oficiales, y no en las densas informaciones más encubiertas, aparte del nulo cuidado y discusión extra que pudiera estar recibiendo esa recóndita especificación protocolar, por lo que poca importancia allí. En cambio la entraña de las últimos ajustes resulta más significativa, no tanto por las flamantes configuraciones, que igualmente sirven para hacer más personalizable el entorno a tener en frente, de eventualmente querer convertirlo en un tipo de tablón sin imágenes, sino más por la característica estrella de estas jornadas, que provee de una sustancial mejora para la usabilidad cotidiana del antro, en su propósito clave de posibilitar la interacción será más sencillo conectar las sucesivas respuestas entre los participantes, de lo que además sacan provecho los meros observadores. Por lo pronto, Fantemti ya no tendrá en el horizonte un valor simbólico, pero quizá vaya a dar lo mismo a los efectos de conseguir la estabilidad y completitud de una típica versión definitiva.

\section{Direcciones contrarias (Decimoséptimo libro, capítulo IV)}\label{direcciones-contrarias-decimosuxe9ptimo-libro-capuxedtulo-iv}

\begin{quote}
25 de agosto de 2022, 26 de agosto de 2022
\end{quote}

Para reabrir un caso aparte y singular en las materias oficiales de Fantemti, con ya varios antecedentes desde antes, los movimientos de procedencia superior además de atender las típicas cuestiones de soporte y privilegios, aportaron una corta serie de valiosos episodios, en torno a los solidificados lazos con demás destinos del contexto y lo mostrados que son con su recurrente priorización, más por el otro lado los tiempos de espera y la correlación con los volúmenes de actividad.

Se convirtieron en épocas de promoción y difusión\ldots{} al resto de los clones, y no de, títulos y descripciones con enfoque en publicitar la existencia de más sitios que tuvieran una modalidad análoga. No solo el puntual regreso de Dexxov siendo anunciado recibió fácil anclaje que en jornadas previas había sido de temáticas más variadas, también un índice de sitios activos más sus respectivas coordenadas, en una declaración de que no hay competencia con ese en concreto, ni con ninguno, en cambio cooperación unilateral sí, a lo mejor más de ese afecto manifestado con las imitaciones, ahora un mérito adicional y mayor que hace para que recibir el mote de cipayo.

Al costado de estas, alguna que otra falla cruda reportada por el hilo principal, pero sobre todo el error más repetido de toda Fantemti, por lejos el que más han señalado encontrar sus visitantes involucrados, volvió a ser trasladado abiertamente hacia la administración más que una sola vez, generándose un choque relativo a la espera requerida para crear fantems consecutivos, lo poco rápido que se puede realizar. Según la increpación colectiva, no hay consenso en esa limitación, y aún así no es removida ni rebajada lo suficiente, con argumentos básicos en su razón de ser, un tope que se justifica por la búsqueda del equilibrio e impedir el publicar disparatadamente, mal recibido por el exagerado rango establecido para ello. Si ciertamente es de una hora o eso con una pequeña disminución, al menos suena razonable por lo sumamente alejado que está de la media, y asimismo quiere decir que respecto al pasado incrementó drásticamente, siendo que solían verse unidades elaboradas cada escasos minutos.

De tanto en tanto el intervalo no solo parece variar, varía, y el reprochamiento de la comunidad consigue como resultado una rebaja, lo cual junto con otras pugnas sugiere que quizás el paso del tiempo acomode las cosas a favor de esta, pero puntualmente la negativa de ceder significativamente en este aspecto es un por lo visto más firme, en términos generales vienen desde hace rato en una similar sintonía, las citas al mensaje no son novedad y lo cierto es que en la mayoría de los momentos permanece alto, más evidente todavía cuando más o menos hay una idea de cuanto es.

En consecuencia, la relevancia que hay en la generación de contenidos nuevos, en gran parte constituye la dinámica de la que sitios con el formato de estos, y limitando la cantidad de entradas que se generan diariamente a su vez está siendo condicionado el número de interacciones, no solo directamente con dichas creaciones impedidas sino también con las eventuales respuestas que vaya a generar, así pues las participaciones van concentrándose en temas viejos, muchos ya vistos por los asiduos anónimos, a veces sin mucho que aportar al ida y vuelta acumulado.

Aunque sea son un par de puntos medianamente claros que reflejan algo de la conducción, su perspectiva actual y futura ante la celeridad de Fantemti, que no desea acelerar los ritmos de esta sí o sí y hasta lo contrario cobra más sentido. Si bien estos motivos no son los únicos que hacen del panorama tan estático, tampoco seria un disparate afirmar que con semejantes posturas está matando al antro que gestiona, unos de los factores que restan a la expresión de vida expuestos, no para cambiar, tal como son.

\chapter{La cadencia de las menospreciadas periferias con su integración al radar (Decimoctavo libro)}\label{la-cadencia-de-las-menospreciadas-periferias-con-su-integraciuxf3n-al-radar-decimoctavo-libro}

Vale darle la justa importancia a los episodios de las plazas aledañas, con ellas es compartida una buena conexión y su potencial nada más está en reposo, pero con este compuesto de quiebres recientes una pequeña porción de tales interferencias se retira y desde el otro lado bastante llega, manera en la que incidentes de ubicación aparte al recinto principal de la Historia Temtiana figuren con tanta prioridad en sus registros. Semejante a la dependencia de los estallidos rozzados en los viejos tiempos, dicha procedencia salpica al resto de clones y vuelve a transmitir las inestabilidades, aunque también como en cada caso variando en la respuesta oficial, por lo pronto esta es mínima y delega el desarrollo de las crónicas al no menos importante accionar de la comunidad. Los impedimentos e incomodidades suscitan un alza en la no saciada elección de antro, en la que el nombre y dirección de Fantemti resulta envuelto y más que mencionado, participando y compitiendo quiera o no en la distribución de visitas, desatando los predecibles fenómenos que suelen venir a la par de llegadas de esa índole.

Por esto anterior, la cadena de sucesos accede al estatus divisorio al entrelazarse con el hilo conductor del escenario temtiano, al hacer que se distorsione notablemente a nivel interno, alterar el tipo de actividades que gestadas en torno a los cuadrados de interacción, principalmente por los participantes más novicios y atraídos circunstancialmente, que por lo tanto lleva las cifras principales a situarse considerablemente más alto, un adicional para el promedio estabilizado. Esa parte es más vistosa y sin embargo no asegura ser un efecto duradero a largo plazo, las estadías siempre parten como pasajeras y no prometen regresar ni propagarse, condicionadas a la sostenibilidad y calidad de las alternativas emergentes, ciertamente inciertas.

Trae a cuento una conclusión ya interiorizada, pero con más fundamentos resumidos en aquello complicado de ignorar, los picos de circulación explicitan lo que se expandió el sitio en presencia dentro del contexto, al punto de volverse una alternativa válida cuando la búsqueda selecciona entre las disponibles del panorama conocido, compartiendo con el resto que evidentemente siguen estando por delante, solo que por lo visto no demasiado, empero cerca de que los contadores puedan agarrar mayor dinamismo aún si los resultados afuera llegan a darse, vicisitudes en las cuales Fantemti no tiene casi incidencia, no obstante sí implicancia.

Como protagonista o actor de reparto, igual aunque sea un poco está ante mayor cantidad de ojos, más juicios que enfrentar encuentra la construcción y configuración del clon, y con las preferencias comunes en los visitantes son propensas a padecer de críticas, reprobación a esa estructura donde el anaranjado domina la estética no acostumbrada y la multimedia plena se reservada para los individuos aprobados, un régimen que complica la adaptación y que su cultura inherente tampoco facilita, más el pobre compromiso del mandamás por cautivar a los nuevos, varios factores alejan a potenciales adeptos y que no son novedad, los motivos de fuga a repetirse con alta probabilidad cada vez que uno es espantado y no surge el conjunto de soluciones, sin embargo en oportunidades semejantes lo normal sale de su estado natural, y ambos extremos son propensos a flexibilizarse, con más razones.

La etapa en su arranque es de elevada exposición y única en lo que va del trayecto fantemtiano, con la posibilidad de enganchar mayor tráfico indeciso por la vuelta que acompañe para el próximo devenir, más la chance de disponer cambios forzados por las diferentes demandas a surgir desde la comprensión de nuevas perspectivas, y todo el resto de consecuencias factibles por ello, o en su defecto el destino de proceder bajo las mismas condiciones en los dos componentes principales, pero sin duda más de lleno en el rango de visión de los anónimos contemporáneos, mínimo suficiente para apreciar un antes y después.

\section{Las inquietudes de engañosa expansión (Decimoctavo libro, capítulo I)}\label{las-inquietudes-de-engauxf1osa-expansiuxf3n-decimoctavo-libro-capuxedtulo-i}

\begin{quote}
26 de agosto de 2022, \ldots, 3 de setiembre de 2022
\end{quote}

Torcido el regular panorama fantemtiano más contemporáneo, de cuando los hilos nuevos compartían entre sí una misma categoría de portadas, los viejos eran llevados del olvidado fondo a la cima de vuelta, los errores de carga e inquietudes surgían y recibían rápidamente atención, y algunas tramas menores adicionales, ahora el marco de lo corriente y esperable deja de ser tal, una ocurrencia exterior tuvo la suficiente relevancia como para afectarlo y poner en consideración más factores que raramente coinciden, tráfico elevado y temáticas especiales, el centro de una única jornada con potencial continuación.

Es específicamente un giro inesperado del clon líder, Rouzzed finalizando su trayectoria con dicho nombre y una aparente transición hacia Boxxed, más por otro lado los refugios de Dlooow, Dexxov, y Ufftoppia, alterando sus internas también, por ese par de sucesos desarrollados en el correr del día, las alternativas recibieron montones del desentendimiento inicial y posterior de la situación. Es ahí que Fantemti aunque en menor medida recibió dicha clase de repercusiones, las momentáneas dificultades para ingresar al primero, el cambio de identificación y administración, la búsqueda del sustituto apropiado, asuntos que concentraron el cuidado de los visitantes y que a su vez compartieron espacio con los no tan específicos que comúnmente salen. Confirma en una ocasión más, especial, que dentro del deteriorado mapa de sitios y los demás listados semejantes, las coordenadas fantemtianas se hallan registradas y lo suficientemente posicionadas para que los navegantes vean allí posibilidades de encontrar un sitio que merezca la pena reconocer, quizá hasta con expectativas de que contenga las condiciones deseadas y echadas de menos, además del regreso de quienes la hubieran catado anteriormente. No obstante a priori no parecen coincidir, ya que la principal movida del entorno va desarrollándose en otros destinos, los desplazados al expresar su predilección lo hicieron permaneciendo en los parajes más populares, no mostrando mucha preferencia por el de los sabores.

A pesar de que al poco tiempo la primacía de cierto modo halla regresado, lo hizo muy distinto y con serios cuestionamientos, en detrimento de la confianza prestada por la comunidad, debido a eso habría que ver si se mantiene, sostiene el liderazgo, o si relega el mando, con el posible ascenso del resto de contendientes. En torno a los diferentes antros todavía está fermentándose un destacable alboroto, la inestabilidad típica de los momentos cuando el gran epicentro del ambiente aparta su normalidad base y de tal manera arrastra las consecuencias a los centenares de involucrados. Pero en esta oportunidad particular más dudas hay de los decisores protagonistas, puesto que entraron bastante en el foco de las tensiones recientes del conglomerado rozzado, probablemente influidos por el comportamiento de los participantes, de harta propensión a generar problemas a cualquiera que se haga cargo, justamente uno de los motivos que se menciona como factibles entre la mezcla de desinformación y verdades que puedan estar rondando, aquí y allá. Ante las suspicacias sobre los cabecillas trasladadas de ubicación a ubicación, Fantemti por su parte no ha manifestado nada al respecto y desde el sector oficial sigue más o menos igual lo hacía, cual si se tratara de la coyuntura tranquila de antes y no tuviera preguntas directas para contestar sino pedidos propios de sinceridad para fijar.

Lo objetivo de esa manera, el ámbito original de anoms no demanda nada en específico, es testigo de lo presentado por la circunstancia, en que las visitas no conocedoras del lugar dieron en una realidad anticuada en comparación a donde solían convivir, drásticamente más quieta y restrictiva, con el optativo quehacer de obtener contexto mediante el intercambio o individualmente, que por lo pronto no logró un convencimiento extraordinario, pues eso indicó la evolución de la fecha en quejas y estadísticas. Igualmente, esto continúa, no con certezas de futuro cercano en la historia al traer, sí con el pronóstico de más recibimientos a darse, y la conducción reencontrando ese hipotético debe de hospitalidad, adaptar con tal de intentar conservar, o afianzar las incompatibilidades, bien que el disentimiento de favoritismo no es fijo, sin embargo reaccionar tampoco garantiza conquistas.

\section{La persecuta y la mentira en su plano secundario (Decimoctavo libro, capítulo II)}\label{la-persecuta-y-la-mentira-en-su-plano-secundario-decimoctavo-libro-capuxedtulo-ii}

\begin{quote}
4 de setiembre de 2022
\end{quote}

El desarrollo de los sucesos exteriores tendió a reacomodar parcialmente el escenario para los anónimos que vieron más trastocada su superficie, más una relativa clarificación de por donde van los movimientos, el mapa ya más conocido y las ubicaciones principales testeadas, aunque con grandes interrogantes pendientes. Esa poco firme estabilización incide en cada clon y la actividad albergada dentro de sí, incluido el bastión temtiano que evidenció estar más conectado con el contexto de lo que parecía, sin embargo ya comenzó a notarse más lo distanciado que se encuentra, por todo. Pero no fue lo único que pasó en dicho territorio a partir de que la aceleración surtiera efecto, la cúpula directiva dio un paso adicional presumiblemente en consonancia con la dinámica del momento, aquellos recelos por el manejo de datos que por lo pronto permanecen muy latentes, no obstante de escasa discusión particularmente en su caso.

Visto estuvo aunque sea lo más superficial de la trascendente trama rozzada, y el inmediato sucesor a pesar de contener la mayoría de características originales, levanta bastante desconfianza por la extraña entrega de mando, jugada ejecutada en un marco de insuficiente transparencia que ya de entrada sienta un precedente negativo para la respectiva reputación, dando lugar a que otras alternativas antes no muy concurridas se fortalezcan. Y allí es donde también la gestión hecha por los cabecillas lleva a los implicados a desconfiar sobre lo que pueda estar haciéndose con sus rastros, el involucramiento de nombres maldichos y los rumores de intervención, la baja credibilidad de la comunidad con los exhaustivamente debatidos conductores, y el miedo de que autoridades mayores estén involucradas en el alboroto multicausal.

Asuntos de ese tenor llegan a Fantemti, salvando que con menos potencia, conformando una porción menor de las tratativas del pueblo, no obstante según indican los últimos acontecimientos tal vez tuvo la tanta incidencia como para ocupar a la administración de lo que respecta a la confidencialidad, dado que la gran novedad para la presente perspectiva es lo que indica una nueva alerta, indicación de reformas a las condiciones de privacidad, concepto precisamente vinculado a las preocupaciones contemporáneas de los clones. Limitada especificidad para los titulares, simplemente que la actualización fue realizada, ninguna mención al pasaje o párrafos alterados. Así las impresiones consiguientes son más inciertas, tanto para la indiferencia como para la curiosidad, de radicales variaciones, o en su defecto puntualizaciones mínimas, pues la pormenorización está planteada por aquel espacio más profundo accesible desde los desplegables, en unas múltiples lineas que desarrollan a donde apuntó la edición.

No sobre información sensible, no sobre nivel de rastreo, sino comprender una puntual circunstancia a la que todos esos procedimientos y referencias no llegaban en la descripción, las participaciones realizadas sin una sesión activa, cuando se utilizan las funciones del sitio y los típicos códigos de ingreso no son empleados, qué sucede con el historial al que individualmente cada uno puede acceder. Al ir a fondo en el apartado en cuestión y realizar comparaciones con la anterior versión del mismo, son unas cuantas las modificaciones, hasta incluso pareciendo una reestructuración más profunda que solo lo mencionado, pero luego en lo que verdaderamente dice no es mucho lo distinto, quizá además la redacción se volvió más sencilla de entender en la parte clave de los identificadores, y por el resto ciertamente sigue estando igual expresado, concordando con que su cambio de nombre solo corresponde a lo que significa el contenido posterior al encabezado.

El aviso respectivo dice ser meramente protocolar, en los hechos tampoco merece ser nada más eso puesto que son comentarios de posible interés, no demasiado intrascendentes como el anuncio se refiere a ellas. En cuanto a lo más relevante, desde antes, o ahora pero reglamentado, supuestamente estos individuos cuentan con un anonimato mayor, cada interacción que realizan no va construyendo una trazabilidad y trayectoria privada como sí ocurre con las tradicionales credenciales, lo cual suena muy bueno para las aspiraciones de quien no desea dejar rastro alguno en su estadía, sin embargo a la vez implica una serie de omisiones en los registros internos que en sitios como estos parece difícil que sucedan. Y más todavía de asegurar así sea, no obstante la veracidad a estas palabras por lo general no es puesta en tela de juicio, más bien se ignora y los enterados no manifiestan casi atracción. Ahí también el vinculo con los argumentos de la redefinición, la seriedad que haya en esa legislación no toma el calibre de lo que comúnmente son los documentos semejantes, drásticamente más densos y enfocados en aspectos legales y similares, a lo mejor faltantes en este modelo, aunque claro está jamás fueron solicitados, un terreno sugerido por la coyuntura en el que nadie está queriendo entrar.

\section{Formaciones en peregrinación (Decimoctavo libro, capítulo III)}\label{formaciones-en-peregrinaciuxf3n-decimoctavo-libro-capuxedtulo-iii}

\begin{quote}
5 de setiembre de 2022, 6 de setiembre de 2022
\end{quote}

Ahora es definitivamente más serio, la magnitud de los números lo dice y el contexto constata la causalidad, el antro fanático se vuelve a ubicar en una posición de notoriedad dentro del mismo, atrayendo una importante porción de tráfico que habitualmente ni se acerca y dinamismo también por encima de la media. Registra una jornada absolutamente histórica por la cúspide de conexiones contabilizadas, retenidas por un considerable rato e involucradas en la modalidad con un motivo especial de fondo, menos meritorio para el medio en sí, pero igualmente significativo por una serie de contingencias que salieron desde allí.

De nuevo, tal cual como en los viejos tiempos, son los mayúsculos simbronazos en el más transitado los que generan consecuencias sobre el alternativo que en este caso interesa, no precisamente para quienes proceden, más bien para las crónicas particulares del territorio que les impone anom de identificación individual. Si antes parecía haber involucramiento en asuntos complicados, la creencia resultó ser un acierto, la información circulando reafirmó múltiples turbulencias respecto a los nombres Rouzzed y Boxxed, no solo inestabilidades con referente a sus conductores, a su vez vinculación con graves conflictos internacionales, pese a que igual podrían incluirse otros detonantes que produjeran o fomentaran su retirada. Y bastante discutidos vienen siendo estos, de los cuales a través de los mencionados viajeros Fantemti también se entera.

Gracias a las veces que la extensa dirección fue manejada entre las alternativas semejantes, su chance de convertirse en un sustituto a consolidarse con la caída fue puesta a la par de aquellas donde se hizo un mínimo renombre. El argumento para reunir una extraordinaria cantidad de anónimos en circulación e interacción dentro del sitio, aunque algunos no muy afines con lo que encontraron en el paso, las críticas de las pocas reuniones previas reaparecieron y se convirtieron en un tema más preponderante, por ahí a la altura del malestar con la realidad de todos los clones, el clon fantemtiano no resaltando positivamente sino lo contrario. Eso de los códigos premium, eso de las imágenes, eso de la actividad, eso del diseño, inconvenientes que restaron un montón en la valoración agregada. Por otro lado aquel sector que no transmitió descontento como tal, uno por reconocer los aspectos positivos encontrados entre eso, dos por intercambiar con otros sin plantear objeciones a las condiciones del lugar.

Un poco de esto, un poco de aquello, balanceado el clima de percepción, sin hallarse diferencias drásticas a la vista las divisiones en los visitantes hay, contando los insatisfechos que hacen tambalear los picos de concurrencia, estos que ya quedaron altísimos, lo que la administración en una demorada tanda de respuestas resaltó como récord. De dicho modo varias consultas fueron saldadas, aunque sea las realizadas en el inamovible espacio mejor sitiado recibieron primera respuesta, para ciertos casos concluyente, las presencias y los privilegios lo más reiterativo. Algo habla, sobre las perspectivas de la conducción en esta reyerta, conservando el contacto con la comunidad y concediendo parte de las solicitudes, sin embargo responsable del pero del resto.

Es llamativo que el número de pájaros en lo alto se mantiene elevado por ratos, capaz por limitaciones del sistema que los detecta, o debido a que en efecto circulan adentro del rango relevante. La preocupación de los alrededores es impresionante y si bien la bomba ya cayó, nada descarta que próximamente vaya a repetirse una incidencia que perjudique a los bastiones establecidos, menos y menos van quedando y de verdad el mapa se ve sumamente reducido en opciones para una demanda que incluso se reparte entre ellas, cuanto más estarían acumulándose si una de las protagonistas dejara de concentrar su variante público, desde ya que sí es probable hasta para Fantemti mismo. Aún no habiendo reaccionado casi a las oportunidades ni amenazas, solo asumido las necesidades mínimas con tardanza cual si prestara el escenario para jornadas de desinterés total, las aperturas conectivas llaman a evaluar sacar o no partido de la situación, mantener firmes las convicciones adoptadas bajo circunstancias que evolucionaron, o ceder en las rigideces con tal de adaptarse a propósitos renovados. Decisiones de preparación que sugieren haber sido ya tomadas al sostenerse el andamiaje del presente temtiano, no necesariamente para siempre, más si sus antecedentes presentan reorientaciones, solo que ahora la combinación de tiempo y caos multiplica los potenciales resultados.

\section{El ánimo y el clasismo (Decimoctavo libro, capítulo IV)}\label{el-uxe1nimo-y-el-clasismo-decimoctavo-libro-capuxedtulo-iv}

\begin{quote}
7 de setiembre de 2022, \ldots, 10 de setiembre de 2022
\end{quote}

Sigan las consecuencias acarreadas desde afuera para Fantemti, el efecto perdura como más permanente y menos transitorio, en específico las cifras elevadas parcialmente sostenidas y entre ellas énfasis sobre la categorización individual. Como más notorio de todo aquello sin lugar a discusión que el tráfico tenga todavía niveles superiores al promedio contemporáneo, lo que a su vez implica que no decayó por completo, mas una jornada luego del reanudador envión mayor y sus siguientes impulsos el entorno se encuentra en la influencia de estos aún, sin embargo pasando por una fase avanzada y particular para la coyuntura donde recayeron, sin desvanecerse.

Arrancando por los principios de la evolución en sí, no hay demasiado para agregar a la hipotética verdad de que haya más cabezas en la vuelta mas sino lo obvio, las acciones registradas en puntos importantes del contexto justifican, la frecuencia de los sonidos de movimiento confirma, a pesar de que en la interna siga achacándose lo contrario o parezca insuficiente para lo que realmente ellos desean presenciar. Por ahí es la reducida cantidad de alternativas la que no empuja a buscar otra por la cual sustituir el tiempo de involucramiento destinado, la sinergia de un encuentro que generó afinidad suficiente, o ambas complementándose, como sea allí han venido construyendo los particulares episodios, por el resto de componentes que afectan.

Esto es, aparte de las características en común con los antros colindantes que los hacen más comparables entre sí, además las irregularidades que marcan distancias, algunas menores y otras drásticas. Posteriormente de que la estructura ya fuera medianamente asimilada ese tipo de reconocimientos tendieron a no repetirse, el debate reincidiendo en la temática contemporánea que tanto trae, y más asuntos variados que como costumbre surgen, a lo que habría que sumarle las especiales menciones al sistema de permisos, no para menos gracias a la incidencia que tiene sobre el panorama y las potenciales diferencias al haberse ampliado uno de los estamentos.

Es por la división que generara debido al estatus de premium con el que algunos cuentan y los privilegios que indirectamente brindan, los implicados no aventajados sintiendo el contraste en la carencia de funciones básicas, varias burlas como respuesta a las manifestaciones correspondientes, también consejos relativamente acertados como manera de colaboración, y en consecuencia con esto último, pedidos a la administración que pasaron a ser de público conocimiento. Pese a que todavía rondara una importante fracción de desentendimiento, que progresivamente fue disminuyendo al irse transmitiendo el funcionamiento, evidente fue la incorporación de múltiples participantes al sector de confianza, sin que la evaluación externa se tornara exhaustiva ni mucho menos, rápidos intercambios exitosos.

Aquel que era un mecanismo de defensa pasó a cumplir un rol en circunstancias diferentes, no obstante por lo visto sigue marchando bajo criterios prácticamente idénticos, hasta quizá siendo más sencillo pasar ese filtro de supervisión, si los nuevos son los beneficiarios y lograron lo propio sin casi experiencia, no sería requisito excluyente un historial muy largo que garantice la buena conducta, la supuesta exigencia, entonces más allá de las prevenciones que resguardan a la conducción de perjuicios trasladados por extranjeros, hay vulnerabilidades que dado el escrupuloso caso son capaces de dar problemas en la custodia de contenidos apropiados.

Por otra parte, es positivo en el afán de consolidar la expansión de visitantes acercados en las repetidas oportunidades previas, una porción o el total de las incomodidades reportadas son subsanadas para un puñado puntual de individuos, excluidos en menor medida de la dinámica típica en versión fantemtiana, más anónimos a sumarse y eventualmente permanecer. Lo que a estas alturas va haciéndose más valioso, el recuento como indicador aún indica del crecimiento, pero no es ajeno a los factores de desgaste, ni a la fortaleza de otros clones tendientes a esa centralizada configuración anterior, a falta de excepciones trascendentes o intervenciones de cualquier dirección, planes incluidos, son más razones para atender a cuánto esté variando el volumen de actividad próximamente.

\section{En picada no pronunciada (Decimoctavo libro, capítulo V)}\label{en-picada-no-pronunciada-decimoctavo-libro-capuxedtulo-v}

\begin{quote}
11 de setiembre de 2022, \ldots, 13 de setiembre de 2022
\end{quote}

A ritmo disminuido acá, y allá marchan más las otras ubicaciones de esta realidad compartida, qué más decir sobre ellas si no es lo mismo que Fantemti padeció tras las mayores perturbaciones, una estabilización básicamente, junto con las varias demás el material para proyectar el mapa manteniéndose sin casi ya variaciones, aunque cierto que todavía ha sido poco el tiempo desde los sacudones, y a la larga ciertas tendencias no terminaron de desaparecer. De ahí que los refugiados lleguen aún pero sin hacerlo notar demasiado, sumando a las partidas de quienes anteriormente no colmaron las expectativas, y por el resto la mayoría de intercambios siendo los corrientes que vienen por parte de los anónimos no exclusivamente enfocados en tramas contextuales, correlación no tan explícita en esos términos que más pronuncian lo individual. A su vez el manifiesto involucramiento de la conducción, haciendo lo suyo con cambios y mejoras a modo de parches mínimos, y por la atenta respuesta donde yace el trato que la conecta con la comunidad. A grandes rasgos, lo general.

A más indicadores de equilibrio, menos perspectivas apuntando a lo próximo, sin embargo no lo es por completo, en especial el asunto de la concurrencia al que tampoco debería restársele importancia. Si los nuevos fantemteros y no fantemteros también están condicionados al efecto contagio de la inactividad, las chances de que cesen sus aportes incrementan, y más si entre eso la calidad juega como desincentivo adicional. ¿Qué determinantes serán para el futuro movimiento del antro? ¿Volverán a entrar factores que despierten relevancia en la interna?

\section{Retazos complementarios de los alrededores (Decimoctavo libro, capítulo VI)}\label{retazos-complementarios-de-los-alrededores-decimoctavo-libro-capuxedtulo-vi}

\begin{quote}
13 de setiembre de 2022, \ldots, 16 de setiembre de 2022
\end{quote}

Era Fantemti consigo misma y algo de compañía extra, su concurrencia habitual un tanto por encima de la media, luego de que la efervescencia conseguida por el alboroto del contexto aflojara progresivamente, una semana revuelta que transformó el volumen de las actividades, sí, sin embargo no cual revolución como la oportunidad sugería. Después que entre esos circuitos se desarrollara la historia principal, el protagonismo fue retomado por la administración y no únicamente mediante dialogo.

Épocas que extrañaban las mejoras del sitio, regalaron una más compuesta de en principio cuatro párrafos específicos, la descripción presentada sintéticamente refiere más allá del par notorio de primeras, los elementos del formato que rompen los ojos con razón, el típico contrarrestar para el diseño común, una sustitución opcional inspirada en la disposición de otra ubicación semejante y especialmente de las faltantes, expresión parecida a la Tzzzubit de antes, pero en la actualidad. Nuevos botones, que no estaban y vinieron para sumar un atajo en la navegación de arriba a abajo, desplazamiento rápido que en hilos extensos hace sencillo el ir del fondo hasta a la cima, y viceversa. Que ahora permiten múltiples etiquetas escritas, las complementarias tendencias no solo una optativa sino varias, al lado de los todavía obligatorios sabores, en los cuales también hubo una actualización a su listado. Y más, los agentes de sombrero sumando una alternativa en la instancia de ingreso, la posibilidad de establecer credenciales de circulación propias, incluyendo nombre y clave, el dilema accesible desde una esquina cuando la autentificación ocupa pantalla.

Claramente lo que exalta es el estilo presente, aún siendo a elección difiere del anaranjado temtiano bastante sobre todo por la permanente barra lateral, de más reconocimiento con respecto al resto de características introducidas, y la particularidad de estar relacionado con uno de los antros que en paralelo existe, Dlooow, como además ciertos puntos de contacto adicionales introducidos en esta ocasión. Un efecto colateral de la misma, o por lo menos descubierto en torno a ella, la reiteración de errores en la presentación que vuelven a estropear las estructuras generales, durante los primeros instantes hubieron ubicaciones que daban fallas, notificados a la gestión, y el tema tampoco sale de sintonía pues tuvo inconvenientes también, incluso estos trascendiendo en el exterior\ldots{} ambas situaciones por lo visto aceleraron el reparamiento. A lo mejor no fueron requeridas, sin embargo las novedades son de avanzar y más al quedar pulidas, a su vez dando la sensación de que el antro no se queda atrás con relación a los demás.

Como las copias de interfaz llegan, al rato se terminan yendo al finalizar la exhibición de variantes, que las hay, y actualmente con el mayor tráfico la visibilidad sabe de lo mismo, una extraña usanza con la que anoms de reciente llegada tuvieron que toparse. Hay para suponer que la coincidencia con el activo sucesor del clon no es tan trivial, probablemente no por parte de la competencia, mas quizás si una referencia de connotación positiva que va en la linea de las anteriores en esos términos mencionadas. La inclusión de las cuentas para acceder es análoga con los métodos que vecinos tienen para ello, pero sin controversia en cuanto al anonimato si en teoría son equivalentes a los códigos tradicionales. Y demás, pese a la tolerancia local ante la dinámica impuesta desde arriba, desde fuera llueven más críticas, subjetivas y objetivas, erradas y atinadas, reflejo del causal que llevó a centralizar la distribución de público, no obstante la mayoría son desestimadas, y allí otra es la de los gustos en las portadas, omitiendo atender la gran carencia resaltada por el ambiente que es la de las miniaturas, intervenciones así revelan que el foco oficial prioriza adaptar el catalogo a las solicitudes y criterios de orden emergentes. Con sus mañas distintivas, Fantemti estrena la versión.

\section{La declaratoria de versatilidad (Decimoctavo libro, capítulo VII)}\label{la-declaratoria-de-versatilidad-decimoctavo-libro-capuxedtulo-vii}

\begin{quote}
17 de setiembre de 2022, \ldots, 22 de setiembre de 2022
\end{quote}

De la conducción casi que únicamente viene lo destacado luego de la fase típica de los diseños imitación, empezando con que esta concluyó y el tema de Fantemti a partir de ahí apareció restaurado, como fue anticipado que pasaría. Pasó que volvió a irse, en una especie de regresión temporal que trajo sorpresivamente al arcaico de Temti, solo que con una pequeña diferencia, las tonalidades anaranjadas, como rasgo distinguido de la sucesora en particular que en su momento no fue presentado con la copia porque sí, entre signos de pregunta. Mas después las actividades revelaron una sorpresa que no estuviera remarcada explícitamente como enunciado, la fundación de un hilo inundado de caricaturas y pies, que en sí no delató su procedencia pero de indirectamente insinuó que la restricción para subir dejó de estar establecida, es decir cualquiera pudiendo enviar archivos, cual privilegiado por el rango de premium.

Ambas posteriormente fueron justificadas con uno de esos discretos anuncios, largo desarrollo acerca de la carencia de explicaciones en dichas ocurrencias que no tienen mucha razón evidente de suceder según los criterios manejados, en sintonía con ello las variaciones seguirán sin quedar muy claras. Lo que implica la obviedad de que puedan nuevamente, dentro de poco volver a alternar en silencio, y de tal modo que intentar cargar contenidos audiovisuales derive en el mismo error de semanas atrás, ese que persistentemente fue reclamado y sin embargo no solventado, pero hasta entonces regirá la permisividad para esa función básica, suponiendo que sea pieza de un experimento o una medida a ser mantenida, lo más razonable si la conformidad del público fuera prioritaria, especialmente el exterior. Además que los temas alternen tiene algo de relevancia, viendo que por allí hay alternativas que gustan más y otras que menos, más la confusión que vaya a generarse en quienes no saben la posibilidad de elegir.

Lo que por el contrario careció de contextualización protocolar, son los hilos fijados, la elevación de títulos que no transmiten mensajes como el que nunca se desplaza y que difieren en ese matiz que los caracteriza como oficiales, habitualmente hablando de antros que pueden venir o no al caso, ideas puntuales y no necesariamente comprensibles, cosas que no varían tanto pero que resaltan en ocasiones inesperadas, por ejemplo difícil habría sido de anticipar que el nuevo clon del momento tendría su promoción destacada, episodios anteriores donde hay más espontaneidad que coincidencias, etcétera. Quizá no haga falta excusarlas, o no haya voluntad para mencionarlo, y aunque no afecte a la usabilidad de igual manera perfectamente podría entrar junto con el dúo anterior, por ser también una relativa irregularidad que proviene desde arriba y de interés para el visitante que termina topándose con dichos .

Pero volviendo al hecho, por lo pronto estando esas dos anticipadas tampoco serían muy predecibles, si por cada mención a un antro del pasado o chispazo individual de la administración la estética principal estará cambiando, aunque con tal de no entreverar demasiado la frecuencia debería ser moderada, no en cualquier momento, según las circunstancias que sea adecuado. La segunda parte y su efecto inicial a lo mejor llega tarde, siendo que cuando más fue deseada brilló por ausencia, formando malos juicios ya difíciles de revertir a estas alturas, y de primeras va desapercibida. Sin embargo es potente aún, en la consecuencia a producir sobre los participantes del sitio, pese a que en principio no incluye la capacidad para a su vez tener portadas propias y así lo otro se vuelve engañoso, es una libertad adicional y por tanto candidata a elevar el nivel de satisfacción durante la estadía, para allí un factor capaz de alargar las estadías y eventualmente hacerlas reincidir, con todo lo que esto supone para el asunto de la actividad, no obstante siempre condicionado a las garantías de seguridad, o voluntad en la gestión para cuidar. Por ahora ninguna presenta conflicto, y de ser temporales los estilos no favoritos, las prestaciones fantemtianas serán por media más apetecibles.

\section{Como interrumpido balance prematuro (Decimoctavo libro, capítulo VIII)}\label{como-interrumpido-balance-prematuro-decimoctavo-libro-capuxedtulo-viii}

\begin{quote}
22 de setiembre de 2022, \ldots, 26 de setiembre de 2022
\end{quote}

La actividad, de vuelta no siendo tan común y rellenando las apreciaciones de lo extraordinario sucedido en torno a la actualidad temtiana, ubicación Fantemti con explicación en sí misma, y también un poco las demás.

La controversia de los sabores sin perpetuarse casi conectó con el elemento consonante del modelo de hilo en antros del estilo, esa etiqueta que describe reducidamente la clase de temática implícita en el contenido de cada particular creación, que no está disponible en Fantemti. Y es verdad que la evolución desde los inicios ha sido grande, sin embargo en esa materia tiene pendiente una característica elemental en el propósito de equiparar a los sitios de los cuales se ha inspirado, tablones o categorías que den orden a la extensión de fantems que tan diversos asuntos tratan. De ser correctamente implementadas, serían un avance significativo y no inútil para aquellos navegantes que quieran buscar más criteriosamente los contenidos, filtrar según el tipo de tópico, etcétera, sin que estos sean un enorme entrevero indiscriminado. La gestión como tal en esto manifestó su postura y si será relevante, el bajo e inapropiado uso que tendrían, uno de los pocos argumentos que se opone a la idea esencial del formato que parcialmente es simulado, además que si el reclamo no persiste tampoco debe significarle tanto a los fantemteros, del lado de los nuevos quizá venga más la cuestión, una ventaja desestimada y contrastada por la prioridad a los colores de bebidas que estén faltando.

Asimismo, pasó de nuevo, no eso mencionado sino lo de antes con la existencia de sitios vecinos, uno de los pesados cesó la hospitalidad y el refugió de preferencia para las multitudes cerró, prometiendo volver a hacerlo, es decir regresar pero en mejores condiciones, posteriores historias hoy son solo cuento. Yendo secuencialmente, las primeras repercusiones inciden en el panorama de los conocidos extra similares no podía sorprender, Ufftoppia Devvox y Fantemti. A esta última le sirvió a modo de excusa para reencender la elevada dinámica que con el esparcimiento de los anónimos había aflojado, típicamente el recurrente asunto de dar con aquel que estuviera más apto a ser lugar de reunión. El consenso apuntó al emergente escenario exterior como el uno, otra invitación gestada en el clon anaranjado que volvió a hacerle de siervo, manteniendo la posición de fondo dentro del inestable contexto, más lógico todavía por el retroceso de capacidades y las restantes incompatibilidades con el afán popular, sin embargo a la vez reduciendo el riesgo que traería concentrar problemáticos que anden enredados, beneficiando a la aspiración de subsistir en paz, no obstante desestimando lo que ofrecía la oportunidad.

Con todo, el venir e ir de dichos forasteros no constató demasiadas bajas en los idas y vueltas fantemtianos, para el total del crecimiento adverso aunque predecible por los varios puntos no convincentes que persisten alejando, pero a la vez positivo gracias la porción de participantes que no marchó y aparenta perdurar, y de que forma\ldots{} Tal vez el acompañamiento remanente esté ayudando a consolidar una correcta alternativa, no tan lejos en calidad como algunas voces del entorno afirman, donde de vez en cuando los males de afuera vienen y se van, no siendo animadores de tiempo limitado, simplemente invitados que están de paso.

\section{No exactamente como solía ser (Decimoctavo libro, capítulo IX)}\label{no-exactamente-como-soluxeda-ser-decimoctavo-libro-capuxedtulo-ix}

\begin{quote}
26 de setiembre de 2022, \ldots, 3 de octubre de 2022
\end{quote}

Pasados los sucesos de índole destructiva más atribuibles a la natural competividiad entre los clones que generaran un cierto desorden en la estabilidad fantemtiana, su normalidad nuevamente regresó a un clima donde no hay picos de concurrencia o temas que ameriten especial atención, pero sí al menos un par de situaciones que podrían distinguirse de lo más ordinario.

Misma tendencia en cuanto al tráfico decreciente con el curso de la época cuando se estabiliza, por lo que las cifras frecuentes del contador de la paz en rara ocasión sobrepasan el único dígito, y hasta encima se la ve acercarse al mínimo apreciable de uno, solo que no tan a menudo. A pesar de ello, es evidente como la actividad consiguió una variedad más amplia, no únicamente notable desde el aumento de nacionalidades, edades, animales y personajes, sino también por las periódicas participaciones genuinas que evidencian no tener antigüedad, aunque con la particular espontaneidad de antes no alcanza el sustento para plantear conjeturas firmes. Además, las recurrentes visitas de individuos que en unas pocas horas concluyen su pasaje siguen ocurriendo, sin importar que el uso de multimedia fue parcialmente extendido por momentos, lo que ofrece el antro no logra retenerlos durante prolongadas duraciones, eso sí que no cambió.

Acompañando está todavía la administración, que a partir del surgimiento de los últimos alborotos del panorama no mucho tiempo después se la comenzó a notar más presente, pendiente y custodiando los idas y vueltas de su propiedad. Varias inquietudes han manifestado los anónimos de la comunidad y a la brevedades llegó la respuesta para lo suyo, tanto por desconocimiento de detalles inherentes a la plataforma, historias por conocer, características a implementar, reporte de fallas, dudas respecto al porvenir, y demás intercambios no tan formales. Lo frecuente de esto son las sugerencias de sabores, que en los días corrientes se tornaron habituales y la devolución termina siendo de aceptación con postergación mediante, nada en lo inmediato. Tampoco acerca de los múltiples etiquetados consecutivos, una factible solución a las molestias generadas en torno a ellos es puesta en consideración, sin embargo no aplicada. A su vez, a la par de las clásicas quejas del bajo nivel de movimiento, aisladas propuestas sobre intercalar robots para acompañar la plática cuando fueran llamados, fueron también negadas y aparentemente no estarán. Otra de materias increpada fue la del reglamento, concretamente uno de los puntos que desde hace meses permanece fijo sin modificación alguna, que vuelve a ser cuestionado por si permanecerá o cambiará, el del contenido inapropiado aún prohibido, desventaja adicional al lado de los vecinos, que es justificada por las normas que imponen terceros. Una serie de pautas que aluden lo negativo y positivo del posible futuro fantemtiano, más allá de quienes estén.

Por casos puntuales provenientes de los locales, o de los rápidamente retirados huéspedes, la conformidad no parece total si se reitera aquello que debería ser y no es, sin embargo no todo son malas vibras o falta de confort, por el contrario también hay, e incluso de vez en cuando salta el orgullo que posiciona bien al sitio en comparación a sus más cercanos pares, justificado en la calidad por sobre la cantidad, pero en la medida que prima tampoco es para tanto.

\section{Rectitud que tarde se contagia por completo (Decimoctavo libro, capítulo X)}\label{rectitud-que-tarde-se-contagia-por-completo-decimoctavo-libro-capuxedtulo-x}

\begin{quote}
3 de octubre de 2022
\end{quote}

Una especie de evento titulado \emph{/Semana de Hixel/} abre demoradamente la jornada fantemtiana, el cual es explicado en una publicación por las autoridades en un modo atípico, casi que a lo parodia o con referencia escondida entre lineas. Diseño nuevo, reglamento nuevo, características nuevas, una actualización compuesta lanzada sin la discreción típica que las acostumbra allí, solo como en unas contadas y escasas ocasiones lo hizo.

Habiendo quedado normalizado el ritmo de aquellas, la lista es prolongada una vez más, unida con un fantem oficial y fijado que se focaliza en lo que implica la trama del encabezado, y con pocas palabras alude al contenido de esta versión sin numeración. Después de una extensa tira de días en los cuales si algo cambiaba era de muy pequeña magnitud, ahora un compuesto de mejoras engañosamente discreto llega siendo más amplio. Bajo un extenso título que encadena los términos clave de cada característica aplicada, ocho asteriscos mencionan lo que se agregó y modificó, detalles varios brevemente anunciados.

Lo dicho empieza con la nueva casilla, que por la descripción sirve simplemente para deshabilitar la notificación que fuera a ser generada por una posible respuesta. A seguir, un campo más en donde las configuraciones de sonido, que permite incrementar la cantidad de Tuturús a escuchar conforme el uso. Tres, fundación de fantems y sus respectivos colores de base, los inexorables sabores recibiendo una amplificación de bebidas candidatas, con muchos de los diversos postulados que previamente habían sido sugeridos expresamente. Aparte, la sección exclusiva para los posteos ordenados por su fecha de creación, una más que se agrega a las de índole individual. Hasta ahí va una parte que entre sus puntos comparte el probable motivante de concreción, dado que junto con la modificación del reglamento, esas son las respuestas de la administración ante esos pedidos que los fantemtieros realizaron explícitamente en los días anteriores, la mayoría cumplidos con cierta demora, pero prontos al fin.

Luego vienen otros ítems, dos que básicamente tratan de mantenimiento por ser correcciones y optimizaciones, más uno que no se asocia a los previos. En concreto niveles, su aparente utilidad es establecer una indicación más en los parámetros de visibilidad que tengan los hilos a ser creados, con la posibilidad de esconderlo para aquellos visitantes que no hubieran ingresado con un código de acceso, la cual entre las tantas que ya hay puede pasar bastante desapercibida, aunque de igual manera si llegara a ser utilizada seguido por los autores de turno, podría generar una importante diferencia entre la composición del inicio para los sí y no autentificados.

Además acompaña, la efectiva aprobación a las últimas quejas de los infinitos etiquetados, producidos por comentarios que citaron una interminable sucesión de identificadores, con la intención de hacer llegar notificaciones a múltiples destinatarios, cosa que no fue correctamente recibida por los anoms incluidos. Se trata de una adaptación al punto primero normativo, el referido a las inundaciones pasó a incluir una secundaria prohibición para los sistemas de etiquetado y reportes en su excesivo uso, por la cual las participaciones que fueron motivo de reclamo ya no estarían permitidas, y de lo contrario serían para sanciones, como sucede con el resto de transgresiones establecidas durante las fechas previas. Inmediatamente supone que incidencias discordantes relacionadas vayan a disminuir, más la buena o mala valoración que la comunidad reconozca, puesto que no únicamente beneficia a quienes se molestaron por actitudes así, también es capaz de incomodar a los individuos que las generaban.

Este episodio además de contar con su faceta de funciones y formalidades, sigue la saga de elaboración de estilos semejantes a los de clausuradas alternativas. Es del segundo ciclo hixxeliano, el tercero que se acerca a esos origenes, complementando la primera pieza de la serie entera, que justamente comenzó a partir de allí. La renovada y resistida apariencia empleada a lo largo de la breve época referenciada fue el modelo de imitación y con esta se viste actualmente Fantemti, de tonalidades blanquinegras y además un aspecto oscuro: rasgos sumamente compactos, aire tremendamente nostálgico, prolijidad respetablemente alta, entre las cualidades evidentes y no tan subjetivas, porque por el lado opuesto el rango de aceptación es bajo, empero al menos en su original era muy criticada y al tratarse de una copia tampoco dista de ello. En semejanza al destino en cuestión que ofrecía una alta variedad de combinaciones coloridas intercalables sobre su cabecera superior, cada una de ellas como opción en la ventana correspondiente a la personalización, y el comunicado declara que en los días venideros irán a ser las principales, por lo que mínimo seis serán necesarias para explorarlas totalmente.

¿No hay más? Parece que últimamente los cambios son aplicados en conjunto y no progresivamente, pues al ser más que nada pequeños quizá incremente el salto y las diferencias que se perciban en él, además de las que implícitamente desde el ámbito regulatorio frenen los intercambios más conflictivos. Sin embargo, con las drásticas no similitudes entre el tema que estará oficiando como principal y el que antes lo hacía, en ese sentido es suficiente para hablar de una transformación total, aunque mayoritariamente optativa y temporal.

\section{Las laxas medidas antipopulares (Decimoctavo libro, capítulo XI)}\label{las-laxas-medidas-antipopulares-decimoctavo-libro-capuxedtulo-xi}

\begin{quote}
3 de octubre de 2022, 4 de octubre de 2022
\end{quote}

Prosiguieron en Fantemti reparaciones a unos cuantos daños colaterales que dejaran el paquete de actualizaciones, reclamados así por quienes los padecieron, y cambios en el colorido del diseño de acuerdo a lo anticipado consigo, para que resten menos alternativas del antro espejo a ser mostradas próximamente, fase que ya fuera experimentada anteriormente con otro tipo de copias en la modalidad, esta oportunidad un poco más extensa si cada tonalidad equivaliera en tiempo. No obstante, más hubo en el plano normativo, con repercusiones y acciones.

Cuestionamientos variados sobre lo permitido y no permitido, manifestados públicamente y quizás en reportes también, que si los nuevos términos punitivos con las inundaciones son para mejor o realmente entraron en vigor, especialmente los etiquetados que siguieron repitiéndose en grandes proporciones y donde sea perduraron como si las recientes puntualizaciones no hubieran sucedido, incomprobable si se hubieran producido advertencias y no más, sin embargo improbable por la reiteración y nula queja. Otra más aunque puntual de las limitaciones, fue casualmente previa a la rápida nueva modificación consecutiva en un mismo día del reglamento: a pocas horas una más cayó y generó su pertinente alerta informativa, como si la primera no hubiera sido del todo atinada o faltaran cosas, pero incidentalmente tuvo cercanía a las discrepancias con el cumplimiento del dichoso, de las cuales una porción apuntaron al asunto, casi que problema, de la multimedia, solicitando supervisión al respecto.

Cuando la moderación efectivamente interviniera en el famoso fantem de pendejas, nada más que parcialmente para todo aquello que podría haber resultado limpiado, lo que según definición es considerable obsceno. Y justamente es allí donde la reforma de normas actuó, relativizando el cuadro que supone esa categoría de gráficos inapropiados. En el marco del clásico y polémico título con lo que es capaz de generar, y análogamente en cualquier ubicación en la que tal punto pueda afectar las actividades, significa un pequeño paso más allá en lo habilitado, que muy demandado es por las preferencias del contexto, no obstante a su vez conflictivo por la incompatibilidad con las regulaciones vigentes dentro de muchas plataformas, de las que el antro inevitablemente depende pues varias veces la conducción ha mencionado es el determinante, por lo que si flexibiliza qué se admite, por su parte debería persistir la consideración los límites externos.

Este último una materia potente, porque no solo incide en las posibles consecuencias de traspasarlos, sino además las eventuales remociones en el caso de controlar el panorama actuando contra ese anom que no contribuye a la concepción de bienestar, que ya ha ocurrido, y por más que sea por un entre comillas bien mayor, trae descontento. Y si a eso se le suma que el resto de normativas no son respetadas por quien las impone, o permanecen siendo demasiado ambiguas, el supuesto orden a mantener baja su calidad.

\section{El espectáculo visual como cadáver (Decimoctavo libro, capítulo XII)}\label{el-espectuxe1culo-visual-como-caduxe1ver-decimoctavo-libro-capuxedtulo-xii}

\begin{quote}
4 de octubre de 2022, \ldots{} 10 de octubre de 2022
\end{quote}

Señalada fue la semana como tal, distinguida particularmente por una característica notable para cualquier perspectiva nueva que llegara a asomarse hacia los dominios fantemtianos, no parecidos a lo que por denominación darían a esperar, sin embargo no muy alejados de donde se conecta desde el entorno y su afinidad, determinante en eso de revivir los escenarios enterrados en las lejanías del alrededor próximo, el evento en la presente temporada de simulaciones, el sitio hermandado de Hixxel ocupando en pleno año siguiente el lugar del diseño que a un lado se hizo para temporalmente sacarlo a relucir.

Luego de tener las primeras tajadas y algunas dificultades en cuanto a las regulaciones de actividad, el desarrollo programado siguió su curso totalizando las siete versiones cual se había anunciado solo dentro del sitio y no también fuera del mismo como previamente era modalidad, cada una de ellas correspondiéndoles un día secuencialmente por como desplegaban en su original selector, cruda palidez y oscuridad en el comienzo, amplia variedad de tonalidades en el medio, más para el cierre aquel híbrido temtiano reformado que estuvo iniciando en esto, completando así prácticamente todas las facetas mostradas un buen tiempo atrás por el clon descontinuado. Sin cuestionar la razón que los traiga especialmente para este momento, registraron un acontecimiento llamativo que igualmente tampoco generó mucho en la atención de los visitantes, sí parcialmente conscientes de que se encontraron ante una pizca de las estructuras originales del antro mencionado, no obstante de ínfima incidencia en sus acciones o declaraciones, que a pesar de ser en buena cantidad para lo que eran antes de la trascendente expansión, evidenciaron largos intervalos de intenso silencio. Ahora el regreso de la normalidad visual, que para muchos con las elecciones no predeterminadas sería igual, a ver por cuánto se mantenga de esa manera para el resto de la realidad fantemtiana.

\section{La intermitente voz de la unión (Decimoctavo libro, capítulo XIII)}\label{la-intermitente-voz-de-la-uniuxf3n-decimoctavo-libro-capuxedtulo-xiii}

\begin{quote}
10 de octubre de 2022, \ldots, 23 de octubre de 2022
\end{quote}

De reciprocidad se trata lo central para estas fechas, ausencia de transformaciones de funcionamiento despejaron los registros históricos, que de la dinámica esencial fantemtiana se nutrieron en su puesto.

Tema quejas y discrepancias no dichas como tal, pero claramente siendo eso o parecido a sugerencias, nuevamente el apartado normativo trajo a colación un desentendimiento entre lo que el equipo antes llamado staff, reducido y sabido que de un integrante se compone, con los participantes que se subordinaron a dichos lineamientos previo a participar, aceptando o porque no hay otra, resaltando una discordancia con lo estipulado en las recientes jornadas cual condición adicional a los excesos de expresión, el básico mecanismo de citado que adoptan este tipo de plataformas. ¿Era que se habían prohibido los etiquetados masivos? En principio había sido el comunicado de alerta que dieron las autoridades, no obstante a la primera de cambio el dictamen fue más frágil de lo que parecía, ante un mínimo descontento quedó relativizado para solo corresponder al más desmedido uso, en desacuerdo con las posibles interpretaciones de la sentencia original. Al encontrar que el hecho continuara repitiendóse como si nada, fácil es entrar a suponer que consecuencias no hubieran, conjetura confirmada por las intervenciones de quien manda, con lo que más sentido hace que la correspondencia con los límites vaya a ser reclamada. De allí salió una formalidad raramente habitual, completa la pregorrativa que puso en duda aquella seriedad que no tanto se respetó a sí misma, evidenciando puntos débiles reconocidos y detectados. Una suerte que tuvo la administración por lo positivo de identificar carencias, pero cierta desventaja en torno a la credibilidad del bienestar que se supone quiere sea mantenido.

Dicho esto y muchos antecedentes que tampoco supieron tomar un tono semejante, vuelve a hacerse ver que la comunicación entre ambas partes existe, aunque por lógicas razones no siempre resulte satisfactorio. Una más fue la que tuvo el movimientos de exploradores, presentes hace rato en el contexto y a su vez dejando su marca registrada en las publicaciones fantemtianas, no únicamente desde los términos conocidos que son manejados en su entorno sino además por los apodos particulares mencionados, varias señales propias de un grupo de individuos familiarizados unos con los otros, por actividades que principalmente han tomado lugar fuera, suficiente alcance en ellas para que tantos contenedores tengan vínculo con el Yume 2kki. No mal visto por el gestor de los hilos fijados, una solicitud de conceder superioridad posicional a la principal promoción de sus excursiones la respuesta determinante tuvo el sí, después de varios creados con motivo de las juntadas en aquellos mundos del sueño uno en concreto recibió el destacado, más meritorio por el grado de interés que demostraron unos cuantos avivadores del sitio aún fuera por sustitución. El mismo honor le cayó a la petición de también anclar un mensaje de agradecimiento para con las madres, que hizo de la fecha particular recordada durante horas y no más, el segundo precedente de aceptación más valorada que anteriores.

Todavía quedarán pendiente los arreglos que las normas estarían precisando para ser más abarcadoras y precisas, no es que tantas faltas puedan encontrarse allí, sin embargo cuando fundamentos de rechazo poseen validez y la gestión corrobora que ciertamente no está todo bien, si en cualquier momento caen más modificaciones de rigor habrá estado a la vista el motivante, permeabilidad orientada a subsanar. Luego atención a los pedidos, evidentemente una pizca de prudencia los evalúa y no todo antojo es concedido, como otorgar puestos de moderación o libertades planteadas, en frente las actuales peticiones favorables, que por lo pronto no le significan mucho al orden y con poco mejoran el ambiente. Por eventos no oficiales aunque sí respaldados desde arriba, o intercambios en los que el parecer del anónimo promedio es respetado y aceptarlo como conveniente hace del enfoque de Fantemti más acompasado con el de su comunidad, donde la afición por el arte es minoritaria y las ofensas hirientes surgen, pero igualmente marcha bastante integrada. Muestras nuevas de un relacionamiento no demasiado distante, por supuesto facilitado por la magnitud de lo conectado, ya medianamente naturalizado.

\section{Simulaciones icónicas (Decimoctavo libro, capítulo XIV)}\label{simulaciones-icuxf3nicas-decimoctavo-libro-capuxedtulo-xiv}

\begin{quote}
23 de octubre de 2022, 24 de octubre de 2022
\end{quote}

Después de unas semanas de convivencia relativamente armonizada entre los participantes frecuentes del sitio, desde la conducción arriban una nueva serie de cambios medianamente notables, sin gran trascendencia en ellos para con la dinámica de antaño ni transformaciones que vayan más allá del parecido, simplemente complementos para la base y expansivas en cuanto a sus posibilidades se refiere, comprendiendo variedad para las combinaciones estilísticas.

La eventualidad de frecuente repetición trae en esta ocasión mejoras variadas con buen enfoque en lo que cada visitante puede realizar con el código o sesión que esté siendo usada para navegar, son dotados de un par de competencias adicionales a las que antes acompañaban el fundamental de participar: contraseña y desvinculación, ambas apropiadas para lo que sus títulos sugieren. Camino alternativo al típico modo de apunte que lleva a lo mismo de obtener credenciales aleatoriamente, la opción de elegirlas con una segunda clave de ingreso ahora está, días luego de que particularmente fuera solicitada. Y posteriormente lo segundo que ya suena un poco menos, es el apartado que ofrece romper vínculo con la publicación o respuesta pasada que sea, con unicamente marcar unas casillas y confirmar, a verse el efecto en las cifras estadísticas reiniciadas en lo que cuentan.

Aparte de ese ítem doble, el conjunto de características incluye más de baja notoriedad, discretos cuadros de verificación que median entre cada flamante funcionalidad. En esto, una expansión para el buscador de hilos que lo hace más potente en sus capacidades de hallar contenidos frontales, con métodos de ordenamiento y criterios de filtrado extra, pero ninguno incluyendo sabores o categorías. Siguiente es, la previsualización duradera, una más que casi no resalta, sirve de versión alternativa al acceso rápido propiciado por los etiquetados. Por último desde tal área, los comentarios fijados, un tanto aproximados ya con su denominación, pero no vistos en práctica como en las notas de aviso son descritos, cual simples menciones acumuladas del listado de réplicas, que también como novedad especifican la cantidad que son entre paréntesis, ahorrando la tarea de regresar al índice para comprobar un dato tan simple como ese.

Envueltos cada uno de esos puntos vienen, con una pieza más de las tantas ya añadidas en el mimado selector de temas alternativos, que como extensión considerable el catalogo ya definitivamente tiene aunque sea los clones principales del entreverado historial de ubicaciones, donde la dinámica del anonimato tras cubos y cuadrados fue revivida en entornos de distinta apariencia, y la que esta circunstancia presenta es una de las más fuertes por el reconocimiento que fue ganando en el ámbito, no a causa de ser considerada de buena manera por cómo es ni nada que se le acerque, básicamente por ser la más acostumbrada al ser adoptada por el sitio líder en concurrencia, muy por encima de cualquier otro que buscara tener un rol semejante. Se convirtieron en los estilos predeterminados para los ambulantes que no escogieran una versión en concreto, sin embargo aún siendo la raíz genuina de Rouzzed, el aspecto es de Dibujazzo, una de las tantas variantes que brinda la personalización extendida.

También tiene párrafo una aclaración que escoltó a las cinco actualizaciones, separadas puntualmente en dicha cuantía dentro del área correspondiente. Nuevamente los aspectos detallistas del registro dejado sobre las evoluciones en el tiempo son preocupación de la dirección, el cometido fue informar una modificación en la modalidad de presentar cada cambio notorio realizado en la plataforma, que pasará a ser una que separe punto por punto y que no los juntará en uno solo, como prácticamente siempre ha sido hasta esta oportunidad. Bastante básico, no obstante de cierta relevancia por ir a compartir espacio junto con los comunicados oficiales de mayor tenor.

Esto y aquello por un lado, aunque sin hacer mención de rumbos específicos declarados o tópicos definidos, confirman que los avances tarde o temprano vuelven a aparecer, y parte de esos viniendo de solicitudes, no todas cumplidas ni tampoco bien son planteadas, pero una que otra recibiendo cabida satisfactoriamente, siendo denegadas más que nada las que implican salirse del formato tradicional, el particular viaje fantemtiano que a las ideas de su contexto mucho se aferra, en veces como esta drásticamente.

\section{Lo mejor de los arqueólogos (Decimoctavo libro, capítulo XV)}\label{lo-mejor-de-los-arqueuxf3logos-decimoctavo-libro-capuxedtulo-xv}

\begin{quote}
25 de octubre de 2022, \ldots, 27 de octubre de 2022
\end{quote}

En las posteriores jornadas a que la actualización dejara los dominios de Fantemti vestidos con las tonalidades beige, no hizo falta pasar del día y monedas para que la adelantada alternancia de los compuestos extra surtiera efecto. Antes de eso, la llegada del diseño típico de los antros rozzado en los papeles podría haber generado un predecible rechazo, una movida relativa a la adopción de elementos característicos y propios de dichos destinos, especialmente en tiempos remotos habría resultado impensada y en todo caso resistida de primeras, como parte de los valores de repudio a la tradicional competencia, sin embargo actualmente es poco lo que perdura de ese dominio de lo estrictamente propio, más con lo tanto que la administración normalizó hacer uso de lo ajeno. Igualmente comprendido en el excesivo aprovechamiento de las riquezas ajenas, hay un componente más de adaptación que es derivada de rasgos originarios o aunque sea altamente ligados al mundo temtiano y fantemtiano, la puntual plantilla de estilos que incluye los distinguidos decorados esqueléticos y anaranjados, una discretamente pintoresca versión propia del tema en cuestión, que por varios días va siendo sostenida, sin que las otras copias prontas entren a ser expuestas, del modo que habitualmente sucede con las recreaciones parcial o totalmente completas.

Dentro de una gran pieza foránea, la variante menos extranjera es la que termina prevaleciendo y por rato largo, quizá la nueva manera de sacar a relucir las imitaciones que estén lanzándose próximamente. Eso como novedad más relevante, porque por el resto, realmente muy pocas situaciones a resaltar, siquiera entre las interacciones hubieron términos que ameriten mención especial, y tampoco los avances en las estructuras que albergan a los anoms renovaron el entusiasmo a permanecer en sus alrededores, un par de reconocimientos positivos han sido los comentarios al respecto, y no mucho más. En comparación a los transcursos sostenidos de mayor movimiento, estos momentos continúan con calma, mientras de nuevo, la personalización va por las suyas.

\section{Conscientes en las profundidades (Decimoctavo libro, capítulo XVI)}\label{conscientes-en-las-profundidades-decimoctavo-libro-capuxedtulo-xvi}

\begin{quote}
28 de octubre de 2022, \ldots, 2 de noviembre de 2022
\end{quote}

En la reiteración de las temáticas ha pasado la ambientación, no por novedades que vengan del día de los trucos o incluso el de los muertos que podría corresponder, sin embargo parecido, fuera del rango de dichas fechas esqueletos que a eso recuerdan vuelven a estar: cambios en las apariencias.

Si bien la expectativa que estén generando las exposiciones de temas no es muy perceptible entre los del ambiente, guiándose por los antecedentes que presentó tal modalidad, cómo se fue desarrollando este episodio resultó impredecible en las primeras horas previas, dado que se salió levemente del patrón estándar y fue más desparejo de lo normal luego, más que cuando las variantes eran probadas equitativamente una por una en su totalidad, o de no haberlas una duración extendida para la única versión de la imitación. Arrancando con que solo dos de las siete opciones tuvieron exposición, y siguiendo con la dispar permanencia de ellas, la restauración del originario diseño fantemtiano rápidamente terminó y a la brevedad las figuras insertos en el estilo rozzado regresaron sin razón evidente, pero sustentados por la discreta aclaración que tiempo atrás justificó las variaciones inesperadas, porque en principio el breve periodo de muestra ya estaba finalizado. Entonces más rato permanecen los estilos nuevos tras lo que fue un reducido descanso, dando la continuidad a la intencionada inestabilidad de unos mismos.

Paralelo a ello, en la interna del sitio además del deporte y la convivencia con altibajos, los hilos fijados fueron parte de lo destacado y sobre todo el actual del juego de los soñadores, con la administración brindando apoyo a dicha iniciativa por tercera vez. Un tanto especial contando que también el canal de grabaciones se sumó a la expedición colectiva, con un largo extracto que el reportero incluyó en su selección de contenidos temtianos, pantallas bidimensionales siendo recorridas en una jornada menos ajena a la realidad del antro, a pesar de no contar propiamente con la distinción oficial. Un caso de varios que representa y engancha mejor a los participantes, que estarán viendo si en la red efectivamente así deben llevarse.

\section{A falta de un hasta mañana (Decimoctavo libro, capítulo XVII)}\label{a-falta-de-un-hasta-mauxf1ana-decimoctavo-libro-capuxedtulo-xvii}

\begin{quote}
3 de noviembre de 2022
\end{quote}

Se había venido la noche y Fantemti lo padeció perdiéndose en la oscuridad, tener inconvenientes en el permitir la entrada y en su lugar no más que emitir una sombría señal de ausencia sin pretexto dedicado por parte de su remoto origen. Durante dicho entonces, el muy circunstancial mensaje del alojamiento ferroviario estuvo presente en el correr de un intervalo demasiado extendido, lo suficiente para extrañar y despertar los cuestionamientos en los alternativos medios de contacto con la administración, sin respuesta mientras el estado crítico se asentaba ocupando los largos dominios del sitio.

La aludida mala coyuntura persistió bastante tiempo así, tomando como punto de comparación los varios episodios en los que una misma expresión del servidor rechazó los pedidos de ingreso a la dirección fantemtiana, no más de unos pocos minutos como lo máximo que permanecían y pronto el andar restauraba su normalidad, pero este caso consiguió distinguirse ampliamente de aquellos por extenderse varias horas. Es referirse al pasado porque el final sin preámbulos llegó en el amanecer respecto al huso más compartido entre los frecuentadores, posterior a ello el antro supo reintegrarse como si nada hubiera sucedido, ni un solo comentario refiriendo a la caída de turno sino luego del mismo, debido a que un par de mensajes revivieron el único hilo por el cual su mandamás implícitamente se dio a conocer como tal, en Devvox, no obstante tampoco desde ahí la noticia tuvo novedades, hasta más tarde que cierta información de primera mano constató lo producido aunque sin ampliar tanto.

Como devolución a la inquietud manifestada en un par de participaciones, el dueño confirmó no por aclaración que aproximadamente un tercio de día el antro anduvo fuera de servicio, pero sin llevar la precisión más allá de la acotada descripción típica, base de datos, además de replicar a la actividad que quedara pendiente en la otra ubicación, como antes similarmente sucedía en destinos ya apagados, una secuencia que muestra el lado para el que van las miradas cuando no están los espacios fantemtianos operativos, hacia la palabra oficial más cercana, siendo que raramente no hay tantos canales inmediatos donde las noticias de ese tipo puedan difundirse en el exterior.

\section{Supremacía comprensiva (Decimoctavo libro, capítulo XVIII)}\label{supremacuxeda-comprensiva-decimoctavo-libro-capuxedtulo-xviii}

\begin{quote}
4 de noviembre de 2022, \ldots, 9 de noviembre de 2022
\end{quote}

Aquellos tramos de casi que nulas incidencias extraordinarias volvieron a repetirse en el curso de los episodios fantemtianos, consecuencia, la lupa debe hacer un poco más de enfoque para dar con lo destacable del momento actual, qué pasa como evidente cuanto mucho está en los intercambios gestados para dentro de fronteras, el microambiente todavía de baja aceleración bien conocido por como es, una extendida versión de lo que era cuando los protagonistas estaban sumamente conexos y hacían parte del, solo que un tanto opacados por los avances del desarrollo, actualmente a grandes rasgos estancados.

Dicho esto, más se nota en la trama del hilo propio de novedades oficiales, no por su función primordial y primera en el orden de los varios propósitos, en cambio sí por la conversa con la conducción, nuevamente asuntos como otros cualquiera de los más terrenales, en un tono normalmente alternado y de presuntas múltiples procedencias, a menudo expresiones amistosas, cuando incluso a veces lo contrario se desencadena. Considerando el contexto del cual provienen, es un tanto esperable que no exista respaldo absoluto por parte de los anoms a pesar de que haya un cierto apartamiento de las costumbres así de remotas, lo cual todavía acostumbra a verse, sin embargo a la par del moderado crecimiento de la comunidad la corriente del respeto ha ido expandiéndose, junto con el tradicional uso del resistido lenguaje inclusivo, siendo que una mayor parte de sus participantes han coincidido con los modos desde antes ocasionalmente transmitidos por la conducción, en lo más reciente reiterado con el anclazo musicalizado, otra insignia del mensaje anti bardo y odio.

Es circunstancial y presenta alguna similitud con anteriores pantallazos donde se veía a la familia fantemtiana relativamente unida, panorama cambiante si los habrá, no obstante viendo esta época de exóticas presencias y criaturas, como chinchillas, bestias, y demás, parece haber una integración más variada y fresca, y descartando que sea la también posible casualidad en un clon animado por desentendidos indiferentes, a lo mejor vaya a estar formándose una comunión más armoniosa y estable entre los anoms, hoy constructores directos de un espacio tranquilo en las actividades periódicas.

\section{La suerte del nominado por descarte (Decimoctavo libro, capítulo XIX)}\label{la-suerte-del-nominado-por-descarte-decimoctavo-libro-capuxedtulo-xix}

\begin{quote}
9 de noviembre de 2022
\end{quote}

Un hogar temtiano vuelve a marcar asistencia en momentos que el contexto presenta inestabilidades y por las circunstancias que fuera lleva a los instalados en antros deshabilitados a reencender su reiterativa búsqueda por refugios o sustitutos, lo que en esta ocasión condujo a los adeptos del recientemente caído a pasarse por las alternativas cercanas, pero justamente no hay muchos candidatos que se estén prestando a dar cobijo a los desplazados actualmente, está el par colindante y luego lo siguiente para encontrarlo hay que apartarse más. El episodio continúa simultáneamente por este lado, con una interna más concurrida e interactiva, y sus mecanismos de transformación visual manuales, accionados por la presente conducción.

Lo más notorio consta de la alternancia en el diseño establecido, que no específicamente fue anticipada unos meses atrás cuando por razones desconocidas el típico y propio del sitio era sustituido con cualquier opción de las disponibles, a menudo porque entre los fantems apareciera una u otra referencia a existencias copiadas. Ahora también pasó en un caso suscitado por sucesos exteriores, que le otorgaron prominencia a un clon del repertorio de imitaciones, el puntual de Devvox que a su vez es versión recreada de su nombre mezclado, que si fuera poco ademas es acompañado con un fantem de invitación hacia allí anclado, el mismo de antes por segunda oportunidad. Considerando las ocurrencias en dicha ubicación como además las del coyuntural contrincante, con algo de sentido llamó la atención y pudo instalar la idea de que Fantemti pudiera estar siendo atacada y por entonces la apariencia de la misma configurada en función del dominante que tiene el momento. Un evento como tal sería sumamente extraño y único en la trayectoria histórica de sí, pero en verdad es un movimiento predecible y más cuando hay vaivenes como los recientes, contrario a una rivalidad que intentase denostar a los competidores, hay simpatía con ellos.

A pesar del cambio favorable a la eventual adaptación de grupos migrantes, o que al menos que disminuye las diferencias entre sitios, es desde el punto de vista estético y hasta allí, por el resto no hay suficientes facilidades, y la administración está medianamente a fin con ello, según puede entenderse con las intenciones manifestadas en intercambios de los hilos oficiales, especialmente el último de los activos en el extranjero. Por esos lares es más común ver la disconformidad y el desprecio a las prestaciones de Fantemti, en parte con justificaciones difíciles de apreciar o inexistentes, y en complemento debidamente fundamentadas con lo que a primeras vistas más rechina, las restricciones para agregar multimedia y en menor medida otros aspectos de inferior trascendencia. Eso desde que el gran público del entorno probó más en profundidad las características ha sido lo achacado y por lejos, sin embargo aún siendo sabido por quien lo determinó, cuanto mucho cedió parcialmente en los permisos, pero no del todo, mostrando resistencia aún así le fuera solicitado.

Vaya a ser temporal o no, lo cierto es que hoy por hoy en el clásico mapa los parecidos puntos verdes son solamente dos, hablando de los antros que ofrecieran el mismo formato de anónimos revivido y fallecido vez tras vez, por lo que los individuos involucrados si es que siguen haciéndolo cuentan con un panorama más acotado, de lo cual Fantemti tranquilamente podría sacar provecho y ganar simpatizantes que estén en diáspora a la espera de tierras prometidas, o nada más abiertos a visitar frecuentemente otro bastión de aquellos más, no obstante las trabas sustanciales para que eso ocurra no son atacadas, sino afirmadas. Todavía con esas desventajas, situaciones como esta la llevan a tener un protagonismo superior en el plano, y a medida que surgen las preguntas sobre si exista algún respaldo cada tanto nombrada, sumando conocimiento de que hay otro dominio funcionando y ofreciendo similares condiciones, y esto conociendo las habituales cifras del contador de presentes y actividad, denota que el tráfico indiscutiblemente sube. De ahí que el incremento se mantenga ya es más complejo, los antecedentes no muy desiguales indican que hay una caducidad, pero tampoco garantizada, las decisiones de los superiores con más subordinados serán decisivas al respecto.

\section{La apatía a la grandeza (Decimoctavo libro, capítulo XX)}\label{la-apatuxeda-a-la-grandeza-decimoctavo-libro-capuxedtulo-xx}

\begin{quote}
10 de noviembre de 2022
\end{quote}

Si irán siendo sumadas oportunidades como estas que un antro de los protagonistas presenta interrupciones en estabilidad, no fueron tantas en el trayecto histórico fantemtiano sino unas pocas, no obstante en cada una de ellas suele suceder más o menos lo mismo, desde que las condiciones más relevantes para sí también permanecen. Según su propia popularidad y la distribución del momento, al restarse una que de amparo a cierta cantidad relevante de individuos, fracción de ellos terminan teniendo aunque sea un breve pasaje por los dominios de tono anaranjado, unos transcursos de notable mayor tráfico y actividad, que va tendiendo a bajar a medida que la estadía no colma las expectativas, en una progresiva vuelta a la normalidad que concluye cuando los candidatos a ser sustitutos se consolidan, o los bastiones originales recuperan la disponibilidad. Ambas contingencias van en perjuicio de la expansión de Fantemti, el contador de visitantes llega a coincidir con altos picos numéricos respecto a lo que son sus intervalos habituales, perceptible por aumentos de la frecuencia en la repetición de los sonidos y destellos, sin casi mas que eso, la característica temporal prevaleciente.

Por el lado positivo, allí fue reflejado lo que ganó en consideración de los anónimos restantes en el contexto, no exclusivamente por una pequeña tendencia a la suba que complementa el involucramiento base presente sin importar las eventualidades, por todas las evidencias que constatan la amplitud de procedencias que pasaron e incluso participaron, al punto de que personajes de la época famosos en los otros sitios parecieron dejar su huella. Aunque claro está, por el resto se mantiene todavía una importante cuota de desconocimiento, sin ir más lejos el debate sobre la competencia entre clones se centra únicamente en dos alternativas, en el que poco se filtra el nombre de una tercera que funcione bajo similar dinámica, desconocido es que la hay.

Así es esta distancia natural y predecible, que bien se deberá a lo complicado de torcer lo establecido que está el dominio de los clones más concurridos, sin embargo también por la postura de la administración ante las situaciones que se presentan, una no muy interesada en sacar tajada de las pertinencias de público por atraer, y por lo tanto tampoco competitiva con el resto de clones que si puedan hacerlo, en cambio si precisamente lo opuesto, contando que además de los hilos de exploraciones y buenos tratos regresando a la cima, la misma conducta de publicitar sus direcciones se estuvo repitiendo más de la cuenta en los recientes grandes desplazamientos, la frecuente usanza de colocar y retirar temáticas que no necesariamente son del interés de la comunidad, solo ciertas materias que desde arriba instalan como destacadas.

Pero aún con el apoyo a las emergencias exteriores, devvoxeras, ufftopperas, voxxeras, no son referencias que funcionen como desvío absoluto ni que lleven a descartar totalmente a la existencia fantemtiana, que hasta por algunos huéspedes ha sido más que positivamente valorada, el estado de relativo abandono calificado como una lástima por las virtudes que seguramente estos encontraron, un despropósito por el supuesto potencial que varias veces atribuyeron tanto directa como indirectamente los críticos de eso que tanto resalta y comúnmente persiguen los antros al diseñarse para recibir interacciones, un tanto esquivo en la baja ambición propagativa demostrada por las declaraciones y decisiones del estabilizado sucesor de Temti, sea por aferrarse a sus principios originarios en modo testarudo, por evaluaciones racionales sobre lo efectivamente conveniente, o por cobardía a continuar creciendo como podría esperarse de sí.

\section{Las grietas en la red de la pureza (Decimoctavo libro, capítulo XXI)}\label{las-grietas-en-la-red-de-la-pureza-decimoctavo-libro-capuxedtulo-xxi}

\begin{quote}
11 de noviembre de 2022, \ldots, 17 de noviembre de 2022
\end{quote}

Dentro de un ambiente así de pequeño y desacelerado como el fantemtiano, viene predominando lo típico y ni los sapos pueden sorprenderse de ello, pero más hay para ampliar cuando el descuido se extiende lo suficiente, es que una contingencia puntual y de escasa ocupación puede convertirse en el foco del escenario y ganar la atención necesaria para protagonizar la o las jornadas correspondientes. He aquí el preámbulo que potencia los motivos para centrar un episodio en torno a lo que uno o dos sujetos con una simple acción provocaron, más luego sus no tan amplificadas consecuencias desencadenadas en el contexto, no alterado en su estado normal, empero marcado con breves percances en el habitual bienestar general hecho factor común.

Fueron esas oportunidades las que el espacio principal recibió en sus celdas las contribuciones subidas de tono para lo que permite el sitio, evidentemente no amparadas por las limitaciones cotidianas impuestas a los intercambios sobre sus direcciones. Conforme con la visibilidad del respectivo insuceso, tuvo la dilatación para llegar a ser registrada por los visitantes del momento, como ciertamente los tuvo y no fue el caso de que rápido los elementos desubicados fueran removidos, y de tal modo también suscitaron algunos comentarios al respecto, tampoco demasiado conmocionados ni excitados en cuanto a los contenidos exhibidos, no así indiferentes de la negativa implicada para el orden pretendido por las autoridades, que en las instancias públicas recibieron reclamos explícitamente objetando las faltas no atendidas, más especialmente los reportes pendientes, esperando respuestas que finalmente llegaron con cierto rato de demora, la retirada más sustitución de los gráficos inapropiados, y conjunta contestación a los colaboradores en el cumplimiento de las normativas.

Sin abordar lo ofensivo contenido en palabras o de nuevo inundaciones, verdad es que al estar a nivel global habilitada la capacidad de cargar multimedia, no hay impedimentos a que los locales y forasteros aprovechen sus medios disponibles y los utilicen libremente por completo, esto hace posible el cero cuidado con las virtuales barreras, por el desconocimiento de las mismas o las intenciones malignas de ignorarlas, lo que anteriormente era de alguna forma prevenido con la aplicación de restricciones sobre las funciones más peligrosas, pero ya no ocurre y trae la parte mala para la perspectiva rigorista del antro, la desventaja de flexibilizar los permisos pretendidos por la mayoría y mantener iguales condiciones por el resto, una inclinación de la balanza que es más reciente en el tiempo. E influye, el grado de estrictez que manejan la norma transgredida, terminantemente adversa al material de adultos, que justamente como mínimo un par de veces fue objetado por la comunidad, más o menos razonable por lo habituado en el entorno de origen. Cual primera referencia, ambas en los demás clones no son cuestionados así, pues además de que el umbral de materiales autorizados lo fijan más amplio, al ser los de cualquier tipo una porción prácticamente indispensable de la usabilidad ofrecida por ellos, la problemática de deber moderar con asiduidad está implícita, más aún con un mínimo de actividad base que exija a la moderación poner orden si es que lo pretende sostener donde oficia, lo que naturalmente se torna más eficiente cuando es un equipo formado por varios, ergo no el caso de los sitios temtianos, que exceptuando las primeras épocas de poder descentralizado y distribuido entre voluntarios comprometidos con la causa, lo evidente ha sido una labor no muy atenta y abarcativa, precisamente para Fantemti cuando el admin se entera, o más bien le hacen enterarse.

\section{Atraco y trama (Decimoctavo libro, capítulo XXII)}\label{atraco-y-trama-decimoctavo-libro-capuxedtulo-xxii}

\begin{quote}
18 de noviembre de 2022, \ldots, 21 de noviembre de 2022
\end{quote}

¿Qué fue lo actualizado? Es primeramente notable en el frente, de aquello que pasa no inadvertido sin tratarse de un cambio radical en el panorama general sitial, para irrumpir un tanto en la monotonía del entorno, añadir un nuevo elemento que acompañe a la dinámica y adaptar el marco de las regulaciones que la endereza. Así es que puede recibirse la reestructuración aunque sea temporal de la cabecera fantemtiana, con iconos y caracteres de más rellenando los espacios de fondo con datos más relevantes, por encima un elevado número escoltando al clásico contador y por debajo un imponente globo de aviso transmitiendo algo seriamente.

Parecido al recuadro de dos valores y figuras de una renovadora presentación en la gran antecesora, el conteo de visitantes y versión juntos secundando al nombre del antro en lo más alto, solo que la novedad estando en la introducción de un segundo que no solía verse allí. Atípicamente este es, un aproximado al mil con vinculo directo hacia otra dirección de anuncios, remodelada, extenso punteo indicando los momentos de cada actualización, numeradas en orden descendente de una en una. Y efectivamente son los detalles de las mismas, que dicho sea de paso tras resignar su modalidad de numeración percibieron alternativa, lineal por ser apuntadas todas a mismo nivel y no destacar las simples correcciones de las vastas implementaciones. Varios párrafos mencionando errores vistos y corregidos, y empezando por la última que trae esto, más referencias a las características introducidas recientemente, pero no en un rango temporalmente muy prolongado, más bien solo hasta cerca de unas cincuenta, luego todo entradas vacías que alcanzan la fecha de fundación, cuando la sucesión comenzó.

A pesar de dicha omisión, la verdadera sucesión de cambios consta de muchísimos asteriscos más que los habidos, incluso faltan los informes anteriormente publicados como aclaración y demás puntualizaciones de entidad. La administración aludió a la situación al decir estar buscando evitar ciertas dificultades en el manejo de esa información, conforme a la cantidad de sentencias que antes se iban en aclaraciones y alertas, que ya no están, con la idea de trasladar el cometido noticiario de mayor vigor al espacio de prioridad fijado. A su vez, el resto de lo esclarecido ante la consulta pública refiere al reglamento, que después de un considerable rato este recibió modificaciones puntuales a su contenido, puntualmente no muy profundas sino que de unas pocas lineas complementarias al eje central.

Sin indicación específica de la posición que fuera trastocada, la guía es dada por la respectiva alerta, un par de pasajes previamente mejorables en cuanto a definición reescritos con mejoras en ese sentido, precisión. En un primer lugar, el concepto de bienestar fundamental, más explicación acerca de lo que significa para quien lo escribió y hará cumplir, lo que a su vez se relaciona con las oraciones posteriores. Y siguiente sobre las acciones que puedan derivar en sanciones, una adaptación que también distinga los infractores con código de los que no. Ambos puntos según indicó el mandamás vienen a cuento por un increpación de tiempo atrás, el descontento manifestado por los criterios aplicados contra algunas incidencias que según esto debían ser penadas y evidentemente no lo fueron, la liviandad en el uso de la autoridad por la preservación del orden en el antro, lo que conllevó a una respuesta de nula reacción más que un extendido entendimiento con los posibles vacíos a compensar, sin embargo finalmente porción de aquellas cuestiones pasa a ser comprendida en las reglas y compañía.

Y lo que falta es, cierto pormenor colateral igualmente contingente a ese grupo de componentes visibles, lo que pone el tapiz por detrás del toque gráfico que cada hilo recibe a elección de su creador, los benditos sabores y un rotundo incidente que hizo desaparecer las decenas y decenas de colores que hacían la peculiar variedad en el obligatorio selector, que todavía siendo opacado por su natural sustituto de las portadas, recibía un mínimo del cuidado del anom promedio durante el proceso de publicar. Pues los muchos que estaban ya no son posibilidad, no en cambio aquellos previamente asignados perduran en el inicio, exceptuando los que fueron aireados episodios del pasado. Consecuencia, únicamente tres gustos sumamente básicos para los futuros fantems, supuestamente más variantes que de costumbre

Sinopsis, una recurrencia en los asuntos ilustrativos, los bastante tratados añaden otra entrada a la lista figurada y también apuntada con citas temporales, en su mayoría conocidas o aunque sea deliberadas cual pormenores en la escena del sitio. Por el lado de las formalidades, probablemente las condiciones vigentes regían ya y no son exclusivamente actuales, pues justamente el comunicado declara el tema esclarecimientos a las pautas difusas, que a lo mejor todavía tengan aspectos a cubrir, no obstante difícilmente trasciendan como tal si no reciben más críticas o reformas próximamente. Pasando, eso quizás tendría el despejado sistema de gradientes, que más allá de su minimizado papel, permanece, más pobre.

Volviendo, lo destacado quizás sea la notoriedad ganada por el historial de mutabilidades, hasta entonces siendo una evolución en paralelo conocida pero igual desapercibida, debiéndose a su colocación junto con los botones descubiertos y su baja incidencia en la cotidianidad, más también la ausencia de alusiones oficiales que ventilen los ajustes. Suponiendo que será mantenido y no abandonado, entre esos hay dos factores que favorecerán a la difusión de su progresión y especificaciones, puesto que a medida vaya aumentando la segunda cifra será más sencillo llegar al informe correspondiente, y que además podrá notarse con el pasar cuando permanezca estática, el avance de las estructuras fantemtianas valorado en cada paso como el peso de los implicados dentro de sí.

\chapter{Los regulares pantallazos del testarudo seguidor viajando al más allá (Decimonoveno libro)}\label{los-regulares-pantallazos-del-testarudo-seguidor-viajando-al-muxe1s-alluxe1-decimonoveno-libro}

Eras complicadas para definir un antes y después trascendental, los motivos simbólicos pasan a ser suficientes en esto de marcar apertura, cuando menos es necesario cerca del puntual instante, a lo mejor con más razón si es contando que las anomalías progresivamente cesaron y de tal su correspondiente periodo especial, únicamente restando episodios con similitudes que se unen a la normalidad. Gracias a ello, un aniversario global de la trayectoria temtiana total establece referencia alrededor de lo que accesibles predicciones dicen tendrá mucha similitud con su anterior, monotonía respecto a lo esencial y modernizador, cambiantes y comprendidas pero lentas entrañas, vecindad que cada tanto sucumbe y acelera, lo típico y de conocimiento.

En cuanto al episodio concreto que supone el quiebre sería inferior en relevancia para la coyuntura dada yendo al rigor de los nombres manejados, no obstante hoy es más la perspectiva histórica la que vale y en eso plantear disyuntivas pasa a otro plano de ínfima consideración. Será inequívoco que sobre Temti y sobre Fantemti suponen temáticas con montones de contenido y entre ellos variantes hay para encontrar, sin embargo la distinción temporal que motiva esta suerte de separación atiene a la primera y se comprueba debido al vinculo compuesto por la segunda, con tanta entidad el papel de la imitación persiste protagonizando la gesta, el fanatismo da una convincente forma de vida y convierte las expresiones de homenaje en la mismísima prosecución, precisamente una gran etapa más de lo mismo, que por cierto no ha dejado de crecer sin parar.

Que a su vez confirma el relegamiento de la difuminada rama original y sus sitios, más y más de novedades a cuenta gotas mezcladas en la nada, señales que facilitan olvidar sus existencias, leyendas que se alejan y no vuelven, expertas en registrar entrevero que hacen extrañar sus rarezas y contrastan con el presente. Estabilizada continuación que pasó a ser drásticamente uniforme, con sus circunstancias propias del entonces y demás, no obstante careciendo de sobresaltos como la tradición había hecho acostumbrar alcanzando poco más de la mitad. Luego es esto, reina la estabilidad a modo de fundamento casi permanente, añadido al estado visible palpable día a día, un todo que en conjunto ha sabido conformar y disgustar con sus virtudes y falencias, empero aferrándose a su principio fundamental no apartado de la dinámica clásica de los orígenes, que hasta aquí y al futuro lleva, a Fantemti.

Lo cual a estas alturas más transmite coherencia considerando la manifestada intención de la partida, comprometida ni bien arrancó con la proyección del bastión y su amplio significado para el mundo que supo llamarse titiritero, un combate a la desaparición o perpetuación en niveles desmedidamente discretos que sin conocer el fracaso siguió y siguió, no por la eternidad, solo un rato y sin finalizar, que al lado de su antecesora significa un montón y más con las proyecciones que en el caso lucen más prometedoras. No renovando el concepto y primordialmente ratificando la consolidación carente de límites drásticos, enfatizando el término de asentarse, allí puede encontrarse la filosofía que multiplica su poder en el curso de la Historia Temtiana, una que a la supervivencia pone harta parte de sus fichas incluso cuando la ocasión el crecimiento de concurrencia más que tímidamente ofrece, el particular sentido común que prósperamente vive con las ocasionales visitas del fragmentado contexto, la alternativa que en incontables sentidos da la nota sin mayor atención, el clon personalizado desde la experiencia e imagen con notable propensión a permanecer cercano a la familia selecta de simpatizantes y activistas, hogar de intercambios lógicos y absurdos, cultura de respeto y ofensas, experiencias soñolientas y realistas, identidad propia y disfrazada copia\ldots{} más fondo y hechura para el cumpleañero espíritu de Temti.

\section{El honor compartido (Decimonoveno libro, capítulo I)}\label{el-honor-compartido-decimonoveno-libro-capuxedtulo-i}

\begin{quote}
22 de noviembre de 2022
\end{quote}

No de pronto sino con anticipación por tratarse de una fecha señalada, el motivo de reconocimiento para el colectivo temtiano y particularmente Fantemti en su papel al involucrarse con el mismo, recibir como es entendido el primer aniversario de Temti, el segundo en agregado.

Involucrados y al pendiente de aquello pese a no haber vivido la previa ni adelantarse como podría ser el antes de un momento especial conocido, en el durante este pueblo realizó algunos reconocimientos del evento en sí conforme la jornada fue avanzando, a lo mejor sin tanto entusiasmo con ello, pero demostrando más que indiferencia mediante las palabras, sobre todo por parte del sector artista que repitió en la elaboración de piezas temáticas con roedores. En el antro activo más allá de las menciones en la interna, se refleja con cambios en la estética principal, que al comienzo simplemente llevaron el colorido de anaranjado a verde, un idéntico tono al ya visto en épocas de los orígenes predominante hasta en los contenedores del espacio de recibimiento, más luego una serie de símbolos puntuales alusivos a la celebración, torta y globo, esos fueron los detalles especiales para la ocasión.

El acontecimiento trastoca la visual en general y más aun los elementos del apartado central, con efectos que sin ser muy decorosos ni elaborados, drásticamente transforman la óptica de la galería habitual, sumando los recuadros de las réplicas no aleatorios sino en un solo color. Quizá basa su matiz por el éxito que conlleva completar una cifra mayúscula de tiempo, para el supuesto propósito de clonar una idea con tal de que no desaparezca es notablemente grato, pero por como la conmemoración fue planteada en un principio resultaba más complicado de interpretar, sin las indicaciones de versión o lo que el contexto comunicara. Ya pormenores típicos de festejo todavía no demasiado explícitos dan a entender apropiadamente el concepto, acompañado por colores que precisamente Temti adoptó en época de fiestas, del cual de cierto modo no hay mucho que escolte, sin embargo virtualmente lo hay bastante, y es su mismo camino el que sigue siendo recorrido.

Vaya que es un hecho resaltable, los anoms además de continuar intercambiando normalmente con una frecuencia baja, últimamente vienen transmitiendo algo de apego al antro o aunque sea un mínimo de aprecio hacia tal, una leve cuota de perpetuación a las costumbres más contemporáneas como la calidad del trato y poco más, no obstante aquel orgullo por la identidad y todo lo propio construido supo pasar a un plano alternativo, más si no fueran consideradas las distinguidas manifestaciones del mandamás, una diferencia con respecto a los viejos tiempos cuando distinguirse era fuente de numerosos discursos multitudinarios. Este hito en particular significa más para aquella minoritaria corriente que valora el contenido cultural y la existencia de las actividades temtianas más allá de un sitio como tal, porque en términos estrictos tendría menos sentido celebrar una vida que pasó a condiciones tan discretas o que incluso finalizó, posible crítica contundente a la presente exaltación, sin embargo la perspectiva dominante es más abierta en ese aspecto, que el año vuelve a cumplirse de inicio a fin, marcando el arranque de uno nuevo que bajo dicha definición difícilmente se vería interrumpido. Será apresurado\ldots{} y sí, e igual, da la sensación de que ahí va el tercero.

\section{Normalidad en libertad condicional (Decimonoveno libro, capítulo II)}\label{normalidad-en-libertad-condicional-decimonoveno-libro-capuxedtulo-ii}

\begin{quote}
23 de noviembre de 2022, \ldots, 29 de noviembre de 2022
\end{quote}

Cortado el ambiente festivo inicializado en justas horas de la madrugada, el colectivo lo hizo más auténtico que artificial, reconociendo el significado de la fecha y acompañando al colorido de congratulación, un hito que quedará para la historia más incluso por el extraordinario reconocimiento en los estilos fantemtianos, confirmando su correspondencia con la causa unificada, adaptaciones para las que quizá no vuelvan a aparecer motivos suficientes. Y las pilas oficiales por un rato hasta allí fueron, cuando las cifras de la cima indicaron una suerte de abandono carente de precedentes, y dicho panorama de mientras lo dinamizó la participación, interacciones que no fallan y participaciones en solitario, entre algunas con la motivación adicional del evento deportivo mundial, exploraciones semanales e intervenciones gráficas censuradas, gatos y chinchillas, algo de amor y demás.

Los llamativos días de inactividad pública en las referencias más notorias finalizaron con el numero cinco ni bien la comunidad consultó por la integridad del administrador, quien presentó un par de novedades aplicadas a la estructura del antro, únicamente por esos dos canales principales donde antes habría sido anunciado que irían las entradas de tal tópico. Acelerado y escueto para mencionar los puntos, de correcciones específicas, sabores añadidos, y una característica compuesta constó, esta tercera la evidentemente más relevante, destacada por tratarse de un avance en las capacidades para los anónimos no privilegiados, elegirle una portada a sus fantems creados, luego de un enredado proceso en el cual aparentemente pasarían por varias pruebas previo a compenetrarse con la galería del inicio, y por entonces contar con una distinción de sabor o parecido, sin revelar por el momento. Serie de pasos complejos, mas todavía la interpretación al oportuno planteo oficial, pero a lo mejor la clave esté en el mensaje alusivo a la descripción en cuestión.

En su versión resumida, sería un asterisco de alta trascendencia y apropiado para darle la estrella a su ítem en el flamante registro de cambios, que con la notoriedad ganada desde la cabecera comenzó a mantener más informados a los visitantes acerca de las actualizaciones implementadas, como mínimo a partir de la progresión numérica. Mediante ello a su vez pasó a ser más sencillo ponerse al tanto de si el desarrollo y mantenimiento se mantiene trancado, contando hasta los mínimos ajustes que entran contemplados, aunque tampoco es una medida apropiada para detectar ausencias absolutas de la autoridad. Curiosamente la novedad llega poco posteriormente a que la disconformidad con los colores volviera a ser planteada en destinos cercanos, y a pesar de ser un reclamo ya casi que clásico, después de abundantes creaciones carentes de chispa visual y el esencial atractivo del contexto, estaría dándose. Acudiendo a sus palabras, la innovación requerirá evaluación, y no solo es por el aspecto funcional u operativo, puesto que de efectivamente cualquiera poder crear hilos del tipo más solicitado, hay un potencial impacto a notarse en la entrada del sitio, si esta posibilidad llegara a conocimiento de los transeúntes menos interiorizados, el uso no unicamente puede ser pertinente, también lo contrario, pero según suele componerse la actividad costaría asimilar mucho por ese lado, y es de suponerse hubo consideración por las vulnerabilidades, asumiendo que la prudencia no desasistió, bastó con que tanto se hiciera esperar.

\section{Próximo al drástico golpe de timón (Decimonoveno libro, capítulo III)}\label{pruxf3ximo-al-druxe1stico-golpe-de-timuxf3n-decimonoveno-libro-capuxedtulo-iii}

\begin{quote}
30 de noviembre de 2022, \ldots, 8 de diciembre de 2022
\end{quote}

Semanas no del todo tranquilas para la existencia fantemtiana, medianamente normal salvando ciertas particularidades no propias del acontecer usual, especialmente una sumamente rara.

Lo que no debería extrañar y se confirma es que, puede notarse como en las oportunidades que hay pausas inesperadas en la dupla de antros principales, un pequeño porcentaje de su cuota termina teniendo periplos de permanencia condicionada al no tardante regreso del origen, últimamente no muy dilatado en ambos casos, favorable a la consolidación del equilibrio contemporáneo, que tiene a la tercera alternativa en peso apartada en su camino paralelo no tan alejado, al todavía prestarse para las contingencias relativas a estos, mientras lo demás es más aleatorio y circunstancial según el momento indique.

En ese sentido, más allá de la típica variabilidad y reiteración que encausa los contenidos generados, no se encuentra una reacción luego del potencial antecedente, tampoco un contraste notorio con respecto a la anterior actualización extraordinaria, la que en teoría removía los mayores impedimentos para aquellos publicadores no acreditados, y más, al contrario de incrementar los factores ilustrados, no han dejado de verse a pleno los degradados en diferentes tonos de color, como si hubiera dado lo mismo.

En un conjunto puntual del asunto actividad fue peculiar lo que siguió a la tanda de nuevos cuadros temáticos, basados en gráficos reconocidos, adaptadas a tramas temtianas. Dilatada e inesperada, reactivó uno de los hilos colección donde no unicamente son albergados los materiales de perspectiva cualquiera, porque a veces se cuelan algunos que encajan en lo relacionado a la cultura correspondiente, y conectan situaciones particulares de la misma con representaciones vistosas. Si habrán sido encontradas ya, hubo épocas cuando la producción de esos llamados memes mostraba resultados con alta frecuencia, pero lo especial en la ocasión actual es la situación generada en torno al canal de difusión oficial, no finalizada con el típico archivado.

Dos caminos de redacciones previamente también llamaron a su aparición, las elaboraciones literarias del registro histórico siguiendo con sus regulares ediciones, y además la no frecuente publicación de los informes de estado, esta última una buena razón para observar la actualidad del ente comunicador, que no demoró mucho en trasladar lo reciente de interés al sitial de privilegio en la plataforma de mensajería. Parcialmente se fue convirtiendo en el medio de Fantemti, antes por el producto del fanatismo, la gran mayoría de asuntos saliendo desde ahí, tanto que supo hacerse un lugar en la descripción de direcciones actuales, ahora lo bastante al punto de relegar el clásico ícono de la t solitaria, dando paso a las iniciales del nombre compuesto actualmente predominante.

Para colmo, la linea previa del último bastión gestionado por la administración original no solo perdió terreno allí, asimismo Titiri Vieja dejó de ser alcanzable con su llamativa secuencia sin descifrar, y de tremendo retroceso la comunidad se enteró, sin datos del inconveniente o motivo que hizo aparecer el inamovible error genérico. Su más símil sucesor sin referenciar en parte compartió la simpatía conservada por el propietario de tales descuidados dominios, el segundo puesto anclado fue transferido a un triste mensaje relativo al distanciamiento de aquel desaparecido, muestra adicional de que el recorrido individual del clon anaranjado insiste con no olvidar los orígenes, por más que el vínculo real que conserven con el presente sea tenue y desentendido.

La caída en cuestión a lo mejor no suponga demasiado, el alejamiento venía siendo amplio sin casi implicación con los anónimos, y poco en lo simbólico, el hecho pone al tanto aquella evolución que tiende a un final cada vez más pronunciado. Igualmente, es una ración mas de la disminución en presencia del antro en sí, que en un pasado no muy lejano era el principal en las andanzas temtianas, los antecedentes que en su momento lo hicieron importante permanecen y claramente sin ellos esta continuación no sería semejante, pero hoy día están minimizadas, y con la contundente asunción de protagonismo consiguiente, menos priman las razones para que su marca particular siga instalada en la imagen frontal del canal, si es del modo que reflejan las actividades transmitidas el enfoque trasciende la denominación de turno, Temti, Titiri, o Fantemti. Aún así, otras identificaciones todavía se deben al primero, a pesar de que el dominio como tal no forme el núcleo común, por lo que si tan fuerte fuera la preponderancia del legatario quizá también el resto de referencias deban estar formuladas en función del mismo, lo cual no es el caso, ¿de momento?

\section{La magia albiceleste (Decimonoveno libro, capítulo IV)}\label{la-magia-albiceleste-decimonoveno-libro-capuxedtulo-iv}

\begin{quote}
9 de diciembre de 2022, \ldots, 19 de diciembre de 2022
\end{quote}

Fantemti tranquila aunque no olvidada enteramente. Por lo que venga del mandato, solo parches y correcciones primando como novedad en su listado de cambios que ni sugerencias incluye, por el resto tomates y demás sin relevancia para el estado global, a no ser que la anclada musicalización de la generación tenga un sentido esperando por atención. De lo último, aquella instalada modalidad de aprobación en las portadas, interiorizada o no difícil sería afirmar la medida en que viene siendo aprovechada a falta de alusiones al respecto, con la diversidad menos predecible por un lado y los personajes repitiendo sus marcas identificables por otro. Pero hay más, no puntualmente las periódicas expediciones que muy alineadas con lo oficial parecen ya estar, en cambio sí qué secuelas regalo el evento transfronterizo que los clones con un mínimo de actividad no ignoran.

La temática especial del mundial nada se relaciona el sitio por su parte, eso dada la indiferencia para con lo organizado en puntos excesivamente lejanos a la ubicación de los origines, no representativa del seguimiento que mantienen los anónimos en general, por lo cual igualmente mucho del dichoso tuvo con la trascendencia que estuvieron dándole los animadores, influyentes en el escenario tal cual demostraron a partir de las acciones esenciales. Etapas primeras de la disputa deportiva tuvieron un cierto nivel de cobertura visible en las comunes creaciones según el momento, antes, pero solo un poco y no para siquiera mencionarlo si como cualquier otro asunto, la cosa cambiaría con el avance del certamen.

Diferencia abismal que dejó a entrever los vamos Holanda, vamos Croacia, vamos Francia, cero de casualidad y todo coincidencias entre los nombres puestos en juego específicamente en el entonces que lo fueron, cada uno contrincante del seleccionado rioplatense que más logró escalar en las posiciones de la competencia observada era apoyado por varias participaciones, pudiendo ser del mismo hincha o saliente de múltiples procedencias, no obstante con resultados análogos en el caso que fuere, respuestas de acompañamiento y a su vez oposición, la que finalmente triunfó en todos los casos, los marcadores de una forma u otra alineándose con la coronación argentina, culminación de dichos hilos que en común también comparten el ida y vuelta de preferencias contrapuestas.

Esto aún cuando desde la administración no hubiera gesto alguno con aquello que tanto interesó y por múltiples semanas perduró, quitando el anaranjado clásico no circunstancial, la pelotita presente dijo en el antro y dando un atípico antecedente más allá de las discordancias entre los bandos que naturalmente se formen en competencias internacionales, no porque sin mucha variedad por los escasos aficionados surgieron igual, sino por los tanteadores completamente adversos al color predominante de turno, la cara de la derrota con imágenes contundentes asociada a quienes integran y construyen Fantemti, insólita o lógicamente yendo en contra de la bandera que más se repite en ellos, un dato que de creer en la mística no debería obviarse, la palabra que subjetivamente insinúa su validez.

\section{El reconocimiento de los conquistadores (Decimonoveno libro, capítulo V)}\label{el-reconocimiento-de-los-conquistadores-decimonoveno-libro-capuxedtulo-v}

\begin{quote}
20 de diciembre de 2022, \ldots, 24 de diciembre de 2022
\end{quote}

Típico episodio de las actualizaciones fantemtianas, venir con diseños nuevos que en realidad son viejos desaparecidos no obstante revividos, ahora el turno de una alternativa ni tan alejada del antro en varios aspectos además de mencionada, justamente con los festejos argentinos la novedad es que Arggnews tocó y por segunda oportunidad, los estilos de bienvenida utilizados también en la presentación decorosa de las mejoras, que la conducción volvió a realizar en la establecida parada devvoxera, del exterior.

Secuencias animadas para mostrar que las interacciones recientes pueden ser colocadas en espera con tal de cargarlas después, no necesariamente cuando sonara el ruido típico de aviso correspondiente, al suceder con respuestas e igual hilos, aunque requiriendo pulsar un par de botones a sabiendas del número que aguarda. Asimismo con esta implementación, el hacer aparecer tiene un tipo de función inversa, desde que el listado excede la cantidad mínima por segmento aquellos de más son descartables a modo de optimización. Aproximado a como tenían alternativas del contexto, solo que no en vías similares por la interfaz, las posibilidades de adaptación aumentan otra vez, independientemente de las demandas.

Las cuales no se dan, sin embargo por ese lado podría plantearse la misma pregunta que hace abundantes semanas, y la respuesta muy probablemente no cambiaría, ni bien esté quedando o retirándose este tema si lo no navideño no se quita ciertamente así sería, aunque la ocasión igualmente con algún gesto que referencie a las fechas, o en su defecto nada memorable para acompañarlas, mas sino la simple presencia del sitio, que tampoco es poca cosa.

\section{Rehabilitación del empoderamiento audiovisual (Decimonoveno libro, capítulo VI)}\label{rehabilitaciuxf3n-del-empoderamiento-audiovisual-decimonoveno-libro-capuxedtulo-vi}

\begin{quote}
24 de diciembre de 2022, \ldots, 28 de diciembre de 2022
\end{quote}

Tuvo un susto o exabrupto, el más que periquete que temporalmente bloqueó pasar al visitante inevitablemente por problemas en la plataforma, a presumir por negligencia de la gestión fantemtiana que sin tardar mucho reemplazó la negación por un letrero elaborado para la ocasión, al mismo tiempo que transmitía el informe desde el hilo de transmisión en terreno de las afueras, el medio de contacto típico cuando no es donde el propio lugar de los hechos. Mantenimiento y periplo de alejamiento intermitente anticiparon lo que posteriormente sería menos noticia, la actualización de escasos progresos aunque sea visibles.

Porque efectivamente, los antros más respondieron al rato de ausencia de plenitud en sus dos diferentes modos, el error interno del servidor y el tiempo de espera frente al mensaje del motivo que al principio no estaba, una circunstancia ideal para confundir pese a los antecedentes de regreso, que al final no fue tanto más que un lapso de bastantes minutos y luego el retorno con las novedades consumado. Pocas igualmente, como primordial que la multimedia no esté topeada en uno pero sí en cinco, que permite cargar más de único archivo o enlace, verlos una vez mandados, no de la manera óptima al ser secuencialmente y no en conjunto, sin embargo por lo demás bien. El segundo asterisco del número estrellado sintetiza y nada concreto dice además de detalles, errores que hayan quedado desde antes y diseño, lo más notorio de aquello cuando los botones entran a demorar por la noche. En acción fueron presenciadas dichas características, esencialmente por las pruebas que expuso la administración donde suele presentar con capturas y eso.

Y listo, breve empero rápido midiendo con referencia las últimas ocasiones así, con lo particular de que pudieran entrar en consideración a partir de sugerencias, sobre todo por el reencuentro con aquel ícono de cabecera en las horas tardías de Temti o Titiri, que cada anom navegando con el tema principal podría tener al accionar ciertos pulsadores interactivos, la luna llena por la mitad rotando sobre la posición, un elemento que no fue trasladado idénticamente, pero que cual referencia cuenta. ¿Avances con frecuencia? ¿Cumplimiento de pedidos? ¿Signos de la tradición? ¿Caídas de servicio? ¿Símbolos del exterior? ¿Cuánto de eso que no repetía tenderá a regresar?

\section{Noble cumbre del anhelo colectivo (Decimonoveno libro, capítulo VII)}\label{noble-cumbre-del-anhelo-colectivo-decimonoveno-libro-capuxedtulo-vii}

\begin{quote}
29 de diciembre de 2022, \ldots, 2 de enero de 2023
\end{quote}

Plazos curiosos para ingresar al contemporáneo bastión de encuentro temtiano, no esfumado y presente cuando muchos se dispersan y otros se ofuscan al resultar enganchados en el universal efecto del momento, y allí lo fácil de ver entre anclajes con o sin correlación.

Las creaciones inamovibles del inicio volvieron a rotar su segundo exponente tras el mensaje de fiestas, llamando así el hambre por encima de los sabores ya sin expansión. Titulando Flantemti, por casual ocurrencia particular repetida o desconocida iniciativa seria, la solidez en cuestión no mermó como el juego de palabras sugiere, y todo el tiempo en línea el clon se encontró. Fue posterior a la nueva característica de multimedia, no utilizada en lo absoluto e indiferente para la normalidad del sitio, una que brevemente se vio alterada por la caída de ubicaciones cercanas, durante esta época que a cuenta gotas temas nuevos recibió.

Siguiendo la programación típica de los espacios donde la interacción o contemplación desarrolla, en este caso puntual un antro que sin destacar en intensidad movimiento alberga, y pese a la frialdad o baja respuesta de antes, igualmente algo habría sido de esperarse, y efectivamente tuvo lo suyo. Con colores, símbolos y animaciones como cualquier otra vez, no al estilo de pocas especiales, el sitio internamente tampoco resultó apático al curso del calendario y las tradiciones que arrastra por sus constructores, aunque sea las más básicas y genéricas que puedan verse a partir de su implicación. Por tal motivo, porción de la comunidad semejante al promedio contabilizado en dilatadas horas se halló dirigida hacia un mismo lado, predominando las buenas intenciones con el destino ajeno y términos dulces representativos de ello, sin cuestionamientos que reduzcan el significado de la conmemoración, un título fijado marcando la frase de común denominador: feliz año.

Aparte de ignoradas actualizaciones y dudas sobre estabilidad estructural, que entran en lo predecible de la realidad fantemtiana, más llaman los destacados episodios de convivencia, cosa que muchas veces sucede por como la dinámica impone, sin embargo con argumentos no tan marcados, de lo que este podría distinguirse al menos un poco, más rememorando el antecedente titiritero frustrado. El precedente permitió la diferida reunión de los anónimos en casa y más todavía, fue concordante con los valores que en ciertas oportunidades fueron promovidos por la gestión de turno, expectativas a futuro que trascienden lo individual y que no surgen desde declaración o similar, solo se asumen de acuerdo a como viene la mano.

\section{El indicador redundancia (Decimonoveno libro, capítulo VIII)}\label{el-indicador-redundancia-decimonoveno-libro-capuxedtulo-viii}

\begin{quote}
2 de enero de 2023, \ldots, 11 de enero de 2023
\end{quote}

No tiene pausa la sucesión de las incidencias fantemtianas, en modo bajo perfil y sin deberse a exabruptos algunos, normalidad como podría llamársele.

Continúan los incrementos en el número que al ícono de ranura acompaña allá arriba, ya un novecientos cincuenta que es alcanzado a base de mejoras internas, detalles de la plataforma sin características llamativas que a la pasada mencionan ciertos puntos, típicamente sin extenderse, de lo que es posible exceptuar a las notificaciones, con específicos desperfectos encontrados antes de evidenciar los avances reales. Deja a entrever que la maquinaria todavía es revisada y trastocada, en ocasiones acorde con la exigencia de los anónimos y su comodidad, pero especialmente por inquietudes internas que a la larga mantienen el reiterativo titular de las correcciones y arreglos, aún cuando el funcionamiento general parezca correcto.

Proveniente de ese lado también, que se quitaran y agregaran encabezados a la par del principal fantem, los cuales como muchos anteriores destacaron aunque con menos sentido manifiesto para el común agente, caso que nuevamente irrumpe introduciendo en la órbita fija ocurrencias de lo más aleatorio, las que no muy extraño sería que así fueran escogidas. Salvando el regreso de las clásicas exploraciones semanales, unidades concretas colmadas de mensajes cifrados al entendimiento textual, ya típicos como las series de hilos de alta correlación entre sus portadas, de vez en cuando llegan a ser priorizados por sobre los demás y de tal manera complejizan lo primero que un visitante ajeno al escenario se encuentra, buen argumento para que luego sitio y sus contribuyentes sean tildados de raros, la medida del exterior acertivamente eso es capaz de concluir.

Hablando de por allí, no será nuevo decirlo, pero la conexión vuelve a hacerse fuerte al entrelazar los asuntos de interés, notorio cuando una cara que estuviera gozando de gran fama y repercusiones además fuera vista a mansalva en Fantemti, casi tipo inundación, que puntualmente generó quejas acerca del exceso de las imágenes llevadas de un antro a otro. Tierras que volvieron a recibir los escritos y difusión de literatura identitaria, y esta tras bastante dejó de ser exclusiva de su hogar, en momentos que el seguimiento de concreciones no va alzando la distancia cronológica, sino lo contrario. Y más al respecto de la actividad, sería que esta permanece viva y no del todo desacelerada, tanto con frases ilustres en francés, música adicional para el listado de colección, los intercambios frente a la administración, o conversaciones de rutina que redacciones familiares protagonizan, veteranos, emoticones, gatos, y demás.

Y allí está lo contemporáneo, coincidiendo con lo eterno, reincidentemente. El paso del tiempo surte efecto y entre condiciones actuales algo se corresponde a ello, sin embargo a veces no es notable sobre determinadas áreas, cuyo enfoque persiste dentro de todo incambiado, eso por los múltiples acontecimientos que siguen viéndose en el más estabilizado contexto del mundo temtiano, sin expectativas a que caduquen.

\section{Retratos en abundancia (Decimonoveno libro, capítulo IX)}\label{retratos-en-abundancia-decimonoveno-libro-capuxedtulo-ix}

\begin{quote}
11 de enero de 2023, \ldots, 3 de febrero de 2023
\end{quote}

Es sin características que lo distingan, para el clon del fanatismo temtiano, otro tramo fluctuante en cuanto a actividad y momentos pico de interacción de vez en cuando reencendido en conexiones y por el resto sumergido en descomunicación. Entre lo oficial se ha dejado ver, además de rotación en las colocaciones ancladas, un nuevo combo de actualizaciones dilatadas en el tiempo, de lo cual no mucho ha salido, el efectivo andamiaje y ciertos inconvenientes que rápido solventados quedaron.

Confirmado que todo está cambiando, por ahí a modo de exageración eso está dicho, pero mediante anuncios llegan las implementaciones de modernización y con ellas el antro avanza en pequeños pasos, mejorando sus componentes en puntuales detalles de desarrollo. Esencialmente la multimedia ha sido la beneficiada, empezando por el complemento del aporte anterior de cargar múltiples unidades, lo que pasa a ser posible en un mismo intento facilitando la selección. A su vez respecto al rango de vínculos soportados hay más miniaturas, de aquellas plataformas que no permitían vista previa la mayoría entraron a hacerlo. Además por ese lado, un flamante apartado para cada usuario de código donde encontrar los contenidos subidos, cual galería con enlaces adheridos a las participaciones conectadas. Y por último en relación a lo notorio, la inauguración de un validador de comodines, espacio en el que un pingüino de fondo mostaza recibe claves en compensación de, en teoría, registros.

Lo demás es mínimo así como lo dicen las sentencias referenciales, lo que factiblemente, o quizá no, esté asociado a los percances de servicio que se han sufrido durante unos ratos no muy largos. Así es, problemas de disponibilidad regresaron a perturbar los dominios fantemtianos aunque con un toque particular, no retornando descripciones vacíos o de terceros, por el contrario una pantalla deliberadamente diseñada para oportunidades de desconexión, un enérgico \emph{/Oh cielos!/} acentuando el panorama crítico, motivo para la rápida manifestaciónen los hilos devvoxeros de origen fantemtiano, los anoms acudiendo a ese medio como un tipo de asistencia o siquiera interrogación. Ya con el asunto solucionado, el clon recuperó su normal funcionamiento, pero con el atípico hecho de haber revelado un letrero de error hasta entonces desconocido.

También del nicho de los bumpeadores fantemtianos, la presencia identificable fuera de fronteras, más allá dentro lo extraño y habituado, la interrelación de diversa índole y en eso algo de convivencia sin ofensas persistente, reflexiones y dioses, dibujos y textos, siguiendo un corto etcétera menos general, en el cual podría destacarse la reiteración de mensajes altamente similares, replicas duplicadas que aún habiendo sido denunciadas la gestión no dejo de admitir, en una determinación al límite de la reglamentación. Y de susodicha parte, de vuelta: promoción a surgimientos extranjeros, ahora con una reciente reincorporación la costumbre repitió, ademas el amparo a los pedidos que buscaron lo propio, y otros que por incierta causa allí se posicionaron, varios casos adicionales para consolidar la realidad de prácticamente siempre tener un segundo fantem fijado en la cima, sin comunicaciones claras asiduamente.

En algunas de tales expresiones, la minoría en verdad, porque lo demás está sabido de antemano, las sorpresas que tanto tras tanto puede dar una breve visita al ocasionalmente incomprensible escenario de Fantemti.

\section{La insólita precisión de los titulares (Decimonoveno libro, capítulo X)}\label{la-insuxf3lita-precisiuxf3n-de-los-titulares-decimonoveno-libro-capuxedtulo-x}

\begin{quote}
3 de febrero de 2023
\end{quote}

En el hogar temtiano de la época, nuevos cambios son al traer luego de dos semanas, no tanta cosa sin embargo un par bastante apreciables, partiendo por el clave, categorías. Respecto al tiempo atrás que se había manejado la posibilidad de acogerlas, esta es la continuación y oficial, después de un largo rato sin finalmente el clon las consiguió y como tal los fantems incluyen ese chico apartado donde seleccionar entres posibles etiquetas que describan el contenido a grandes rasgos, pero en este caso solamente hay dos candidatas, lo general y lo no general. Dentro de esa bajísima especificidad es que han entrado los motones ya creados, indiscriminadamente, y las posteriores que vayan a sumarse en cambio sí podrán ser clasificadas en esa típica temática tan abarcativa. Y como complemento de la característica, opciones para invisibilizar la aparición de los contenidos pertenecientes a cada cual, allí permitiendo adaptar el inicio a la elección del anom, todo o nada.

Respecto al ámbito publicar también está referido algo de guardado automático, se ven pequeñas variaciones en la fracción donde eso sucede, no obstante discreto. Por otro lado y notable, mientras no haya elección tapando lo que rige estéticamente, hay más excepciones a la normalidad del tema principal, sigue siendo este mismo, aunque mínimamente variado. Las pantallas ahora mucho se parecen a lo del comienzo absoluto, Fantemti dando sus primeros pasos, pero con todos los chiches del desarrollo posterior. Clásica ocurrencia de que sea modificada la estética predeterminada, esta oportunidad actuando para retrotraer lo esencial de un año atrás, única columna conteniendo cada componente de los hilos y su listado central ademas, aún viéndose atípico o inusual en cuanto a lo que supone el formato, una manera de conmemorar el cumplimiento según fue señalado en el deteriorado espacio del tablón paralelo, a mantenerse temporalmente.

Y con un poco más que tampoco es perceptible, moderación y múltiples funciones que no están al alcance, cierra la lista de novedades, la cual cuenta con sus detalles particulares. Fuera de aquello, llamativo lo del dato calificativo que contienen y contendrán los fantems, en parte por lo tarde que emergieron o la entre comillas negación de antes, y más todavía dado el limitado catalogo dual, que tiene más de referencia a la antigua problemática temtiana con el popular subtitulo sobresaliente con diferencia del resto, que utilidad efectiva o rol de separación como los antros suelen darle. Por ello quizá pronto estén siendo añadidas más y acertadas, en su defecto retiradas de la plataforma, a su vez de recibir cuestionamientos que pretendan sugerir sentido común, o simplemente nada. Y en qué momento, momento que sin casi menciones al respecto viene avecinándose la fecha aniversario del sitio en sí, aparecen indicadores de que no será una jornada común y corriente, si bien lo manifestado es que para el sábado no había nada especial preparado, en principio los hechos podrían terminar contradiciendo tal dicho, habrá que ver llegado el entonces, que no falta nada.

\section{La causa validando su medida preferida (Decimonoveno libro, capítulo XI)}\label{la-causa-validando-su-medida-preferida-decimonoveno-libro-capuxedtulo-xi}

\begin{quote}
4 de febrero de 2023, \ldots, 9 de febrero de 2023
\end{quote}

Y llegó nada más, el aniversario específico del clon de Temti, Fantemti. Sin mucho sustancial para sumar a lo que había sido por adelanto, el reencuentro con la reproducible interfaz originaria y consigo las menos óptimas condiciones para lo que el formato ofrece, o de otro modo la distribución más unitaria, el salto de antes conservó las principales características tal cual eran y de dicho modo es que por anticipado comenzó a vivirse, con casi exactitud prácticamente un día entero antes. Aunque con el mote de oficial más entre comillas, también hizo parte de la jornada rememorativa un especial título dentro de las expediciones semanales, el viaje luego subido por el singular corresponsal en efecto tuvo dicha especial dedicatoria. Más además ni bien las horas hicieron llegar el día, la mención directa a lo que por fecha estaba concretándose, y las respuestas tirando para el mismo lado. Esas fueron condiciones en las cuales el cumpleaños fue reconocido, pronunciado y celebrado aunque discretamente, fecha que lleva a recordar un acontecimiento que para propios evidentemente vale la pena destacar positivamente. Y con lo primero más notorio permaneciendo, una nueva suerte de paralelismo con la temática que tan valorada es comenzada en el microambiente, volver hacia atrás manteniendo un pie en el presente.

Pero no se trató simplemente de eso en cuanto al último retoque en profundidad, este dio para más, la escasa variedad categórica dio para hablar sea o no con seriedad, eso de que solo hubieran dos dejó en claro un montón de posibilidades que los anónimos creadores a lo mejor habrían incluido en el catalogo de ellos haber elegido, la exagerada simplificación representa mucho y demasiado, exceptuando lo intuitivo, no obstante esas aparentan ser las definitivas al menos a corto plazo, junto con la permanencia del estático catálogo de obligatorios sabores, la particularidad de este extraño antro es que realmente están, sin estar a la vez.

Trama adicional que tendrían las expresiones culturales para plasmar por la vía que sea, los típicos estilos que habitualmente acompañan tarde o temprano el acontecer siguen apareciendo al menos pocas unidades y en especial un gráfico que atípico se hizo, entre expectativas y multicolores extraviados la colección de elementos característicos que pudo remarcar como esenciales una perspectiva, mitos y verdades que no son de afuera sino acá, Fantemti, el sitio y las no convencionales distinciones del administrador, conjuntamente con las reiterativas actividades de su comunidad no muy magnificada sino lo contrario, personajes con renombre y momentos de posicionar, más un corto etcétera. Es muestra de que se tienen más ojos interpretando los sucesos corrientes y hay el suficiente contenido como para recrear a partir de ello, y no es poco a decir, pues si al contrario de como teoriza el iceberg todo tiene una misma procedencia, las construcciones colectivas de hecho lo son, y el año subsiguiente más almas han de presenciar.

\section{La serie del trasnochado en extremo (Decimonoveno libro, capítulo XII)}\label{la-serie-del-trasnochado-en-extremo-decimonoveno-libro-capuxedtulo-xii}

\begin{quote}
10 de febrero de 2023
\end{quote}

Cuestiones estéticas envuelven Fantemti, nuevamente, la noticia que vende aunque pocas o nulas editoriales y esencialmente para la perspectiva oficial, más consecuentemente algún despistado que no esté al tanto de la movida, aquella propia de cuando son implementados diseños alternativos adicionales, ahora de primeras visualmente drástica. El anuncio está y allí podrá ser la posibilidad de regresar a temas tradicionales, pero como siempre al no asentado sí o sí le modifica el entorno y de ellos sigue habiendo, cosa que igual le dará a los planes de exhibición y su cadena rotativa de estilos. Por el mismo lado de la oportunidad previa y de vuelta sin razón puntual, el portal de Arggnews, de alta variedad en este sentido, complementa las dos opciones ya copiadas con más combinaciones de épocas enlistadas por mes, entre las cuales una fue escogida para ser la expresión predeterminada del sitio y desde enormes diferencias en los elementos fundamentales marcar quizás el contraste mayor a partir de que la modalidad de imitaciones corre, todo llevado al mínimo indispensable, todavía usable, sin embargo fuertemente simplificado.

\section{Cultura de retribución (Decimonoveno libro, capítulo XIII)}\label{cultura-de-retribuciuxf3n-decimonoveno-libro-capuxedtulo-xiii}

\begin{quote}
11 de febrero de 2023, \ldots, 14 de febrero de 2023
\end{quote}

Nada raro en Fantemti, o sí. Entre los normales términos de actividad es la prolongación de un periodo esencialmente tranquilo, ahora con enlaces descontextualizados en los espontáneos hilos anclados, y un evento anterior que no repitió sino que estando ausente varió, más la prosecución del despliegue estético con orígenes extranjeros. Cero de ver para creer, solo esperar para confirmar.

En el exterior dichos de la conducción fantemtiana ratificaron la programación típica de las imitaciones, las vigentes en el presente marchando como a la usanza lo han hecho, en esta oportunidad con tres noches para la primera de las diez pertenecientes al clon arggero, que llamativamente además de lo mencionado tuvo un particular enfoque recopilatorio y artístico, no concerniente al sitio que en la actualidad lo recuerda, mas si quizás parte de un pasar obsesivo u homenaje especial que en este se origina. Regresando allí, son los estilos recientemente implementados que continúan su extensa linea de exhibición, la que tiene a las copias turnándose como predeterminadas, en este caso progresivamente. Y como estado propenso a causar ciertas impresiones siempre y cuando haya asistencia, el combinado de apariencias previo con su básica interfaz logro provocarlas, y elevar la fama de rareza que en los últimos tiempos a ha caracterizado al comportamiento de la alternativa en su conjunto.

Y más días completan sus jornadas de un lado con las cabeceras rojas, que no resultan totalmente ajenas al transcurso de las fechas aunque la tradición no sepa mucho al respecto. Sin referenciar al Día del amor a Temti, los corazones igualmente saltaron a la escena, siendo respaldados por la plataforma mediante condiciones globales, en el nombre de cada anom tal símbolo entró y sustituyó, nadie desde que la hora inicial llegó escapó, así es que la data mundialmente reconocida coincidió junto al nuevo gesto del sitio artista, magnificado en los ocultos recuadros a partir de un cambio aplicado. Simultáneamente, la comunidad sin manifestar discordancias, alzó la voz en sintonía con la tendencia que al contexto incumbió, en un sentido fraterno. Ante la fría actividad corriente, el cálido rellenar momentáneo.

\section{Las cariñosas elaboraciones sintéticas (Decimonoveno libro, capítulo XIV)}\label{las-cariuxf1osas-elaboraciones-sintuxe9ticas-decimonoveno-libro-capuxedtulo-xiv}

\begin{quote}
15 de febrero de 2023, \ldots, 24 de febrero de 2023
\end{quote}

Sigue el amor en la red de la comprensión, el atributo básico apuntado por eventualidad sin salir de su lugar, por mucho rato modificando el atípico seudónimo del común emisor fantemtiano, es ese sugerente de corazón que todavía está entre el corto término de anónimo. Luego de su inesperada ocurrencia al principio de aquel día donde en teoría correspondía, el único detalle aparte de lo hablado, es de cierta manera un término que vino para quedarse, o que va demorando la retirada. A favor de dicha primera posibilidad, las referencias a la permanencia implícitas en recientes ediciones de la administración, y también, la verdad del antro y su simpatía a darle ese toque afectivo a los intercambios, modo que de vez en cuando se vuelve fuerte ante las no erradicadas influencias opuestas, además de no haberse generado disconformidad considerable en el ambiente. Elementos con los que concluir y tal vez criticar después, que quizá vaya a sostenerse por buen tiempo.

Más por el lado del sitio y sus condiciones, no pasa naranja y si otros colores ajenos, que dan para la muchedumbre que de los exteriores llegue a mirar, si es que una ínfima porción se logra a asomar, con lo poco que suele pasar, allí donde las apariencias parecen cambiar. Siguientes opciones del paquete extranjero repleto de imitaciones, la rotación de tales temas mínimamente dispares desde la copiada base, asignados a perdurar durante múltiples jornadas en cada opción individual, y producir postales adicionales con las que el conductor y el canal una galería serían capaces de hacer. Sin finalizar todavía la programación de esperar, ya menos de la mitad marcó un tramo cronológico histórico, las exhibiciones de estilos que progresivamente van creciendo en duración, y no necesariamente en atención.

Una vez más acontece en paralelo, es nula la respuesta a la iniciativa y dinámica que la gestión propone, los fantemteros andan en las suyas, no obstante tampoco se alejan entre sí. La remoción de contenidos estuvo reiterandose, a juzgar por lo que el escenario deja, es de lo prohibido que aparece y desaparece, ahora normal al ser unos casos como cada tanto ocurren, pero ya relativamente más raro sería si las sucesivas repeticiones aumentan, así que habrá de verse. Otro factor que a lo mejor ocupa y esconde, las participaciones de presunta procedencia programada y automatizada, esas que con el tiempo han sido repetidas y extrañan un montón al visitante crítico, son una tendencia a veces notoria, justamente cuando las inteligencias artificiales van en auge. Pese a que estas en particular no demuestren ser muy listas, siempre en algo llevarán razón, quien desee afirmar lo contrario, realmente no puede saberlo.

\section{Los arcoíris de los arcoíris (Decimonoveno libro, capítulo XV)}\label{los-arcouxedris-de-los-arcouxedris-decimonoveno-libro-capuxedtulo-xv}

\begin{quote}
25 de febrero de 2023, \ldots, 13 de marzo de 2023
\end{quote}

El periodo de la nombrada exposición finalizó, ese sería el enunciado que el orden sitial subrayaría por la ocasión, y tras la despedida del último estilo argento, la versión predeterminada denominada como linda anunció su vuelta, Fantemti, entre más noticias producidas durante la alternancia del semejante antro homenajeado, dentro y afuera.

Mientras las opciones estéticas eran desplegadas en su mes particular, el evento de diseños más extenso consumado, la gestión también mandó aunque sea un par de complementos funcionales, en las que puntuales componentes brillaron coloridamente por progresar, cubrir faltas antes no contempladas. Primero como característica, un versátil comportamiento dota el área de responder, acompañando cuando la visión lejos de su posición inicial vaya. Segundo, el doble ordenamiento que comprimió los puntos menos necesarios, optimizando así el uso de espacio y demás. Tercero, las numerosas omisiones corregidas en el área de modernizaciones, cientos de anotaciones de la senda evolutiva pasan a completarla. Cuarto, la carga automática de secuencias gráficas activada globalmente, en sintonía con las publicaciones habituales. Quinto, la repuesta a un pedido que solicitaba ayuda para redimensionar los cuadros de escribir, a lo que salió otra opción útil en las circunstancias aludidas. Y sexto, más avances no esenciales, o invisibles.

Hay un párrafo adicional en relación a la sección más retocada, dedicada a mencionar los detalles de la clase de situaciones que atravesó recientemente, la ampliación del registro hacia atrás de la actualidad, ahora incluyendo múltiples palomitas que esconden novedades de diversa categoría y no exclusivamente la homónima del apartado en cuestión, además de que las muchas sin descripción recibieron lo propio, de principio a fin, con una elevada disminución de entradas, el recuento luego referenciado sobre la cabecera, por cierto no más visible. La parte extraña y no advertida de esto es justamente la baja de perfil que padeció el seguimiento comprendido allí, aquello que hasta entonces era prioridad señalarlo dejó de tener su amplificación numérica y por ende cuando los cambios surten no cuenta de indicador inmediato que en todo el sitio advierta al respecto, solo cuando un comentario reportaje fuera emitido o la correspondiente lista fuera visitada, únicamente en esos casos habría manera de enterarse, y objetivamente resulta ser un paso atrás, aunque no estrictamente peor, dependerá de si realmente pormenoriza algo relevante o a nadie le importa.

Ya de lleno en el relleno, empieza por el ejemplar del conejo hipnótico, que es a destacar, de la información y desinformación que han visibilizado los planteos anclados, otra vez titulares con eco posteriormente moldean siguientes discursos, las palabras de las palabras, las copias de las copias, el etcétera del etcétera. A su vez en similar materia, el consolidado centro emergente de la movida exploradora, del que no asombra su recurrente posicionamiento prácticamente cada sábado, una carta fija y apreciada que hace de nexo con el sinfín de cosas generadas en torno a los subyacentes mundos de vasto tipo y tono, apoyo para la galería de gráficos estáticos y secuenciales donde reviven los momentos vividos allí. Y qué decir acerca del típico anhelo de regreso, es la conservada gratitud oficial, parcialmente escoltada, a su vez el confirmar de que aún hay ganas de ver volver al antro antecesor y de origen, junto con el fundador que menos se lo olvida, no obstante pocas expectativas lo encuentran cerca. Y por ese lado seguramente haya más, en un asunto que puede crecer, mientras la conducción permanezca eligiendo y rotando tendrá lugar que dichos mensajes se conviertan en elementos emblemáticos y de que una forma u otra perduren en el tiempo, como hay de los ya instalados en la cultura fantemtiana.

Aunque sin tratar paréntesis trascendentes de más, suceden sucesos que no son vacíos, la renovación estructural después del año cumplido habla de las energías que todavía percibe el clon por parte de quien lo mantiene, y que eso consista de más que simplemente mover portadas da mejor concepto sobre el labor realizado, y que conste hubo felicitaciones entre la gente. Y siendo eso independiente a la introducción de nuevas caratulas, los participantes del escenario traen consigo regularmente más temas de discusión, animados y variados en cuanto a contenido, como hace rato viene pasando, dando a suponer que continuará pasando, valga la redundancia detectada y escrachada, pero igual replicada.

\section{El sueño desde la punta flotante (Decimonoveno libro, capítulo XVI)}\label{el-sueuxf1o-desde-la-punta-flotante-decimonoveno-libro-capuxedtulo-xvi}

\begin{quote}
13 de marzo de 2023, \ldots, 16 de abril de 2023
\end{quote}

La estabilizada y extendida corriente fantemtiana con sus cíclicos argumentos marca presencia cuando la novedad se reduce a viajar y bumpear, consolidando los mensajes de la trayectoria cercana en circunstancias que esta los extraña, entre los propios conocedores e incluso gringos desentendidos.

Con el atenuante de al menos dos caídas reclamadas demorando su esperada resolución, el dilatado rato que viene siendo atravesado no suma tampoco desde el desarrollo expansivo para el sitio en cuestión, únicamente mínimos detalles han reformado las estructuras en los meses que van, y la gestión solo actúa un poco más allá de su rol básico, anclajes semanales y respuestas irrelevantes que ratifican algo de disposición. Que por la otra parte también lo hay, la comunidad todavía percibe intercambios bajo la frecuencia contemporánea, correspondiente a la época que ocasionalmente se prende y hace notar al recuento de pájaros como un estimador verosímil, entre los demás intervalos fundamentalmente saturados de silencio que lo vuelven incierto.

Levemente discreto en términos culturales a su vez, lo propio de la identidad ha regresado enfoques surgidos anteriormente, los créditos sociales y la revisión e interpretación de cosas llamativas. Mientras no hay comandos o registros internos de dicha primera medida, el pueblo fantemtiano evalúa la actitud de los camaradas e informa cuando lo ve pertinente, visto que últimamente cada participación puede ser razón de retiros en la cuenta del anónimo que replica, siéndole exigida autocrítica por los contenidos pronunciados, y más probabilidades de descuentos complementarios que de justificaciones o devoluciones son las que suceden a tales avisos. Las demás costumbres tampoco se pierden, distintos momentos de la historia son insertados en nuevas imágenes temáticas lentamente generadas y difundidas afuera, ampliando la base de citas gráficas, cada tanto mezcladas con humor de dudosa calidad.

Al parecer un poco distanciado de aquello están, los hallazgos del reflotado iceberg, porque nuevamente sus conceptos han sido empleados en el contexto, logrando que esa modalidad ascienda en atractivo. Esto incluye creencias populares de un supuesto gran autor de narrativas apodado Temtiano, de quien saldría la mayor o total parte del relleno visible, en ciertas menciones conectado con el administrador, a menudo el o los únicos usuarios asiduos según también a veces dicen por ahí, atípica correlación entre pares además repetida con el clon de Temchianu. Como teorías acerca de verdades hay referencias profundas adicionales, elementos de mística transformada y tendencias sostenidas, varias sobre las cuales la vigente literatura no redunda al igual que la realidad. Comentarios y notificaciones distinguidos por funcionar raramente, los comportamientos de corte conciliador entre reuniones y lenguaje, el reiterado acto de revivir y desvirtuar tópicos a gusto, otros motivos más obvios, es lo que supieron destacar en tal esmerado armado.

Pues no parece raro, a juicio de los que habitúan encontrarlas y apenas o nada criticarlas, que estas podrían ser las crónicas del sitio en el siguiente largo plazo, haciendo cada vez menos comunes los eventos extraordinarios o cambios oficiales, así queda de protagonista el ingenio colectivo y el relativo interés de sus no exclusivos animadores, los cipayos escucharán lo que tengan para decir, rechazando sus hipótesis difamatorias, o formulando más de ellas.

\section{El refuerzo abandónico (Decimonoveno libro, capítulo XVII)}\label{el-refuerzo-abanduxf3nico-decimonoveno-libro-capuxedtulo-xvii}

\begin{quote}
17 de abril de 2023, \ldots, 19 de abril de 2023
\end{quote}

A partir de lo que son accidentes ya un tanto comunes, de los que Fantemti rápido se entera, la interna recibe sus impulsos de predecible caducidad, no obstante poco acostumbra a cambiarse más allá de eso, si es que a semejante circunstancia corresponde, una vez como esta podría decirse lo contrario, y dejar registro de un antecedente aunque de leve relevancia, único.

La mala fortuna de Fran Velvet sirve a los efectos de justificar y más polemizar, cae Devvox y lo vienen a contar al refugio, para algunos sujetos predilecto, pero solo hasta ahí, claro e implícito manifiestan que es por el momento y de mientras algo más puede hablarse, ver por ejemplo el autodeclarado rey. Si el dueño no duerme tal vez festeja por el tráfico sin declararlo, porque tampoco debería preocupar cuando en principio el comportamiento es apropiado, aquellos no tan entendidos solo con la presencia inflan la valoración colectiva del antro, de hecho las dos cifras son más convincentes que habituales y bastante suelen apreciarse.

Vibras entusiastas que se comparten junto con el tiempo, dan lugar incluso a que típicas actividades recreativas sean iniciadas e involucren participantes que hasta escogen un nombre para competir. Juegos de unicornios parlantes, redondeles futbolistas, y teléfonos descompuestos, unas de las tramas que la gestión subraya con morado, propuestas que aparte difundieron en el sitio vecino cuando este fue restaurado, lograron generar gratos encuentros donde también interesantes ideas quedaron capturadas.

Y sobre anclados consiste la eventualidad destacada por demás, de vuelta originada en una solicitud individual y validada por el mandamás, resaltar todos los fantems, cada uno de ellos. Supo verse y bastante que títulos específicos recibieran la dicha, sin embargo unitariamente, de a dos o tres, en esas bajas cantidades el arte de poner y sacar stickys era, no a los montones existentes, con lo polémico y rechazado que se incluye en los muchos allí. Aunque en orden parecieran posicionarse igual y solo un número limitado fuera visible de primeras, los pins abundaron y resultó raro verlo así, además de que al sostenerse la medida emergió otro inconveniente que cuando fuera identificado tampoco fue tratado como tal.

Porque al encontrarse los cuatro mil y pico creados con prioridad en el ordenamiento cuadriculado, los nuevos que se sumaran no adoptaban el mismo nivel y surgían casi que invisibles, dado que para el inicio eran sus últimas entradas. Así las publicaciones posteriores a la extensa determinación directamente no aparecían y eso tuvo punto de discusión en el medio oficial, pero aun así quedó la problemática o gracia vigente, un par de días los recientes estuvieron prácticamente sin estar, condicionando negativamente una fuente de interacción.

Hasta que por casualidad la dinámica limitada coincidió con algunas puteadas, un pedido de disculpas acompañó a la normalidad retomada, la meseta fantemtiana que sigue sin mayores desordenes, aparte de los que siempre tiene.

\section{El bocado que los técnicos reparten (Decimonoveno libro, capítulo XVIII)}\label{el-bocado-que-los-tuxe9cnicos-reparten-decimonoveno-libro-capuxedtulo-xviii}

\begin{quote}
20 de abril de 2023, \ldots, 26 de abril de 2023
\end{quote}

Un par de antecedentes especiales marcan al clon temtiano y fantemtiano, más que tiempos de simplemente hablar y callar, dos demostraciones más de lo rico que es hasta en su definición, Flantemti.

Con lo que bajaron su frecuencia, que se den amerita más todavía su reconocimiento: cantidad de cambios aplicó la gestión. No debería ser el caso de esas veces que manda mucha descripción diciendo poco, si vale la pena enunciar más que un breve punteo. Esto es, Fantemti ya tiene audios, hay aunque escondida una posibilidad de grabar notas de voz o parecido, y de eso rellenar una participación como si de cualquier material tratara. También incluye ahora un botón de cruz para remoción de lo dicho, si uno publicó durante un rato puede deshacerlo y que nadie más lo vea, función atípica en el contexto que es semejante a otra antes pedida y parcialmente aceptada en el medio oficial. Adicional entre aquellos es la alternativa a generar nuevas credenciales cuando al momento de proceder era sugerido o requerido, dado que esa ventana sumó la opción de acceder con unas previamente creadas, y continuar más rápido.

Sigue lo notable de esta actualización, por lo menos de lo curioso y según se lo vea, cómico. Si antes los temteros le daban a los pancitos, los fantemteros, en inclusivo por cierto, hacen lo propio con el flan. El refrigerio preferido por el anom promedio resulta venir de postre en la flamante imitación de estilos, precisamente correspondiente a la época y locación que tuvo pintadas de amarillo las cabeceras degradadas, por alguna razón entre los disponibles cuatro coloridos del esquema, el seleccionado para ser el principal y único establecido, luego de dos años y pico. En conjunto esta oportunidad las características produjeron más sensaciones manifestadas de lo usual, por lo menos los sonidos tuvieron sus primeras apariciones en el debate, eso sí, mudos y de ruido ambiente, además de que el detalle del diseño viene simpático para lo que varios zombis dijeron querer, por segunda vez la demanda lo ve servido en la esquina superior.

Y unos días después comienza lo entre comillas normal, solo que de manera no habitual, aunque suene contradictorio, los típicos refugiados que caen cuando el antro preferido carece de lugar para ellos. Siempre declarando lo que efectivamente ocurrió, estimulan la actividad de forma extraordinaria y aceleran la tasa de intercambios, lo mismo de cada ocasión similar, pero en esta la situación se prolongó casi un día, entre la incertidumbre de sí realmente regresaría y los supuestos anuncios de allí que determinada confianza recibieron, hasta encontrar la demorada restauración, luego los foráneos siguieron el destino de partida, sin error de origen. No únicamente al completar la diáspora, durante esta también, el contraste de contenido hizo resaltar la modificada conformación de público, cuando ciertamente parte de este se impresionó con lo que encontró en su visita temporal, además de desconocer aspectos básicos del funcionamiento local a veces considerado mejor, no obstante luego opacado desde la decisión tomada por la mayoría, volver.

La predicción de los próximos episodios, además de la desacelerada evolución cimental, pasa a tener mayor probabilidad de que lo exterior altere a lo interior. Ya era de darse, sin embargo se está reiterando más, a medida que los desperfectos crecen en protagonismo, de lo que Fantemti tampoco escapa, porque a su vez supo verse que falla. Quitando eso, la experiencia de hospedar a numerosas familias conviviendo sin casi restricciones pareció correcta y visto el desenlace, podría ser un estado deseado a mantener, pero sí que sería más improbable que otra cosa, por lo menos la historia cíclica así lo ha dicho, incluso con los desequilibrios que más serios suponían ser, el sitio es apático a tales escenarios.

\section{El rincón comunal no autosostenible (Decimonoveno libro, capítulo XIX)}\label{el-rincuxf3n-comunal-no-autosostenible-decimonoveno-libro-capuxedtulo-xix}

\begin{quote}
27 de abril de 2023, \ldots, 22 de mayo de 2023
\end{quote}

Poniendo el énfasis nuevamente en el revuelto y ordenado Fantemti, interesa mencionar lo más notorio de los sucesos no tan típicos como para pasar por alto, de hecho recién extraordinariamente destacados del tipo que antes no lo habían sido, además de alguna otra cosa que no registra antecedentes en la continuidad que tanto viene protagonizando este enfoque.

Saliendo del impulso que las vecindades repitieron sin quedarse una vez más, la rareza episódica igual tuvo de quienes apoyarse para repercutir, individuos que descubrieron el descalabro y dejaron más que un testimonio sobre lo que bajo la mayoría de puntos de vista sería un problema a resolver. Andaban mal las imágenes, lo cual no se dice en modo irónico o ignorante con las condiciones corrientes del antro, sino por efectivamente ser una falla en la visibilidad de ese componente clave. Ni las portadas ni lo interno daban señal, lo que aparentemente era un percance parcial del proveedor, siendo que este de origen tenía sus constituyentes operativos, pero para su utilidad fundamental no estaba sirviendo del todo. Esto dicho así porque la gestión inicialmente fue indiferente al inconveniente, al ser incapaz de detectarlo y deducirlo por sus propios medios, dependió del aviso anónimo para luego de un extenso intercambio determinar respuestas, alternar el servicio de las subidas ulteriores. Asimismo cargas anteriores permanecieron invisibles según el colectivo, hasta aproximadamente un día después, cuando los afectados comentaron que Imgur ya funcionaba bien, la administración restauró el uso común y pronto, como si nada hubiese ocurrido.

Viendo más de lo que supo decirse, asunto serio por los potenciales efectos a generar sobre el contexto referido, aunque relativizado por las reales consecuencias todavía manifestadas en la coyuntura global, está la difusión del sitio y su relleno, llevado más allá del reducido sector que los clones delimitan. Complejidad que otros supieron y saben confrontar, por más inviable que parezca, sale como posible oportunidad y amenaza también para el creativo portal de Fantemti, siendo aplicado el polémico modelo en la red del pajarito. Sabidas las incompatibilidades entre antros, el resultado parecería desfavorable para la parte promocionada, y de vuelta entraría la incumbencia de la custodia, en esos pedidos de auxilio, responder a la tradicional frase de hagan algo.

Por otro lado, a su vez concerniente a intervenciones, surgieron nuevos desordenes provocados por la comunidad. No es que tache el buen concepto que para varios se ha formado a partir del correcto involucramiento presente en una considerable porción de la actividad más próxima, que quizá sea exagerado definirlo como tal, sin embargo las excepciones son eso, incidentes aislados. La cuestión es que pasaron, ahí es notorio el vacío de contenidos tras la remoción que sus gráficos indican, y en particular un caso adicional únicamente patente al momento de producirse y no mucho más luego, inundación de expresiones demasiado largas. Si durante horas estuvieron y después ya no, tendría que deberse a la eliminación integra dictaminada desde arriba, que para las maneras de tratar con lo ilícito mantenidas hasta entonces, resulta inusual.

En virtud de sostener un concreto orden en la dinámica del sitio, lo evidente es que solo no se mantiene y la potestad pertinente tiene lo suyo por realizar cada tanto, lo cual en la configuración actual lleva a un estado que pende más aún de quien o quienes se encuentren manejando estos dramas. Respecto al plural, pese a lo sumamente entendido en el ambiente acerca de la conformación del staff, la falta de aclaraciones contundentes, y los dichos que aluden a supuestos poderosos, no permiten cerrar la disyuntiva. Que igualmente carece de sustento suficiente para introducir el tema, como sucede en otras cuestiones donde lo rebuscado incita a polemizar, lo que posteriormente no alcanza tomar mayor dimensión. Pero dadas las circunstancias cercanas de demanda, y teniendo en cuenta experiencias de los antros semejantes, especialmente aquel rol en el sentido singular o múltiple podría volverse más relevante, si cualquier factor cambió o lo está por hacer. En base a tales condicionales, los ciclos y reiteraciones ganan más importancia, por la cantidad de eventualidades derivadas de ellos.

\section{Los procesos ratificados (Decimonoveno libro, capítulo XX)}\label{los-procesos-ratificados-decimonoveno-libro-capuxedtulo-xx}

\begin{quote}
23 de mayo de 2023, \ldots, 6 de junio de 2023
\end{quote}

Ánimos a la escena de Fantemti, originados y difundidos por medio del ambiente de quietud no permanente, lo reciente que este acumula y exhibe temporalmente da una serie de titulares extraordinarios, como producto de un sitio aferrado a sus valores clásicos y el complemento de unos autores inspirados.

Lo nuevo proveniente de la parte de mando es más de importancia simbólica y menos de trascendencia evidente. Algo se menciona sobre procesos automatizados, aunque no hay mucho sabido de ello, así que tal vez solo incumba para la interna. Por lo demás, lo externo, el foco está en la antigüedad, el plazo transcurrido desde el registro inicial, el cambio hace que dicha información pase a estar con más precisión para cada quien que tenga un código, en la sección de estadísticas. Es un detalle que sale a pedido, y a su vez resulta curioso que haya más atención al tiempo atravesado en el sitio, ya que como fue criticado del manejo de permisos, en oportunidades anteriores pudo ser un criterio de restricción para el uso de funciones, y considerando las numerosas llegadas de participantes inexperientes, quizás adquiera mayor relevancia.

Pasando, la actualización especialmente trata de un elemento atípico para su respectiva modalidad, los estilos de tipo copia pero ideados en un exponente indefinido como se lo referencia, que de conocimiento público no llegó a ser lo completo que en este caso es, sino un concepto aplicado a pocas piezas. Sin embargo de vuelta, el clon temtiano lleva lo original más allá de lo que era, y esto dio lugar a una estética simple y oscura semejante a la de aquella extraña y entreverada época, que tiene nula relación aparente con la corriente, pero salvando el punto de la improvisación, que a su vez hace más fuerte el conjunto de homenajes a las raíces, sucede en el marco de la misma práctica que fabricó las imitaciones previas. Insertada en dominios fantemtianos, supo mantenerse varias jornadas, para las perspectivas que dieran con el tema preestablecido.

En sintonía con la dinámica de crear y mejorar, el sitio como tal se ha visto implicado en la evolución de un nuevo ejemplar de clase similar a los suyos, otro espacio donde sin seudónimos la generación y visualización de contenido es lo principal para hacer, aunque tampoco ajustándose íntegramente a la clasificación de clon. El vínculo arranca en la difusión inaugural de este proyecto, extendida en las cercanías afines a su propulsor y recibida por los anónimos, que conocieron y pusieron a prueba la plataforma en sí, unos viendo potencial, incentivando, aportando, y sugiriendo, además de lo contrario, pero bien que lo primero sobresalió, las prestaciones básicas han ido avanzando rápidamente desde entonces. Siendo esta iniciativa apoyada repetidamente por Fantemti, en su política de anclajes y parecer replicante, Simmplesocialmedia encontró allí un lugar en el que contar con mayor interés las ideas concernientes a su presente y futuro.

Sin ser el único tema llevado a las posiciones elevadas, como marca la costumbre y no la casualidad, múltiples hilos obtuvieron ubicación de fijado, y aparte de ser rarezas que la mística local ve con preferencia, también hubieron fechas puntuales, más que simples jueves. Entre ellas, incidencias de reconocimiento a lo señalado excepcionalmente por el calendario no francés sino argentino, para esa convivencia internacional en ocasiones indicada desde la selección de banderas, concentrada alrededor del sol del quinto mes, emblema que viene a cuento cuando un recuerdo pertinente lo trae de trasfondo. El día del locro sería el encabezado dado a una especie de evento que el sitio declaró tardíamente, motivado por el menú común que algunos compartieron y lograron promover hasta el catálogo de sabores, que durante la jornada conmemorada exclusivamente admitió ese gusto en concreto, para dotar al menos una decena de fantems preparados con las influencias del orgullo o tradición, en un contexto que normalmente los valora.

Respecto a la concurrencia e involucramiento de aquel reparto, sigue ocurriendo que las procedencias tienden a variar acorde el número de enchufados crece, a su vez que las etapas nocturnas perciben esa actividad vigorizada, pero en una de esas la movida sumó para la cultura del sitio en lo que puede decirse del mismo. Roles de inherentes a la gestión inciertamente oficiales fueron asociados al nombre del antro, antes un monarca designado por voluntad individual, después un presidente proclamado por elección colectiva: cada procedimiento tuvo sus polémicas y así igualmente lo resuelto posteriormente modeló las discusiones, en torno a lo que cada personaje metido transmite sobre sí, y fundamentalmente la impresión dejada al resto de anoms sin denominación establecida, entre aceptación y rechazo, gratitud y deslealtad, etcétera. En particular, la determinación popular naturalmente se manifestó con más alcance y contundencia, aun cuando los resultados contestados estuvieran sujetos al vulnerable mecanismo de voto que el clon dispone, así la trama de un tal Cromax y su comprometido respaldo durante días ha venido impulsando las andanzas fantemtianas tanto dentro como fuera de sus propios tablones.

Minimizando altibajos y juntando todo, a grandes rasgos distinto sería eso, de momento. Aparte de lo incluso eventualmente fijo o periódico, son circunstancias que por cierto inciden más allá de su puntual surgimiento, a veces integrando un plan en ejecución que supera la inmediatez, o tan solo ampliando algún pensamiento que la coyuntura tuvo la suerte de quedarse.

\section{Clarificaciones de oscura legitimidad (Decimonoveno libro, capítulo XXI)}\label{clarificaciones-de-oscura-legitimidad-decimonoveno-libro-capuxedtulo-xxi}

\begin{quote}
6 de junio de 2023, \ldots, 14 de junio de 2023
\end{quote}

Sin contar lo tanto otro, en Fantemti para la administración hubo una simple sugerencia, y dio lugar a más, pudo verse a partir de la misma modalidad de antaño en su presentación estándar. La novedad llega de donde poco se sabe, refiriendo a las condiciones que implícitamente operan día a día, pero en su expresión directa y redactada. Según esta transmisión de presunta relevancia dice, hubieron cambios intrascendentes, puras aclaraciones de cosas que ya eran aplicadas en los protocolos de orden y gestión, los criterios de permisividad y moderación, y además anotaciones informativas. El contenido de susodichas secciones aunque de manera rebuscada puede revisarse y eventualmente compararse con el anterior, para ver las modificaciones puntuales y lo que estas impliquen, las consideraciones adicionales que fueran agregadas, y no más que eso, si es que de verdad lo nuevo es viejo.

Teniendo en cuenta esa pauta y creyendo que así es, no existirían mayores razones para interiorizarse con las adaptaciones, en caso de darle importancia a la parte práctica del desarrollo en cuestión. Y al menos por lo evidente esa fue la respuesta mayoritaria, cero cuestiones trataron al relleno específico, solo atendieron a lo cómico de la alerta en sí y aludieron al rigor que determinaciones del estilo tienden a generar. Tal vez luego hayan leído y no señalado nada al respecto, pero el supuesto llevaría subestimar la prudencia o interés de los sujetos, y por consiguiente a creer que nadie o casi nadie lo habrá hecho, y esa es la incidencia sobre la escena presente, nula.

De todas formas, allí podría haber alguna precisión interesante para el sitio en sí mismo. Recordar la versión previa mostraría que las redacciones varían más que en simples pormenores, además de las aseveraciones en lo concerniente al antro y su sentido como tal, entre las que estaría comprendido el planteo motivante, también las áreas sensibles se vieron un tanto expandidas, trastocando e incluso añadiendo. Pero, todo eso y lo no dicho ya estaba siendo efectivo, sin ser declarado\ldots{} habría que ir más al detalle y tener demasiado en consideración, no obstante tal vez haga falta dudar y desconfiar del resumen e identificación de diferencias, si hay atrás omisiones pertinentes, o truco escondido.

\section{De lo excepcional a lo corriente (Decimonoveno libro, capítulo XXII)}\label{de-lo-excepcional-a-lo-corriente-decimonoveno-libro-capuxedtulo-xxii}

\begin{quote}
15 de junio de 2023, \ldots, 3 de julio de 2023
\end{quote}

Extenso boletín fantemtiano a contar, de esos que dicen mucho y a la vez no mucho, otra vez se vinieron ciertas novedades a incidir en el funcionamiento no únicamente nocturno con el que los anónimos coincidirán.

Hablando de aquellos, pero con forma general en los sustantivos como vuelve a ponerse de moda, el medio de acceso y involucramiento es una de las cosas afectadas, los dichosos códigos. Primero al querer entrar, al menos por el lado de adentro, son saludados con un noble \emph{/hola/}, y una vez autentificados más capacidades de baja prevalencia aguardan. La evidente y solicitada, las credenciales de ingreso pasan a ser más flexibles al no ser solamente para claves, también alias está permitido, será posible ponerse un nombre que nadie más vea.

Sobre lo no visto o difundido, varias anotaciones más integran el inciso informativo: Un tipo de rango particular, que excedería a los ya conocidos. Modificaciones en los detalles simbólicos de espera al interactuar, que dadas las circunstancias pueden ponerse incluso soleados. Y sucesos determinados por probabilidades desconocidas, cantidad de referencias en su amplio sentido parecen ser las enunciadas, en esta actualización de alto componente aleatorio. Es tanto lo del a veces sí y a veces no, que además los declarados cambios introdujeron características de frecuencia más alta, seguido al lanzamiento pudo notarse lo imposible en los nuevos recuadros de respuesta. Un viejo color alternativo entre la variedad cromática, que después de bastante rato es reintegrada a la lista de posibilidades no reveladas en número, el anom anaranjado.

Siguiendo con las contestaciones, más de las que también son aleatorias. Idea similar a los comentarios entre comillas random, es otra de las reproducidas tomando como referencia al concepto clásico del antro antecesor, pero implementada distintamente, pues tiene que ser manual, un botón debe ser pulsado para ver los remotos dichos que en determinado entonces eran nuevos. Volviendo a ese procedimiento, una vez el signo de interrogación recibe la respectiva señal, una réplica distorsionada se añade al principio con el resto, dando una atípica y acelerada impresión, eventualmente memorable.

Aparte y raramente tal vez, porque algún precedente habría y poco es lo suficientemente atípico, las fallas internas aunque fueron puntuales, también vienen siendo recurrentes, así el sitio mostró sus carencias al momento de devolver determinados hilos. En cada caso de ellos salta el error y de repente un agente fúnebre monta la escena crítica, acorde a las épocas precarias, pero con el renovado marco del clon y su modalidad parcialmente retrospectiva. Pese a la gracia, evidentemente trata de un inconveniente y tal cual fue considerado por quienes oportunamente padecieron sus consecuencias, repetidas ocasiones, y sin embargo la gestión no parece haber logrado erradicarlo del todo, este pormenor negativo que al bienestar global resta.

La próxima incidencia de calificativos especulativos también se desprende de lo resuelto reiteradamente desde la moderación, a priori sin especificar el origen puntual de ello. La misma índole de recuadros oscuros correspondientes al relleno censurado aparece a la brevedad de que un contenido simplemente picante sale, una pizca de atención alcanza para notar la transformación que inadvertidamente ocurre con ellos. La inmediatez reconocida en esta clase de acciones resultó interpretada como una reacción sistemática del sitio, y los recientes anuncios de novedades han mencionado a la ligera determinados procesos automáticos en torno al tema, más de los que ya estaban. Esas señales sugieren que lo subido puede ser retirado al instante. Considerándolo así, implica que las cargas desubicadas prontamente sean removidas, en virtud de sostener el orden reglamentario, a su vez que los falsos positivos son un probable desajuste: ambas parecen haber estado aconteciendo en la actualidad, para bien y mal.

Bajo ese resguardo y el sueño del dueño en horas de luz moran los participantes, que en lo cercano trajeron sus propias temáticas relativamente asociadas a la cultura local, unas armadas dentro, otras traídas de afuera, con el leve rezago literario figurando. La sensación de éxito es una de ellas, la promueven específicamente quienes encuentran entusiasmo en los indicadores de tráfico crecientes, a su vez que la valoración cualitativa del ambiente también se extiende conforme la comodidad es hallada, más cuando surgen las comparaciones frente a las alternativas denostadas, la disparidad en grados de normalidad y toxicidad allí influye en la consideración. Además los personajes continúan destacando en este contexto, algunos disputan su aceptación en torno a las conflictivas demostraciones de no exclusividad, mientras otros agrandan el círculo virtuoso de los llamados amiguitos, colectivo al que supieron insertarse los identificados junto a la sigla LGI, viejos conocidos de correcto vínculo con el antro.

Así y con lo anterior, el ámbito fantemtiano tiene sus puntos de expansión, más allá de que sean un tanto irregulares y sepan apagarse seguido, posteriormente habrán dejado sus rastros de posible conservación. Pero en la medida que sean presente, más individuos disfrutarán de ello, y más valoraciones del tipo bueno serán las generadas, por encima de lo malo que también pueda estar.

\section{El énfasis extrafuncional (Decimonoveno libro, capítulo XXIII)}\label{el-uxe9nfasis-extrafuncional-decimonoveno-libro-capuxedtulo-xxiii}

\begin{quote}
7 de julio de 2023, \ldots, 27 de julio de 2023
\end{quote}

En el marco de las especialidades locales, conjuntamente con lo sacado de afuera, una nueva oleada de detalles trasciende en el entorno del fanatismo temtiano, estas veces no a partir de apariencias o propiedades del estilo, el repertorio viene variando en su manera de homenajear y recordar, si es que son los calificativos más indicados.

Corolario de las discretas novedades, ya no tan desapercibido como en un principio habían sentado los titulares secundarios, aparentemente aquellas rarezas en efecto tienen una incidencia cuando hay lugar para la situación oportuna, las proporciones coincidiendo con los valores puntuales que suscitan las ocurrencias. Con atención a lo que realmente sucede, muestras gráficas levantan la sospecha de posibles delirios, que en la ubicación son ampliamente factibles y sin embargo parece ser otra cosa, las noventinueve notificaciones y su robótico pájaro animado, que en algún lado fueron vistas antes. Los testimonios también afirman haber escuchado ruidos, no como corrientemente suenan, anomalías de esa índole habrían sustituido al clásico sonido de interacción y a la vez generado un importante susto a quienes padecieron la casualidad. En teoría puede faltar darse a conocer alguna sorpresa adicional, aparte de las oficialmente declaradas, pero lo sustancial hasta la fecha sería eso, el resto está dentro de lo normal.

Si se fuera a pensar en referencias, en eso el clon temtiano parece cargar con una consideración agregada al meter lo que sea. La actualización sugiere tener un doble sentido consigo, entre el mote asociado a la autoridad miliquera y la figura simplificada de un roedor, la intervención de orden reglamentario tiene imagen de ratón policía. Por un lado aquello, el complemento cuenta de otra presentación y también afín a la administración, relativo a lo que fácilmente recuerda una interminable sucesión de zetas, para los chismosos, el don de quien en su ausencia no insinúa más que simplemente andar durmiendo, mientras en el antro algún contenido espera por ser revisado. Marcada diferencia con los gráficos de tipo removido que llegaron a verse, provechosa en términos de personalización e incluso en lo esclarecedor, esto último siempre y cuando haya entendimiento.

Esta clase de elementos y los novedosamente traducidos por huevo de pascua, son de encontrar a menudo en determinados sitios, solo que adaptado a las circunstancias propias de cada cual, en sí no seria mayor innovación. Sin embargo llevado a los clones y sus respectivas limitaciones, ya es más atípico dar con dichas características en la esencia del desarrollo, la media sería la de plataformas más básicas, siendo las distinciones menos significativas en cuanto a formato, alcanza repasar el grado de impronta que los antros han puesto sobre sus réplicas parciales. La linea temtiana y fantemtiana resultó ser excepcional en la categoría, el precursor que genuina e influenciadamente adoptó la dirección de implementar componentes fuera de los estándares y desde ahí construir un montón, entre aquello lo siguiente, el sucesor que eligió en parte dedicarse a la continuidad de las peculiaridades, una inclinación de a ratos sabe dar para hablar. Al ser así de cultural, más subjetivo el enfoque colectivo, que en síntesis es proclive a la aprobación de tales particularidades, también próximo a la indiferencia, luego capaz existan otros puntos a evaluar o comparar, como el informativo y el efectivo, lo cual seguramente tenga sus momentos de prioridad, más episodios de naturaleza distinta, quizá similar.

\section{Visiones de desubicada revolución (Decimonoveno libro, capítulo XXIV)}\label{visiones-de-desubicada-revoluciuxf3n-decimonoveno-libro-capuxedtulo-xxiv}

\begin{quote}
28 de julio de 2023, 29 de julio de 2023
\end{quote}

Es una cuestión cíclica, no precisamente el ciclo de los antros denominado por algunos como tal, sí un ciclo que sucede con los antros de manera correlacionada, el efecto sustitución, que tampoco tiene mucha ciencia, pero se reitera e incide en la realidad corriente. En la medida que cada episodio de estos dura más de la cuenta y repercute en la realidad correspondiente más de lo habitual, siguen siendo situaciones primordialmente debidas al proceso de los refugiados, sin embargo tal vez ameriten distinciones, especulaciones, en cada caso podría suscitarse algo de aquello.

Empezando porque llama la atención, es un gran tema y al sitio le interesa que asiduamente estén haciendo referencia a sus circunstancias, en este caso trata del uso y su cuantificación en dos cifras, la cantidad elevada de esos está en boca de unos cuantos, puesto que incluso supieron aproximarse a los supuestos máximos valores históricos. Aunque tampoco dice demasiado, efectivamente la mayoría de ser real tiende a mantener una participación reducida, que después pudo ir incrementando un poco, pero todavía resultando desproporcionada para el promedio usual. Luego hay más, sin embargo más de lo mismo, el estado de movilización distorsionando las temáticas a favor de lo desordenado, compartiendo espacio con las relegadas preferencias locales, aquellas que tan variadas que a veces se ven estimuladas por la alta visibilidad que alcanzan sus planteos.

Dado que el retorno del caído valga la redundancia estaría al caer, hipotéticamente y más cuando el bochorno mancha la confianza relativa, la restauración de las magnitudes normales no tendría porqué demorarse tanto, si los patrones de siempre repiten por ese lado, los actores mantienen sus tendencias, en frente las razones recurrentes también vuelven. Es igualmente valorable el estado que logra formarse desde que la concentración fantemtiana crece, y por el motivo que sea, mientras el hogar propio y prestado ofrece condiciones gratas, los anónimos lo van percibiendo y con ello pueden quedarse a gusto, lo que en principio no debería presentar problemas, lo cual no quiere decir que falten mejores opciones, ni hablar de ser mala la aceptación inicialmente. Presumiblemente la mayor parte de los elementos irán a encontrarse en orden pronto, a no ser que la ventana justamente sea más extraordinaria todavía, por el nuevo producto de las múltiples conexiones, o vaya a saberse porqué.

\section{Devoradores de la improcedencia (Decimonoveno libro, capítulo XXV)}\label{devoradores-de-la-improcedencia-decimonoveno-libro-capuxedtulo-xxv}

\begin{quote}
30 de julio de 2023, \ldots, 18 de agosto de 2023
\end{quote}

Y un poco sí importa lo sucedido en medio de lo regular, el contexto fantemtiano tuvo más aspectos para destacar, también entre palabras mayores, la jerarquía de los hechos es muestra de ello. No del aprendido ciclo semanal, este no dejó de ser tan similar en su patrón de variabilidad, mas simplemente la novedad es complementaria, la gestión redobló la apuesta, precisamente con un doble paquete de mejoras. Siendo que uno de los tópicos insertos en este es la profundización sobre características, lo demás puede tratarse por aparte, como generalmente se mantienen independientes, primero va el desarrollo.

Las dos tandas tienen eso, por cada viñeta una linea de descripción para aunque sea contar con un intento adicional si la idea fuera entender lo que en susodichas entradas fue redactado, cuestión que habitualmente no resulta sencillo. En caso de ser posible, podrá decirse que hay más casillas de adaptación, notificaciones y etiquetados las áreas en las cuales estas ventajas aplican, para enterarse de más maneras, y para citarse de más maneras, valga la redundancia porque esas elecciones luego darán efectos repetidos. Asimismo los agentes ganan control por el lado de sus códigos, dado que fueron sumadas un par de secciones, donde los lazos de la cuenta anónima asociada serían en cierto modo más gestionables. Los avances también se manifiestan en el formato que las interacciones son registradas en el listado de las campanas, si el pretexto era poco claro, las respuestas ya estarían un poco más explícitas de antemano. Y no conforme con semejante facilidad, una serie de puntos no operativos e igual difíciles de constatar, pormenores y peculiaridades que capaz, seguramente, estén mejor exhibidas en alguna escena recóndita.

Siguiendo y pasando a puntualizaciones de diferente calibre, lo restante estuvo contado por quienes presumiblemente no procesaron todo esto. Retomando la situación previa, de vuelta el antro supo acomodarse sin la menos respetada mayoría de otros clones, en especial aquellos que fallan al visualizar una evidente invitación. Como vagamente aproximaron ciertos gatos de culto atípico, además de ellos y su propagado ejército, solo quedaron los que no comentan nada: momentáneamente para significativas extensiones temporales, o en términos comparativos con magnitudes alejadas. Al final la actividad permanece acumulando sus normales y extraños intercambios, no solo para variar entre cuentos y preguntas oscuras, también enfocarse sobre las circunstancias que el sitio presenta. Los incómodos números de anónimos en su despersonalizada abreviación, las complejas expresiones generadas bajo la simplicidad de los sabores, las galerías integradas al sistema de carga múltiple, las aisladas críticas agresivas al acompañamiento adverso, las intensivas evocaciones alrededor del seudónimo admin, y van, tampoco se extiende tanta movida auténtica más.

Salvo lo del cierre, si ellos no se dirigen a ninguna parte, Fantemti tiene que presenciar como nuevamente los espacios oficiales se prestan para dar cabida a unos charlatanes, dinámica que a lo mejor corresponde a un medio alternativo y sin embargo no lo hace. De tal forma por lo pronto ha carecido de discrepancias puntuales, así el baile sigue hacia la próxima cifra milenaria, cinco decenas de centenares, raramente alcanzados fuera de antros anaranjados. Por cada detalle rescatable de ahí, tal vez sean tenga muchos, pero por resaltar uno funcional y no reiterar lo central, lo dicho hace referencia a un incidente técnico, las sesiones: justamente en días de aparentes retoques, cuyos problemas habrían sido atendidos, junto a más desperfectos de la ocasión, sea apropiado considerarlos así o no.

La puerta quedará abierta, y esto en el sentido amplio, porque a veces la relación entre una cosa y otra es complicada de encontrar, más incluso que demostrarle a un robot que está equivocado, y no hay verdad estricta a la cual llegar\ldots{} será que los anoms ciertamente pueden extender su búsqueda, y de ella esperar lo que sea.

\section{Expertos del reciclaje (Decimonoveno libro, capítulo XXVI)}\label{expertos-del-reciclaje-decimonoveno-libro-capuxedtulo-xxvi}

\begin{quote}
19 de agosto de 2023, \ldots, 8 de setiembre de 2023
\end{quote}

Entre el relleno de antecedentes fantemtianos las sorpresas vienen siendo nulas, hay cambios, y hay charla, fundamentalmente. Manteniendo el orden lo primario consiste de actualizaciones, que el sitio mediante su numeración de cabecera secundaria anunció, sin recaer en reducciones, y con explicaciones de más. Pasando de aquello menos evidente, o resumiendolo a incrementos en el listado del azar y restauraciones en el catalogo de sabores, más variaciones a la estructura arribaron como extensión, y justamente a los efectos de calibrar lo sobrante a lo esencial, en una dinámica tan específica como también complementaria. Acerca de la antedicha característica, esta surge bajo al denominación de seriedad, teniendo la función de disgregar lo especial de algunas rutas y convertir el espacio que atañe en una versión sencilla de este, sin tramas implícitas y con cometidos explícitos.

Igualmente, tal abordaje significa poco para el asunto del entendimiento, al darse sobre un área de baja relevancia y tampoco quitar preponderancia a las cosas confusas del antro. El concepto en su conjunto también incorpora ironías de su propio sentido, y lo hace puntualmente desde las direcciones internas sin cobertura formal, que implican transicionar del vacío al extremo opuesto, lo cual más allá del pretexto, causa ruido por donde se lo mire, más todavía recordando el botón que lleva hacia allí. Es más abarcativo aun, en dicha pantalla nuevamente hay contenidos guiños a los sitios antecesores, las portadas por defecto y las rotaciones de revés, dos distinciones del legado temtiano. Lo otro a resaltar e imposible de menospreciar por utilidad y recepción, viene a ser la implementación de la funcionalidad nominada como momento, pues además de elevar la precisión para posibles interesados en la cronología, trajo finalmente eso tan pero tan solicitado, oficialmente el calendario republicano francés.

Del ciento siete contado y sus desarrollos se desprende un punto adicional, dada la apertura a una sección que en su encabezado refiere a elementos llamados intervenciones, siendo en teoría un registro de auditoría o similar, por cierto con información vagamente descriptiva de las situaciones indicadas. Y a propósito de la moderación, su proceder regresó a zona de reproches y motivó lágrimas de mamíferos, haciendo menos discreta la ausencia de un hilo donde inmenso cariño fue compartido. Es más lo cuestionable incluso, el componente de gestión habría tenido una omisión en su rol integrador adentro del círculo de clones, mientras que los bucles de priorización mantuvieron su constancia, la difusión de invitaciones resultó dispar al tratar de girasoles.

Por otros lados de allí, los medios de interacción fantemtiana, la actividad valida que las etiquetas de desierto hayan disminuido, y habla, de unos cuantos aspectos. Lo destacado en linea del clásico que es lo reiterado corresponde a las palabras, literalmente en múltiples casos han habido combinaciones textuales que en más de un tema quedaron plasmadas cual espejo, en principio no porque un mismo autor así lo quisiera, en cambio sí un supuesto individuo aparte, quien sería el responsable de dichas coincidencias, un ladrón que de esa práctica típica generó su seudónimo. Respecto a nombres, por añadidura hay orígenes un tanto inciertos, además de apodos crecidos, una tal Deborah sostiene su fama dentro de dominios entre comillas anónimos, aumentando las menciones, tácitamente conectadas con aquella presencia que en todos los hilos del antro tiene determinada injerencia. Para variar y acercarse también a los menos interiorizados, los indicios del ambiente sugieren un alto tránsito y eventualmente como consecuencia que crece la masa de actores: los intercambios culturales exceden lo típico y característico localmente, las a veces despreciadas referencias que apuntan al exterior persisten, y suplementan con frecuencia la variedad que carece de identidad cercana.

En suma, la realidad fantemtiana encuentra más de lo que ya tenía, a su vez fortaleciendo vínculos y consolidando procesos, favorables en proporción a las viejas incidencias, las que sugieren estar diciendo hasta pronto en breves, para que sin perderse sobre el camino vuelvan a ser encontradas.

\section{Importaciones de brusquedad y capacidad (Decimonoveno libro, capítulo XXVII)}\label{importaciones-de-brusquedad-y-capacidad-decimonoveno-libro-capuxedtulo-xxvii}

\begin{quote}
9 de setiembre de 2023, \ldots, 1 de octubre de 2023
\end{quote}

Como hay que señalar hechos salidos de lo común, inevitablemente quedaría subrayado lo que hicieron los de etiqueta anónima, las que eran en teoría excepciones al rango de lo tolerado, variaron su repertorio y con ello su repercusión. En el marco de repetidas violaciones al principio comunitario de rechazo a la agresividad, un enfoque particular de ellas las tuvo formuladas en torno a Fantemti y sus componentes, lo cual fue asociado a manifestaciones anteriores de tono mayúsculo, similares en argumentación esencialmente crítica. Hilos ampliados a base de actividad perversa y trata maliciosa de datos privados, mostraron una suerte de enojo con algo compuesto del antro, que creció hasta alcanzar contextos exteriores, encontrando confusión e indiferencia, las reacciones evidentes y tal vez complementarias al incierto control interno.

Los procedimientos implicados opacan lo que puede ser una perspectiva válida de los rasgos aproximados al discurso de lo pacífico, recientemente atacado desde numerosos sinónimos parciales, aquel que coincide en reiterarse bastante sin tibieza. Son parte de las condiciones que cambiando de percepción ya posiblemente sean más raras, al menos de acuerdo con las características resaltadas en clones vecinos, donde el contraste de lo normal sube en notoriedad, para a menudo ser discriminado mal. Las dedicatorias al ir también a la gestión tocan un punto más, que no es tanto sobre el intercambio de discrepancias, más bien atiene a la conversación centrada en figuras individuales, relativamente frecuente aunque bajo menor amplificación y conflictividad, e igual nuevamente parece haber comprensión.

El retorno por ese lado sucedió un cacho después y considerado con las ocurrencias del mismo escenario, aparte de otras que viniendo de afuera extrañarían más por los momentos dados. En cuanto a estas que son de mayor contenido, las inspiraciones declaradas acentúan el vinculo referido al resto de clones, y simplemente así un par de sus virtudes fueron regeneradas, una manera de ordenar el inicio y una forma de participar desde atajos. A su vez lo novedoso incluye una adaptación en los accesos de registro, por haber sido aumentados los requerimientos para ello, e implícitamente el énfasis fue puesto sobre dos cuestiones clave, que al solicitarse atención al respecto factiblemente ayuden a cubrir carencias manifiestas: la reglamentación, que oportunamente suele resultar transgredida, y la lectura, que asiduamente destaca cuando falta.

Más que eso, contando los detalles que acompañan, la complejización no terminaría allí. Una deriva de los botones dinámicos y las bienvenidas, a través de ello el sitio parece pretender volverse más animado y cálido. Los errores son otro asunto, aquellos incorporaron información, destinada a ser notable además en las veces que su copia replicada logra más trascendencia que la falla correspondiente, dicha precisión debería elevar la eficiencia del proceso reconstructivo. Y hablando de colaboración, por allí habría una señal más significativa sobre la aplicación de los valores locales, aunque sea rebuscado verlo de esa forma, los redondeos pudieron dejar un buen ejemplo de ello. Es así, cada tanto es decir poco sobre lo seguido que surgen aspectos dependientes del criterio, consecuentemente posibilidades de discusión, pero en términos fantemtianos se sabe que hay alternativas de resolución, sus convencidos vienen enseñando una vía conveniente: recapacitar.

\section{Accidentes de reserva (Decimonoveno libro, capítulo XXVIII)}\label{accidentes-de-reserva-decimonoveno-libro-capuxedtulo-xxviii}

\begin{quote}
2 de octubre de 2023, \ldots, 10 de octubre de 2023
\end{quote}

El sistema tuvo complicaciones, pero con una particular discreción, y la competencia para ilustrar sus torpezas, ambas de alguna manera decorosa. Justo cuando los conceptos sobre errores fueron introducidos en el contexto fantemtiano, tanto resultaron interiorizados que de hecho el antro los padeció. Favoreciendo a los dichos que un mal andar asociaban a él, las declaraciones del estilo poco tendrían de difamatorias en esta coyuntura, siendo que en efecto las escenas indeseadas estuvieron, como distorsiones. Por allí lo demostraron anonimamente, lo cierto de ello, las transcripciones de diferentes incedencias adversas, a veces estéticas, otras oportunidades funcionales, en parte globales, complementariamente individuales.

Cual asunto no menor, la conducción operativa pudo encontrarlo, y sin embargo careció de pronta solución, viendo la persistencia de las tomas y sus devoluciones a medias, disculpas que solo quedaban ahí\ldots{} durante varios días en cadena la secuencia simplemente supo reiterarse, y así permitió que se enterasen quienes mínimamente o nulamente dieron con las afectaciones. Respecto a las suficientes razones para activar las causas temtianas en plazas extranjeras, estos indicadores demoraron unas cuantas jornadas ni bien volverion a la normalidad, como reflejo de que las fallas habrían cesado, conforme tampoco pudieron verse más murales de referencia al cielo y lo inesperado.

Sin llevarlo a términos exagerados, situado en la trayectoria fantemtiana cambia de relevancia. Distinto a los desperfectos entendidos a partir de características intencionales, los de verdad son raros, y según la trascendencia terminan siendo compensados, así que lo excepcional viene a ser medianamente evidente. Si de acuerdo con la tradición moderna ya está cubierta la cosa, la observación y la espera podrían minimizar su atención, para mejor concentrarse en otras rutas y horizontes, las verificaciones a realizar más tarde estarían volviendóse menos necesarias.

\section{Contrastes de lo esperado (Decimonoveno libro, capítulo XXIX)}\label{contrastes-de-lo-esperado-decimonoveno-libro-capuxedtulo-xxix}

\begin{quote}
11 de octubre de 2023, \ldots, 5 de noviembre de 2023
\end{quote}

A veces es difícil no reconocer las características fantemtianas cuando estas se manifiestan, hasta al cruzarse con otras, sus acciones dan la noticia de que seguramente está escribiendo un anom, sin la necesidad de ser explícitos consigo mismos, ni tampoco comprensibles, o congruentes.

En contextos de intercambio, parte de su idiosincrasia entra en cuestionamiento, pues la hospitalidad aparenta ser de baja disposición a repartir sonrisas, a los obvios foráneos con gentilicio priman los pedidos de salida. Sin embargo eso da casi igual, sus estadías incluso superan los ciclos críticos del exterior, pareciendo que el bastión anaranjado fuera más que solo un candidato descartable, variando en la compatibilidad evidenciada. En virtud de la expansión, el ampliamente mencionado antro opuesto también, tiende a largar reacciones de apartamiento, hasta intentos deliberados de trascender encuentran contradicciones, siendo que la atención captada muestra más argumentos discordantes con la integración desde ese lado, por ejemplo la contaminación del ambiente, pasividades que espantan, expresiones en código, prejuicios de interpretación, negaciones del disgusto, y demás tipicidades. Empero, a modo de curiosidad más indirecta, los casos que implican formas de entretenimiento separadas logran mayor adhesión, en especial si los participantes son insuficientes, nuevamente títulos de experiencia compartida a su vez sirven de excusa para unir comunidades, y además confirmar lo relativo que es todo.

Continuando con el potente significado de aquel atributo, la inconsistencia es bilateral, las maniobras conectadas a la conducción han sido desconcertantes, confusas a un nivel de difícil justificación, todavía cualidades de su impronta distintiva, pero más pasmosas que de costumbre. La rareza de turno es de orden en una intencionada versión fantemtiana, el desorden, que para estos momentos se vio aplicado numerosas veces al espacio inicial, a partir de las marcas temporales que sus componentes reflejan, incongruentes con la verdadera actividad contenida en cada cuadrado. Ello evidentemente distorsiona la participación sobre tales unidades vigentes en primera plana, complicando su destaque al tener menos indicios ser recientes. A su favor, que pueda calificar como fenómeno extraordinario, incluso en plan referencia a los posicionamientos irregulares de plataformas temtianas. Para enredar el asunto más aún, la discusión vio deteriorada la confianza de los dichos deslocalizados presuntamente oficiales, gracias a declaraciones insinuantes cargadas de información errada. Pese a ser de baja frecuencia, la típica horizontalidad sienta la base para que fácilmente ocurra, sobre todo si la deshonestidad busca jugar con la jerarquía, las baratas loterías de anzuelos interesan.

Respecto al proceder predecible, las novedades arribaron igual, con un retraso de reducida carga estructural, básicamente estética con simbología y alternativas de discusión, luego de un extenso periodo vacío en dicho enfoque. Esto implicó reincorporar un clásico formato a las prestaciones precedentes, el color acuoso y su molde calcado del antro número uno, hipotéticamente motivado desde la viva memoria de este. En adición solitaria desde el sentido práctico, surge un medio por el cual hacerle honor al concepto típico de la modalidad esencial, los hilos como clon de aquellos tablones tradicionales. Y en complemento al desarrollo propio, la relevancia del ajeno aumenta y bajo respaldo de la mismísima gestión, esa es una impresión posible con el nuevo proyecto Simmpleshouts, familiar por la sencillez que ya supo verse en nombres cercanos. Asumiendo desde la pronta difusión, también en montajes muy sugerentes, los apoyos previos hacen anticipar que una relación semejante repita, en la medida que el crecimiento cumple proyecciones. Y por ahí va la perspectiva promovida, acompasando en tono verde ritmos dispares, denotando sus ganas de acompañar.

\section{Para informar e incomodar (Decimonoveno libro, capítulo XXX)}\label{para-informar-e-incomodar-decimonoveno-libro-capuxedtulo-xxx}

\begin{quote}
4 de noviembre de 2023, \ldots, 21 de noviembre de 2023
\end{quote}

Semanas fantemtianas y un poco más que de cierta forma irían a culminar, tomando lo que ya se supo ajustar y entre interrogantes progresar, que para arrancar a los hilos termina por afectar. Al menos aquellos muertos a ser revividos, los demás a lo mejor, o capaz a lo peor, tal vez sumen otra complicación. Parecido a los nuevos, los viejos también incorporan su distinción, y unos cuantos van quedando marcados con su respectivo símbolo, las elecciones en eso de adelantar lo de atrás, cual si estuviera mal que pasaran como cualquier unidad más, siempre habiendo ocurrido y generando mínimos reproches, pero sucedió esto.

Hay más, las novedades siguen su principio que las lleva a ser declaradas, los principales medios pues luego las tendrán replicadas, por tal es de orden dejarlas mencionadas. Entonces pormenores, esto son observaciones contestables y categorías filtrables, resubidas y errores. Aunque el funcionamiento no permitió verlos suficiente, quizá lo último sí sea a resaltar, por algún ínterin de extraño andar cuando una inédita pantalla supo hacerse notar, esta con el título de sobrecarga fue puesta a estrenar, lo que podía ser un momento de inestabilidad, o meramente un pretexto para probar. Y hasta ahí, sin embargo exceptuando un punto digno de tratar separadamente, las implicancias son distintas y potencialmente mayores.

Bajo el nombre de inicio vacío o simple, el espacio que en condiciones normales recibe al visitante, también conocido como home, ahí está la clave. Por lo pronto ya no lo es más, la clásica galería se habría vuelto una alternativa del conjunto fantemtiano dispuesto por defecto, una cosa adicional a corregir manualmente. Un sencillo punto de partida ocupa su lugar, de ínfimo relleno y elevada literalidad, al comenzar maximizando la parquedad, luego revelando enlaces varios. Así en una segunda etapa la reforma fue aplicada, sinceras palabras ocupan el liso frente a su vez con rutas esenciales, un nexo que antecede al diverso portal de interacciones y demás complementos.

La coherencia y la dichosa justificación es factible que exista, no obstante los enfoques menos demandados persisten, medidas que producen cierta disconformidad entre voluntades reiteradas, aun teniendo estas la libertad de elegir, fundamentalmente el inicio mostró respuestas apáticas con la flamante versión del mismo, sin mucho más tras la expansión consiguiente. Ante eventuales modificaciones, especulaciones sobre sensaciones tampoco son definitivas, mientras el habituarse o adaptarse alcanza para conformar, similarmente a decisiones indeseadas anteriores, considerando que si bien en el fondo las intenciones de mandar a freír espárragos puedan estar, de vez en cuando posteriores improvisaciones en cambio han sugerido esmero por ello.

Semejante a otras dificultades planteadas desde la estructura y configuración, diferencias a los estándares típicos de cada clon hay más de una, y aparte del propósito, tienden a entenderse como filtros, sin ser la excepción en el caso presente. Las explicaciones parciales también entreveran y las pretensiones permanecen ambiguas, pero las trampas igual se mantienen operativas, y a menudo dejan en evidencia individuos ajenos. Todavía siendo contra intuitivo, dicha clase de características son vistas con buenos ojos en el ámbito local, poca información alcanza para activar el estado deseable no concordante con el despiste y la inexperiencia, lo que al final vale lo suyo.

En ese marco y aunque cueste encontrar, todavía el desarrollo de actividades tiene sitio, refugiados y veteranos, auténticos y artificiales, que los términos oficiales llamaron comunidad. La idea de cuidar lo que se pueda hallar es difícil de interpretar, un poco por quienes de afuera sepan llegar, también respecto a los propios y sus creaciones valorar. Pero mientras entre aquello sea posible explorar, atención a lo que vayan a preguntar y más todavía adelantar, que en una de esas la suerte vuelve a estar, no solo para en sueños idealmente proyectar, que incluso el futuro nuevamente logren anticipar.

\section{El triunfo preparado (Decimonoveno libro, capítulo XXXI)}\label{el-triunfo-preparado-decimonoveno-libro-capuxedtulo-xxxi}

\begin{quote}
22 de noviembre de 2023, \ldots, 9 de diciembre de 2023
\end{quote}

Es tiempo de una causa típica, de resaltar un tema honorífico, de dar lugar a un fantasma enorgullecido. Como bien podía adelantarse y como cierto pudo anticiparse, el antro mandó los cambios temporales para confirmar que la temática honoraria de verdad le parece importante, bastante más que otras posibles conmemoraciones. Con signos de exclamación y celebración, Temti cumplió y Fantemti recordó, los tres añitos que este ya acumuló.

Una torta y el correspondiente numero de velas, acompañado por más verde y más anclajes, las adaptaciones oficiales que el clon reiteró y aplicó para la ocasión. Rememorando cada día de aquellos, las novedades se reducen al mero incremento del digito, siendo mas que la indiferencia del primer momento, pero salvando los colores vigentes menos del segundo momento, por ser en esencia la misma idea puntual que de ese caso. Y con una similar recepción popular ante los destacados y demás, la tranquilidad nuevamente fue lo corriente durante esa pasada, tanto que los saludos y dedicaciones fueron buena proporción de su actividad total.

Cuando ciertas cuestiones podrían plantearse al respecto, también se reitera que la trama recibe respeto, poco se pregunta sobre que haya suficiente excusa para romper la normalidad, habrían planteos válidos, pero no llegan a generarse, y de hacerlo tal vez serían contestados casi sin aprobación. Sin mas cháchara, rápido vino y de igual manera se fue, en principio justificado por la jornada al ser simplemente una. En el marco del mes aniversario, el mayor reconocimiento explicó porqué las tonalidades anteriores, y presumiendo que para el rey quiere la cuarta\ldots{} también quedaría descubrir cuán posteriores estarían siendo.

\section{Compartiendo a distinta escala (Decimonoveno libro, capítulo XXXII)}\label{compartiendo-a-distinta-escala-decimonoveno-libro-capuxedtulo-xxxii}

\begin{quote}
10 de diciembre de 2023, \ldots, 18 de diciembre de 2023
\end{quote}

¿Se pueden esperar resultados diferentes haciendo lo mismo? Lo correcto sería contestar que depende, cuando tiene sentido negarlo, la moderna respuesta temtiana se justifica, y\ldots{} un poco sí. La influencia característica de los clones posibilita esto, de no ser por uno será el otro, y gracias a eso lo corriente deja de serlo, pero en perspectiva no será tan atípico, aunque cada experiencia de cierta forma tiende a sus impresiones únicas, las reiteraciones son medianamente conocidas, en particular para quienes no sean pasajeros, los vecinos ya están en condiciones de hacer memoria. Como jornada duradera, por momentos incierta y trascendente para decenas de tipos, el antro fantemtiano también lo vivió de manera especial y al lado de su nombre comenzó a manifestarlo, las recordadas máximas de tráfico se mantuvieron e internamente los estímulos sin llegar a ser proporcionales acompañaron.

Hasta confirmar la predeterminada caducidad, costó que el tiempo real pudiera informar, con lo cual el componente oficial tampoco contribuyó, ni para sugerir desvíos un tanto más lejanos, sin embargo al final se vio a la mayor porción partir, así produjeron un antecedente adicional para afirmar eso difícil de negar. Es lógico que los ordenamientos de una manera u otra cuentan con su motivación, o incluso alegato, las virtudes que hacen ver superior al entorno fantemtiano son insuficientes dentro de tal criterio, resultando solamente cumplidos que en el común de los casos hablan de alguna simpatía sumada, y ya. Contrapuesto al sinsabor de no poder sostener el ilusorio crecimiento reivindicado en las oportunidades del incidente, el consenso local sobre dicho final parcial en cada diáspora tampoco lo ve igual de negativo por sus razones, la revolución que causalmente coincide con las visitas desproporcionadas está en parte mal vista, lo que de vez en cuando se ve en el día a día y llama al pronunciamiento de quienes defienden la supuesta calidad, eso, pero potenciado en cantidad.

A propósito de las debilidades, aquella vez más pudo notarse que resta el límite gráfico. Sin saberlo o por gusto, que repetidas violaciones al aspecto normativo del antro sucedieron y por una reacción u otra luego fueron corregidas, atípico según el uso normal entre los foráneos, que incluso tal vez siendo adultos no obtuvieron del visto bueno para traer imágenes de esa clase. La cuestión se acredita a los proveedores, en determinados casos la remoción pareciera venir de dicho lado, aunque a su vez son de aparecer las leyendas con barras que llevan a una dirección interna, señales que dan a entender que la represión viene por ambos extremos. A lo mejor los acostumbrados estén comprendidos en un posible conjunto de oposición a ello, poco notorio en la práctica y a suponer tampoco cercano a la gestión, que con esta condición empeora la esencia de la expresión libre que los clones pretenden replicar, las excepciones existen y esta sería una de ellas, quizá menos mala si atrás de admitir vinieran arrastrados otros factores de atención.

Entre remanentes el entretenimiento pudo haberse resentido, sin embargo como una cuestión de momentos largos y no totalmente, interesarse en las temáticas propuestas por el colectivo sigue siendo habitual y hasta superar la brecha de las ambigüedades. Los encuentros virtuales que tuvieran certificación aportaron mayor variedad al rótulo de las juntadas fantemtianas, eventualmente casuales sin casi indicios de convertirse en actividades periódicas similar a las ya destacadas, aunque semejantes por formar grupos reducidos a desenvolverse entre numerosos desconocidos. Además respecto a lo externo del antro pero en el sentido contrario, por entrar en lugar de salir, lo sucedido bajo modalidades que implican no denominarse anom o anon permanece extendido y trasladado a donde sí son apodados así. Lejos de ser novedad esta circunstancia que a veces genera la resistencia de los más discretos, las referencias vienen aumentando de hace rato y también pareciendo que fueron a parar a un escenario secundario, sobre el que por cierto de querer pueden incidir como si de primarios se tratara.

\section{Marcadores en trayectoria (Decimonoveno libro, capítulo XXXIII)}\label{marcadores-en-trayectoria-decimonoveno-libro-capuxedtulo-xxxiii}

\begin{quote}
19 de diciembre de 2023, \ldots, 16 de enero de 2024
\end{quote}

Caía el año nuevo y el escenario fantemtiano adelantaba repetirse, y sí, no hubo lucesitas ni estrellitas, tampoco arbolito, ni mencionar nieve. Pero no estando ausente, si tenía que suceder algo entonces fue de lo típico e idéntica resultó ser su fácil vía de transmisión, en semejanza al titular de los jueves que tanta prioridad recibe en su debido momento, lo mismo para las demás fechas de inferior ocurrencia pero superior en predicibilidad. En cada día puntual que tampoco fueron más de dos estuvo presente un amplio despliegue de saludos compartidos, casi que ninguno faltó en ser destacado, para después regresar a la recurrente selección inesperada de temas.

Más de lo resaltado fue una clase de serie, tres listas de novedades que cada cinco días activó el contador incrementando uno, incluso con al menos un ítem notorio esa cantidad de veces. Empezando por esto, el de mayor complejidad a pesar de sus pretensiones de simplificar, repite la modalidad de accionar desde una ventana, y llevó la formulación de fantems para ahí mismo, prácticamente igual a la dinámica de toda la vida, acotada, sin el largo panel de casillas que flexibiliza las características base entre ellos, pero todavía manteniendo un par de accesos a estas adaptaciones, que de vez en cuando se ven puestas y resaltadas en el índice. A suponer por las referidas sugerencias que han de ser las deconstructivas con el exceso antedicho, el pronunciado marco comparativo vuelve a reforzar el concepto de clon.

Sigue habiendo orientación a las posibilidades por el contrario no principales y hasta inaccesibles, el hilo de comunicaciones muestra entre sus destellos productivos lo entreveradas que son las ideas intercambiadas. Flamante ejemplo compuesto, el par de mejoras asociadas al asunto enlaces, partiendo de un estado hipotético que solo pudo verse desde un pantallazo, su primera reacción encuentra otro sentido por el cual aplicar ajustes, cubrir las direcciones que pretendan dirigir hacia fuera del antro. Lo que contribuye a seguir dentro del mismo describiendo lo que ocurre, en creaciones por debajo en visibilidad que quizás ameritan menor seriedad sus declaraciones, pues negativo, cuestión que se habló mal de los sabores y no se tomó a la ligera, pronto la mayoría se tornaron pesados ni bien se contempló la etiqueta saludable, se enriqueció el catálogo a base de jugos. En tercer lugar pero segundo en cuanto a dos funcionalidades, las maneras de buscar con un siguiente nivel de precisión sobre los extensos archivos de fantems viejos todavía activos, y a su vez el listado de las cifras diarias en la suma de aportes a la actividad. Cuando la indagación del pasado del espacio presente es posible pero dificultosa, la misma mantiene sus rigideces convencionales implícitas de formato y esto habrá de servir para propósitos más específicos o meramente curiosidades.

Lo demás al preciado sector de recovecos complementarios, los que ocupan una porción de sentido clarificativo, a su vez dotado por detalles figurados, a veces notorios en cuanto a referencias. Una de las novedades las incluye claramente, con alta distancia cronológica viene un cuadro animado de la antecesora, aunque en cambio por una variante que el colectivo prefirió para representar eso del cero en las notificaciones estables, el silencio. También las que ya estaban entran en consideración tras tener renovación, con expresiones análogas en significado cuentan en este sentido las pantallas de revisión y de homenaje, también la vibrante de los cuatro recuadros invertidos y el esqueleto con la señora rebobinando su secuencia. De una forma u otra los rescates de cultura temtiana siguen estando allí, creciendo en número de tanto en tanto.

No solo limitada a la linea propia, que los gestos por los demás clones volvieron y en abundancia, para alargar el gran selector de copias incorporando recuerdos más difíciles todavía, que Kiiubex, que Dexxov o Ufftovvox, que Arggenchan, variedad de sobra que activa la confusa dinámica estética, alternar drásticamente la cubierta del sitio. De corrido en el calendario según documentan los registros de novedades, termina una semana y enseguida arranca la siguiente, hasta que los estrenos pendientes se agotan, y con la mínima incidencia sobre la dinámica allí contenida, el esquema cotidiano toma de vuelta su lugar.

\section{Para seguir estando (Decimonoveno libro, capítulo XXXIV)}\label{para-seguir-estando-decimonoveno-libro-capuxedtulo-xxxiv}

\begin{quote}
17 de enero de 2024, \ldots, 3 de febrero de 2024
\end{quote}

El titular fantemtiano de arranque con mayúsculas y exclamaciones avisa del avance en su carrera individual. ¿Por qué consultar para qué? Si la respuesta equivale a nada, como si en cambio la interrogante preguntara por cuándo, suponer es el recurso favorito hasta que finalmente sucede, el sitio temtiano volviendo a dejar de las suyas.

Las excusas sitúan a la demanda en primer lugar, si los ingenios estuvieran siendo ideados para complacerla, sin embargo no resulta un absoluto, teniendo en cuenta que el filtrado abarca pedidos. Hablar de excepciones deriva de creer que usualmente hay voluntad, o esa impresión queda, un caso reciente mostró la utilidad de los comentarios fijados y su disposición completa, acompañada por detalles más atípicos de lo normal, una de las cosas que quisieron traer del exterior, solo que adecuada con algún motivo poco intuitivo. Para no variar, este clon también acostumbra maneras más explícitas de dar entender lo que es copiar, y así se ve: menos anoms teniendo la experiencia en anaranjado y más anoms conociendo los diseños de alternativa. Los seleccionados fueron dos, el par que parcialmente tuviera distinciones en la difusión propia, después de meses y semanas recibe otro guiño de la gestión sin agregarle anotaciones a su pretexto.

Pasando de punto para volver a lo común, de las formas habidas y por haber el calificativo de irrelevante estuvo, y factiblemente siga correspondiendo, no obstante ahora hay aunque sea un motivo para adaptarlo, a eso que se entiende de las categorías. En su caracterización vuelven al tratamiento, no porque su sentido entre a notarse cotidianamente, tampoco a causa del movimiento interno que las recalque. Una más aparece con la dupla de generales, una que en cambio tiene de prefijo casi, y una de las primeras varía por separado, en teoría adoptando un limitante de posicionamiento para sus componentes. Cosa que suele ocurrir, al mantenerse un largo recorrido de intercambios que llega íntegramente hasta los comienzos, a falta de reinicios y reciclados tampoco surgen los contratiempos de querer revivir lo descrito en números resumidos, aquello que ya fue pero no del todo. Pinta ser una objeción a susodicho fenómeno adherida al rótulo de los hilos viejos, directamente consiste de un impedimento para replicarlo, sin embargo su cumplimiento de momento permanece ambigua.

Por eso y más de lo sabido se están poniendo menos permisivos, otra medida antipática con la capacidad de responder introducida en el inicio, aunque no en un sentido tan frecuente y pertinente, o a lo mejor sí ubicando al antro entre los suyos, nuevamente un asterisco rojo ni bien se trata de responder en masa. Antes normativo, ahora estructural, una práctica llamativa y eventualmente digna de múltiples adjetivos que en general los fantemtianos tenderán a guardarse, semejante a la cuestión anterior ya de por sí inesperada. Que así los límites tácitos figuran en un entorno más reducido, acompañando las demás restricciones de aplicación protocolar impuestas recientemente, sin embargo lo primero incluso con sus quisquillosos dejos de sensibilidad, coincide en las cercanías como sucede cuando de clones consiste la cosa.

A propósito de los demás y a su vez oportuno por lo rechazado, notar la concurrencia en su composición más reconocida que de costumbre, no gracias a los que promueven la asociación de considerable margen de confusión, sí por los que cero problema ven al explicitar ser los mismos de algún intercambio anterior. Siendo alta la proporción de hipotéticos participes repetidos, al menos tres firmas ratifican presencia continua en los hilos fantemtianos, sus ilustraciones supieron verse multiplicadas hasta sin haber percances devvoxeros, aunque las discrepancias con lo de aquellos tienen lo suyo que ver. Remitirse a los antecedentes lejanos trae que personajes así encajan incluso bien, recordando que llegar a eso requiere un afianzamiento mayor en el ámbito local, cuando el común denominador en los foros remarca prohibitivamente la contradicción al carácter anónimo, pero por lo pronto la semejanza no es tanta en el clon de los clones.

Aún expectante por los de similar estilo también referenciando, este mantiene las tramas de su parte medianamente enlentecidas, favoreciendo que abunde más lo aleatorio sin tampoco tanta variedad, haciendo valer la orientación marcada por sus categorías. Aunque ciertamente los enfoques contemporáneos admiten una segmentación de mayor precisión, pero antes que ver eso llevado a términos superiores, pareciera más probable que los gatos asados obtuvieran su propio espacio. Con altibajos y todo junto a la larga lo dicho marcha sin frenar, el discreto apartado estadístico para darle número a esa noción confirmaría, el silencio puntual tampoco se traslada a lo global durante el suficiente rato. Para lo rigurosos que son, difícil sería demostrar la inserción de principiantes, o la persistencia de veteranos, pero las sospechas afirmativas fácilmente pueden justificarse. Una entre tantas dimensiones curiosas a contrastar por lo menos explícitos que se ven sus registros cronológicos, justo cuando Fantemti está por alcanzar los dos años de existencia.

\section{Cruces conmovedores (Decimonoveno libro, capítulo XXXV)}\label{cruces-conmovedores-decimonoveno-libro-capuxedtulo-xxxv}

\begin{quote}
4 de febrero de 2024, \ldots, 16 de febrero de 2024
\end{quote}

En la vieja trayectoria temtiana, la nueva ruta fantemtiana dispuso que sus navegantes atravesaran un sector de estrellas y dedicatorias, que con naturalidad y un clásico ajustar, la voluntad que hubiese si ansiaba por cumplirse pudiera tener su anhelo, simplemente la espera en condiciones estables haciendo llegar lo anticipado. Solo para entendimientos que así lo tuvieran en cuenta, además de allegados que no lo evitaran por su cuenta, del cumpleaños algo tuvieron, en principio únicamente su nostálgica tradición oficial, o a lo mejor un tanto más que la ocasión traiga como valoración, de muchas cosas que podrían venir al caso, por experiencia, por actualidad, aparte de todo lo generado en cada acompañante, manifestado más allá de un particular. Incluso sin reivindicar sus implicancias normalizadas, la importancia se asume hasta cuando lo anterior ocurre en una siguiente oportunidad, el impulso de la causa ya lo reconoce y lo demás depende de ello.

Visto lo visto, la ola de desplazados no tomó las recomendaciones de engancharse a Tssubit, otro desajuste más que hubo, y Temti en su versión sucesora dando alternativas accesibles para continuar. El sentido acumulativo de estos episodios tan recurrentes tal vez exista para tarde o temprano notarse, la hospitalidad aun con su porción antipática saca a relucir un poco de cariño entre unos cuantos, el refugio que suele prestarse para bancar a más que un segmento civilizado. Avivadas sobre el potencial del sitio para configurar, la reunión es constructiva hasta en las muestras de cultura, resaltando características inherentes de turno, bajo la perspectiva que se acusa por sí sola, mientras respecto a limitaciones repercuten menos los achaques, el contraste de incomodidades tampoco logrando acaparar tanto entre tendencias, lo que lleva a conversar más que acerca de sanciones injustamente aplicadas y públicos normales indeseados, también motivos ajenos que sirven para exagerar.

Hablando de circunstancias que fueron haciéndose comunes, la establecida en la raíz del componente participativo del espacio pasó a ser cosa pasada, las miles y miles de letras vocales sustituidas por un similar deformado dejaron de multiplicarse. El seudónimo dominante recupera la simplicidad de composición, perpetuada por la facilidad de su reproducción de costumbre en los desvirtuados intercambios internos. Igual así estuviera esa adaptación, también sigue siendo una expresión semejante por diferencia con la estándar, que debería significar parecido a la clásica versión corta de anónimo, salvando los detalles que el antro en particular les agregara. Manteniendo dicha terminación que retrocede un paso en el alfabeto no georgiano, el distintivo de larga data en el recorrido fantemtiano hace ver como una obviedad lo fijo que se encuentra allí.

Volviendo a la jornada, la chincheta protocolar de directa referencia al momento tampoco faltó, a través de mensajes que se recordaron ya lejos del presente, más el raro añadido de ser un medio cerrado el que saludara. De aquellos que de verdad abundan poco, ejemplares únicos de un pequeño tramo histórico propio del sitio fantemtiano, lo más cercano a un archivo a encontrar entre lo que este de sí mismo ofrece. Sumado a eso, junto a la aclaración de supuesta caducidad del apodo antedicho, cayó la contraparte, no concluyente con los símbolos del día atípico, menos para los eventuales ciclos que estarían dando a conocer su segunda etapa. Con la noticia dada sin mayor severidad, los planes definidos que tendrían su salvedad, por quizás no tener esa cualidad. Pero de quedar así podrá registrarse otro periodo de estilo propio, contemplado factibles interrupciones solo eso, para el resto ser una época especial por tonalidad, y quién sabe qué más.

\section{Corrientes y contaminaciones (Decimonoveno libro, capítulo XXXVI)}\label{corrientes-y-contaminaciones-decimonoveno-libro-capuxedtulo-xxxvi}

\begin{quote}
17 de febrero de 2024, \ldots, 10 de marzo de 2024
\end{quote}

Tópicos complejos y repercusiones aseguradas, lo que trae Fantemti en un subsiguiente tramo es interesante por eso combinado, pero dada su alta variedad y baja claridad, sería turno de quedarse con el resumen, aprovechando que se repite en cantidad, sobre lo que tampoco es novedad.

Primero y no por seguir un orden concreto, la evaluación del ambiente es puesta a discusión en recurrentes ocasiones, como más entreverado podía ser, gracias a los numerosos ojos y diferentes perspectivas que en ello tienen lugar para dar su parecer. Entre propios responsables de formar eso continuamente, uno de los cuestionamientos por excelencia destaca nuevamente, además de la crítica a distancia que habitualmente solo aparece tras las citas a ese antro lejano, en aquellos clones cercanos. No solamente respecto a un estado armónico, lo contrario incluso, desde la misma interna que con insistencia enfatiza sus discrepancias no unificadas, la promoción del suicido es una expresión común en ese enfoque, que en esencia aqueja las comprensibles dificultades en la comunicación.

Sin ser un vale todo, la preferencia oficial favorece al criterio de anónimos dispuestos a influir y que efectivamente lo hacen, así sea si la ambigüedad domina, las minorías deberían poder acudir su derecho de oponerse como en tales ejemplos. El hipotético sería que pretensiones semejantes alcancen su cometido pronto, en contraposición con tendencias visibles bajo varias coyunturas que totalizan una larga extensión, porque lo que entra en dicha clasificación lleva construyendo mucho más, hasta generando esa fama tan polemizada, que en circunstancias normales no termina en debate alguno. Lo mencionado aplica para la propagación de respuestas clásicas y del momento también, cuando llegan a niveles de abundancia comunes de ver tampoco están libres de calificar como exceso, según si ese adjetivo tuviera validez. Incluyendo las que implican un desorden profundo en el listado relegado de hilos, caso aparte de la posible actividad dando igual que tenga velocidad de caracol, tampoco ve limitada la reutilización selectiva de temáticas, la marca de alienigena lo demuestra de paso.

Tanto antes como después, la llamada mano blanda de mando contribuye a que menos se especule acerca de movimientos en los límites, mientras las revisiones cubren imperfectamente antedicha área sensible, lo funcional y social parece importar más. Un rol activo que en su regularidad característica de vez en cuando hace excepciones, sin llegar a los desastres, quizás. Ejemplo: los enlaces que no cargan, las respuestas que no suben, el antro que no anda, y tal vez un etcétera reportado por las supuestas inestabilidades en la conducción del tren, que igualmente después sumó una lista de mejoras orientadas a la acumulación.

Parece simple sintetizarlo así, que si bien podría ser suficiente según el lector, hay detalles adicionales que no daría para omitir. Lo dicho trae dos alternativas sonoras, de referencias quizá rebuscadas, en comparación a los tres que ya se encontraban en reproducción cada tanto, que por cierto mantienen su curioso efecto sorpresa sobre quienes menos supieron escuchar. Para ambos pendientes por descubrir, primero es desconocida la vigencia integra de la memoria, a no ser por tal alusión puntual al servidor temxelero, cuando en segundo lugar, el ruido de felino responde a una presencia notoriamente reciente. Otro de relativa importancia va a las columnas del inicio apartado, una más que lo aleja de su rótulo de simple, la cual al contrario de ser revertida, viene sumando secciones desde su implementación. Y que las extensas posibilidades de ajustes tanto utilitarios como estéticos en dinámica sostiene su dirección: una que curiosamente completa con botones la otra versión de inicio, y una de estilos sumamente exóticos entre los clones voxxeros.

En esa linea de los que por sinónimo son refugios, es asunto del denso hilo que tantos vínculos contiene, cual pese al arduo laburo de identificar uno por uno, de las que refieren a Fantemti hay y con especial connotación reciente, desde ahí otra posibilidad gana consideración hasta pareciendo insignificante. Ya supo estar la que en un principio contenía al antro como tal, hoy en condiciones idénticas mandando en cometa a la actual, pero luego el índice se agrandó, sin una mínima verificación oficial, además de las presunciones nada disparatadas respecto al contexto. El hecho es descubrir uno de aquellos prácticamente en foja cero, de más quietud que la acostumbrada y solo un punto de intercambio, ese siendo el único fantem formulado a su vez en posible alusión a los malos rendimientos acontecidos. Más allá de lo sospechoso e incierto del ejemplo, no despreciable pero poco ampliable, tampoco vienen al caso las ventajas de contar con un respaldo operativo, menos para la subsistencia que solo padece quiebres efímeros, pero olvidarse del asunto dista de ser acertado, hay algo sobresaliente entre lo bastante que todavía no está contestado. ¿Y luego?

\section{Inmensidades inquietantes (Decimonoveno libro, capítulo XXXVII)}\label{inmensidades-inquietantes-decimonoveno-libro-capuxedtulo-xxxvii}

\begin{quote}
11 de marzo de 2024, \ldots, 4 de abril de 2024
\end{quote}

Se continuó encontrando en dinamismo al actual de los sitios temtianos, lo que pudiera ser considerado así desde las condiciones que con constancia integra, sus mecanismos de vuelta adaptaron según lo que quisieron los protagonistas de turno. El modo discreto que mantienen las novedades cerca está de los tópicos críticos: así fue planteada una manera adicional con sus salvedades en incentivos, cuando la excepción excede la voluntad individual, de acercarse al modelo imaginado equiparando en unidades el espacio principal. Sobre el privilegio que llevan los sabores, ese estandarte de tradición propia y también de rol identificador, es una reconsideración de su tensión entre consolidarse y debilitarse, que hoy por hoy conlleva tener un inicio mixto, aunque desbalanceado gracias a la responsabilidad distribuida.

Por un sentido central de actualizar, la practicidad de los botones crece levemente dando soluciones específicas, y lo demás del paquete quedó algo más difuso entre cambios incomprobables. Para verle la gracia a parte de este tercer punto, después el clásico primer y segundo fantem fueron a combinarse en uno extendido, tanto que si falta lectura la cantidad de texto resultaría enorme, aunque en defensa de quienes así lo vean, tampoco habría mucho nuevo por lo que enterarse. Pudiendo tomarlo como formulación protocolar o indirecta orientadora, esta síntesis del contexto manifestada desde la perspectiva oficial da lo normalmente dispuesto en presentaciones, pero bajo el particular enfoque fantemtiano, también inspirador para imaginar las referencias contenidas y distinguir las construcciones colectivas, cuidados pertinentes ante la imposibilidad de hacer saber directamente todo sobre donde se está.

Lo demás podría ser menos feliz para determinado bienestar, desde un lado por lo subjetivo y el restante debido a lo incierto. Son los bucles contemporáneos y sus implicancias, que negros impongan sobre blancos más reduce las sensaciones de paz, también el entendimiento mediante excesos deteriora la calidad del ida y vuelta, a su vez tanto mal estomacal como síntoma tampoco mejora las señales. Intercambios que por frecuencia quizá sean despreciables, teniendo desnivel tomarían otra dimensión, si es que la gestión confirma las creencias o conforma la interrogante. De las prácticas trasladadas esta sería otra, la actuación abusiva ya desde arriba para abajo, sin transparencia que responda sobre posibilidad alguna, mas solo indicios que de sospechas no quedan exentos. En perspectiva para el diálogo sostenido en los medios relevantes, no son frecuentes los indicios de trato inadecuado con los participantes o sus expresiones, asimismo siendo factible que si lo hubiera este no trascendiera. Para el silencio forzado, en ríspideces del estilo o las recientemente acusadas, es claramente atípico enterarse de que cualquier anom lo padeció, también siendo clave darse cuenta del formato poco conocido.

Respecto al resto de justificaciones para la quietud, más adelante otra vez un atenuante se sumó a su contradicción, nada extraño si se revisaran los antecedentes cercanos. Yendo de una a lo excepcional, grandiosos números fueron más que lo previsible, la máxima de enchufados como primer medida superando el sesenta y sus siguientes pasos sobre dominios fantemtianos, el entretenimiento se volvió intenso en términos relativos y variado hasta donde las transiciones habitúan. Especialmente para los antros y su público concentrado, el conjunto de preocupaciones que por su inminente evolución al vecino tienen que interesarle, esto son las caídas y levantadas, posteriormente los rumores que desarrollen la pertinencia de los sucesos, sin la represión aparente de un escucha paciente. Más parecido a un simple clon, porque eso que tratan también transmiten, no solo actividad que hasta propios podrían querer, demás aspectos conflictivos además, agentes dedicados a exportar creatividad son un ejemplo incluido sin casi aceptación, entre el complemento de factores comprometedores para la diferenciación del ambiente.

Aparte del fin predecible, los restos del movimiento dieron a entender otra expectativa de migración asociada, por encima de los sitios vigentes que más cerrados permanecen, superando a la fuerza de episodios previos por multiplicación incluso posterior. Sin reiterar que la sostenibilidad explica por sí sola la porción fundamental, junto a su estructura debería facilitarse contestar, que la alternativa en cuestión esté alejada más que por una tendencia de la costumbre y el movimiento, de hecho lo demás tampoco está reseñado positivamente por completo y el sustento a veces queda casi explícito, pero hay que añadirle la intención que esté acompañando. Referencias ancladas proponen las respectivas conjeturas sobre la salida de visitantes, dichos locales reflejan el desinterés por expandirse, y la influencia de sus responsables repercute. A lo mejor en la misma linea, faltó decir que durante el desenlace hubo un incidente para Fantemti, tal vez acelerador del proceso antedicho, la demorada restauración no aportó datos del caso y los errores no son la única respuesta posible. Del modo que fuera, contagiarse también de las imperfecciones resta desde bastantes puntos de vista, y los precedentes además de sentar oportunidades constructivas, similarmente adelantan circunstancias que puedan estar viniendo, aunque la estabilización sea la próxima más creíble.

\section{Clarificaciones de composición (Decimonoveno libro, capítulo XXXVIII)}\label{clarificaciones-de-composiciuxf3n-decimonoveno-libro-capuxedtulo-xxxviii}

\begin{quote}
5 de abril de 2024, \ldots, 30 de abril de 2024
\end{quote}

En camino al reencuentro con lo típico, las expectativas fantemtianas tuvieron un corto desfasaje en referencia a su realidad cercana, de magnitudes excedidas pero tampoco desconocidas. De aquello primero antecede una enésima experiencia que hace ver crecido un factor de hostilidad e iniciativa vinculadas al fenómeno movilizador, aspiraciones desmesuradas candidatas a perderse sin la influencia de pulgares hacia abajo. La demora pasa a ser un detalle disminuido entre inactividad, esta a niveles extremos toma su lugar reconocido para múltiples sobrevivientes con la noción incorporada más allá del momento, junto a los negros, los pedros, las mujeres, y otras variaciones no tan de repertorio.

Asociado al hipotético guion y quizás incluyendo lo que no fue pensado para formar parte de este, algo más a conocer hay sobre dicho rollo, sin implicar que baste con tal de conformar, aunque no haya punto señalable como final. La razón de los llamados stickys parece existir y persiste estando arriba de las simples coincidencias, motivadas por los mensajes ambiguos y simpáticos. Con tal de concentrar temáticas deseables, el argumento del gusto particular se torna relativo al haber definición de orientación, no obstante siendo poco lo obvio otro cantar es lograr saber lo que pensar o incluso intercambiar. Poniendo en tela de juicio parte del propósito pero no solo desacreditando, la lógica constructiva atrás es factible, encima fijar y aplicar un criterio implica sacar vacantes a otros esfuerzos, o generar un marco de impresión para quien tiende a discrepar.

En la serie del exento, la modalidad de exhibición por demostración vuelve a las hileras oficiales, y nuevamente en mención directa al combinado rouzzero, que suma otra característica compartida ya un amplio tiempo después. Un clásico en la cultura de los antros, aquella que llena estos de pelados coloridos, ese sería el nuevo producto de la fábrica de copias que a la causa nostálgica global también le suma. Lo siguiente es más de la casa y tratándose de configuraciones, no por agregar de ellas que igual es el caso en una puntual, sino por la posición que toman las mencionadas en conjunto.

El primer ingreso ya sugiere pasar por el sitio de ajustes y revisar su largo tendido de posibilidades, semejante a como muchas circunstancias de bienvenida supieron intentar transmitir, pero en formato oficial y sin puntos específicos. En cuanto a las complejidades definidas por el medio fantemtiano, el historial con este tendría una entrada más y siendo en ambos sentidos, de afloje y refuerzo, de respaldo para lo predeterminado y comprensivo con su oposición. Y por tanto para continuar el mensaje sobre lo complejo es simple, o no se cambia más, o se sigue cambiando.

\section{Encimados particulares (Decimonoveno libro, capítulo XXXIX)}\label{encimados-particulares-decimonoveno-libro-capuxedtulo-xxxix}

\begin{quote}
1 de mayo de 2024, \ldots, 30 de junio de 2024
\end{quote}

Hay constancia para evaluar más que un criterio, los ejemplares temporalmente inamovibles expresaron parte de la trama colectiva, de las posibles una en concreto por recibir los destaques de la conducción fantemtiana. De lo mucho para decir hay una reducida distancia, donde la situación corriente tampoco es que quede demasiado manifiesta y toma distintos lugares según se lo mire. Allí la justificación de motivador aplicaría al menos cuando hay interrogantes, a su vez la neutralidad de numerosos fantems sugiere similarmente, pero también puede ser que no, incluso ni siquiera la lectura. Los que salen de ciclo concuerdan más, la coyuntura de los antros arrastra la disposición que cada tanto logra plasmarse mediante desorden, el pronunciamiento de valores locales repite con cierta espontaneidad, mientras que otros casos resultan excepcionales por frecuencia y propiedad, el tema de las mascotas por traer a cuento uno. Las similitudes persisten de nuevo cuando el día trae algo especial, para adecuarse al estilo leyenda de bienvenida, un simple señalamiento por ejemplo suele alcanzar hasta sin que los tiempos parezcan ameritarlo, como si lo feliz de cada jornada no pudiera estar ausente.

Para el resto del panorama, sigue la desaceleración fantemtiana, más allá de lo cambiante y raro, viendo por cantidades no extrañará asumir que la tendencia ganando fortaleza, aunque sea una precisión propensa a ser omitida. Digna del título Frantemti y semejantes en determinada proporción, la conversación un poco pasó por los nombres, los personajes que sobresalen sin gran estima sobre el anonimato menos favorecido por el nivel de masividad, el antro los ve un tanto afianzados hasta en un lugar que respecta a su propio presente. Los ejemplos son capaces de dar una posible explicación a realidades como los enlentecimientos, relacionados con la falta de anoms conocidos por así decirlo. Amigos, amigos, amigos y disputas, entre eso ha ido lo asociado a los grupos, primer exponente de ello trae a los que se identifican por Magios, de menciones sostenidas adaptadas al contexto que reflejan de una presencia incidente. La discusión es vieja para los clones y que haya buenas contribuciones para estos es una posibilidad, así como también que el rechazo de quienes ajenos a ello sean igual cuenta, sin embargo llevar lo primero a conflictos interminables cambia la cuestión, el interés general puede verse más preocupado si tanto espacio resulta dedicado al desacuerdo.

La gestión hasta siendo nombrada fue indiferente a todo lío, lo típico de su cometido fue responder a lo directo, más presentar unos pocos puntos adicionales, desde su hilo renovado que en los detalles escritos volvió a tener movimientos. Hay factores que vienen por ese lado, un guiño por el medio casi cien porciento destinado a ello, las categorías tienen un lugar adicional para la moda y sin mayores especificaciones sumadas a lo dicho. En esencia sobre las referencias numéricas superiores, las que acompañan en toda ubicación del sitio, hablando de información nada se echó a perder, empero no se entiende mucho, menos si previamente tampoco lo hacía. Semejante a bastantes cosas en Fantemti, un paso extra para ponerse en sintonía con lo siguiente, que cada dígito tiene un concepto declarado, entre los cuales está el de mayor consulta, preservado. Y el remanente de mejoras restan en algún recoveco reforzando más todavía lo suplementario, desde la caducidad del relleno tanto en su integridad como prioridad: lo primero pareciera servir para desaparecer, lo que cualquier anom quisiera hacer expirar luego de haber hecho, más lo segundo que directamente permite encontrar, aquello con antecedentes de permanecer fijo en la galería del antro.

Las partes menos obvias cuentan más, la sencillez de los números si es por evolución aumenta, no proporcional con su significado conjunto, aunque sea desde las dos adiciones que sintetizan enfoques familiares, notoriedad van a tener más. La polémica paloma también vista como cangrejo fue víctima de esta reforma, no en desmedro del propósito pacífico, pero muy austero con la representación de este. Y para los que ya andaban ahi, equivalencia en cuanto a contenido y tal vez distinto por prefijo, lo novedoso es más grande, tanto que el mil supera, referencia a los días de trayectoria que Temti lleva\ldots{} ¿vigente? Muchos términos hay para adjudicar, y con lo discutible que entra en consideración igual, oficialmente son solo los pasados desde la primerísima fecha. Pero además dicha acumulación tiene su respaldo en una serie de mensajes, representativos de la carrera al infinito, la hoja de ruta más evidente del sitio. En esto, mínimo es el crédito que asume Fantemti, considerando que los demás indicadores hablan del curso inherente, como que la preocupación vuelve a recaer en la causa originaria, y más allá del contenido o los parecidos, por lo menos algo del culto identitario allí se perpetúa, el de seguir siendo cercano al principio sin fin.

\section{Antagonismos incorporados (Decimonoveno libro, capítulo XL)}\label{antagonismos-incorporados-decimonoveno-libro-capuxedtulo-xl}

\begin{quote}
1 de julio de 2024, \ldots, 17 de agosto de 2024
\end{quote}

En presencia de una progresión que avanza de forma interrumpida, el vínculo entre lo que ya parecieran ser dos secciones conformadas por partícipes queda un tanto manifiesto. Lo declarado a nombre de uno expresa por la mayoría, incluso sin haber representación extendida, pero si fuera el caso, tampoco estaría tan lejos. Rivalidad, enemistad, conflictividad, esto se contrapone con la inclusión y hospitalidad, ambos sentidos presentes no siempre equiparados. La expresividad e intencionalidad a veces clara entre ellos habla sobre lo que quieren, tal vez alejar amenazas, ampliar relaciones, o simplemente seguir valores. Y de cualquier modo, dichos actos construyen imagen, significativa para un amplio común, local y ajeno, sin que esto último deba implicar indiferencia.

Contemplando los desperfectos de casa, la sinergia con lo de afuera igualmente suscita cuestiones, en especial si el crecimiento conlleva destrucción, la adversidad tampoco cae siempre donde mejor viene, y es relevante entendiendo que haya circunstancias con chances de revertirse próximamente. Seriedad, respeto, tranquilidad, dimensiones que por algún lado tal vez se expandieron, por otro retrocedieron, al menos según una porción percibe viene sucediendo. Tales cualidades que en parte atraen al público en ascenso, tienen su reconocimiento en la percepción y por ello también permanecen rondando, sin embargo, el punto es la falta de consenso, dinámica en los días que corren y relevante como para poner en duda la pasividad o llevadero en cada origen participativo.

El marco funcional trae variedad para acompañar y condicional a la movida antedicha. Con superiores distancias entre adición y adición, solo una aparece en cartelera oficial, pero de particular consonancia con el momento: un mecanismo de respaldo a la intolerancia, de utilidad justamente sujeta a eso, a la eventualidad de no verla. En esta materia sigue el remanente del rol mandatario, nuevamente inmerso en sospechas de accionar incierto, desempeño en medida negativa por manejar limitantes, las que niegan voluntades, las que suprimen libertades. Tomando declaraciones públicas de información, hay abundante exceso, lo máximo para afirmar mientras mucho de ello corre bajo un manto de baja certidumbre. Porque las experiencias valen conforme la transparencia también, postulando el asunto sanciones en términos que denotan una cercanía disminuida, dudosa por los motivos en tratativas, los modos, direcciones, o vaya a saberse qué.

Con poco margen de superación respecto a discernimiento en los asuntos internos, el panorama no está cerrado, donde lo predecible no cubre, lo faltante puede revelar nuevos elementos, cosa que cuando hay dudas merece mayor interés, al referirse a los asuntos fantemtianos.

\section{Procesos con repuestos (Decimonoveno libro, capítulo XLI)}\label{procesos-con-repuestos-decimonoveno-libro-capuxedtulo-xli}

\begin{quote}
18 de agosto de 2024, \ldots, 5 de setiembre de 2024
\end{quote}

Con foco en lo base, no era cosa de todos los días mas sino la mayoría de ellos, pero un episodio cambió la constante, la que antes hubo regresó con nueva declaración de precaución para la ocupación de la estructura funcional, lo que quede de ella en condiciones teñidas de negro. Desde las sesiones al no tan más allá conocido, fue difícil moverse en determinados puntos, también contar lo que estuviese en frente o presente, así el crédito de la gestión cayó un tanto en virtud de una inoperancia que tampoco extraña en antecedentes. Paciencia dicen por ahí, mientras los errores seguían apareciendo ante una recepción de términos predecibles, justamente durante un periodo pertinente. Lo hecho y su incierto sentido fue una etapa, o parte de una, sus ajustes invisibles dejaron de ser el punto extremo en el curso oficial, habiendo arreglos y agregados después. Fuera de lo ordinario y más para la media dispuesta por los clones, el menosprecio al rechazo de scripts conoce un emparejamiento de posibilidades, aquellos agentes distinguidos, si vieran oportuna la alternativa, no estarían privados de un constituyente adquirido clave, las imágenes.

Y todo eso para el público que cumple su rol, siendo un tanto caracterizable con varias excepciones de momento, entre signos de renovación y más de lo mismo. Unos ven paz, otros buscan guerra, ambas realidades comprensibles y no estrictamente contradictorias, tampoco necesariamente compatibles. Si de la política el antro no era libre, del desacuerdo menos lo es, con lo discutible que es calificar la calidad de los dichos, poco orgullo sobre ello manifiesta el total de camaradas. Facetas contrarias conviven, el colectivo también muestra una constancia con el interés por los buenos días, los saludos frecuentes y su considerable acompañamiento manifiestan cierta continuidad para valores puntuales del medio. En la línea colaborativa, la identidad local motivó más las contribuciones, y así aquella tradición de los dibujos pero como fantemtianos, las tentativas de definición colectiva e individualmente sentaron un precedente, para el recuerdo y quizás incluso eventuales continuaciones.

Esperando lo suficiente, las proporciones de lo nuevo y lo viejo tenderían a reestablecerse, a no ser que\ldots{} cambie, y es que las hipotéticas renovaciones mantienen ese potencial, aunque la superficie descuente relevancia a dicha posibilidad. En materia de colorido los retrocesos tienen su vía para adquirir dimensión, por tener momentos oscuros donde las inundaciones demuestren las limitaciones del panel, o con más credibilidad sobre lo textual, el menospreciado compuesto de seres creativos, indiferentes, y detractores. Con el factor artificial menos considerado en los días nombrados que transcurren, son los componentes que hay y corren los riesgos típicos que vienen de prestar su voluntad, y con lo importante que es, que otro cansancio pueda interferir en ello representa una contingencia a tener en cuenta.

\section{Espejismos dinámicos (Decimonoveno libro, capítulo XLII)}\label{espejismos-dinuxe1micos-decimonoveno-libro-capuxedtulo-xlii}

\begin{quote}
6 de setiembre de 2024, \ldots, 31 de octubre de 2024
\end{quote}

La estadía se alarga, compartiendo y aprendiendo entre medio, eventualmente todos los días, al también ser posible que no. Excluir lo familiar indirectamente trae esa expectativa, de que nunca pase nada, pero creerlo así resulta un tanto desconsiderado con el silencio fantemtiano y su complemento, el paisaje completo es demasiado hasta para sus seguidores especializados, no solo por atraso, además lo sería por omisión.

No obstante, allá donde tanto actualizan, o donde en su momento lo hacían, algo debería quedar, y en efecto, los remanentes están presentes. Como clon de referencias, la demanda ya sorprende menos, o si extraña deja de ser asunto interno, pero la función conectiva está activa, bilateralmente se pusieron al día, haciendo el aguante a la sensación del ámbito, revelando oportunidades adicionales de compañerismo, etcétera. Pase lo que pase con Dunna, esta como mínimo habrá tenido presencia de anoms, más evidente por las distancias funcionales que hacen su integración dinámica y aparte de los cebos que antros pares a lo mejor conocieron. A su vez, tras registrada la eventual desaparición de un vínculo conocido, siguió su caso singular en la cadena de copias locales, pues acorde a los estándares de estilos, una selección de propiedades looopperas fue integrada a la red oficial, y sin polémicas en el paso.

Tras un punto y aparte, lo propio repercute más, sobre todo al notarlo la comunidad, incluso siendo a modo parcial. Si las tendencias marcaban lo poco común de la ida y vuelta, el antro lo vuelve más complicado todavía, agregado a lo ya sabido, nuevas condiciones especiales. Tan solo trata de responder, que no es tan simple como eso, al tener la posibilidad de que cueste más, o que en su defecto no suceda. Por ello hubo más descargos negativos, reflejos de un problema medianamente insólito, Fantemti sumando características, para expandir capacidades, pero a su vez, para retornar errores. A propósito de lo discreto, el contexto cumple un rol comunicativo y pese a la obviedad, mas los contenidos corren el riesgo de no corresponder, el descuido no se reduce a evitar inconvenientes, también los genera, y que las razones logren justificar ya tiene otra complejidad, semejante como lo es solucionar.

La discusión deja más actividad para enganchar con las novedades oficiales, por ejemplo, el ave blanca y su regreso. Justo por encima de antagonismos ajenos, aquella descarta ser víctima definitiva, escalando en posiciones al tomar una más notoria que su original, sin embargo difícil creer que esto logre cambiar su consideración, validar mensajes aún depende de voluntades propensas a desalinearse. De manera similar el frío destaca, justo en plena temporada de congelamientos, sugiriendo otra incógnita que permanece latente, esa que amplifica el eco del lado infeliz que el antro esconde atrás de sus mecanismos, conjunto enemigo de anoms que pretendan operar bajo temperaturas contrarias. Concierne a la intensidad, cuestión que combinada con la numerosidad, tiene efectos mayores que personajes concretos, por experiencia aunque breve, un ejército de tapires puede ejemplificarlo. En general, el crecimiento relativo de las interacciones es un factor clave para los posibles distintivos, y no todo son hileras de observadores, tal vez los resten puedan seguir encontrando trabajo.

\chapter{Lo único que nunca cambió (Desenlace)}\label{lo-uxfanico-que-nunca-cambiuxf3-desenlace}

¿Y\ldots?

Luego transcurrir tanto o cuanto tiempo posterior al último segundo comprendido entre comillas formalmente por la más reciente edición de Diario Temtiano, se puede decir con toda certeza que ahora mismo su ubicación temporal, sin tener que especificarle una fecha concreta, se encuentra muy bien definida. Entre medio de la prolongación y de su conclusión, allí en ese ambiguo instante está. Qué tan cerca de uno u otro, eso lo estaría anticipando el contexto, aunque considerando al particular pero a veces predecible misterio que caracteriza a estas crónicas en su conjunto, los pronósticos están sujetos a romperse.

Por lo que como en contadas y numerosas ocasiones ya ha sucedido, con la intención de seguir realizando su labor, cumpliendo su responsabilidad, manteniendo viva la tradición, y saciando su deseo de superación, sin saber con precisión cuando pero con algunos indicadores que tendrían que comunicarlo y aproximarlo\ldots{} Diario Temtiano podría volver a hacer una aparición en el marco dónde y cómo la situación lo amerite, y así renovarse una vez más.

Pero también esta podría ser la última vez, porque bien es cierto que entre lo amplio que es el pasado, el presente y el futuro, existe el momento en el que Diario Temtiano se estancó y chocó con barreras temporales mientras intentaba aferrarse sus propósitos. Quizás ese singular acontecimiento se caracterice por ser bastante explícito, ya sea por circunstancias narradas dentro de la obra, por cualquier episodio experimentado o por vivir que no esté en ella pero que sí la afecte, por algo más peculiar aún, o a lo mejor es factible que no haya o vaya a haber claridad respecto a cuándo, cómo, ni porqué. Y para no dejar de serle fiel a uno de los principios temtianos más fuertes y duraderos, más allá de como pueda llegar a ser, el desenlace, por llamarlo de alguna forma, es abierto, inconcluso, porque luego hay más.

Esa o esta última aparición vigente y con vida en la escena correspondiente no significa ni atenta contra la existencia de la Historia Temtiana, pues luego de esa puntual ocasión, su camino seguirá siendo recorrido eternamente, solo que ya sin la misma retardada compañía. Ni durante la bonanza más prospera, ni cuando se cree que se está viviendo la coyuntura más crítica de todas, nunca la continuidad de la Historia Temtiana corre peligro de extinción, siempre de alguna manera u otra esta se continuará desarrollando. Para conocer cómo, cuándo, quiénes o qué, y por qué, será necesario estar ahí. Y eso es lo más importante, porque en definitiva Diario Temtiano trata de ella, no de sí mismo: es un elemento más contenido en el interior de lo tan indescriptible que es.

En síntesis, no es un tema más, el tiempo no sabe terminarlo. Aún así, después de tanto texto, incluidas reiteraciones más reiteraciones entre idas y vueltas, es posible que no solo la pregunta de qué prosigue no esté respondida correctamente, sino que también seguramente nada esté quedando claro, pero no importa, lo fundamental es lo siguiente: no existe punto final que de cierre a la Historia Temtiana.

\end{document}
